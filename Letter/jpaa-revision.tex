\documentclass[reqno,10pt,a4paper,oneside,draft]{amsart}

\setcounter{tocdepth}{1}
\usepackage[parfill]{parskip}

\usepackage{uniform-kan-prelude}

\begin{document}

\title{Comments on the revision} 

\author{Nicola Gambino \and Christian Sattler} 

\maketitle

We are very grateful to the referee for the comments on our paper. We took them into consideration and revised the paper accordingly.  Please see below for a description of the changes that have been made.

{\bf General changes}
\begin{itemize} 
\item We changed the title of the paper to ``The Frobenius condition, right properness and uniform fibrations'', so as to reflect the new organization of
the paper and the increased focus on right properness. 
\item We reorganized the paper as suggested. We start 
by fixing the setting (Section 1). After that, two main parts: the first (Sections 2 and 3) treats the
non-algebraic case and the second (Section 4 to 7) treats the non-algebraic case.
We end the paper comparing algebraic and non-algebraic notions (Section 8). 
\item The paper is now shorter (35 pages, in contrast with 45 of the original submission).
\item Introduction: this has been revised so as to reflect the changes in the body of the paper.
\item We have modified some of our notation to help the readers' intuition. For example, we now use $(\mathsf{Cof}, \mathsf{TrivFib})$ and $(\mathsf{TrivFib}, \mathsf{Fib})$ to denote
certain weak factorization systems, so as to suggest the analogy with the two weak factorisation systems of a model category.
\end{itemize}


{\bf Changes in sections} 
\begin{itemize}
\item Section 1: the setting for the development of the paper is now fixed once and for all in the first technical section of the paper. 
\item Section 2: we introduced the definition of the notion of a `suitable' (algebraic) weak factorisation system (Definition 2.1 and Definition 6.1, respectively). These encapsulate the necessary 
assumptions needed to define (uniform) fibrations from (uniform) cofibrations. This allows us also separate the problem of constructing examples of weak factorisation systems
of `cofibrations' and `trivial fibrations' (which is now discussed  in Example 2.2 for the non-algebraic setting and in Theorem 8.1 for the algebraic setting) from the main development of the paper, which is devoted to constructing (algebraic) weak factorisation systems of
`trivial cofibrations' and `fibrations' (Proposition 2.6 and Theorem 6.5) and 
establishing the 
Frobenius property for them (Theorem 3.8 and Theorem 7.6).
\item Section 3: this section now contains the proof of the Frobenius condition in the non-algebraic setting. The proof of the key lemma (Lemma 3.7), which was in an appendix in the original submission,  is now included in the body of the paper.
\item Section 4: we made some minor revisions to the material on categories of orthogonal maps;
in particular, we made the presentation a bit shorter and removing a few minor 
lemmas which were not  used  in the remainder of the paper.
\item Section 5: the material on the functorial Frobenius condition  has been throughly revised, so as to avoid multiplying entities beyond necessity (as suggested in the referee report). We only work with the fundamental notion (functorial Frobenius condition) and a technically useful notion (generalized functorial Frobenius condition). This allowed us to streamline the presentation 
and make it clearer what the important notions are and what the main result is. 
\item Sections 6 and 7: the development of the results on uniform fibrations 
is now done so as to mirror the non-algebraic case. This allows us to omit the 
the proofs that are essentially analogous to the non-algebraic ones and 
give more explanations of the proofs that are different.
\item Section 8: the comparison between algebraic and non-algebraic notions is now at the end of the paper. The main result has been simplified greatly thanks to a
suggestion of Thierry Coquand and Andr\'e Joyal. 
\end{itemize}


\end{document}
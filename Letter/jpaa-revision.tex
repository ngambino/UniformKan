\documentclass[reqno,10pt,a4paper,oneside,draft]{amsart}

\setcounter{tocdepth}{1}
\usepackage[parfill]{parskip}

\usepackage{uniform-kan-prelude}

\begin{document}

\title{Comments on the revision} 

\author{Nicola Gambino \and Christian Sattler} 

\maketitle

We are very grateful to the referee for the comments on our paper. We took them into consideration and revised the paper accordingly. We also benefited from
discussions with colleagues, especially Thierry Coquand, over the summer. We believe that the revised version is
more widely accessible and reads better than the original one. 

Please see below for a description of the changes we made.

\begin{itemize}
\item The setting for the development of the paper is now fixed once and for all in the first technical section of the paper (section 2). 
\item The proof of the Frobenius condition in the non-algebraic setting is now presented in the first part of the paper (sections 3 and 4). 
The key lemma, which was in an appendix in the original submission, has now been moved inside the main body of the paper (it appears as Lemma 4.7). 
\item We made some minor revisions to the material on categories of orthogonal maps (now in Section 5). We made the presentation a bit briefer and removing a few minor 
lemmas which were not actually used  in the remainder of the paper.
\item The material on the functorial Frobenius condition (section 6) has been throughly revised, so as to make the presentation more streamlined, and making it 
clearer what the important notions are and what the main result is. 
\item The development of the theory in the algebraic case (sections 7 and 8) is now done so as to mirror the non-algebraic case. This allows us to omit the 
the proofs that are essentially analogous to the non-algebraic ones and give more explanations of the proofs that are different.
\item The comparison between algebraic and non-algebraic notions is now at the end of the paper (section 9). The main result has been simplified greatly thanks to a
suggestion of Thierry Coquand. 
\item The introduction (section 1) has been revised so as to reflect the changes. 
\item We would like  to change the title of the paper to ``The Frobenius condition, right properness, and uniform fibrations'', so as to reflect the new organization of
the paper and the increased focus on right properness. We hope that this acceptable.
\item In various places, we have modified the notation. For example, we now use $(\mathsf{Cof}, \mathsf{TrivFib})$ and $(\mathsf{TrivFib}, \mathsf{Fib})$ to denote
certain weak factorization systems, so as to suggest the analogy with the two weak factorisation systems of a model category.
\end{itemize}


\end{document}
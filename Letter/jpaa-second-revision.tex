\documentclass[reqno,10pt,a4paper,oneside,draft]{amsart}

\setcounter{tocdepth}{1}
\usepackage[parfill]{parskip}

\usepackage{uniform-kan-prelude}

\begin{document}

\title{Comments on the 2nd revision} 

\author{Nicola Gambino \and Christian Sattler} 

\maketitle

We have addressed all the comments made by the referee. Below, we provide some explanation for the most substantial changes that have been made. 

\begin{itemize}
\item {\bf Comments preceeding Proposition 5.6.} The referee wrote that the claim did not obviously follow from our Proposition 4.7, as stated in the paper, and that the proof
needed instead a combination of 
Proposition 21 in [7] and Proposition 4.3. 

But we believe that this is not necessary, so we added some further explanations to justify the claim.
\item {\bf Comments on the proof of Proposition 5.7.} The referee wrote that also the claim did not obviously follow from our Proposition 4.7, as stated in the paper. 

TO BE ADDED.

\item {\bf Comments on Theorem 6.4.} The referee noted a mistake in the proof of Theorem~6.4 (Theorem 6.5 in the previous version), where we 
confused the category of coalgebras for a comonad with the category of coalgebras for its underlying copointed endofunctor. 

We fixed this mistake by modifying the definition of a uniform fibration (Definition 6.3), which is now given in terms of a set of generating trivial cofibrations, rather than of
the class of cylinder inclusions. The generating trivial cofibrations, in turn, are obtained by Leibniz product between endpoint inclusions and generating cofibrations. 

In this way, the uniform fibrations are the algebras for a monad and they match Coquand's uniform fibrations in cubical sets. The statement
of Theorem 6.4, its proof and the rest of the paper have been modified accordingly. 

In order to keep the symmetry between the non-algebraic and the algebraic part of the paper,
we  revised  the non-algebraic part of the paper in a similar way. 

An alternative way of fixing the mistake would have been to first obtain the awfs $(\mathsf{C_t}, \mathsf{F})$ of Theorem 6.4 (along the lines of the proof of Theorem 6.5 in the version of the paper), and then define uniform fibrations as the algebras for the pointed endofunctor $\mathsf{F}$.
But the resulting notion of a uniform fibration would not match the one in Coquand's work.



\end{itemize}

\end{document}
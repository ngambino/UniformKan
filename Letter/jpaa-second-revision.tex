\documentclass[reqno,10pt,a4paper,oneside,draft]{amsart}

\setcounter{tocdepth}{1}
\usepackage[parfill]{parskip}

\usepackage{uniform-kan-prelude}

\begin{document}

\title{Comments on the second revision} 

\author{Nicola Gambino \and Christian Sattler} 

\maketitle

We have addressed all the comments made by the referee and made some further changes  which, we believe, improve the
paper. Below, we provide some explanation for the most substantial changes that have been made. 

\section*{Proposition 5.6 and Proposition 5.7}

The referee wrote that Proposition 5.6  did not obviously follow from our Proposition 4.7, as stated in the paper, and that the proof needed instead a combination of Proposition 21 in [7] and Proposition 4.3.
The claim concerns a certain lift of the map $f^* \co \cal{E}_{/Y}^\to \to \cal{E}_{/X}^\to$, given a certain lift of the map $f_! \co \cal{E}_{/X}^\to \to \cal{E}_{/Y}^\to$.

We believe that our application of Proposition 4.7 is sufficient.
Note that $\cal{E}_{/X}^\to$ and $\cal{E}_{/Y}^\to$ can be viewed as the arrows categories of $\cal{E}_{/X}$ and $\cal{E}_{/Y}$, respectively, and the above maps $f_!$ and $f^*$ coincide with the functors $f_! \co \cal{E}_{/X} \to \cal{E}_{/Y}$ and $f^* \co \cal{E}_{/Y} \to \cal{E}_{/X}$ lifted to arrow categories (we have added a clarifying sentence to the paper).
This is precisely what is required for an application of Proposition 4.7 to get the needed lift.

Similarly, the referee wrote that also Proposition 5.7  did not obviously follow from our Proposition 4.7, as stated in the paper.
This analogous to the previous item, only concerning a lift of the map $(v_j)_* \co \cal{E}_{/C_j}^\to \to \cal{E}_{/D_j}^\to$ to a map $\liftr{\cal{K}}_{/C_j} \to \liftr{\cal{I}}_{D_j}$.
We believe that our application of Proposition 4.7 is sufficient, for essentially the same reasons as outlined in the preceeding item.

\section*{Sections 6 and 7} 

The referee noted a mistake in the proof of Theorem~6.5 where we 
confused the category of coalgebras for a comonad with the category of coalgebras for its underlying copointed endofunctor. 

We fixed this mistake by modifying the definition of a uniform fibration (Definition 6.3), which is now given in terms of a set of generating trivial cofibrations, rather than of
the class of cylinder inclusions. The generating trivial cofibrations, in turn, are obtained by Leibniz product between endpoint inclusions and generating cofibrations. 

In this way, the uniform fibrations are the algebras for a monad and they match Coquand's uniform fibrations in cubical sets. 
An alternative way of fixing the mistake would have been to first obtain the awfs $(\mathsf{C_t}, \mathsf{F})$ of Theorem 6.4 (along the lines of the proof of Theorem 6.5 in the version of the paper), and then define uniform fibrations as the algebras for the pointed endofunctor $\mathsf{F}$.
But the resulting notion of a uniform fibration would not match the one in Coquand's work.


The statement of Theorem 6.4, its proof and the rest of the paper have been modified accordingly, as detailed below. 

A related notable change is the following. Previously, the property that the Leibniz product of an endoint inclusion with a cofibration is
a trivial cofibration was immediate from the definition of a trivial cofibration. But now this has to be proved explicitly, and we do
so in Corollary 6.7. Writing down the proof of Corollary 6.7 required us to make explicit the functorial aspects of Proposition 6.4 
(previously stated only as a logical equivalence). For this, we make use of Proposition 4.11, Corollary 4.12 and Remark 4.13,
which expand on the old Remark 4.10, making explicit its use (which was unclear in the previous version of the paper). We
also added Corollary 4.10 (which implies Corollary 4.12). 

A point related to this is the new version of Proposition 7.5, which now refers to the generating trivial cofibrations instead of the
open cylinder inclusions.

With this notion of uniform trivial fibration and uniform fibration, there is no change needed in Theorem 8.1. 

\section*{Sections 2 and 3} 

In order to keep the symmetry between the non-algebraic and the algebraic part of the paper,
we  revised  the non-algebraic part of the paper in a similar way. 







\section*{Other changes} 


\begin{itemize}
\item
We simplified the definition of a suitable awfs slightly: condition (S3) has been replaced by an easier condition directly equivalent to it.
\end{itemize}

\end{document}
\documentclass[reqno,10pt,a4paper,oneside]{amsart}

\usepackage{uniform-kan-prelude}

\begin{document}

\section*{Potential Generating Left Classes}

In the following, we will examine different choices for the subcategory of generating left maps $\cal{I} \subseteq \SSet^{\to}$.
Objects in $\cal{I}$ will always at least be classical trivial cofibration, and hence also monomorphisms.
Most of the time, we will actually choose $\cal{I}$ as a subcategory of $\SSet_{\text{cart}}^{\to}$ consisting of arrows and \emph{cartesian} commutative squares for reasons now explained.

\subsection*{The problem with non-cartesian choices of $\cal{I}$}

Consider a non-cartesian morphism $j \to k$ in $\cal{I}$
\[
\xymatrix{
  A
  \ar[rrr]
  \ar[dd]_{j}
  \ar[dr]
&&&
  C
  \ar[dd]^{k}
\\&
  C'
  \ar[urr]
  \ar[dl]_{k'}
  \pullback{dr}
\\
  B
  \ar[rrr]
&&&
  D
}
\]
where we have drawn in the pullback for comparison.
Note that $A \to C'$ and $C' \to B$ are all monomorphisms, and that $A \to C'$ will not be invertible.

Fix a lifting problem
\[
\xymatrix{
  A
  \ar[r]
  \ar[d]_{j}
&
  X
  \ar[d]^{p}
\\
  B
  \ar[r]
  \ar@{.>}[ur]^{d}
&
  Y
}
\]
for a generating left map $j : A \to B$ and a right map $p : X \to Y$.
The algebraically generated solution is denoted $d$.
For any factoring
\[
\xymatrix{
  A
  \ar[r]
  \ar[d]_{j}
&
  C
  \ar[r]
  \ar[d]_(0.35){k}
&
  X
  \ar[d]^{p}
\\
  B
  \ar[r]
  \ar@{.>}[urr]^(0.3){d}
&
  D
  \ar[r]
  \ar@{.>}[ur]_{e}
&
  Y
}
\]
through $j \to k$ of this lifting problem, we have coherence of the algebraically generated lifts $d$ and $e$ as indicated.
It follows that $d$ is a lift for the inner square in
\[
\xymatrix{
  A
  \ar[rrr]
  \ar[dd]_{j}
  \ar[dr]
&&&
  X
  \ar[dd]^{p}
\\&
  C'
  \ar[urr]
  \ar[dl]_{k'}
\\
  B
  \ar[rrr]
  \ar@{.>}@/_2em/[uurrr]_{d}
&&&
  Y
}
\]
as well.

If $B \to D$ is a monomorphism, it is usually part of a section-retraction pair (for example, if $B$ and $D$ are both representable).
Given \emph{any} solution $d'$ to the above lifting problem $j \pitchfork p$, this can be used to define a factoring of $j \to p$ through $j \to k$ as above, using $d'$ to define the map $C \to X$.
\end{document}

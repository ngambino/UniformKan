\documentclass[reqno,10pt,a4paper,oneside]{amsart}
\usepackage{amssymb,amsmath,amsthm,stmaryrd,enumerate,geometry}
\usepackage[all]{xy} 

\SelectTips{cm}{}
\newdir{ >}{{}*!/-7pt/@{>}}
\newdir{m}{->}
%\newdir{m}{{}*!/-1pt/@{o}}
\newcommand{\xycenter}[1]{\vcenter{\hbox{\xymatrix{#1}}}}
%%% Pullback symbols
\newcommand{\ulpullback}[1][ul]{\save*!/#1-1.2pc/#1:(-1,1)@^{|-}\restore}
\newcommand{\dlpullback}[1][dl]{\save*!/#1-1.2pc/#1:(-1,1)@^{|-}\restore}
\newcommand{\urpullback}[1][ur]{\save*!/#1-1.2pc/#1:(-1,1)@^{|-}\restore}
\newcommand{\drpullback}[1][dr]{\save*!/#1-1.2pc/#1:(-1,1)@^{|-}\restore}

\newcommand{\ie}{\text{i.e.\ }}
\newcommand{\eg}{\text{e.g.}}
\newcommand{\resp}{\text{resp.\ }}
\newcommand{\myemph}{\textit} 
\newcommand{\changed}{\todo[noline]{Changed}}


\newtheorem{theorem}{Theorem}[section]
\newtheorem*{theorem*}{Theorem}
\newtheorem{lemma}[theorem]{Lemma} 
\newtheorem{proposition}[theorem]{Proposition} 
\newtheorem{corollary}[theorem]{Corollary}  
\newtheorem{apptheorem}{Theorem}



\theoremstyle{definition}
\newtheorem{definition}[theorem]{Definition}	
\newtheorem*{definition*}{Definition}	



\newtheorem{remark}[theorem]{Remark} 
\newtheorem*{remark*}{Remark} 
\newtheorem{example}[theorem]{Example}
\newtheorem{examples}[theorem]{Examples}
\newtheorem*{example*}{Example}
\newtheorem*{examples*}{Examples}



\newcommand{\defeq}{=_{\mathrm{def}}}
\newcommand{\co}{\colon}
\newcommand{\iso}{\cong} 
\newcommand{\rev}{\mathit{\vee}}
\newcommand{\op}{\mathrm{op}}
\newcommand{\catequiv}{\simeq} 
\newcommand{\cateq}{\simeq} 
\newcommand{\coend}{\int}




\newcommand{\cat}[1]{\mathbb{#1}}
\newcommand{\catA}{\cat{A}}
\newcommand{\catB}{\cat{B}}
\newcommand{\catC}{\cat{C}}
\newcommand{\catD}{\cat{D}}
\newcommand{\catK}{\cat{K}}
\newcommand{\catM}{\cat{M}}


\newcommand{\SSet}{\mathbf{SSet}}
\newcommand{\UU}{\overline{\mathsf{U}}}
\newcommand{\U}{\mathsf{U}}
\newcommand{\Weq}{\mathrm{Eq}}

\author[]{Nicola Gambino}
\address{School of Mathematics, University of Leeds, Leeds LS2 9JT, United Kingdom}

\email{n.gambino@leeds.ac.uk}



\title[Simplicial univalence]{Notes on simplicial univalence \\ (After Voevodsky, Streicher, Joyal and Cisinski)}

\date{March 12th, 2015}



\begin{document}

\maketitle


We work with the category $\SSet$ of simplicial sets. We consider $\SSet$ as equipped with the Quillen model structure $(\mathcal{W},
\mathcal{F}, \mathcal{C})$ in which the class of weak equivalences  $\mathcal{W}$ consists of the weak homotopy equivalences, the class of fibrations $\mathcal{F}$ consists of the Kan
fibrations and the class of cofibrations $\mathcal{C}$ consists of the monomorphisms. We refer to
elements of $\mathcal{W} \cap \mathcal{F}$ as trivial fibrations and $\mathcal{W} \cap \mathcal{C}$
as trivial cofibrations. Recall that the model structure is right proper, \ie that weak equivalences are
preserved by pullback along fibrations.


\section{Preliminaries}


\begin{lemma} \label{thm:minfib} \hfill 
\begin{enumerate}[$(i)$]
\item Every fibration can be factored as a trivial fibration followed by a minimal fibration, \ie 
for every fibration $p \co X \to A$ there exists a factorisation of the form
\[
\xymatrix{
X \ar[rr]^{t} \ar[dr]_p & & X' \ar[dl]^{m} \\
 & A & }
 \]
 where $t$ is a trivial fibration and $m$ is a minimal fibration.
 \item Minimal fibrations are closed under pullback, i.e. for every pullback diagram of the
 form
 \[
 \xymatrix{
 X \ar[r]^j \ar[d]_{m} & X' \ar[d]^{m'} \\
 A \ar[r]_i  & A' }
 \]
 if $m'$ is a minimal fibration, so is $m$.
 \item Weak equivalences between minimal fibrations are isomorphisms, \ie for every 
 commutative diagram of the form
 \[
\xymatrix{
X \ar[rr]^{w} \ar[dr]_m & & X' \ar[dl]^{m'} \\
 & A & }
 \]
if $m$ and $m'$ are minimal fibrations and $w$ is a weak equivalence, then $w$ 
is an isomorphism.
\end{enumerate}
\end{lemma}

By a \emph{minimal resolution} of a fibration $p \co X \to A$ we mean a minimal fibration $m \co Y
\to A$ equipped with a trivial cofibration $i \co Y \to X$ making the following diagram commute:
\[
\xymatrix{
Y \ar[rr]^{i} \ar[dr]_{m} & & X \ar[dl]^{p} \\
 & A & }
 \]
 

\begin{lemma} \label{thm:minres} Let $p \co X \to A$ be a fibration and let
\[
\xymatrix{
Y \ar[rr]^{i} \ar[dr]_{m} & & X \ar[dl]^{p} \\
 & A & }
 \]
 be a minimal resolution of $p$. Every retraction of the trivial cofibration $i$, \ie every map $r \co X \to Y$ such that
\[
\xymatrix{
Y \ar[r]^{i} \ar@/_1pc/[dr]_{1_M} & X \ar[d]^r \\
 & Y }
 \]
commutes, is a trivial fibration.
 \end{lemma}

Note that, given a minimal resolution of a fibration, a retraction $r$  as in Lemma~\ref{thm:minres} exists always, since it can be obtained as
a diagonal filler of the diagram
\[
\xymatrix{
Y \ar@{=}[r] \ar[d]_{i} & Y \ar[d]^{m} \\
X \ar[r]_{p} & A }
\]
since $i$ is a trivial cofibration and $m$ is a fibration. 


\begin{proposition}[Minimal fibrations can be extended along trivial cofibrations]
\label{thm:minfibalongtrivcof}
 For every
minimal fibration $m \co Y \to A$ and trivial cofibration $i \co A \to A'$, there exists a pullback
square
\[
\xymatrix{
Y \ar[r]^{j} \ar[d]_m & Y' \ar[d]^{m'} \\
A \ar[r]_{i} & A'}
\]
where $m'$ is a minimal fibration. 
\end{proposition}

\begin{proof} By the model category axioms, we can factor the composite $i \circ m \co Y \to A'$ as
a trivial cofibration followed by a fibration, \ie there exists a diagram
\[
\xymatrix{
Y \ar[d]_{m} \ar[r]^{u}  & Z \ar[d]^{p} \\
A \ar[r]_{i} & A'}
\]
where $u$ is a trivial cofibration and $p$ is a fibration. By Lemma~\ref{thm:minfib}, we can factor the 
fibration~$p \co Z \to A'$
as a trivial fibration followed by a minimal fibration, \ie there exists a diagram
\[
\xymatrix{
Z \ar[rr]^t \ar[dr]_{p}  & &  Y' \ar[dl]^{m'} \\
  & A' & }
  \]
  where $t$ is a trivial fibration and $m'$ is a minimal fibration. Thus, we get the following diagram:
  \begin{equation}
  \label{equ:cart}
\xycenter{
Y \ar[r]^{u} \ar[d]_m & Z \ar[r]^{t} & Y' \ar[d]^{m'} \\
A \ar[rr]_{i} & & A'}
\end{equation}
We claim that this gives us the required diagram (by letting $j = t \circ u$). For this, we only need to
 show that the diagram is pullback. For this, take the pullback of $m'$ along $i$, we have the diagram
\[
\xymatrix{
Y \ar@/^1pc/[drr]^{t \circ u} \ar@/_1pc/[ddr]_m  \ar@{.>}[dr]^{w} &  & \\
 & Y'' \ar[r]^(.5){d} \ar[d]^-{m''}  & Y' \ar[d]^{m'}  \\
  & A \ar[r]_{i} & A'}
  \]
  Since $m'' \co Y'' \to A$ is the pullback of a minimal fibration, it is a minimal fibration by Lemma~\ref{thm:minfib}.   
 Next, note $t \circ u$ is a weak equivalence (since it is the composite of the trivial cofibration $u$ and the trivial fibration $t$). Since $i$, being a trivial cofibration, is a weak equivalence and the model structure is right proper, $d \co Y'' \to Y'$ (being the pullback of the weak equivalence $i$ along the fibration $m'$) is a weak equivalence. Therefore, by 
 the 3-for-2 property of weak equivalences, the canonical map $w \co Y \to Y''$ is a weak equivalence. Thus, we have a diagram
\[
\xymatrix{
Y \ar[rr]^{w} \ar[dr]_m & & Y'' \ar[dl]^{m''} \\
 & A & }
 \]
where $m$ and $m''$ are minimal fibrations and $w$ is a weak equivalence. Hence, by Lemma~\ref{thm:minfib}, $w$ is an
isomorphism and hence the square in~\eqref{equ:cart} is a pullback square, as required.
\end{proof}


\begin{lemma} \label{thm:trick} Let $i \co A \to A'$ be a map of simplicial sets.
\begin{enumerate}[(i)]
\item The functor $i_* \co \SSet/A \to \SSet/A'$ preserves trivial fibrations.
\item If $i$ is a cofibration, then the counit of the adjunction $i^* \dashv i_*$, 
\[
\varepsilon \co i^* i_* \Rightarrow 1_{\SSet/A'}
\]
is a natural isomorphism.
\end{enumerate}
\end{lemma}

\begin{proof} \hfill
\begin{enumerate}[(i)]
\item By adjointness, since $i^*$ preserves cofibrations.
\item If $i$ is a cofibration, $i_{!} \co \SSet/A \to \SSet/A'$ is fully faithful, so 
$i^* \circ i_{!} \iso 1_{\SSet/A}$. By adjointness, $i^* i_* \iso 1_{\SSet/A'}$. \qedhere
\end{enumerate}
\end{proof}

\begin{proposition}[Trivial fibrations can be extended along cofibrations]
 \label{thm:trivfibalongcof}
For every
trivial fibration $t \co X \to Y$ and every cofibration $j \co Y \to Y'$ there exists a pullback
square
\[
\xymatrix{
X \ar[r]^k  \ar[d]_t & X' \ar[d]^{t'} \\
Y \ar[r]_{j} & Y'}
\]
where $t' \co X' \to Y'$ is a trivial fibration. 
\end{proposition}

\begin{proof} Define $t' = j_*(t)$. Since $t$ is a assumed to be a trivial fibration, 
by part (i) of Lemma~\ref{thm:trick}, also $t'$ is a trivial fibration. The required pullback square 
exists since 
\[
j^*(t') = j^* j_* (t) \iso t 
\]
where the last isomorphism is given by part (ii) of Lemma~\ref{thm:trick}.
\end{proof}

\begin{proposition}[Fibrations can be extended along trivial cofibrations] \label{thm:joyal}  For every
fibration $p \co X \to A$ and trivial cofibration $i  \co A \to A'$ there exists a a pullback
square
\[
\xymatrix{
X \ar[r]^k \ar[d]_p & X' \ar[d]^{p'} \\
A \ar[r]_i & A'}
\]
where $p' \co X' \to A'$ is a fibration.
\end{proposition}

\begin{proof} By Lemma~\ref{thm:minfib}, we can factor $p \co X \to A$ as a trivial fibration followed by a minimal
fibration, \ie there exists a diagram
\begin{equation*}
\xycenter{
X \ar[rr]^{t} \ar[dr]_p & & Y \ar[dl]^{m} \\
 & A & }
 \end{equation*}
 where $t$ is a trivial fibration and $m$ is a minimal fibration. By Proposition~\ref{thm:minfibalongtrivcof}, we can extend the minimal fibration
 $m$ along the trivial cofibration $u$, so as to get a pullback diagram
 \begin{equation}
 \label{equ:cart1}
 \xycenter{
 Y \ar[d]_{m} \ar[r]^{j} & Y' \ar[d]^{m'} \\
 A \ar[r]_i & A'}
 \end{equation}
 where $m'$ is a trivial fibration. Note that $j$ is a cofibration since cofibrations (being the 
 monomorphisms) are closed under pullback.  By Proposition~\ref{thm:trivfibalongcof}, we can  extend the trivial fibration $t$ along the cofibration $j$,
 so as to get a pullback diagram
 \begin{equation}
  \label{equ:cart2}
 \xycenter{
 X \ar[r]^k \ar[d]_t & X' \ar[d]^{t'} \\
 Y \ar[r]_{j} & Y'}
 \end{equation}
 By pasting the diagrams in~\eqref{equ:cart1} and~\eqref{equ:cart2}, we obtain the 
 required diagram:
 \[
 \xymatrix{
 X \ar[r]^k \ar[d]_t & X' \ar[d]^{t'} \\
 Y \ar[r]_{j} \ar[d]_{m} & Y' \ar[d]^{m'} \\ 
 A \ar[r]_i & A'}
 \]
since $p = m \circ t$. Note that $m' \circ t' \co Y \to A'$ is a fibration since it is the composite of
the trivial fibration $t'$ and the minimal fibration $m'$.
\end{proof}


\begin{proposition}[Weak equivalences between fibrations can be extended along cofibrations]
\label{thm:weakalongcof}
For every diagram
 \[
 \xymatrix{
 X_1 \ar[ddr]_{p_1}  \ar[drr]^{w}  &  & &  & & \\
  & & X_2 \ar[dl]^{p_2} \ar[rrr]^(.35){u_2} & & &  X_2' \ar[dl]^(.35){p'_2} \\
  & A \ar[rrr]_{i}  & & &A' & & }
  \]
  where $p_1$ and $p_2$ are fibrations, $w$ is a weak equivalence and $i \co A \to A'$ is a cofibration,  
  there exists a diagram
   \[
 \xymatrix{
 X_1 \ar[ddr]_{p_1}  \ar[drr]^{w} \ar[rrr]^{u_1} &  & & X_1' \ar[drr]^{w'}  \ar@{.>}[ddr]^(.7){p_1'}  & & \\
  & & X_2 \ar[dl]^{p_2} \ar[rrr]^(.35){u_2} & & &  X_2' \ar[dl]^(.35){p'_2} \\
  & A \ar[rrr]_{i}  & & &A' & & }
  \]
where $p'_1$ is a fibration, $w'$ is weak equivalence and all squares are pullbacks.
\end{proposition}

\begin{proof} Let us consider the pullback diagram
\begin{equation}
\label{equ:frontface}
{\vcenter{\hbox{\xymatrix@C=2cm{
X_2 \ar[r]^{u_2} \ar[d]_{p_2} & X'_2 \ar[d]^{p'_2} \\
A \ar[r]_{i} & A'}}}}
\end{equation}
 In order to construct the required diagram, we begin by 
 factoring the fibration $p_2' \co X_2' \to A'$ as  a trivial fibration followed by a minimal fibration, so that we have
a diagram
\[
\xymatrix{
X_2' \ar[rr]^{t_2'} \ar[dr]_{p_2'} & & Y' \ar[dl]^{m'} \\
 & A' & }
 \]
 where $t_2'$ is a trivial fibration and $m_2'$ is a minimal fibration. Then, we form the pullback square
 \[
 \xymatrix@C=2cm{
 Y \ar[r]^j \ar[d]_{m} & Y'  \ar[d]^{m'} \\
  A \ar[r]_{i} & A'}
 \]
Since $m'$ is a minimal fibration, by Lemma~\ref{thm:minfib}, also $m$ is a minimal fibration. We can then decompose the pullback  in~\eqref{equ:frontface} as follows:
 \[
 \xymatrix@C=2cm@R=0.8cm{
 X_2 \ar[r]^{u_2} \ar[d]_{t_2} & X'_2 \ar[d]^{t'_2} \\
  Y \ar[r]^{j} \ar[d]_{m} & Y' \ar[d]^{m'} \\ 
  A \ar[r]_{i} & A'}
 \]
Note that $t_2 \co X_2 \to Y$ is a trivial fibration, since it is the pullback of the trivial fibration $t'_2$. Now define
$t_1 \co X_1 \to Y$ as the composite of $t_2 \co X_2 \to Y$ and the weak equivalence $w \co X_1 \to X_2$  
that is part of the given data:
\[
t_1 = w \circ t_2 \, .
\]

 \medskip

We claim that $t_1 \co X_1 \to Y$ is a trivial fibration. 
Assuming that this is the case, the proof can be completed as follows. First, we apply Proposition~\ref{thm:trivfibalongcof}, to extend the trivial fibration $t_1$ along the cofibration $j$ and obtain a pullback diagram
\[
\xymatrix@C=2cm{
X_1 \ar[r]^{u_1} \ar[d]_{t_1} & X_1' \ar[d]^{t'_1} \\
Y \ar[r]_j & Y'}
\]
where $t_1'$ is a trivial fibration. Secondly, we obtain $w'$ as the diagonal filler of the following diagram
\[
\xymatrix{
X_1 \ar[r]^{w} \ar[d]_{u_1} & X_2 \ar[r]^{u_2} & X_2' \ar[d]^{t'_2} \\ 
X'_1 \ar[rr]_{t'_1} \ar@{.>}[urr]^(.42){w'} & & Y'} 
 \]
 Here, $u_1$ is a cofibration since it is a pullback of the cofibration  $j$, while $t'_2$ is a trivial fibration by
 construction. Observe that $w'$ is a weak equivalence by the 3-for-2 property since $t'_1$ and $t'_2$ are
 weak equivalences (since they are trivial fibrations). Putting all this data together, we obtain the required
 diagram as
 \[
 \xymatrix{
 X_1 \ar[ddr]_{t_1}  \ar[drr]^{w} \ar[rrr]^{u_1} &  & & X_1' \ar[drr]^{w'}  \ar@{.>}[ddr]^(.7){t_1'}  & & \\
  & & X_2 \ar[dl]^{t_2} \ar[rrr]^(.35){u_2} & & &  X_2' \ar[dl]^(.35){t'_2} \\
  & Y \ar[rrr]_j  \ar[d]_m  & & & Y' \ar[d]^{m'} & & \\
  & A \ar[rrr]_{i}  & & &A' & & }
  \]
  
  \medskip
 
 It remains to prove that $t_1 \co X_1 \to Y$ is a trivial fibration. For this, let 
 \[
 \xymatrix{
 M \ar[dr]_{m''} \ar[rr]^{u} & & X_1 \ar[dl]^{p_1} \\
  & A & }
  \]
  be a minimal resolution of $p_1$.      If we consider the diagram
  \[
  \xymatrix{
  M \ar[dr]_{m''} \ar[r]^{u} & X_1 \ar[r]^{t_1} \ar[d]^{p_1} & Y \ar[dl]^{m} \\
   & A &   }
   \]
   we have that $t_1 \circ u$ is a weak equivalence between minimal fibrations and so it is
   an isomorphism by Lemma~\ref{thm:minfib}. If $v \co Y \to M$ is the inverse of $t_1  \circ u$,
   we have the diagram
   \[
   \xymatrix{
   M \ar[r]^u \ar@/_1pc/[ddr]_{1_M} & X_1 \ar[d]^{t_1} \\
     & Y \ar[d]^{v} \\
     & M}
     \]
which shows that $v \circ t_1$ is a retraction of $u$. Since $u \co M \to X_1$ 
is a minimal resolution, by Lemma~\ref{thm:minres} we have that $v \circ t_1$ is
a trivial fibration. But now the diagram 
 \[
 \xymatrix{
 X_1 \ar@{=}[r] \ar[dd]_{t_1} & X_1 \ar[d]^{t_1} \\
  & Y \ar[d]^{v} \\
  Y \ar[r]_v & M }
  \]
shows that $t_1$ is isomorphic a trivial
fibration, and hence so is $t_1$.
\end{proof}


\section{Universal fibrations and univalence}

% \newcommand{\catC}{\mathbb{C}}
\newcommand{\pshC}{\widehat{\catC}}
\newcommand{\Set}{\mathbf{Set}}
\newcommand{\yon}{\operatorname{y}}

We review the construction of a classifier for small fibrations, exploiting 
work of Hoffmann and Streicher. 
Recall that for every simplicial set $X$ we have an equivalence of categories
\[
\textstyle
\SSet/X \simeq \widehat{\int X} \, ,
\]
where $\SSet/X$ denotes the slice category of $\SSet$ over $X$ and 
$\int X$ is  the category of elements of $X$. When $X = \Delta[n]$,
we have and equivalence of categories $\int \Delta[n] \simeq \Delta /[n]$ and therefore we have
\begin{equation}
\label{equ:equivslice}
\SSet/ \Delta[n] \simeq \widehat{\Delta/[n]} \, .
\end{equation}
Let us now fix a cardinal $\kappa$. We say that a map of simplicial sets is \emph{small} if and only if all its
fibers have cardinality less or equal to $\kappa$. We wish to construct a classifier for small maps,
i.e. a map of  simplicial sets $\overline{\mathsf{V}} \to \mathsf{V}$ such that for every small map $p \co Y \to X$ there 
exists a pullback diagram of the form
\[
\xymatrix{
B \ar[r] \ar[d]_{f} \drpullback & \overline{\mathsf{V}} \ar[d]^{v} \\
A \ar[r]_{\chi} & \mathsf{V}}
\]
A first attempt would be to define $\mathsf{V}$ by mapping $[n] \in \Delta$ to the set of presheaves over $\Delta[n]$
with fibers of cardinality less or equal to $\kappa$, with restriction given by pullbacks. This idea suffers
from the problem that it is not strictly functorial. To remedy this, we exploit the equivalence in~\eqref{equ:equivslice}
and define
\[
\mathsf{V}_n  \defeq [ (\Delta/[n])^\op \, ,  \; \Set_\kappa] \, ,
\]
where $\Set_\kappa$ is the full subcategory of $\Set$ spanned by sets of cardinality less or equal to $\kappa$. We then let $\overline{\mathsf{V}}_n$ to be the set of pairs $(F, x)$ consisting of an element $F \in \mathsf{V}_n$ and an
element~$x \in F(\mathrm{id}_{[n]})$. We write $\U$ 
for the simplicial subset  of $\mathsf{V}$ whose value at $[n]$ consists of all the presheaves $F \co  (\Delta/[n])^\op \to \Set_\kappa$ such that the map~$p \co X \to \Delta[n]$ corresponding to $F$ under the equivalence in~\eqref{equ:equivslice} is a Kan fibration. We then define the map $u \co \UU \to \U$ via the 
pullback diagram
\[
\xymatrix{
\UU \ar[r] \ar[d]_{\pi} &  \overline{\mathsf{V}} \ar[d]^{v}  \\
\U \ar[r]_{i}  & \mathsf{V} }
\]
where $i$ is the inclusion.




\begin{theorem} The map $\pi \co \UU \to \U$ is a fibration.
\end{theorem}

\begin{proof} This follows immediately by Proposition~\ref{thm:classifier}, but we 
spell out the details by means of illustration. We need to find a diagonal filler for the square
\begin{equation}
\label{equ:given}
\xycenter{
\Lambda^k[n] \ar[r] \ar[d]_i & \UU \ar[d]^\sigma \\
\Delta[n] \ar[r]_f & \UU }
\end{equation}
We can factor this diagram via the pullback of $e$ along $f$, as follows:
\[
\xymatrix{
\Lambda^k[n] \ar[r] \ar[d] & X \ar[r] \ar[d]^p &  \UU \ar[d]^\sigma \\ 
\Delta[n] \ar@{=}[r] & \Delta[n] \ar[r]_f & U }
\]
By the very definition of $\sigma \co \UU \to U$, the map $p \co X \to \Delta[n]$ is a fibration. 
Since  $i  \co \Lambda^k[n] \to \Delta[n]$ is a trivial cofibration, by
the model category axioms, we can find a diagonal filler 
\[
\xymatrix{
  \Lambda^k[n]   \ar[r] \ar[d]_i &  X \ar[d]^e \\ 
 \Delta[n]  \ar@{=}[r] \ar@{.>}[ur] & \Delta[n] }
\]
This diagonal filler can be used in an evident way to define a diagonal filler for the diagram in~\eqref{equ:given}.
\end{proof}


\begin{proposition} \label{thm:classifier} \hfill
\begin{enumerate}[(i)]
\item For every small fibration $p \co Y \to X$, there exists a pullback
\[
\xymatrix{
X \ar[r] \ar[d]_p  \drpullback & \UU \ar[d]^{u} \\
A \ar[r]_{\chi} & \U}
\]
\item (Composition)
\end{enumerate}
\end{proposition}





\begin{theorem} The simplicial set $U$ is a Kan complex.
\end{theorem}

\begin{proof} We need to find a filler
\[
\xymatrix{
\Lambda^k[n] \ar[r] \ar[d]_{i} & U \\
\Delta[n] \ar@/_1pc/@{.>}[ur] & }
\]
By the definition of $U$, this is equivalent to extending a fibration $p \co X \to \Lambda^k[n]$ 
along the inclusion $i \co \Lambda^k[n] \to \Delta[n]$, \ie to construct a diagram of the form
\[
\xymatrix{
X \ar[r] \ar[d]_{p} & X' \ar[d]^{p'} \\ 
\Lambda^k[n] \ar[r]_{i} & \Delta[n] }
\]
But this follows by Proposition~\ref{thm:joyal}, since $i  \co \Lambda^k[n] \to \Delta[n]$ is a trivial cofibration.
\end{proof}




\begin{theorem} The fibration  $p \co \UU \to U$ is univalent.
\end{theorem}

\begin{proof} We need to show that the map $w$ in the diagram
\[
\xymatrix{
U \ar[rr]^w \ar[dr]_{\Delta} & & \Weq(\UU) \ar[dl]^{(s,t)} \\
 & U \times U }
 \]
 is a weak equivalence.  By composing with the second projection $\pi_2 \co U \times U \to U$, this diagram
 becomes 
 \[
\xymatrix{
U \ar[rr]^w \ar[dr]_{1_U} & & \Weq(\UU) \ar[dl]^{t} \\
 & U  }
 \]
 Hence, by the 3-for-2 property of weak equivalences, it suffices to show that $t \co \Weq(\UU) \to U$ is
 a weak equivalence. Since $t$ is a fibration, we show that it is a trivial fibration. So, we need to prove
 the existence of a diagonal filler in a diagram of the form
 \begin{equation*}
 \xymatrix@C=2cm{
 A \ar[d]_i \ar[r]^-{w}  & \Weq(\UU) \ar[d]^t \\
 A' \ar[r]_{f} & U }
 \end{equation*}
where $i$ is a cofibration. By the definition of $t \co \Weq(\UU) \to U$, this amounts to a diagram
 \[
 \xymatrix{
 X_1 \ar[ddr]_{p_1}  \ar[drr]^{w}  &  & &  & & \\
  & & X_2 \ar[dl]^{p_2} \ar[rrr]^(.35){u_2} & & &  X_2' \ar[dl]^(.35){p'_2} \\
  & A \ar[rrr]_{i}  & & &A' & & }
  \]
  where $p_1$ and $p_2$ are fibrations, $w$ is a weak equivalence and $i \co A \to A'$ is a cofibration,  
  while the diagonal filler amounts to having a diagram of the form
 \[
 \xymatrix{
 X_1 \ar[ddr]_{p_1}  \ar[drr]^{w} \ar[rrr]^{u_1} &  & & X_1' \ar[drr]^{w'}  \ar@{.>}[ddr]^(.7){p_1'}  & & \\
  & & X_2 \ar[dl]^{p_2} \ar[rrr]^(.35){u_2} & & &  X_2' \ar[dl]^(.35){p'_2} \\
  & A \ar[rrr]_{i}  & & &A' & & }
  \]
where $p'_1$ is a fibration, $w'$ is weak equivalence and all squares are pullbacks. The claim then
follows by Proposition~\ref{thm:weakalongcof}.
 \end{proof} 





\end{document}


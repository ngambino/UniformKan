\documentclass[reqno,10pt,a4paper,oneside]{amsart}

\usepackage{uniform-kan-prelude}

\begin{document}

\begin{abstract}
  The purpose of this file is to examine the appropriate constructive rephrasing of the notion of Kan fibrations.
\end{abstract}

The purpose of this file is to examine the appropriate constructive rephrasing of the notion of Kan fibration.
We work in a constructive setting.
In classical logic, all presented enhanced notions of Kan fibrations will be equivalent to the usual one. 

\section{Abstract Leibniz Nonsense}

Recall the Leibniz construction and its properties \cite[Section 4]{riehl-verity:reedy}.
For a bifunctor $H : \catC \times \catD \to \catE$, we will denote its associated Leibniz construction by $\hat{H} : \catC^{\to} \times \catD^{\to} \to \catE^{\to}$.
A standard example is provided by the product bifunctor $\arghole \times \arghole : \catC \times \catC \to \catC$ for categories $\catC$ with finite products and finite colimits and its associated Leibniz construction, the Leibniz product $\arghole \hattimes \arghole : \catC^{\to} \times \catC^{\to} \to \catC^{\to}$ (also known as the smash product or pushout-product).
If $\catC$ has exponentials and finite limits, it is closed with pointwise right adjoint given by the Leibniz exponential $\hatexp : (\catC^{\op})^{\to} \times \catC^{\to} \to \catC^{\to}$.

The names for the notions in this section are inspired by model categories which have an object $I$ such that $I \times X$ is a cylinder object and $X^I$ is a path object for any $X$.
I do not know what names to use on this abstract level.

Let $\catC$ be a category with finite products and $(I, \bot, \top)$ be a bipointed object in $\catC$.

\begin{definition}
\label{def:homotopy-equivalence}
A \emph{homotopy equivalence} consists of maps $f : A \to B$ and $g : B \to A$ together with maps $h : I \times A \to A$ and $k : I \times B \to B$ such that
\begin{itemize}
\item[(1)] $h \cc \angles{\bot \cc \canonical_{A \to 1}, \id_A} = \id_A$,
\item[(2)] $k \cc \angles{\bot \cc \canonical_{B \to 1}, \id_B} = \id_B$,
\item[(3)] $h \cc \angles{\top \cc \canonical_{A \to 1}, \id_A} = g \cc f$,
\item[(4)] $k \cc \angles{\top \cc \canonical_{B \to 1}, \id_B} = f \cc g$.
\end{itemize}
We call it \emph{strong} or \emph{co-strong} if in addition we have $f \cc h = k \cc (I \times f)$ or $g \cc k = k \cc (I \times g)$, respectively.

Such a homotopy equivalence is called a \emph{deformation retract} if $h = \pi_2$ (note that his makes $f$ and $g$ into a section-retraction pair and conditions (1) and (3) hold vacuously).\footnote{really needed?}
\end{definition}

When we speak below of a map $f : A \to B$ being a homotopy equivalence, what we always mean is that further data $g, h, k$ is present so as to comprise a proper homotopy equivalence as per \cref{def:homotopy-equivalence}.
Be aware that under this convention a deformation retract $f : A \to B$ has $f$ a section.

The notion of homotopy equivalence admits an obvious notion of duality under which the properties of strength and co-strength are interchanged.

\begin{example}
\label{de-morgan-deformation-retract}
Let $I$ be a De Morgan algebra in $\catC$.
The maps $\bot, \top : 1 \to I$ are each part of a strong and co-strong deformation retract.
\end{example}

\begin{proof}
By duality, it suffices to examine the map $\top : 1 \to I$.
Clearly, this is a section to the canonical map $\canonical_{I \to 1}$.
The remaining data is given by disjunction, noting that it has $\bot$ as left unit and $\top$ as left and right annihilator.
\end{proof}

Now let $\catC$ also have finite colimits.
Consider the following diagram (note the underlying symmetry!):
\[
\xymatrix@C+2em{
  0
  \ar[r]^{\canonical_{0 \to 1}}
  \ar[d]_{\canonical_{0 \to 1}}
  \ar[dr]^{\canonical_{0 \to A}}
&
  1
  \ar[d]^{\top}
\\
  1
  \ar[r]^{\bot}
&
  I
}
\]
Let $\theta_1 : \canonical_{0 \to 1} \to \canonical_{0 \to I}$ and $\theta_2 : \canonical_{0 \to I} \to \top$ denote the individual triangles and $\theta = \theta_2 \cc \theta_1 : \canonical_{0 \to 1} \to \top$ denote the composite square, all in the arrow category $\catC^{\to}$.

Recall that $\canonical_{0 \to 1}$ is a unit for the Leibniz product, and that Leibniz product with $\canonical_{0 \to I}$ corresponds to the objectwise product with $I$. 

\begin{lemma}
\label{strong-h-equiv-same-as-retraction}
Strong homotopy equivalences $f : A \to B$ are in bijective correspondence with retractions to the map of arrows $\theta \hattimes f$ from $\canonical_{0 \to 1} \hattimes f$ to $\top \hattimes f$.\footnote{
A strong deformation (or co-deformation) retract under this correspondence is a retraction $r$ such that the square $r \cc (\theta_2 \hattimes f) : I \times f \to f$ has the second projection as top (or bottom, respectively) morphism.
}
\end{lemma}

\begin{proof}
To give a map from $\top \hattimes i$ to $\canonical_{0 \to 1} \hattimes i$ is to give maps $\alpha : I \times A \to 1 \times A$, $\beta : 1 \times B \to 1 \times A$, and $\gamma : I \times B \to 1 \times B$ such that $\alpha \cc (\top \times A) = \beta \cc (1 \times f)$, $(1 \times f) \cc \alpha = \gamma \cc (I \times f)$, and $(1 \times f) \cc \beta = \gamma \cc (\top \times B)$.
Constituting a section to $\theta \hattimes f$ means $\alpha \cc (\bot \times A) = \id_{1 \times A}$ and $\gamma \cc (\bot \times B) = \id_{1 \times B}$.

Let us replace $1 \times A$ and $1 \times B$ by the respective isomorphic object $A$ and $B$ in the above type signatures.
We write $h : I \times A \to A$, $g : B \to A$, and $k : I \times B \to B$ for the respective image of $\alpha$, $\beta$, and $\gamma$ under this bijective transformation.
The first three equations turn into condition (3), strength, and condition (4) of \cref{def:homotopy-equivalence}, while the two equations for the section constraint turn into conditions (1) and (2).\footnote{
For the addendum on strong deformation and co-deformation retracts, simply note that the top and bottom morphisms in the square $r \cc (\theta_2 \hattimes f)$ are given by $h$ and $k$.
}
\end{proof}

If $\catC$ furthermore has pullbacks and is cartesian-closed, we can dualize the statement of \cref{strong-h-equiv-same-as-retraction} as follows:

\begin{corollary}
\label{strong-h-equiv-same-as-section}
Strong homotopy equivalences $f : A \to B$ are in bijective correspondence with sections to the map $\hatexp(\theta, f)$ of arrows from $\hatexp(\top, f)$ to $\hatexp(\canonical_{0 \to 1}, f)$.\footnote{
A strong deformation (or co-deformation) retract under this correspondence is a section $s$ such that the square $\hatexp(\theta_2, f) \cc s : f \to f^I$ has the constant map as top (or bottom, respectively) morphism.
}
\end{corollary}

Purely formaly, this yields the following closure property of Leibniz exponentials:

\begin{proposition}
\label{strong-h-equiv-closed-under-exp}
If one of the maps $f$ and $u$ is a strong homotopy equivalence, then so is $\hatexp(f, u)$.
\end{proposition}

\begin{proof}
Assume first that $f$ is a strong homotopy equivalence.
By \cref{strong-h-equiv-same-as-retraction}, we know that $\theta \hattimes f$ has a retraction.
By contravariance of $\hatexp$ in its first argument, this means $\hatexp(\theta \hattimes f, u)$ has a section.
Exploiting the adjunction between Leibniz product and exponential, this is equivalent to $\hatexp(\theta, \hatexp(f, u))$ having a section.
By \cref{strong-h-equiv-same-as-section}, this means $\hatexp(f, u)$ is a strong homotopy equivalence.

Assume now that $u$ is a strong homotopy equivalence.
By \cref{strong-h-equiv-same-as-section}, we know that $\hatexp(\theta, u)$ has a section.
By covariance of $\hatexp$ in its second argument, this means $\hatexp(f, \hatexp(\theta, u))$ has a section.
Now continue as before, exploiting the adjunction between Leibniz product and exponential.
\end{proof}

\section{Basic simplicial notions}

Let $\Delta$ denote the simplex category and $\Delta_+$ the augmented simplex category.
Recall that the restriction $\Set^{\Delta_+} \to \Set^{\Delta}$ has a left adjoint, given by augmenting over the set of connected components, and a right adjoint, given by augmenting over a singleton set.
Consider the the canonical morphism from the initial to the terminal object in $\Set^{\Delta}$.
We denote its value under the left adjoint by $h_0^0 : \Lambda_0^0 \to \Delta^0$ and under the right adjoint by $i^0 : \partial \Delta^0 \to \Delta^0$.

Recall the monoidal structure of $\Delta_+$ given by ordinal addition.
This structure gives rise to a Day convolution on augmented simplicial sets called the join and denoted $\arghole \star \arghole : \Set^{\Delta_+} \times \Set^{\Delta_+} \to \Set^{\Delta_+}$.

Let us now define the boundary inclusion $i^n : \partial \Delta^n \to \Delta^n$ of dimension $n$ as the ($n+1$)-fold Leibniz join $i^n = (\hatjoin)^{n+1}(i^0$) for $n \geq 0$.
By associativity of the join, it follows that $i^a \hatjoin i^b = i^{a+1+b}$ for $a, b \geq 0$.

Similarly, let us define the $i$-th horn inclusion $h_k^n : \Lambda_k^n \to \Delta^n$ of dimension $n$ as the ($n+1$)-fold Leibniz join $h_k^n = (\hatjoin)^k(i^0) \hatjoin h_0^0 \hatjoin (\hatjoin)^{n-k}(i^0)$.
By associativity of the join, it follows that $h_k^a \hatjoin i^b = h_k^{a+1+b}$ for $0 \leq k \leq a$ and $i^a \hatjoin h_l^b = h_{a+1+l}^{a+1+b}$ for $0 \leq l \leq b$ where $a, b \geq 0$.

The join operation on augmented simplicial sets induces an associative operation on simplicial sets by virtue of terminal augmentation.
Observing that the only simplicial set in the preceeding paragraphs not terminally augmented is $\Lambda_0^0$, all identities not involving $h_0^0$ still hold in simplicial sets.

\begin{definition}
The class of \emph{constructively generating anodyne maps} is given by closing $\braces{h_0^1, h_1^1}$ under Leibniz product with monomorphisms.
A \emph{constructive Kan fibration} is a map that has the right lifting property with respect to constructively generating anodyne maps.
Finally, a \emph{constructive anodyne maps} is one that has the left lifting property with respect to constructive Kan fibrations.
\end{definition}

Recall that $i \hattimes i'$ is a monomorphism whenever $i$ and $i'$ are.
By associativity of the Leibniz product, it follows that the above closure of $\braces{h_0^1, h_1^1}$ is reached after a single iteration, \ie the constructively generating anodyne maps are precisely those of the form $h_k^1 \hattimes i$ where $k = 0, 1$ and $i$ is a monomorphism.

Since constructively generating anodyne maps are closed under Leibniz product with monomorphisms, it follows by adjointness that constructive Kan fibrations are closed under Leibniz exponential with monomorphisms.
Again by adjointness, constructive anodyne maps are closed under Leibniz product with monomorphisms.

\begin{lemma}
\label{horn-constructive-anodyne}
The horn inclusion $h_k^n$ is constructive anodyne where $0 \leq k \leq n$ with $n \geq 1$.
\end{lemma}

\begin{proof}
Without loss of generality, assume that $k < n$.
Recall~\cite[Proposition~2.1.2.6]{lurie:htt} that $h_k^n$ is a retract of $h_0^1 \hattimes i^n$.
\end{proof}

\begin{proposition}
A map $p$ is a constructive Kan fibration if and only if it has the right lifting property with respect to $h_k^n \hattimes i$ for $0 \leq k \leq n$ with $n \geq 1$ and $i$ a monomorphism.
\end{proposition}

\begin{proof}
The non-trivial direction of the equivalence follows from \cref{horn-constructive-anodyne} and the comments preceeding it.
\end{proof}

Awodey's notion of uniformity for Kan fibrations corresponds --- in the simplicial setting --- to the right lifting property with respect to $h_k^n \hattimes i$ for only those monomorphisms $i$ with domain the initial object.
Our reason for stipulating this property for arbitrary monomorphisms is twofold.
First, it seems to lead to a smoother development.
Second, and more important, it also captures --- in the simplicial setting --- Coquand's notion of regularity for Kan fibrations.
\section{Interval object}

Let $I$ denote the cofree simplicial set on two points $\ptA$ and $\ptB$.
The $n$-simplices of $I$ consist of lists $v_0 \ldots v_n$ of length $n+1$ where each entry is either $\ptA$ or $\ptB$.
The $i$-th face or degeneracy map acts by deleting or duplicating, respectively, the entry at index $i$.
It follows that $I$ has exactly two non-degenerate simplices in dimension $n$, consisting of the alternating lists $\ptA \ptB \ptA \ldots$ and $\ptB \ptA \ptB \ldots$ of length $n+1$.
Note that equality and degeneracy of simplices is decidable.

\begin{lemma}
The object $I$ forms a Boolean algebra in simplicial sets.
Specifically, it admits nullary operations $\bot, \top : 1 \to I$ given by $\bot = \ptA$ and $\top = \ptB$, binary operations $\arghole \vee \arghole, \arghole \wedge \arghole : I \times I \to I$, and a unary operation $\neg : I \to I$, all subject to the usual laws.
\end{lemma}

\begin{proof}
The cofree functor from sets to simplicial sets preserves products since it is a right adjoint.
Since the theory of Boolean algebras is algebraic, Boolean algebra in sets are mapped to Boolean algebras in simplicial sets.
Now simply observe that $I$ was defined as the cofree simplicial set on two points $I_0 = \braces{\ptA, \ptB}$, and that this is the underlying set of the standard example of a Boolean algebra in sets.
\end{proof}

For $n \geq 0$, let $I_n$ denote the simplicial subset of $I$ consisting of the non-degenerate simplex $\ptA \ptB \ptA \ldots$ of dimension $n$ and its generalized faces.
We can characterize the simplices $v_0 \ldots v_k$ in $I_n$ as those for which the list $\ptA v_0 \ldots v_k$ contains at most $n$ alternations between $\ptA$ and $\ptB$.
With this description, it is easily verified that the only non-degenerate simplices in $I_{n+1}$ not contained in $I_n$ are the list $\ptA \ptB \ptA \ldots$ of length $n+2$ and the list $\ptB \ptA \ptB \ldots$ of length $n+1$.
Observing that the latter is the $0$-th face of the former, it follows that the monomorphism $I_n \to I_{n+1}$ is a cobase change of the horn filling $h_{0}^{n+1} : \Lambda_0^{n+1} \to \Delta^{n+1}$.
Taking the $\omega$-colimit $\braces{\ptA} = I_0 \to I_1 \to \ldots$, it follows that $\braces{\ptA} \to I$ is constructive anodyne.
Dually, the same holds for $\braces{\ptB} \to I$.

\section{Results}

Let us now eat the fruits of our abstract Leibniz nonsense.
All homotopy equivalences will be with respect to the Boolean algebra $I$ in simplicial sets.

\begin{lemma}
\label{strong-h-equiv-mono-is-anodyne}
Any strong homotopy equivalence that is a monomorphism is constructive anodyne.
\end{lemma}

\begin{proof}
Let $i : A \to B$ be the given morphism.
Recall that $\top$ is constructive anodyne.
By closure under Leibniz product with monomorphisms, we know $\top \hattimes i$ is constructive anodyne.
But $i$ is a retract of $\top \hattimes i$ by \cref{strong-h-equiv-same-as-retraction}, making it constructive anodyne as well.
\end{proof}

\begin{lemma}
Any strong homotopy equivalence that is a constructive Kan fibration has the right lifting property with respect to monomorphisms.
\end{lemma}

\begin{proof}
Let $p : X \to Y$ be the given morphism and fix any monomorphism $i$.
Our goal is to show that $i \pitchfork p$.

Recall that $\top$ is constructive anodyne.
By closure under Leibniz product with monomorphisms, we know $\top \hattimes i$ is constructive anodyne.
It follows that $\top \hattimes i \pitchfork p$.
By adjointness, we have $i \pitchfork \hatexp(\top, p)$.
But $p$ is a retract of $\hatexp(\top, p)$ by \cref{strong-h-equiv-same-as-section}, implying $i \pitchfork p$.
\end{proof}

\begin{corollary}
\label{strong-deformation-retract-is-anodyne}
Any strong deformation retract is constructive anodyne.
\end{corollary}

\begin{corollary}
\label{refl-is-anodyne}
Given any simplicial set $A$, the constant map $A \to A^I$ is constructive anodyne.
\end{corollary}

\begin{proof}
By \cref{de-morgan-deformation-retract}, the pair of maps $\bot : 1 \to I$ and $\canonical_{I \to 1} : I \to 1$ forms a co-strong deformation retract.
By \cref{strong-h-equiv-closed-under-exp}, the constant map $A \to A^I$ --- represented as the Leibniz exponential $\hatexp(\canonical_{1 \to I}, \canonical_{A \to 1})$ --- is a strong homotopy equivalence.
Since it is also a section, we can conlude by \cref{strong-h-equiv-mono-is-anodyne}.
\end{proof}

\begin{corollary}
\label{path-object}
Given a constructive Kan complex $A$, the chain of maps $A \to A^I \to A \times A$ factors the cartesian diagonal into a constructive anodyne map followed by a constructive Kan fibration.
\end{corollary}

\begin{proof}
A consequence of \cref{refl-is-anodyne} and the fact that $\hatexp(i, \canonical_{A \to 1})$ is constructive Kan where $i : 2 \to I$ is the obvious inclusion.
\end{proof}

\begin{lemma}
\label{2-out-of-3}
Strong homotopy equivalences between constructive Kan complexes satisfy 2-out-of-3. 
\end{lemma}

\begin{proof}
[TODO: do the Graduate lemma] 
\end{proof}

\begin{proposition}
\label{fibrant-cod-factorization}
Any map between constructive Kan complexes can be factored into a constructive anodyne map followed by a constructive Kan fibration.
\end{proposition}

\begin{proof}
The usual argument.

Fix $f : A \to B$ with $A$ and $B$ constructive Kan complexes.
Construct $P$ as a pullback as indicated:
\[
\xymatrix{
  A
  \ar[r]
  \ar[d]^{f}
&
  P
  \ar[r]
  \ar[d]
&
  A \times B
  \ar[r]^-{\pi_1}
  \ar[d]^{f \times B}
&
  A
  \ar[d]^{f}
\\
  B
  \ar[r]
&
  B^I
  \ar[r]
&
  B \times B
  \ar[r]^-{\pi_1}
&
  B
}
\]
The remaining squares also form pullbacks and the top and bottom composite rows are identities.
The map $P \to A$ is constructive Kan because it is a base change of $B^I \to B$, which is constructive Kan by \cref{path-object}.
The map $A \to P$ is a strong homotopy equivalence by \cref{2-out-of-3} and mono because it is a base change of $B \to B^I$, thus it is construcive anodyne by \cref{strong-h-equiv-mono-is-anodyne}.
\end{proof}

\begin{corollary}
\label{wfs-anodyne-fibration}
The full subcategory of constructive Kan complexes admits a weak factorization system where the left class consists of constructive anodyne maps and the right class consists of constructive Kan fibrations.
\end{corollary}

\begin{lemma}
Constructive anodyne maps between Kan complexes are stable under pullback by constructive Kan fibrations.
\end{lemma}

\begin{proof}
??
\end{proof}

With the last lemma, this would give a constructive account of the simplicial set model (without universes) as a type-theoretic fibration category in Shulman's sense.

\bibliographystyle{plain}
\bibliography{../common/uniform-kan-bibliography}

\end{document}






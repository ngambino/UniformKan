\documentclass[reqno,10pt,a4paper,oneside]{amsart}

\usepackage{uniform-kan-prelude}

\begin{document}

\section{Introduction}

The purpose of this file is to examine the appropriate constructive rephrasing of the notion of Kan fibration.
We work in a constructive setting.
In classical logic, all presented enhanced notions of Kan fibrations will be equivalent to the usual one. 

The results so far can be read as abstractly (and thus constructively) proving the following:

\begin{theorem}
\label{abstract-homotopical-tribe}
Fix a finitely complete and cocomplete category $\catC$ with exponentials such that monomorphisms are closed under Leibniz product.
Consider a monomorphism $2 \to I$, which we will view as a bipointed object $\bot, \top : 1 \to I$.
Assume $!_{I \to 1} : I \to 1$ is a strong homotopy equivalence.

Let $\classleft$ be a class of morphisms at least including $\bot$ and $\top$ and closed under Leibniz product with monomorphisms.
Call fibrations those maps with the right lifting property with respect to $\classleft$.
This turns the full subcategory of fibrant objects of $\catC$ into a homotopical tribe.
\end{theorem}

\section{Abstract Leibniz Nonsense}

Recall the Leibniz construction and its properties \cite[Section 4]{riehl-verity:reedy}.
For a bifunctor $H : \catC \times \catD \to \catE$, we will denote its associated Leibniz construction by $\hat{H} : \catC^{\to} \times \catD^{\to} \to \catE^{\to}$.
A standard example is provided by the product bifunctor $\arghole \times \arghole : \catC \times \catC \to \catC$ for categories $\catC$ with finite products and finite colimits and its associated Leibniz construction, the Leibniz product $\arghole \hattimes \arghole : \catC^{\to} \times \catC^{\to} \to \catC^{\to}$ (also known as the smash product or pushout-product).
If $\catC$ has exponentials and finite limits, it is closed with pointwise right adjoint given by the Leibniz exponential $\hatexp : (\catC^{\op})^{\to} \times \catC^{\to} \to \catC^{\to}$.

The names for the notions in this section are inspired by model categories which have an object $I$ such that $I \times X$ is a cylinder object and $X^I$ is a path object for any $X$.
I do not know what names to use on this abstract level.

Let $\catC$ be a category with finite products and $(I, \bot, \top)$ be a bipointed object in $\catC$.

\begin{definition}
\label{def:homotopy}
A \emph{homotopy} $h : f_1 \sim f_2$ between $f_1, f_2 : A \to B$ consists of a map $h : I \times A \to B$ such that
\begin{align*}
h \cc \angles{\bot \cc \canonical_{A \to 1}, \id_A} &= f_1
,\\
h \cc \angles{\top \cc \canonical_{A \to 1}, \id_A} &= f_2
.\end{align*}

Given $f : A \to B$, there is a \emph{trivial homotopy} $h : f \sim f$ given by $h = \pi_2$. 
\end{definition}

\begin{definition}
\label{def:homotopy-equivalence}
A map $f : A \to B$ is called a \emph{homotopy equivalence} if it comes with $g : B \to A$ together with homotopies $h : \id_A \sim g \cc f$ and $k : \id_B \sim f \cc g$.
This notion is symmetric, admitting an obvious duality.
Such a homotopy equivalence is \emph{strong} if $f \cc h = k \cc (I \times f)$ and \emph{co-strong} if its dual is co-strong, \ie if $g \cc k = h \cc (I \times g)$.

A \emph{deformation retract} is a homotopy equivalence as above where the homotopy $h$ is trivial (note that this makes $f$ and $g$ into a section-retraction pair).
Dually, a \emph{co-deformation retract} has the homotopy $k$ trivial (with $g$ and $f$ a section-retraction pair).
\end{definition}

We say that $I$ has \emph{disjunction} if it comes with an associative binary operation with unit $\bot$ and annihilator $\top$.

\begin{example}
\label{disjunction-deformation-retract}
If $I$ has disjunction, the map $\top : 1 \to I$ is a strong and co-strong deformation retract.
\end{example}

\begin{proof}
Clearly, the map $\top$ is a section to the canonical map $\canonical_{I \to 1}$.
The remaining data is given by disjunction, noting that it has $\bot$ as left unit and $\top$ as left and right annihilator.
\end{proof}

\begin{remark}
\label{h-equiv-exponent}
For an object $A$ and $f : X \to Y$ a (strong, co-strong) homotopy equivalence (deformation retract, co-deformation retract), we have $A^f : A^Y \to A^X$ is a (strong, co-strong) homotopy equivalence (co-deformation retract, deformation retract).
\end{remark}

\begin{comment}
\begin{example}
\label{de-morgan-exp-deformation-retract}
Let $I$ have the structure of a De Morgan algebra in $\catC$.
For any object $A$, the section-retraction pair $A^{\canonical_{I \to 1}} : A^1 \to A^I$ and $A^{\top} : A^I \to A^1$ forms a strong and co-strong deformation retract.
The same holds with $\top$ replaced by $\bot$.
\end{example}

\begin{proof}
Combine \cref{de-morgan-deformation-retract} with \cref{h-equiv-exponent}.
\end{proof}
\end{comment}

Now let $\catC$ also have finite colimits.
Consider the following diagram:
\[
\xymatrix@C+2em{
  0
  \ar[r]^{\canonical_{0 \to 1}}
  \ar[d]_{\canonical_{0 \to 1}}
  %\ar[dr]^{\canonical_{0 \to A}}
&
  1
  \ar[d]^{\top}
\\
  1
  \ar[r]^{\bot}
&
  I
\\
  \canonical_{0 \to 1}
  \ar[r]^{\theta}
&
  \top
}
\]
Here, $\theta$ denotes the square as a morphism in the arrow category $\catC^{\to}$.
Recall that $\canonical_{0 \to 1}$ is a unit for the Leibniz product. 

\begin{lemma}
\label{strong-h-equiv-as-retraction}
Strong homotopy equivalences $f : A \to B$ are given by retractions $r$ to $\theta \hattimes f$:
\[
\xymatrix@C+2em{
  \canonical_{0 \to 1} \hattimes f
  \ar[r]_{\theta \hattimes f}
  \ar@/^2em/[rr]^{\id}
&
  \top \hattimes f
  \ar@{.>}[r]_{r}
&
  \canonical_{0 \to 1} \hattimes f
}
\]
Furthermore, strong co-deformation retracts correspond to this being a lift of

\[
\xymatrix@C+2em{
  1 \times B
  \ar[r]_{\bot \times B}
  \ar@/^2em/[rr]^{\id}
&
  I \times B
  \ar[r]_{\canonical_{I \to 1} \times B}
&
  1 \times B
}
\]
with respect to the codomain fibration.
\end{lemma}

\begin{proof}
To give $r$ from $\top \hattimes i$ to $\canonical_{0 \to 1} \hattimes i$ is to give maps $\alpha : I \times A \to 1 \times A$, $\beta : 1 \times B \to 1 \times A$, and $\gamma : I \times B \to 1 \times B$ such that $\alpha \cc (\top \times A) = \beta \cc (1 \times f)$, $(1 \times f) \cc \alpha = \gamma \cc (I \times f)$, and $(1 \times f) \cc \beta = \gamma \cc (\top \times B)$.
The map $r$ constituting a section to $\theta \hattimes f$ means $\alpha \cc (\bot \times A) = \id_{1 \times A}$ and $\gamma \cc (\bot \times B) = \id_{1 \times B}$.
\[
\xymatrix@C+1em{
  0 \times A
  \ar[dr]
  \ar[dd]
&&&
  1 \times A
  \ar[dr]^{\top \times A}
  \ar[dd]^(0.7){1 \times f}
\\&
  1 \times A
  \ar[dd]^{1 \times f}
  \ar[rrr]_(0.4){\bot \times A}
&&&
  I \times A
  \ar[dd]^{I \times f}
  \ar@{.>}@/_2em/[lll]_{\alpha}
\\
  0 \times A
  \ar[dr]
&&&
  1 \times B
  \ar[dr]^{\top \times B}
  \ar@{.>}@/^1em/[llu]^{\beta}
\\&
  1 \times B
  \ar[rrr]_{\bot \times B}
&&&
  I \times B
  \ar@{.>}@/_2em/[lll]^{\gamma}
}
\]

Let us replace $1 \times A$ and $1 \times B$ by the respective isomorphic object $A$ and $B$ in the above type signatures.
Write $h : I \times A \to A$, $g : B \to A$, and $k : I \times B \to B$ for the respective image of $\alpha$, $\beta$, and $\gamma$ under this bijective transformation.
With respect to the requirements of \cref{def:homotopy-equivalence}, the first three equations turn into endpoint $\top$ for $h$, co-strength, and endpoint $\top$ for $k$, while the two equations for the section constraint turn into endpoint $\bot$ for $h$ and $k$.

For the statement about strong co-deformation retracts, note that $\cod_{\catC}(r) = \gamma$ in the above picture, and $\gamma = \canonical_{I \to 1} \times B$ corresponds to $k = \pi_2$.
\end{proof}

Let us furthermore assume that $\catC$ has pullbacks and exponentials.
We can then add a dual property to \cref{strong-h-equiv-as-retraction}:

\begin{corollary}
\label{strong-h-equiv-slick}
Given any map $f$ in $\catC$, the following are equivalent:
\begin{itemize}
\item The map $f$ is a strong homotopy equivalence.
\item The map $\theta \hattimes f$ is a section.
\item The map $\hatexp(\theta, f)$ is a retraction.
\end{itemize}
\end{corollary}

The following closure properties are purely formal, working entirely on the level of arrow categories.

\begin{proposition}
\label{strong-h-equiv-closed-under-prod}
If one of the maps $f$ and $g$ is a strong homotopy equivalence, then so is $f \hattimes g$.
\end{proposition}

\begin{proof}
By \cref{strong-h-equiv-slick}, this is equivalent to showing that if $\theta \hattimes f$ or $\theta \hattimes g$ is a section, then so is $\theta \hattimes f \hattimes g$.
But sections are preserved by functors, in this case the Leibniz product in one variable.
\end{proof}

\begin{proposition}
\label{strong-h-equiv-closed-under-exp}
If one of the maps $f$ and $g$ is a strong homotopy equivalence, then so is $\hatexp(f, g)$.
\end{proposition}

\begin{proof}
Let us show that if $f$ is a strong homotopy equivalence, then so is $\hatexp(f, g)$.
By \cref{strong-h-equiv-slick}, this is equivalent to showing that if $\theta \hattimes f$ is a section, then $\hatexp(\theta, \hatexp(f, g)) = \hatexp(\theta \hattimes f, g)$ is a retraction.
But the functor $\hatexp$ is contravariant in its first argument, mapping sections to retractions.

Let us show that if $g$ is a strong homotopy equivalence, then so is $\hatexp(f, g)$.
By \cref{strong-h-equiv-slick}, this is equivalent to showing that if $\hatexp(\theta, g)$ is a retraction, then so is $\hatexp(\theta, \hatexp(f, g)) = \hatexp(f, \hatexp(\theta, g))$ is a retraction.
But the functor $\hatexp$ is covariant in its first argument, mapping retractions to retractions.
\end{proof}

\begin{proposition}
\label{strong-h-equiv-closed-under-retract}
Strong homotopy equivalences and strong co-deformation retracts are closed under retracts.
\end{proposition}

\begin{proof}
Apply \cref{strong-h-equiv-slick} and use that functors preserve and sections are closed under retracts.
\end{proof}

\begin{lemma}
\label{strong-h-equiv-base-change-along-fibration}
Strong homotopy equivalences are stable under base change along morphisms having the right lifting property with respect to $\bot \hattimes \canonical_{0 \to X}$ for $X : \catC$.
Strong co-deformation retracts are stable under arbitrary base change.
\end{lemma}

\begin{proof}
Let us start with the first statement.
Let $f : A \to B$ be a strong homotopy equivalence and $v : B' \to B$ be a map having the lifting property mentioned in the statement.
We consider the following pullback square:
\[
\xymatrix{
  A'
  \ar[r]^{u}
  \ar[d]^{f'}
  \pullback{dr}
&
  A
  \ar[d]^{f}
\\
  B'
  \ar[r]^{v}
&
  B
}
\]
The goal is to show that $f'$ is a strong homotopy equivalence.

According to \cref{strong-h-equiv-as-retraction}, we must construct $(s', t')$ from $(s, t)$ in the following commutative diagram in the arrow category $\catC^{\to}$:
\[
\xymatrix@C+3em{
  \canonical_{0 \to 1} \hattimes f'
  \ar[r]_{\theta \hattimes f'}
  \ar[d]_{\canonical_{0 \to 1} \hattimes (u, v)}
  \ar@/^2em/[rr]^{\id}
&
  \top \hattimes f'
  \ar@{.>}[r]_{(s', t')}
  \ar[d]^{\top \hattimes (u, v)}
&
  \canonical_{0 \to 1} \hattimes f'
  \ar[d]^{\canonical_{0 \to 1} \hattimes (u, v)}
\\
  \canonical_{0 \to 1} \hattimes f
  \ar[r]^{\theta \hattimes f}
  \ar@/_2em/[rr]_{\id}
&
  \top \hattimes f
  \ar[r]^{(s, t)}
&
  \canonical_{0 \to 1} \hattimes f'
}
\]
Since the codomain fibration has $(u, v)$ as cartesian morphism, it will suffice to solve this problem when projected to its base $\catC$:
\[
\xymatrix@C+2em{
  1 \times B'
  \ar[r]_{\bot \times B'}
  \ar[d]_{1 \times v}
  \ar@/^2em/[rr]^{\id}
&
  I \times B'
  \ar@{.>}[r]_{t'}
  \ar[d]^{I \times v}
&
  1 \times B'
  \ar[d]^{1 \times v}
\\
  1 \times B
  \ar[r]^{\bot \times f}
  \ar@/_2em/[rr]_{\id}
&
  I \times B
  \ar[r]^{t}
&
  1 \times B
}
\]
This is a lifting problem of the form $\bot \hattimes \canonical_{0 \to B'} \pitchfork v$, which is solvable by assumption.

To verify the second statement, relaxing the lifting property assumption for $v$, note that the second diagram in the statement of \cref{strong-h-equiv-as-retraction} is natural in $B$.
\end{proof}

\section{Proof of \cref{abstract-homotopical-tribe}}

Fix a category $\catC$ together with the structure listed in the assumptions of \cref{abstract-homotopical-tribe}.
Recall this includes a class $\classleft$ of morphisms at least including $\bot$ and $\top$ and closed under Leibniz exponential with monomorphisms.

\begin{definition}
\label{def:fibration}
We define:
\begin{itemize}
\item
A \emph{fibration} is a map with the right lifting property with respect to $\classleft$.
\item
A \emph{trivial cofibration} is a map with the left lifting property with respect to fibrations.
\end{itemize}
\end{definition}

Since $\classleft$ is closed under Leibniz product with monomorphisms, it follows by adjointness that fibrations are closed under Leibniz exponential with monomorphisms.
Again by adjointness, trivial cofibrations are closed under Leibniz product with monomorphisms.

\begin{lemma}
\label{strong-h-equiv-mono-is-anodyne}
Any strong homotopy equivalence that is a monomorphism is a trivial cofibration.
\end{lemma}

\begin{proof}
Let $i : A \to B$ be the given morphism.
Recall that $\top$ is a trivial cofibration.
By closure under Leibniz product with monomorphisms, we know $\top \hattimes i$ is a trivial cofibration.
But $i$ is a retract of $\top \hattimes i$ by \cref{strong-h-equiv-slick}, making it a trivial cofibration.
\end{proof}

\begin{lemma}[Following~\cite{joyal-tierney:simplicial-homotopy-theory}]
\label{anodyne-between-fibrant-is-strong-def-retract}
Any trivial cofibration between fibrant object is a strong deformation retract.
\end{lemma}

\begin{proof}
Let $j : A \to B$ be the given map.
Since $\canonical_{A \to 1}$ is a fibration, we can construct a retraction $r$ to $j$ as a lift in the following diagram:
\[
\xymatrix{
  A
  \ar[r]^{\id}
  \ar[d]_{j}
&
  A
  \ar[d]^{\canonical_{A \to 1}}
\\
  B
  \ar[r]
  \ar@{.>}[ur]^{r}
&
  1
}
\]
Now let $i : 2 \to I$ denote the monomorphism $i = [\bot, \top]$.
Since $\canonical_{B \to 1}$ is a fibration, so is $\hatexp(i, \canonical_{B \to 1})$.
Thus, we may construct the remaining data of a strong deformation retract as a lift in the following diagram:
\[
\xymatrix{
  A
  \ar[r]
  \ar[d]_{j}
&
  A^I
  \ar[r]^{j^I}
&
  B^I
  \ar[d]^{\hatexp(i, \canonical_{B \to 1})}
\\
  B
  \ar[rr]_{B \times (r \cc j)}
  \ar@{.>}[urr]
&&
  B^2
}
\]
\end{proof}

\begin{corollary}
\label{anodyne-between-fibrant-characterization}
In the full subcategory of fibrant objects, the following classes of maps are identical:
\begin{itemize}
\item
trivial cofibrations,
\item
strong homotopy equivalences that are monomorphisms,
\item
strong deformation retracts.
\end{itemize}
\end{corollary}

\begin{lemma}
\label{fibration-base-change-of-anodyne-between-fibrant}
Trivial cofibrations between fibrant objects are stable under base change along fibrations.
\end{lemma}

\begin{proof}
First note that the pullback takes place entirely in the full subcategory of fibrant objects.
In there, we work with the characterization of trivial cofibrations as strong homotopy equivalences that are monomorphisms.
The proof is now a combination of two ingredients:
\begin{itemize}
\item
By \cref{strong-h-equiv-base-change-along-fibration}, strong homotopy equivalences are stable under base change along fibrations.
\item
Monomorphisms are always stable under base change.
\end{itemize}
\end{proof}

A \emph{path object} for a fibration $X \to Y$ is a factorization of the diagonal $X \to X \times_Y X$ into a trivial cofibration followed by a fibration.
A path object for a fibrant object $X$ is a path object for the fibration $\canonical_{X \to 1}$.

\begin{lemma}
\label{leibniz-exp-factorization}
Let $f : A \to B$ be a morphism and $p : X \to Y$ be a fibration.
Assume we are given a factorization $A \xto{i} K \xto{r} B$ of $f$ into a monomorphism $i$ followed by a strong homotopy equivalence $r$ that is also a retraction.
Then the Leibniz exponential $\hatexp(f, p)$ factors into a trivial cofibration followed by fibration as follows:
\[
\xymatrix{
  X^B
  \ar@{>->}[r]^-{\sim}
&
  X^K \times_{Y^K} Y^B
  \ar@{->>}[r]
&
  X^A \times_{Y^A} Y^B
}
\]
Furthermore, this factorization is functorial in $p$ as an object of the arrow category.
\end{lemma}

\begin{proof}
Recall that the Leibniz construction acts on composition in one of its argument by ``skewed'' composition of individual Leibniz constructions.
In this case, the Leibniz exponential $\hatexp(f, p)$ is given by $\hatexp(r, p)$ followed by a base change of $\hatexp(i, p)$.

The map $\hatexp(r, p)$ is a strong homotopy equivalence by \cref{strong-h-equiv-closed-under-exp}.
It is also a section since $r$ is a retraction.
By \cref{strong-h-equiv-mono-is-anodyne}, it follows that it is a trivial cofibration.
The map $\hatexp(i, p)$ is a fibration because $i$ is a monomorphism and $p$ a fibration.
Since fibrations are defined by a right lifting property, any base change of it is a fibration as well.

This provides the desired factorization of $\hatexp(f, p)$.
Functoriality in $p$ follows from functoriality of the Leibniz construction.
\end{proof}

\begin{proposition}
\label{path-objects}
Any fibration $p : X \to Y$ admits a path object functorially in $p$ as an object of the arrow category.
In detail, we have the following factorization:
\[
\xymatrix@C+4em{
  X
  \ar@{>->}[r]_-{\sim}^-{\angles{X^{\canonical_{I \to 1}}, p}}
&
  X^I \times_{Y^I} Y
  \ar@{->>}[r]^-{\angles{X^{\bot}, X^{\top}} \cc \pi_1}
&
  X \times_Y X
}
\]
\end{proposition}

Note that this factorization already appears in Warren~\cite{warren:thesis} and Awodey and Warren~\cite{awodey-warren:homotopy-idtype} in the context of simplicial model categories where $I = \Delta^1$ and the exponential takes the form of a power.

\begin{proof}
Let $p : X \to Y$ be a fibration.
Our goal is to factor the diagonal $\angles{\id_X, \id_X} : X \to X \times_Y X$ as a trivial cofibration followed by a fibration.
Note that the diagonal $X \to X \times_Y X$ is equivalently obtained as the Leibniz exponential $\hatexp(\canonical_{2 \to 1}, p)$.

Now $\canonical_{2 \to 1}$ factors as $i = \bracks{\bot, \top} : 2 \to I$ followed by $!_{I \to 1}$.
By assumption, we have $i$ a monomorphism and $\canonical_{I \to 1}$ a strong homotopy equivalence that is a retraction (\eg to $\bot$).
We thus conclude with an application of \cref{leibniz-exp-factorization}.
\end{proof}

\begin{lemma}
\label{path-objects-stable}
The factorizations provided by \cref{path-objects} are stable (up to isomorphism) under pullback.
In detail, consider a base change of fibrations as follows:
\[
\xymatrix{
  U
  \ar[r]
  \ar@{->>}[d]^{p}
  \pullback{dr}
&
  X
  \ar@{->>}[d]^{q}
\\
  V
  \ar[r]
&
  Y
}
\]
Then the induced morphism of factorizations has pullback squares as indicated:
\[
\xymatrix{
  U
  \ar[r]
  \ar@{>->}[d]_{\sim}
  \pullback{dr}
&
  X
  \ar[d]
  \ar@{>->}[d]_{\sim}
\\
  U^I \times_{V^I} V
  \ar[r]
  \ar@{->>}[d]
  \pullback{dr}
&
  X^I \times_{Y^I} Y
  \ar@{->>}[d]
\\
  U \times_V U
  \ar[r]
&
  X \times_Y X  
}
\]
\end{lemma}

\begin{proof}
The Leibniz construction commutes with base change.
Given a pullback square written as a cartesian morphism $s : p \to q$ in the arrow category, the action of the Leibniz exponential $\hatexp(s, f)$ on $s$ is again a cartesian morphism (\ie a pullback square) for any morphism $f$.
\end{proof}

This means our path objects are stable in the sense of Warren~\cite{warren:thesis} and Awodey and Warren~\cite{awodey-warren:homotopy-idtype}.

\begin{proof}[Proof of \cref{abstract-homotopical-tribe}]
This is now merely an exercise in relating different notions of models of type theory.

Let $\catC_f$ denote the full subcategory of fibrant objects of $\catC$.
Recall that pullbacks in $\catC_f$ are inherited from $\catC$.
Fibrations are stable under base change as they were defined via a right lifting property.
This makes $\catC$ into a tribe.

The result now follows as an application of \cref{wfs-from-path-objects} to $\catC_f$.
First note that base change along fibrations preserves trivial fibrations by \cref{fibration-base-change-of-anodyne-between-fibrant}.
Path objects are provided by \cref{path-objects}, and they are stable under base change by \cref{path-objects-stable}.
\end{proof}

We also have:

\begin{theorem}
If $\catC$ in addition has right adjoints to base change along fibrations, then it is a type-theoretic fibration category in the sense of Shulman~\cite{shulman:inverse-diagrams}.
\end{theorem}

For this, note that Shulman~\cite{shulman:inverse-diagrams} claims their conditions~(5') and~(6') are weaker than the stable path objects of Warren~\cite{warren:thesis} and Awodey and Warren~\cite{awodey-warren:homotopy-idtype}.
\textbf{Warning: check this claim!}

\subsection{Further abstract results}

\begin{lemma}
Any strong homotopy equivalence that is a constructive Kan fibration has the right lifting property with respect to monomorphisms.
\end{lemma}

\begin{proof}
\label{strong-h-equiv-kan-is-trivial}
Let $p : X \to Y$ be the given morphism and fix any monomorphism $i$.
Our goal is to show that $i \pitchfork p$.

Recall that $\top$ is a trivial cofibration.
By closure under Leibniz product with monomorphisms, the same holds for $\top \hattimes i$.
It follows that $\top \hattimes i \pitchfork p$.
By adjointness, we have $i \pitchfork \hatexp(\top, p)$.
But $p$ is a retract of $\hatexp(\top, p)$ by \cref{strong-h-equiv-slick}, implying $i \pitchfork p$.
\end{proof}

\begin{lemma}[Following~\cite{joyal-tierney:simplicial-homotopy-theory}]
Any constructive Kan fibration that has the right lifting property with respect to monomorphisms is a strong co-deformation retract.
\end{lemma}

\begin{proof}
Completely dual to \cref{anodyne-between-fibrant-is-strong-def-retract}.

Let $p : X \to Y$ be the given morphism.
Since $!_{0 \to Y}$ is a monomorphism, we can construct a section $s$ to $p$ as a lift in the following diagram:
\[
\xymatrix{
  0
  \ar[r]
  \ar[d]_{\canonical_{0 \to Y}}
&
  X
  \ar[d]^{\id}
\\
  Y
  \ar[r]_{p}
  \ar@{.>}[ur]^{s}
&
  Y
}
\]
Again, let $i : 2 \to I$ denote the monomorphism $i = [\bot, \top]$.
Since $\canonical_{0 \to X}$ is a monomorphism, so is $i \hattimes \canonical_{0 \to X}$.
Thus, we may construct the remaining data of a strong co-deformation retract as a lift in the following diagram:
\[
\xymatrix{
  2 \times X
  \ar[rr]^{Y + (p \cc s)}
  \ar[d]_{i \hattimes \canonical_{0 \to X}}
&&
  X
  \ar[d]^{p}
\\
  I \times X
  \ar[r]_{\pi_2}
  \ar@{.>}[urr]
&
  X
  \ar[r]_{p}
&
  Y
}
\]
\end{proof}

\begin{corollary}
A constructive Kan fibration has the right lifting property with respect to monomorphisms precisely if it is a strong co-deformation retract.
\end{corollary}

\section{The simplicial setting}

Let $\Delta$ denote the simplex category and $\Delta_+$ the augmented simplex category.
Recall that the restriction $\Set^{\Delta_+^{\op}} \to \Set^{\Delta^{\op}}$ has a left adjoint, given by augmenting over the set of connected components, and a right adjoint, given by augmenting over a singleton set.
Consider the the canonical morphism from the initial to the terminal object in $\Set^{\Delta^{\op}}$.
We denote its value under the left adjoint by $h_0^0 : \Lambda_0^0 \to \Delta^0$ and under the right adjoint by $i^0 : \partial \Delta^0 \to \Delta^0$.

Recall the monoidal structure of $\Delta_+$ given by ordinal addition.
This structure gives rise to a Day convolution on augmented simplicial sets called the join and denoted $\arghole \star \arghole : \Set^{\Delta_+^{\op}} \times \Set^{\Delta_+^{\op}} \to \Set^{\Delta_+^{\op}}$.

Let us now define the boundary inclusion $i^n : \partial \Delta^n \to \Delta^n$ of dimension $n$ as the ($n+1$)-fold Leibniz join $i^n = (\hatjoin)^{n+1}(i^0$) for $n \geq 0$.
By associativity of the join, it follows that $i^a \hatjoin i^b = i^{a+1+b}$ for $a, b \geq 0$.

Similarly, let us define the $i$-th horn inclusion $h_k^n : \Lambda_k^n \to \Delta^n$ of dimension $n$ as the ($n+1$)-fold Leibniz join $h_k^n = (\hatjoin)^k(i^0) \hatjoin h_0^0 \hatjoin (\hatjoin)^{n-k}(i^0)$.
By associativity of the join, it follows that $h_k^a \hatjoin i^b = h_k^{a+1+b}$ for $0 \leq k \leq a$ and $i^a \hatjoin h_l^b = h_{a+1+l}^{a+1+b}$ for $0 \leq l \leq b$ where $a, b \geq 0$.

The join operation on augmented simplicial sets induces an associative operation on simplicial sets by virtue of terminal augmentation.
Observing that the only simplicial set in the preceeding paragraphs not terminally augmented is $\Lambda_0^0$, all identities not involving $h_0^0$ still hold in simplicial sets.

\begin{definition}
\label{def:constructive-kan-simplicial}
The class of \emph{constructively generating trivial cofibrations} is given by closing $\braces{h_0^1, h_1^1}$ under Leibniz product with monomorphisms.
A \emph{constructive Kan fibration} is a map that has the right lifting property with respect to constructively generating trivial cofibrations.
Finally, a \emph{constructive trivial cofibration} is one that has the left lifting property with respect to constructive Kan fibrations.
\end{definition}

Recall that $i \hattimes i'$ is a monomorphism whenever $i$ and $i'$ are.
By associativity of the Leibniz product, it follows that the above closure of $\braces{h_0^1, h_1^1}$ is reached after a single iteration, \ie the constructively generating trivial cofibrations are precisely those of the form $h_k^1 \hattimes i$ where $k = 0, 1$ and $i$ is a monomorphism.

Since constructively generating trivial cofibrations are closed under Leibniz product with monomorphisms, it follows by adjointness that constructive Kan fibrations are closed under Leibniz exponential with monomorphisms.
Again by adjointness, constructive trivial cofibrations are closed under Leibniz product with monomorphisms.

\begin{lemma}
\label{horn-constructive-anodyne}
The horn inclusion $h_k^n$ is a constructive trivial cofibration where $0 \leq k \leq n$ with $n \geq 1$.
\end{lemma}

\begin{proof}
Without loss of generality, assume that $k < n$.
Then $h_k^n$ is a retract of $h_0^1 \hattimes h_k^n$~\cite[Proposition~2.1.2.6]{lurie:htt}.
Note that $h_k^n$ is a monomorphism.
\end{proof}

\begin{proposition}
\label{constructive-kan-alternative}
A map $p$ is a constructive Kan fibration precisely if it has the right lifting property with respect to $h_k^n \hattimes i$ for $0 \leq k \leq n$ with $n \geq 1$ and $i$ a monomorphism.
\end{proposition}

\begin{proof}
The non-trivial direction of the equivalence follows from \cref{horn-constructive-anodyne} and the comments preceeding it.
\end{proof}

Classically, monomorphisms are the saturated class of boundary inclusions $i^n : \partial \Delta^n \to \Delta^n$.
Recall that the Leibniz product of a horn inclusion with a boundary inclusion lies in the saturated class of horn inclusions~\cite[Proposition~2.1.2.6]{lurie:htt}.
Using \cref{constructive-kan-alternative}, it thus follows that the notions of \cref{def:constructive-kan-simplicial} classically coincide with the usual ones.

Awodey's notion of uniformity for Kan fibrations corresponds --- in the simplicial setting --- to the right lifting property with respect to $h_k^n \hattimes i$ for only those monomorphisms $i$ with domain the initial object.
Our reason for stipulating this property for arbitrary monomorphisms is twofold.
First, it seems to lead to a smoother development.
Second, and more important, it also captures --- in the simplicial setting --- Coquand's notion of regularity for Kan fibrations.

It can be checked that simplicial sets together with $\classleft$ the class of constructively generating trivial cofibrations and $I = \Delta^1$ satisfy the assumptions of \cref{abstract-homotopical-tribe}.
Thus we have proved constructively:

\begin{corollary}
Simplicial sets with constructive Kan fibrations form a homotopical tribe.
\end{corollary}

Since they also admit right adjoints to base change, we furthermore have:

\begin{corollary}
Simplicial sets with constructive Kan fibrations form a homotopical $\Pi$-tribe and a type-theoretic fibration category.
\end{corollary}

\section{Interval object}

Let $I$ denote the cofree simplicial set on two points $\ptA$ and $\ptB$.
The $n$-simplices of $I$ consist of lists $v_0 \ldots v_n$ of length $n+1$ where each entry is either $\ptA$ or $\ptB$.
The $i$-th face or degeneracy map acts by deleting or duplicating, respectively, the entry at index $i$.
It follows that $I$ has exactly two non-degenerate simplices in dimension $n$, consisting of the alternating lists $\ptA \ptB \ptA \ldots$ and $\ptB \ptA \ptB \ldots$ of length $n+1$.
Note that equality and degeneracy of simplices is decidable.

\begin{lemma}
The object $I$ forms a Boolean algebra in simplicial sets.
Specifically, it admits nullary operations $\bot, \top : 1 \to I$ given by $\bot = \ptA$ and $\top = \ptB$, binary operations $\arghole \vee \arghole, \arghole \wedge \arghole : I \times I \to I$, and a unary operation $\neg : I \to I$, all subject to the usual laws.
\end{lemma}

\begin{proof}
The cofree functor from sets to simplicial sets preserves products since it is a right adjoint.
Since the theory of Boolean algebras is algebraic, Boolean algebras in sets are mapped to Boolean algebras in simplicial sets.
Now simply observe that $I$ was defined as the cofree simplicial set on two points $I_0 = \braces{\ptA, \ptB}$, and that this is the underlying set of the standard example of a Boolean algebra in sets.
\end{proof}

For $n \geq 0$, let $I_n$ denote the simplicial subset of $I$ consisting of the non-degenerate simplex $\ptA \ptB \ptA \ldots$ of dimension $n$ and its generalized faces.
We can characterize the simplices $v_0 \ldots v_k$ in $I_n$ as those for which the list $\ptA v_0 \ldots v_k$ contains at most $n$ alternations between $\ptA$ and $\ptB$.
With this description, it is easily verified that the only non-degenerate simplices in $I_{n+1}$ not contained in $I_n$ are the list $\ptA \ptB \ptA \ldots$ of length $n+2$ and the list $\ptB \ptA \ptB \ldots$ of length $n+1$.
Observing that the latter is the $0$-th face of the former, it follows that the monomorphism $I_n \to I_{n+1}$ is a cobase change of the horn filling $h_{0}^{n+1} : \Lambda_0^{n+1} \to \Delta^{n+1}$.
Taking the $\omega$-colimit $\braces{\ptA} = I_0 \to I_1 \to \ldots$, it follows that $\braces{\ptA} \to I$ is a constructive trivial cofibration.
Dually, the same holds for $\braces{\ptB} \to I$.

\section{Appendix}

Recall that a \emph{tribe} is a category $\catC$ with a subcategory of morphisms called \emph{fibrations} denoted by $X \fib Y$ satisfying the following:
\begin{itemize}
\item
There exists a terminal object $1$.
The canonical map $\canonical_{X \to 1}$ is a fibration for any $X$.
\item
Base changes of fibrations exist and are fibrations.
\end{itemize}

Morphisms that have the left lifting property with respect to fibrations are called \emph{trivial cofibrations} and are denoted $A \acof B$.
A tribe $\catC$ is called \emph{homotopical} (or an \emph{h-tribe}) if:
\begin{itemize}
\item
Base change along fibrations preserves trivial cofibrations.
\item
Any map factors as a fibration followed by a trivial cofibration.
\end{itemize}

Recall that a \emph{path object} for a fibration $A \fib B$ is a factorization $A \acof P_B A \fib A \times_B A$ of the diagonal $\angles{\id_A, \id_A}$ as in the above definition.
A path object for an object $A$ is a path object for $\canonical_{A \to 1}$.

Something like the following theorem was first proved by Gambino and Garner~\cite{gambino-garner:idtypewfs} in a type-theorical phrasing.
A categorical transcription was given by Shulman~\cite[Lemma~2.4]{shulman:inverse-diagrams} 
Emmenegger~\cite{emmenegger:stable-path-objects-wfs} gave a slightly generalized categorical account.

We only need the following corollary to the above efforts:

\begin{theorem}[Gambino and Garner, Shulman, Emmenegger]
\label{wfs-from-path-objects}
Let $\catC$ be a tribe such that base change along fibrations preserves trivial cofibrations.
Assume that any fibration $p$ in $\catC$ admits a path object $P$ such that for any base change $q$ of $p$, the associated base change of $P$ is a path object for $q$.
Then any map in $\catC$ factors as a fibration followed by a trivial cofibration.
\end{theorem}

\begin{proof}
By inspection of Lemma~2.4 in Shulman~\cite{shulman:inverse-diagrams}.
Let us switch to that notation.
It is shown that condition~(5') together with a certain instance of condition~(6') implies condition (5).

We are given a map $f : A \to B$ to be factored (we only need the case $C = 1$).
The only path objects needed are $B \acof P B \fib B \times B$ for $B$ and $P_{B \times B} (P B)$ for $P B \fib B \times B$.
The mapping space $P f$ for $f$ and a further path object $P_{A \times B} (P f)$ for $P f \fib A \times B$ are constructed from these by pullback (the only use of condition~(6') is taken care of by stability of our path objects by pullback).

Furthermore, the concept of a type-theoretic fibration category without conditions~(5) and~(6) is identical to that of a tribe together with a partial right adjoint to pullback along fibrations, defined on and valued in fibrations.
However, this property is not used in this proof, only its adjoint implication that base change along fibrations preserves trivial cofibrations.
\end{proof}

\begin{proof}[Alternative proof]
This is Theorem~2.8 in Emmenegger~\cite{emmenegger:stable-path-objects-wfs}.
As mentioned there, their notion of stability for path objects is a strictly weaker notion than our stability under base change.
\end{proof}

% Copied from Mike Shulman
\begin{comment}
\newcommand{\too}[1][]{\ensuremath{\overset{#1}{\longrightarrow}}}

\begin{proposition}
Assume that any map $A \to 1$ has a path object $P(A)$ such that the map $P(A) \to A \times A$ has itself a path object.
Then any morphism factors as a trivial cofibration followed by a fibration.
\end{proposition}

\begin{proof}
Let $f : A \to B$ be any morphism.
Define $P f$ as the pullback in the following diagram:
\[
\xymatrix{
  A
  \ar[r]^{f}
  \ar[ddr]
  \ar@(d,ul)[dddr]_{f}
  \ar@{.>}[dr]
&
  B
  \ar[dr]
\\&
  P f
  \ar[r]^{q}
  \ar@{->>}[d]
  \drpullback
&
  P B\ar@{->>}[d]
\\&
  A\times B
  \ar[r]_{f\times 1_B}
  \ar@{->>}[d]
&
  B\times B
\\&
  B
}
\]
We claim that $A \to Pf \fib B$ is a factorization of $f$ as desired.
For this, it remains to be checked that $i : A \to Pf$ is a trivial cofibration.

  First, we need some basic operations on paths.
  Let $B \xto{r} P B \fib B\times B$ be a path object for $B$.
  Consider the following square of solid arrows:
  \begin{equation}
  \vcenter{\xymatrix@C=3pc{
      P B\ar@{=}[r]\ar[d]_{(1,r)} &
      P B\ar@{->>}[d]\\
      P B \times_B P B\ar[r]_-{\pi_1 \times\pi_3} \ar@{.>}[ur]^c &
      B\times B
    }}\label{eq:cdef}
  \end{equation}
  where the pullback $P B \times_B P B$ is over the ``middle'' copies of $B$.
  Then the left-hand map is the pullback of $r$ along the fibration $P B\fib B$, hence is an acyclic cofibration.
  Thus there exists a lift which we have called $c$; we think of it as a ``concatenation'' operation on paths.
  The commutativity of the upper triangle in~\eqref{eq:cdef} means that $c(1,r) = 1_{P B}$, i.e.\ post-concatenating with a constant path is the identity.
  Pulling~\eqref{eq:cdef} back along $f\times 1: A\times B \to B\times B$, we obtain
  \[\vcenter{\xymatrix@C=3pc{
      P f\ar@{=}[r]\ar[d]_{(1,r)} &
      P f\ar@{->>}[d]\\
      P f \times_B P B\ar[r]_-{\pi_1 \times\pi_3} \ar@{.>}[ur]^c &
      A\times B.
    }}\]
  Now the following square of solid arrows commutes:
  \begin{equation}
  \vcenter{\xymatrix{
      B \ar[r]^-r \ar[d]_r &
      P B\ar[r] &
      P_{(B\times B)} (P B)\ar@{->>}[d]\\
      P B\ar[rr]_-{(c(r,1),1)} \ar@{.>}[urr]^(.4)\psi &&
      P B \times_{(B\times B)} P B
    }}\label{eq:psidef}
  \end{equation}
  (where we have chosen a particular path object for $P B$ over $B\times B$).
  Of course, $r$ is an acyclic cofibration, so there exists a lift which we have called $\psi$.
  Commutativity of the lower triangle in~\eqref{eq:psidef} means that $\psi$ is a path from $c(r,1)$ to $1_{P B}$, i.e.\ pre-concatenating with a constant path is homotopic to the identity.
  Pulling~\eqref{eq:psidef} back along $f\times 1$, we obtain
  \begin{equation}
  \vcenter{\xymatrix{
      A \ar[r]^-i \ar[d]_i &
      P f\ar[r] &
      P_{(A\times B)} (P f)\ar@{->>}[d]\\
      P f\ar[rr]_-{(c(r,1),1)} \ar@{.>}[urr]^(.4)\psi &&
      P f \times_{(A\times B)} P f.
    }}\label{eq:psipb}
  \end{equation}
  Note that by~\ref{item:cat8p}, the pullback of $P_{(B\times B)} (P B)$ along $f\times 1$ is a valid path object for $P f$, so we have denoted it $P_{(A\times B)} (P f)$.

  To show that $i$ is an acyclic cofibration, we must show that it has the left lifting property with respect to fibrations.
  However, since fibrations are stable under pullback, in fact it suffices to find a lift in any commutative square
  \begin{equation}
  \vcenter{\xymatrix{
      A\ar[r]^d\ar[d]_i &
      D\ar@{->>}[d]\\
      P f\ar@{=}[r] &
      P f
    }}\label{eq:desired}
  \end{equation}
  with an identity arrow on the bottom.
  In this case, the composite $D \fib P f \fib B$ is a fibration, so in the following commutative square of solid arrows:
  \[\vcenter{\xymatrix{
      D\ar@{=}[rr]\ar[d]_{(1,r)} &&
      D\ar@{->>}[d]\\
      D\times_B P B\ar[r] \ar@{.>}[urr]^m &
      P f \times_B P B \ar[r]_-c &
      P f
    }}\]
  the left-hand map $(1,r)$ is an acyclic cofibration; thus there exists a lift which we have called $m$.
  Therefore, the following diagram commutes:
  \[\vcenter{\xymatrix@C=3pc{
      A \ar[r]^d\ar[d]_i &
      D\ar@{->>}[d]\\
      P f\ar[r]_-{c(r,1)} \ar[ur]|{m(d,q)} &
      P f
    }}\]
  (recall that $q: P f \to P B$ is the map induced by $f$).
  Finally, since $D\fib P f$ is a fibration, the left-hand map in the following square is an acyclic cofibration:
  \[\vcenter{\xymatrix@C=3pc{
      D\ar@{=}[r]\ar[d] &
      D\ar@{->>}[d]\\
      D\times_{P f} P_{(A\times B)}(P f)\ar[r]_-{\pi_2} \ar@{.>}[ur]^\tau &
      P f
    }}\]
  so we have a lift $\tau$.
  Now the composite
  \[ P f \xto{(m(d,q),\psi)} D\times_{P f} P_{(A\times B)}(P f) \too[\tau] D \]
  is a lift in~\eqref{eq:desired}.
  This proves~\ref{item:cat7}.
\end{proof}
\end{comment}

\bibliographystyle{plain}
\bibliography{../common/uniform-kan-bibliography}

\end{document}






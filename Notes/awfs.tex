\documentclass[reqno,10pt,a4paper,oneside]{amsart}
\usepackage{amssymb,amsmath,amsthm,stmaryrd,enumerate,geometry}
\usepackage[all]{xy} 

\SelectTips{cm}{}
\newdir{ >}{{}*!/-7pt/@{>}}
\newdir{m}{->}
%\newdir{m}{{}*!/-1pt/@{o}}
\newcommand{\xycenter}[1]{\vcenter{\hbox{\xymatrix{#1}}}}
%%% Pullback symbols
\newcommand{\ulpullback}[1][ul]{\save*!/#1-1.2pc/#1:(-1,1)@^{|-}\restore}
\newcommand{\dlpullback}[1][dl]{\save*!/#1-1.2pc/#1:(-1,1)@^{|-}\restore}
\newcommand{\urpullback}[1][ur]{\save*!/#1-1.2pc/#1:(-1,1)@^{|-}\restore}
\newcommand{\drpullback}[1][dr]{\save*!/#1-1.2pc/#1:(-1,1)@^{|-}\restore}

\newcommand{\catE}{\mathcal{E}}

\newcommand{\RAlg}{R\text{-}\mathrm{Alg}}
\newcommand{\Ralg}{\R\text{-}\mathrm{alg}}
\newcommand{\LAlg}{L\text{-}\mathrm{Alg}}
\newcommand{\Lalg}{\L\text{-}\mathrm{alg}}
\newcommand{\LLAlg}{L'\text{-}\mathrm{Alg}}
\newcommand{\LLalg}{\L'\text{-}\mathrm{alg}}


\newcommand{\ie}{\text{i.e.\ }}
\newcommand{\eg}{\text{e.g.}}
\newcommand{\resp}{\text{resp.\ }}
\newcommand{\myemph}{\textit} 
\newcommand{\changed}{\todo[noline]{Changed}}




\newtheorem{theorem}{Theorem}[section]
\newtheorem*{theorem*}{Theorem}
\newtheorem{lemma}[theorem]{Lemma} 
\newtheorem{proposition}[theorem]{Proposition} 
\newtheorem{corollary}[theorem]{Corollary}  
\newtheorem{apptheorem}{Theorem}



\theoremstyle{definition}
\newtheorem{definition}[theorem]{Definition}	
\newtheorem*{definition*}{Definition}	



\newtheorem{remark}[theorem]{Remark} 
\newtheorem*{remark*}{Remark} 
\newtheorem{example}[theorem]{Example}
\newtheorem{examples}[theorem]{Examples}
\newtheorem*{example*}{Example}
\newtheorem*{examples*}{Examples}



\newcommand{\defeq}{=_{\mathrm{def}}}
\newcommand{\co}{\colon}
\newcommand{\iso}{\cong} 
\newcommand{\rev}{\mathit{\vee}}
\newcommand{\op}{\mathrm{op}}
\newcommand{\catequiv}{\simeq} 
\newcommand{\cateq}{\simeq} 
\newcommand{\coend}{\int}




\newcommand{\cat}[1]{\mathbb{#1}}
\newcommand{\catA}{\cat{A}}
\newcommand{\catB}{\cat{B}}
\newcommand{\catC}{\cat{C}}
\newcommand{\catD}{\cat{D}}
\newcommand{\catK}{\cat{K}}
\newcommand{\catM}{\cat{M}}


\newcommand{\SSet}{\mathbf{SSet}}
\newcommand{\UU}{\overline{\mathsf{U}}}
\newcommand{\U}{\mathsf{U}}
\newcommand{\Weq}{\mathrm{Eq}}

\author[]{Nicola Gambino}

\address{School of Mathematics, University of Leeds, Leeds LS2 9JT, United Kingdom}
\email{n.gambino@leeds.ac.uk}


\author{Christian Sattler}
\address{School of Mathematics, University of Leeds, Leeds LS2 9JT, United Kingdom}
\email{c.sattler@leeds.ac.uk}




\title[]{Draft Notes on Uniform Kan fibrations}



\date{June 30th, 2015}



\begin{document}

\begin{abstract}
The aim of this short note is to show how Garner's small object argument can be applied to
obtain an algebraic weak factorisation system in which the right class of maps are what we
call uniform strong Kan fibrations, i.e.\,Kan fibrations equipped with a choice of lifts for horn
inclusions which satisfy a uniformily condition. 
\end{abstract}


\maketitle

 
 \section{Review of algebraic weak factorisation systems}
 
 
 
 Let $\catE$ be a category. 
 
 \begin{definition} An \emph{algebraic weak factorisation system} on $\catE$ consists of a monad 
 $(R, \ldots , \lambda)$ and a comonad $(L,  \ldots , \rho)$ on $\catE^2$ such that
 \begin{itemize}
 \item the pointed endofunctor $(R, \lambda)$ and the copointed endofunctor $(L, \rho)$ 
 form a functorial factorisation system on $\catE$,
 \item the canonical map $\delta \co LR \Rightarrow RL$ is a distributive law.
 \end{itemize}
 \end{definition}
 
 
 Given an algebraic weak factorisation system as above, for a map $f \co A \to B$ we have a factorisation
 \[
 \xymatrix{
  A \ar[dr]_f \ar[rr]^{\lambda_f} & & P_f \ar[dl]^{\rho_f} \\
   & B & }
   \]
   The functor part of the comonad $L$ sends $f \co A \to B$ to $\lambda_f \co A \to P_f$ and the components of the counit are  the commutative squares
 \[
 \xymatrix{
 A \ar@{=}[r] \ar[d]_{\lambda_f} & A \ar[d]^f \\
 P_f \ar[r]_{\rho_f} & B \, . }
 \] 
 Dually,   the functor part of the monad $R$ sends $f \co A \to B$ to $\rho_f \co P_f \to B$ and the components of the the unit are given  by the commutative squares
 \[
 \xycenter{
 A  \ar[d]_f \ar[r]^{\lambda_f}  & P_f \ar[d]^{\rho_f} \\
 B \ar@{=}[r] & B \, . }
 \]
 We write  $\LAlg$ for the category of coalgebras for the comonad $L$, and $\RAlg$ for the category of algebras for the monad $R$. 
 To give an $L$-coalgebra structure on a map $i \co A \to B$ amounts
 to giving a diagonal filler in the square
 \[
 \xymatrix{
 A \ar[r]^{\lambda_i}  \ar[d]_{i} & A \ar[d]^{\rho_i} \\
 B \ar@{=}[r]  \ar@{.>}[ur]^s & B }
 \]
 Dually, to give an $R$-algebra structure on a map $f \co X \to Y$ is to give
 a diagonal filler in the square
 \[
 \xymatrix{
 X  \ar@{=}[r]    \ar[d]_{\lambda_f} & X \ar[d]^f \\
 P_f \ar[r]_{\rho_f}  \ar@{.>}[ur]^t & Y }
 \]
Now, given an $L$-coalgebra $(i, s) \co A \to B$ and an $R$-coalgebra $(f, t) \co X \to Y$,
 every commutative square
 \[
\xymatrix@C=1cm{
 A  \ar[d]_i \ar[r]^u  & X \ar[d]^{f}  \\
 B \ar[r]_v & Y}
 \]
  admits a canonical diagonal filler, given by the composite
 \[
 \xymatrix{ B \ar[r]^{s} & P_i \ar[r]^{P_{u,v}} & P_f \ar[r]^{t} & X}
 \]
 where $P_{u,v}$ is obtained by the functoriality of the factorization. 
 
 We write $\mathcal{L}$ and $\mathcal{R}$ for the
 closures under retracts of the classes of maps in $\catE$ admitting an $L$-algebra 
 or an $R$-algebra structure, respectively. Then $(\mathcal{L}, \mathcal{R})$ is
 a weak factorisation system.
 
 
 
\section{Uniform Kan fibrations}

For $k, n$ we write $i_{k,n} \co \Lambda^k_n \to \Delta_n$ for the corresponding horn inclusion. Given 
$m$, we write 
\[
j_{k,n,m} \co \Lambda_{k,n} \times \Delta_m \to \Delta_n \times \Delta_m
\] 
for the 
product of the horn inclusion $i_{k,n}$ with the identity map on $\Delta_m$. We call these maps the
\emph{parametrized horn inclusions}. A morphism of parametrized horn inclusions is a commutative square
of the form
\[
\xymatrix{
 \Lambda_{k_1,n_1} \times \Delta_{m_1} \ar[r] \ar[d] &  \Lambda_{k_2,n_2} \times \Delta_{m_2} \ar[d] \\
 \Lambda_{k_1,n_1} \times \Delta_{m_2} \ar[r] &   \Delta_{n_2} \times \Delta_{m_2}}
 \]
We write $\mathbf{J}$ for the category of parametrized horn inclusions and morphisms between them. This is a subcategory of $\SSet^2$ and we write
\[
 I \co \mathbf{J} \to \SSet^2
 \]
 for the inclusion functor.
 
 \begin{definition} A \emph{uniform Kan fibration} is a map of simplicial sets $f \co X \to Y$ 
 equipped with a function $\phi$ that provides a filler for every diagram of the form
 \[
\xymatrix@C=1.8cm@R=1.5cm{
 \Lambda_{k,n} \times \Delta_m \ar[r]^-u \ar[d]_{i_{k,n,m}} & X \ar[d]^f \\
 \Delta_{n} \times \Delta_m \ar[r]_-v  \ar@{.>}[ur]^{\phi_{u,v}}&  Y}
 \]
 such that the uniformity condition  expressed by the following diagram holds:
 \[
\xymatrix@C=1.8cm@R=1.5cm{
 \Lambda_{k',n'} \times \Delta_{m'} \ar[r] \ar[d] &  \Lambda_{k,n} \times \Delta_m \ar[d] \ar[r]^-u & X \ar[d]^{f}  \\
\Delta_{k',n'} \times \Delta_{m'} \ar[r]  \ar@{.>}[urr]^{\phi_{u',v'}} &   \Delta_{n} \times \Delta_{m}  \ar[r]_-v \
 \ar@{.>}[ur]_{\phi_{u,v}}
 & Y \, .}
 \]
 \end{definition}

 
 
 \section{Garner's small object argument}
 
 
 
 
 
 
 We wish to apply Garner's small object argument so as to obtain an algebraic weak factorisation system
 in which the right class of maps are exactly the uniform Kan fibrations.  In order to apply Garner's results, one has to show the following.
 
 \begin{proposition} The inclusion functor $I \co \mathbf{J} \to \SSet^2$ preserves $\omega$-filtered 
 colimits.
 \end{proposition} 
 
 The main theorems in Garner's paper then give us the following result.
 
 \begin{theorem} The generalized horn inclusions generate a free algebraic weak factorisation system on $\SSet$. \end{theorem}
 
 More precisely, the statement of the theorem expresses that there exists an algebraic weak factorisation system
 $(L, R)$ on $\SSet$ and a functor
 \[
 \eta \co \mathbf{J} \to \LAlg
 \]
 over $\SSet^2$, which is universal in the sense that 
 for every algebraic weak factorisation system $(L', R')$ on $\SSet$ and every $F \co \mathbf{J} \to \LLAlg$ there exists a unique
 morphism of algebraic weak factorisation systems $F^\sharp \co (L, R) \to (L', R')$ such that the diagram
 \[
 \xymatrix{
 \mathbf{J} \ar[r]^{\eta} \ar@/_1pc/[dr]_{F} & \LAlg \ar[d]^{F^\sharp\text{-}{\mathrm{Alg}}}  \\
  & \LLAlg}
  \] 
  commutes. What follows is simply an unfolding
 of Garner's construction in this special case. 
 
 \medskip
 
  \noindent
 {\itshape First step.} We form the so-called density comonad $(M, \delta, \varepsilon)  \co \SSet^2 \to \SSet^2$ 
 associated to the functor $I \co \mathbf{J} \to \SSet$.  This is done by taking the left Kan extension of $I$ along itself, 
 \[
 \xymatrix{
 \mathbf{J} \ar@/_1pc/[dr]_I \ar[r]^I  \ar@{}[dr]|{\quad \Rightarrow}  & \SSet^2 \ar[d]^{M} \\
  & \SSet^2 \, . }
  \]
 For $f \co X \to Y$, we have the coend formula
\[
M(f) = 
\int^{(k,n,m)} \SSet^2( j_{k,m,n}, f )  \cdot j_{k,m,n} \, .
\] 
Here, note that $\SSet^2( j_{k,m,n}, f )$ is  the set of squares of the form
\[
\xymatrix@C=1.2cm@R=1.2cm{
 \Lambda_{k,n} \times \Delta_m \ar[r]^-u \ar[d]_{i_{k,m,n}} & X \ar[d]^f \\
 \Delta_{n} \times \Delta_m \ar[r]_-v   &  Y \, .}
 \]
 The counit $\varepsilon$ of the comonad has component at $f$ given by a diagram of the form
 \[
 \xymatrix@C=1.2cm{
  \bar{X} \ar[r]  \ar[d]_{M(f)} & X \ar[d]^f \\
  \bar{Y} \ar[r]  & Y }
  \]
 
 
 \medskip
 
 \noindent
 {\itshape Second step.} We define functors $L \co
 \SSet^2 \to \SSet^2$ and $R_0 \co \SSet^2 \to \SSet^2$    via the following pushout diagram:
\[
   \xymatrix@C=1.2cm{
  \bar{X} \ar[r] \ar[d]_{M(f)} & X \ar[d]^{L_0(f)} \ar@/^1.5pc/[ddr]^{f} \\
  \bar{Y} \ar[r] \ar@/_1.5pc/[drr]  & P_0(f) \ar@{.>}[dr]_-(.3){R_0(f)}  \\ 
   & & Y  }
 \]
The functor $L_0 \co  \SSet^2 \to \SSet^2$ inherits a comonad structure from the 
 comonad~$(M, \delta, \varepsilon)$, but we do not spell  the details out.  Also note that the endofunctor
 $R_0 \co \SSet^2 \to \SSet^2$ acquires the structure of a pointed endofunctor by considering the natural
 transformation~$\lambda_0 \co 1  \Rightarrow R_0$ with components:
 \[
 \xymatrix@C=1.4cm{
 X \ar[d]_f \ar[r]^{L_0(f)} \ar[d] & P_0(f) \ar[d]^{R_0(f)} \\
 Y \ar@{=}[r]  & Y \, .}
 \]
 
 
 \medskip

\noindent 
 {\itshape Third step.} We now form the free monad $(R,  \ldots, \lambda)$ on the pointed endofunctor $(R_0, \lambda_0)$. In this case, we start by forming a diagram in $\SSet^2$ of the form
 \[
 \xymatrix{
 X_0 \ar[r]^{\sigma_1}  \ar[d]^{f_0} & X_1 \ar[r]^{\sigma_2} \ar[d]^{f_1} & \ldots \ar[r] & X_n \ar[r] \ar[d]^{f_n} 
 & X_{n+1} \ar[r] \ar[d]^{f_{n+1}} & \ldots & & P(f) \ar[d]^{\rho_f} \\
 Y \ar[r] & Y \ar[r] & \ldots \ar[r] & Y \ar[r] & Y \ar[r] & \ldots & & Y }
\] 
We then define $P_f$ as the colimit 
 \[
 \xymatrix{
 X_0 \ar[r]^{\sigma_1} \ar@/_1pc/[drrr]_{\lambda_0} & X_1 \ar@/_0.5pc/[drr]^{\lambda_1} \ar[r]^{\sigma_2}  & \ldots \ar[r] & X_n \ar[r]  \ar[d]^{\lambda_n} & X_{n+1} \ar[r] \ar@/^1pc/[dl]_{\lambda_{n+1}} & \ldots  \\
  & & & P(f) & & }
  \]
Since we will define $f_0 = f$, the required map $\lambda_f \co X \to P(f)$ is defined as $\lambda_0 
\co X_0 \to P(f)$. The map $\rho_f \co P(f) \to Y$ is then obtained by the universal property of $P(f)$.


\medskip
 
  For every $n \in \omega$, we define maps $f_n \co X_n \to Y$, $f_{n+1} \co X_{n+1} \to Y$, and $a_{n} \co R_0(f_n) \to f_{n+1}$. This
  allows us to define $\sigma_{n}$ as the composite below:
  \[
  \xymatrix@C=1.5cm{
  f_n \ar[r]^{\eta f_n}  \ar@/_1pc/[dr]_{\sigma_n} & R_0(f_n) \ar[d]^{a_{n}} \\ 
   & f_{n+1} }
   \]
   Note that this diagram dictates how to define $\sigma_n$. We proceed recursively as follows:
 
 \begin{itemize}
 \item Base case: Let $f_0 = f$, $f_1 = R_0(f)$ and $a_0 = 1_{R(f)}$.
 \item Successor step: assuming that we have defined $f_n$, $f_{n+1}$ and $a_{n}$,  recall that the arrow $R_0(f_{n+1})$ is given by the pushout
 \[
   \xymatrix@C=1.2cm{
 \bar{X}_{n+1} \ar[r]  \ar[d]_{M(f_{n+1})} & X_n \ar[d]^{L(f_{n+1})} \ar@/^1.8pc/[ddr]^{f_{n+1}} \\
\bar{Y} \ar[r] \ar@/_1.5pc/[drr]  & P(f_{n+1}) \ar@{.>}[dr]_-(.3){R_0(f_{n+1})} \\ 
   & & Y  }
 \]
We then define $f_{n+2}$ and $a_{n+1}$ via  the coequalizer diagram
 \[
 \xymatrix@=2cm{
 R_0(f_n) \ar@<1ex>[r]^{R_0 (a_n) \cdot \eta_{R_0(f_n)}}  \ar@<-1ex>[r]_{ R_0(a_n) \cdot R_0(\eta_{f_n})}& R(f_{n+1}) 
 \ar[r]^{a_{n+1}} & f_{n+2} }
 \]
 \end{itemize}
 
 The free algebraic weak factorisation system associated to $I \co \mathbf{J} \to \SSet^2$ is then
 given by the pair $(L, R)$. A further application of Garner's results, we get the following.
 
 \begin{theorem} The algebraic weak factorisation system $(L, R)$ is algebraically-free, in 
 the sense that the functor
 \[
 \xymatrix@C=1cm{
  \RAlg \ar[r]^-{\textup{lift}} &
    \big( \LAlg \big)^\pitchfork \ar[r]^-{\eta^\pitchfork} &
   \mathbf{J}^\pitchfork}
  \]
  is an isomorphism. In particular, the algebras for the monad $R$ are exactly the
  uniform Kan fibrations.
  \end{theorem} 
 
 


\end{document}


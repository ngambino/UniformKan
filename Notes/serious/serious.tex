\documentclass[reqno,10pt,a4paper,oneside]{amsart}

\usepackage{uniform-kan-prelude}

\newcommand{\SSetCart}{\SSet_{\text{cart}}^{\to}}

\begin{document}

\title{Leibniz-product-uniformity with respect to one-dimensional horns}

\maketitle

\section*{Trivial fibrations}

Let $\cal{I}$ be a full subcategory of $\SSetCart$ consisting only of monomorphisms.
For example, we might consider the following increasing sequence of choices:
\begin{itemize}
\item
Boundary inclusions: $\cal{I}_1 = \braces{i^n : \partial \Delta^n \to \Delta^n}$.
Note that this category is discrete.
It is also the only class of monomorphisms considered here not defined entirely in terms of the allowed codomains.
\item
Subobjects of representables: $\cal{I}_2 = \braces{A \hookrightarrow \Delta^n}$.
Other than the previous option, this seems like the most natural small choice.
\item
Subobjects of products of representables: $\cal{I}_3 = \braces{A \hookrightarrow \Delta^{n_1} \times \ldots \times \Delta^{n_k}}$.
\item
Subobjects of finite simplicial sets: $\cal{I}_4 = \braces{A \hookrightarrow B \mid \text{$B$ finite and finite-dim.}}$.
\item
All monomorphisms $\cal{I}_5$.
Note that this is not a small class.
In addition, all other classes contain only decidable monomorphisms $i : A \to B$ (\ie every $i_n : A_n \to B_n$ has a decidable image) with decidable codomain (\ie degeneracies in $B$ are decidable).
\end{itemize}

Taking $\cal{I}$ as generating left category, we get an algebraic weak factorization system --- except in case of $\cal{I}_5$, which is not small; however, the notion of category of algebras over a virtual right pointed endofunctor can still be interpreted in this case, although it will not be locally small.
The right maps will be called \emph{$\cal{I}$-trivial fibrations}.

After previous experiences, let us start with some sanity checks.

\begin{definition}
A \emph{regular trivial fibration} is a map $p : X \to Y$ with designated fillers $d$ for squares $i^n \to p$ such that whenever the square factors as shown below,
\[
\xymatrix{
  \partial \Delta^n
  \ar[r]
  \ar[d]^{i^n}
&
  \Delta^{n-1}
  \ar[r]
  \ar[d]^{\id}
&
  X
  \ar[d]^{p}
\\
  \Delta^n
  \ar[r]_{s_k^{n-1}}
  \ar@{.>}[urr]^(0.3){d}
&
  \Delta^{n-1}
  \ar[r]
&
  Y
}
\]
the composite filler is coherent with respect to the trivial filler in the right square.
\end{definition}

Regular trivial fibrations are the right class for the algebraic weak factorization system with generating left category having objects $\cal{I}_1 \cup \braces{\id : \Delta^n \to \Delta^n}$ and non-trivial morphisms as above:
\[
\xymatrix{
  \partial \Delta^n
  \ar[r]
  \ar[d]^{i^n}
&
  \Delta^{n-1}
  \ar[d]^{\id}
\\
  \Delta^n
  \ar[r]_{s_k^{n-1}}
&
  \Delta^{n-1}
}
\]

Classically, every ordinary trivial fibration is a regular trivial fibration, simply because we can choose (using the axiom of choice) designated fillers for $i^n \to p$ based on (using excluded middle) whether that square factors through some $s_k^{n-1} : i^n \to \id$.
Note that it does not matter which degeneracy we choose if multiple are available --- the resulting diagonal filler will cohere with all possible choices.

\begin{lemma}
Every regular trivial fibration is an $\cal{I}_2$-trivial fibration.
\end{lemma}

\begin{proof}
For the rest of the proof, fix a regular trivial fibration $p : X \to Y$.
Consider a square $i \to p$ with $i : \cal{I}_2$.
We define a diagonal filler by decomposing $i$ into a finite composition of cobase changes of boundary inclusions, filling each of these according to the previous paragraph.
Crucially, this process is independent of the actual order of the boundary fillings (note that this is not true for the analogous situation of horn fillings).
For each morphism in $\cal{I}_2$ and commuting triangle of squares
\[
\xymatrix{
  A'
  \ar[r]
  \ar[d]
  \pullback{dr}
&
  A
  \ar[d]
  \ar[r]
&
  X
  \ar[d]
\\
  \Delta^{n'}
  \ar[r]
  \ar@{.>}[urr]
&
  \Delta^n
  \ar[r]
  \ar@{.>}[ur]
&
  Y
}
\]
we need to exhibit coherence of fillers as indicated.
By ``vertical'' induction and the remark on order invariance of boundary fillings, it will suffice to study the case where the middle vertical map is a boundary inclusion $i^n : \partial \Delta^n \to \Delta^n$.
Working ``horizontally'', it suffices to study the situation where the map $\Delta^{n'} \to \Delta^n$ is a face or degeneracy map as $\Delta$ is generated by these.

Let us first examine the case of a face operation.
\[
\xymatrix{
  \Delta^n
  \ar[r]
  \ar[d]
  \pullback{dr}
&
  \partial \Delta^{n+1}
  \ar[d]
  \ar[r]
&
  X
  \ar[d]
\\
  \Delta^n
  \ar[r]_{d_k^{n+1}}
  \ar@{.>}[urr]
&
  \Delta^{n+1}
  \ar[r]
  \ar@{.>}[ur]
&
  Y
}
\]
Since the left vertical map is necessarily the identity, the filler for the composite square is uniquely determined, so there is no coherence to be verified.

Let us now examine the case of a degeneracy operation.
\[
\xymatrix{
  2 \times \partial \Delta^n
  \ar[r]
  \ar[d]
  \ar@/^2em/[rr]^(0.3){\pi_2}
  \pullback{dr}
&
  \bigcup_{i \neq k, k+1} \Delta^{[n+1] - i}
  \ar[r]
  \ar[d]
  \pullback{dr}
&
  \partial \Delta^n
  \ar[d]
  \ar[r]
&
  X
  \ar[dd]
\\
  2 \times \Delta^n
  \ar[r]
  \ar@/_2em/[rr]_(0.3){\pi_2}
&
  \partial \Delta^{n+1}
  \ar[r]
  \ar[d]
  \ar@{.>}[urr]
  \pullback{ul}
&
  \Delta^n
  \ar[d]
  \ar@{.>}[ur]
\\&
  \Delta^{n+1}
  \ar[r]_{s_k^n}
  \ar@{.>}[uurr]
&
  \Delta^n
  \ar[r]
  \ar@{.>}[uur]
&
  Y
}
\]
The pullback of the boundary inclusion $\partial \Delta^n \to \Delta^n$ along $s_k^n$ decomposes as a cobase change of two parallel boundary inclusions of dimension $n$ followed by a boundary inclusion of dimension $n+1$ as indicated.
The two parallel boundary fillings are identical copies of the original right square boundary filling, so they cohere as indicated.
Finally, the filling for the boundary inclusion $\partial \Delta^{n+1} \to \Delta^{n+1}$ coheres as indicated by how boundary filling was originally defined for degenerate squares.
\end{proof}

\begin{lemma}
Consider a class $\cal{I}_2 \subseteq \cal{I} \subseteq \cal{I}_5$.
The first inclusion induces a restriction functor of categories of algebras over (virtual) pointed endofunctors from $\cal{I}$ to $\cal{I}_2$.
This functor is an isomorphism.
In particular, the classes $\cal{I}_2$ to $\cal{I}_4$ generate the same algebraic weak factorization systems.
\end{lemma}

\begin{proof}
We will show that the restriction functor is bijective on objects and morphisms.

Let us first look only at the objects.
Fix an algebra over the right pointed endofunctor for $\cal{I}_2$.
This consists of a map $p : X \to Y$ and coherent fillers for every square $i \to p$ with $i : \cal{I}_2$.
Let us show that there is a unique $\cal{I}$-algebra restricting to it.

For this, consider a square $i \to p$ with $i : \cal{I}$:
\[
\xymatrix{
  A
  \ar[r]^{u}
  \ar[d]^{i}
&
  X
  \ar[d]^{p}
\\
  B
  \ar[r]^{v}
  \ar@{.>}[ur]
&
  Y
}
\]
Write $B$ as a colimit of representables:
\[
\xymatrix{
  \colim_{\Delta^n \to B} \Delta^n \times_B A
  \ar[r]^-{\simeq}
  \ar[d]
&
  A
  \ar[r]
  \ar[d]^{i}
&
  X
  \ar[d]^{p}
\\
  \colim_{\Delta^n \to B} \Delta^n
  \ar[r]^-{\simeq}
  \ar@{.>}[urr]
&
  B
  \ar[r]
&
  Y
}
\]
For each representable, we are forced by $\cal{I}$-coherence to solve the lifting problem using the algebra structure for $\cal{I}_2$.
Since these solutions are stable under pullback, they coherently extend to a unique global solution to the lifting problem $\colim_{s : \Delta^n \to B} s^* i \to p$.

Let us verify that the fillers for $i \to p$ with $i : \cal{I}$ are themselves stable under pullback.
Consider the following situation with $i', i : \cal{I}$:
\[
\xymatrix{
  A'
  \ar[r]^{a}
  \ar[d]_{i'}
  \pullback{dr}
&
  A
  \ar[r]^{u}
  \ar[d]_(0.3){i}
&
  X
  \ar[d]^{p}
\\
  B'
  \ar[r]_{b}
  \ar@{.>}[urr]
&
  B
  \ar[r]_{v}
  \ar@{.>}[ur]
&
  Y
}
\]
Again writing $B$ and $B'$ as colimits of representables and exploiting that $i' = b^* i$, this becomes the following diagram in the arrow category:

\[
\xymatrix{
  \colim_{s' : \Delta^n \to B'} (bs')^* i
  \ar[rr]
  \ar[dr]
&&
  p
\\&
  \colim_{s : \Delta^n \to B} s^* i
  \ar[ur]
}
\]
Here, the map between the colimits is induced by the functor $y \downarrow B' \to y \downarrow B$ given by composition with $b$ translating between the shape categories: for each fixed representable $s' : \Delta^n \to B'$, the lifting problems coming from $i'$ via $s'$ and $i$ via $bs'$ are identical.
This proves the desired coherence.

The argument for bijectivity on morphisms of algebras is similar.
Note that injectivity is trivial since the restriction functor leaves the carrier part of the algebras untouched.
So let us verify that every morphism $w : p \to p'$ of $\cal{I}_2$-algebras is also a morphism $p \to p'$ of $\cal{I}$-algebras as constructed above.
Fix a lifting problem $i \to p$ with $i : A \to B$ in $\cal{I}$.
We need to show that the constructed fillers for $i \to p$ and $i \to p'$ coincide when mediated by $w$:
\[
\xymatrix{
&
  p
  \ar[dr]^{w}
\\
  \colim_{s : \Delta^n \to B} s^* i
  \ar[ur]
  \ar[rr]
&&
  p'
}
\]
By $w : p \to p'$ being a morphism of $\cal{I}_2$-algebras, this holds for each representable separately.
Hence, it also holds globally.
\end{proof}

\begin{question}
There must be a more high-level categorical way of writing the above proof.
Nicola?
\end{question}

\begin{question}
What is the relationship between the classes $\cal{I}_1 \subseteq \cal{I}_2$? 
Is the restriction functor of the right algebra categories really proper (classically, say)?
What is the relationship between $\cal{I_2}$-trivial fibrations and regular fibrations?
\end{question}

After the preceeding lemma, we will freely intermingle $\cal{I}_2$ to $\cal{I}_5$, using whatever is most suitable to the situation at hand.

\section*{Fibrations}

Fix a class $\cal{I}$ as in the previous section.
Recall the endofunctors $h_0^1 \hattimes \arghole$ and $h_1^1 \hattimes \arghole$ on $\SSetCart$.
Let $\cal{J}$ be the disjoint sum of these functors restricted to $\cal{I}$.
Taking $\cal{J}$ as left generating functor, the right maps will be called \emph{$\cal{I}$-fibrations}.
\footnote{The rationale for omitting higher dimensional horns is that those (at least in bare form) are indirectly included as retracts of one dimensional horns by Leibniz product with certain subobjects of representables.}
Note that a $\cal{J}$-algebra structure on a map $p$ corresponds precisely to $\cal{I}$-algebra structures on $\hatexp(h_0^1, p)$ and $\hatexp(h_1^1, p)$.
Thus, by definition, a map $p$ is an $\cal{I}$-fibration exactly if $\hatexp(h_0^1, p)$ and $\hatexp(h_1^1, p)$ are $\cal{I}$-trivial fibrations.

\begin{lemma}
Classically, every ordinary fibration is an $\cal{I}_5$-fibration.
\end{lemma}

\begin{proof}
Let $p : X \to Y$ be an ordinary fibration.
Then $\hatexp(h_k^1, p)$ is an ordinary trivial fibration for $k = 0, 1$.
Classically, it can be made into regular trivial fibration.
By previous lemmata, we then have $\hatexp(h_k^1, p)$ an $\cal{I}_2$-trivial and $\cal{I}_5$-trivial fibration.
But that means $p$ is an $\cal{I}_5$-fibration.
\end{proof}

\begin{definition}
For the purpose of this definition, fix a direction $k \in \braces{0, 1}$ of filling.
Consider a simplicial map $i : A \to B$ and a square $h_k^1 \hattimes a \to p$:
\[
\xymatrix{
  \braces{1-k} \times B +_{\braces{1-k} \times B} \Delta^1 \times A
  \ar[r]
  \ar[d]_{h_k^1 \hattimes i}
&
  X
  \ar[d]^{p}
\\
  \Delta^1 \times B
  \ar[r]
  \ar@{.>}[ur]
&
  Y
}
\]
A \emph{filling} for this square is a diagonal map as indicated.
The simple, though underutilized map of arrows $\theta : \canonical_{0 \to \braces{k}} \to h_k^1$ induces a second square $i \to p$ via precomposition:
\[
\xymatrix{
  \canonical_{0 \to \braces{k}} \hattimes i
  \ar[rr]
  \ar[dr]_{\theta \hattimes i}
&&
  p
\\&
  h_k^1 \hattimes i
  \ar[ur]
}
\]
A \emph{composition} for the original square $h_k^1 \hattimes i \to p$ is a diagonal map for the square $i \to p$ as indicated below:
\[
\xymatrix{
  \braces{k} \times A
  \ar[r]
  \ar[d]_{i}
&
  X
  \ar[d]^{p}
\\
  \braces{k} \times B
  \ar[r]
  \ar@{.>}[ur]
&
  Y
}
\]
A filling \emph{extends} a given composition if the diagonal maps for $h_k^1 \hattimes i \to p$ and $i \to p$ commute in the obvious way.
\end{definition}

\begin{definition}
\emph{Coherent filling} (respectively, \emph{coherent composition)} for a simplicial map $p$ consists of a choice natural in $i : \cal{I}$ of fillings (compositions) for squares of the form $h_k^1 \hattimes i \to p$.
\end{definition}

Observe that having coherent fillings is the definition of an $\cal{I}$-fibration.

\begin{lemma}
Assume $\cal{I}$ is closed under Leibniz product with $h_k^1$ for $k = 0, 1$ (in the sense of $h_k^1 \hattimes \arghole$ restricting to an endofunctor on $\cal{I}$).
Then any map $p : X \to Y$ with coherent composition has coherent filling.
Furthermore, the coherent filling can be made to extend the coherent composition.
\end{lemma}

\begin{proof}
The crucial ingredient (``connections'') is the fact that $h_k^1$ is a strong deformation retract.
Recalling our previous characterization of strong homotopy equivalences, this implies that $\theta \hattimes h_k^1$ is a section in the arrow category.

Suppose we are given a square $v : h_k^1 \hattimes c \to p$ to fill.
Since $\theta \hattimes h_k^1$ is a section, so is $(\theta \hattimes h_k^1) \hattimes c \simeq \theta \hattimes (h_k^1 \hattimes c)$.
We thus may factor the given square as follows:
\[
\xymatrix{
  h_k^1 \hattimes c
  \ar[rr]^{v}
  \ar[dr]_{\theta \hattimes (h_k^1 \hattimes c)}
&&
  p
\\&
  h_k^1 \hattimes (h_k^1 \hattimes c)
  \ar[ur]
}
\]
So a filling for the square $v : h_k^1 \hattimes c \to p$ is nothing more than a composition for the derived square $h_k^1 \hattimes (h_k^1 \hattimes c) \to p$!
By assumption, the map $h_k^1 \hattimes c$ is again an object of $\cal{I}$ and has such a composition available.
Observe that this argument is entirely natural in $c : \cal{I}$.
Thus, coherent composition induces coherent filling.

Let us now show that the defined filling extends the given composition.
For this, consider the below situation:
\[
\xymatrix{
  c
  \ar[dr]^{\theta \hattimes h_k^1}
  \ar[dd]_{\theta \hattimes h_k^1}
\\&
  h_k^1 \hattimes c
  \ar[dd]_(0.25){h_k^1 \hattimes (\theta \hattimes c)}
  \ar[dr]^{w}
\\
  h_k^1 \hattimes c
  \ar[rr]_(0.35){v}
  \ar[dr]_{\theta \hattimes (h_k^1 \hattimes c)}
&&
  p
\\&
  h_k^1 \hattimes (h_k^1 \hattimes c)
  \ar[ur]
}
\]
Recall that the square $h_k^1 \hattimes (h_k^1 \hattimes c) \to p$ was defined in terms of $v$ via a retraction $r : h_k^1 \hattimes h_k^1 \to h_k^1$ to $\theta \hattimes h_k^1$.
Note that $r$ is automatically also a retraction to $h_k^1 \hattimes \theta$.
It follows that $w = v$ in the above diagram.
We need to show that the diagonal map for the square $c \to p$ induced by the filling of $v$ (the new composition) coincides with the given composition of $v$.
As seen in the diagram, this is now just coherence of the given coherent composition with respect to $\theta \hattimes c$.
\end{proof}

\section*{Dependent Product}

\begin{lemma}
Dependent product $\Pi_u$ along any map $u : Y \to X$ preserves $\cal{I}_5$-trivial fibrations.
Furthermore $\Pi_u$ extends to a functor $\cod^{-1}(X) \to \cod^{-1}(Y)$ between fibers of the algebra category for $\cal{I}_5$ and is preserves by base change.
\end{lemma}

\begin{proof}[Proof (sketch)]
This boils down by adjointness to maps in $\cal{I}_5$ being stable under base change.
Coherence with respect to cartesian squares is easy to verify.
\begin{comment}
Fix an arbitrary map $u : Y \to X$ and let $q : Z \to Y$ be a uniform trivial fibration.
We want to show that $\Pi_u q : \Pi_u Z \to Y$ is a uniform trivial fibration.
Coherently solving a lifting problem $(s, t) : i \to \Pi_u q$ with $i : \cal{I}$ is equivalent to coherently solving $u^*(i) \pitchfork q$ where $u^*(i) : u^* A \to u^* B$.

\[
\xymatrix{
  A'''
  \ar[rr]
  \ar[dd]
  \ar[dr]
&&
  A''
  \ar[rr]
  \ar[dd]
  \ar[dr]
&&
  Z
  \ar[dd]
\\&
  A'
  \ar[rr]
  \ar[dd]
  \pullback{dr}
&&
  A
  \ar[rr]
  \ar[dd]
&&
  \Pi_u Z
  \ar[dd]
\\
  B'''
  \ar[rr]
  \ar[dr]
&&
  B''
  \ar[rr]
  \ar[dr]
&&
  Y
  \ar[dr]
\\&
  \Delta^{n'}
  \ar[rr]
&&
  \Delta^n
  \ar[rr]
&&
  X
}
\]
\end{comment}
\end{proof}

\begin{question}
Does the adjunction of base change and dependent product transfer from fibers of the arrow category to fibers of the algebra categories?
If so, then preservation of dependent products under base change would mean Beck-Chevalley.
\end{question}

\begin{lemma}
Dependent product $\Pi_u$ along $\cal{I}_5$-fibrations $p : Y \to X$ preserves $\cal{I}_5$-fibrations.
Furthermore $\Pi_u$ extends to a functor $\cod^{-1}(X) \to \cod^{-1}(Y)$ between fibers of the algebra category for $\cal{J}_5 = \braces{h_k^1 \hattimes i \mid k = 0, 1, i : \cal{I}_5}$ and is preserved by base change.
\end{lemma}

\begin{question}
Similar questions as for the corresponding lemma for trivial fibrations apply.
The presentation of the following proof can still be improved.
\end{question}

\begin{proof}
Let $p : Y \to X$ be a uniform fibration.
For any uniform fibration $q : Z \to Y$, our goal is to show that the dependent product $\Pi_p q : \Pi_p Z \to X$ is a uniform fibration as well.
By the previous lemma, it will suffice to define coherent composition for $\Pi_q p$.
Without loss of generality, we will omit treating $h_0^1$.

Let $i : A \to B$ be an object of $\cal{I}$ and consider a square $h_1^1 \hattimes i \to \Pi_q p$.
\[
\xymatrix{
  \braces{1} \times A
  \ar[r]
  \ar[d]
&
  T
  \ar[r]
  \ar[d]
&
  \Pi_p Z
  \ar[d]
\\
  \braces{1} \times B
  \ar[r]
  \ar@{.>}[urr]
&
  \Delta^1 \times B
  \ar[r]
&
  X
}
\]
Here, we have abbreviated $T = \braces{0} \times B +_{\braces{0} \times A} \Delta^1 \times A$.
We want to define a composition as indicated.
Furthermore, this composition should be coherent in the usual sense.

Let us draw the base changes of the triangle $q : Z \to Y$ over $X$ along the various maps pointing into $X$ from the previous diagram:
\[
\xymatrix{
  Z''''
  \ar[rrr]
  \ar[dd]^(0.7){q''''}
  \ar[dr]
&&&
  Z'''
  \ar[dd]^(0.7){q'''}
  \ar[dr]
\\
&
  Z''
  \ar[rrr]
  \ar[dd]^(0.7){q''}
&&&
  Z'
  \ar[rrr]
  \ar[dd]^(0.7){q'}
&&&
  Z
  \ar[dd]^(0.7){q}
\\
  Y''''
  \ar[rrr]
  \ar[dd]
  \ar[dr]
&&&
  Y'''
  \ar[dd]
  \ar[dr]
  \ar@{.>}@/^1em/[uu]
&&
  \Pi_{\canonical \times B} Z'
  \ar[dd]^(0.7){\Pi_{\canonical \times B} q'}
\\&
  Y''
  \ar[rrr]
  \ar[dd]
  \ar@{-->}@/^1em/[uu]
&&&
  Y'
  \ar[rrr]
  \ar[dd]^{p'}
&&&
  Y
  \ar[dd]^{p}
\\
  \braces{1} \times A
  \ar[rrr]
  \ar[dr]
&&&
  \braces{1} \times B
  \ar[dr]
&&
  \Pi_{\canonical \times B} Y'
  \ar[dd]
\\&
  T
  \ar[rrr]
&&&
  \Delta^1 \times B
  \ar[rrr]
  \ar[dr]
&&&
  X
\\&&&&&
  \braces{1} \times B
}
\]
Here, all squares are pullbacks.
We have added a certain dependent product for later use.

By adjointness, the problem of defining composition becomes as follows: given a section to $q'' : Z'' \to Y''$ over $T$, find a section to $q''' : Z''' \to Y'''$ over $\braces{1} \times B$, pulling back to the same section to $q'''' : Z'''' \to Y''''$ over $\braces{1} \times A$.
Furthermore, the defined section needs to be stable under pullback of the whole situation given a cartesian square from $A' \to B'$ to $A \to B$ in $\cal{I}$.

For comparison, the problem of defining filling is as follows: given a section to $q''$, find a section to $q'$ pulling back to it.
For oblique reason, we are unable to solve this directly.
Instead, we decompose this problem into two parts.
Suppose we are given a section to $q''$ (over $T$):
\begin{enumerate}
\item Construct a section to $\Pi_{\canonical \times B}(q')$ (over $\braces{1} \times B$).
\item Construct a section to to $q'''$ (over $\braces{1} \times B$).
\end{enumerate}
Part~(1) will make use of $q$ as a fibration, while part~(2) will make use of $p$ as a fibration.

\paragraph{Part (1)}

Decompose $q'$ as follows:
\[
\xymatrix{
  Z'
  \ar[rrr]
  \ar[ddd]_{q'}
  \ar@{-->}[dr]^{\hatexp_{\Delta^1 \times B}(\canonical_{h_1^1 \hattimes i \to 1}, q')}
&&&
  \ar[ddd]^{\exp_{\Delta^1 \times B}(h_1^1 \hattimes i, q')}
  \exp_{\Delta^1 \times B}(h_1^1 \hattimes i, Z')
\\&
  \arghole
  \ar[ddl]
  \ar[urr]
  \pullback{dr}
\\\\
  Y'
  \ar[rrr]
&&&
  \exp_{\Delta^1 \times B}(h_1^1 \hattimes i, Y')
}
\]
The dotted line is the Leibniz exponential of $q'$ with the canonical map from $h_1^1 \hattimes i$ to $\id_{\Delta^1 \times B}$, all in the slice over $\Delta^1 \times B$.
Let us do our best to individually find sections for each of the two maps $q'$ decomposes into here.

Recall that exponentiating with the object $h_1^1 \hattimes i$ decomposes into first the dependent product along $h_1^1 \hattimes i$ and then base change along $h_1^1 \hattimes i$.
By assumption, we have a section to the base change $q''$ of $q'$ along $h_1^1 \hattimes i$.
By functoriality, with thus also have a section to $\exp_{\Delta^1 \times B}(h_1^1 \hattimes i, q')$ and its base change in the above diagram.

For the other map, the Leibniz exponential $\hatexp_{\Delta^1 \times B}(\canonical_{h_1^1 \hattimes i \to 1}, q')$, we are unable to construct a section directly.
The best we can do is construct a section for it after changing from the slice over $\Delta^1 \times B$ to the slice over $\braces{1} \times B$ via the functor $\Pi_{\canonical \times B}$.
This is the reason for the form of the intermediate step in the original decomposition of the problem.
Finding a section to
\[
\Pi_{\canonical \times B} (\hatexp_{\Delta^1 \times B}(\canonical_{h_1^1 \hattimes i \to 1}, q'))
\]
follows from solving
\[
\canonical_{0 \to u} \pitchfork \Pi_{\canonical \times B} (\hatexp_{\Delta^1 \times B}(\canonical_{h_1^1 \hattimes i \to 1}, q'))
\]
for monomorphisms $u$.
By adjointness of dependent product and pullback, this is equivalent to solving:
\[
\canonical_{0 \to \Delta^1 \times u} \pitchfork \hatexp_{\Delta^1 \times B}(\canonical_{h_1^1 \hattimes i \to 1}, q')\]
By Leibniz adjointness, this is in turn rewrites to
\[
\underbrace{\canonical_{0 \to \Delta^1 \times u} \hattimes_{\Delta^1 \times B} \canonical_{h_1^1 \hattimes i \to 1}}_{(\Delta^1 \times u) \times_{\Delta^1 \times B} \canonical_{h_1^1 \hattimes i \to 1}} \pitchfork q'
\]
where the Leibniz product simplifies as one factor has initial codomain.
Solving lifting problems is invariant under slicing, so let us switch from the slice over $\Delta^1 \times B$ to the underlying category:
\[
(\Delta^1 \times u)^* (h_1^1 \hattimes i) \pitchfork q'
\]
Exchanging Leibniz product and base change, we get
\[
h_1^1 \hattimes u^* i \pitchfork q'
,\]
or alternatively $u^* i \pitchfork \hatexp(h_1^1, q')$.
But $u^* i$ is again a monomorphism and $\hatexp(h_1^1, q')$ carries an $\cal{I}_5$-algebra structure by assumption.

\paragraph{Part (2)}

Let us focus on the following section-retration pair $(s, r)$:
\[
\xymatrix@C+2em{
  \braces{1} \times B
  \ar[r]^{\overbrace{h_1^1 \times B}^s}
  \ar@/_2em/[rr]_{\id}
&
  \Delta^1 \times B
  \ar[r]^{\overbrace{\canonical \times B}^r}
&
  \braces{1} \times B
}
\]
It induces a natural transformation $v$
\[
\xymatrix@C+2em{
  \Pi_r
  \ar[r]^-{\Pi_r \eta}
&
  \Pi_r \Pi_s s^*
  \ar[r]^-{\simeq}  
&
  s^*
}
\]
using the unit $\eta$ of the adjunction $s^* \dashv \Pi_s$.
We are going to show that $v_{Y'}$ (or $v_{p'}$, more precisely) has a section.
This suffices to diagrammatically transfer a section to $\Pi_s q'$ to a section to $q''' = s^* q'$.

Using the adjunction $r^* \dashv \Pi_r$, finding a section to $v_{p'} : \Pi_r p' \to s^* p'$ is equivalent to finding $d : r^* s^* p' \to p'$ such that $s^* d$ is a section to the identity.
Using the adjunction $\Sigma_s^* \dashv s^*$, the latter condition can be written as the following diagram:
\[
\xymatrix{
  \Sigma_s s^* p'
  \ar[r]^{\epsilon p'}
  \ar[d]_{\epsilon s^* p'}
&
  p'
\\
  r^* s^* p'
  \ar@{.>}[ur]_{d}
}
\]
Looking at this diagram not in the slice over $\Delta^1 \times B$, but in the underlying category,
\[
\xymatrix{
  Y'''
  \ar[d]_{h_0^1 \times Y'''}
  \ar[r]
&
  Y'
  \ar[d]^{p'}
\\
  \Delta^1 \times Y'''
  \ar[r]
  \ar@{.>}[ur]^{d}
&
  \Delta^1 \times B
}
\]
we have a left map $h_0^1 \hattimes \canonical_{0 \to Y'''}$ with $\canonical_{0 \to Y'''} : \cal{I}_5$.
Since $p'$ is an $\cal{I}_5$-fibration, we have a lift as required.

\paragraph{Conclusion}

Some straightforward diagram chasing should show that composing the constructions of part (1) and part (2) yields a section that coheres with the starting section in the required sense.
All the added functoriality stuff and preservation by base change still needing to be verified in the obvious manner hints that there should be a more abstract way of presenting the above proof.
\end{proof}

\end{document}
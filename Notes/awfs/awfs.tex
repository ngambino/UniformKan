\documentclass[reqno,10pt,a4paper,oneside]{amsart}
\usepackage{amssymb,amsmath,amsthm,stmaryrd,enumerate,geometry}
\usepackage[all]{xy} 

\SelectTips{cm}{}
\newdir{ >}{{}*!/-7pt/@{>}}
\newdir{m}{->}
%\newdir{m}{{}*!/-1pt/@{o}}
\newcommand{\xycenter}[1]{\vcenter{\hbox{\xymatrix{#1}}}}
%%% Pullback symbols
\newcommand{\ulpullback}[1][ul]{\save*!/#1-1.2pc/#1:(-1,1)@^{|-}\restore}
\newcommand{\dlpullback}[1][dl]{\save*!/#1-1.2pc/#1:(-1,1)@^{|-}\restore}
\newcommand{\urpullback}[1][ur]{\save*!/#1-1.2pc/#1:(-1,1)@^{|-}\restore}
\newcommand{\drpullback}[1][dr]{\save*!/#1-1.2pc/#1:(-1,1)@^{|-}\restore}

\newcommand{\catE}{\mathcal{E}}

\newcommand{\RAlg}{R\text{-}\mathrm{Alg}}
\newcommand{\Ralg}{\R\text{-}\mathrm{alg}}
\newcommand{\LAlg}{L\text{-}\mathrm{Alg}}
\newcommand{\Lalg}{\L\text{-}\mathrm{alg}}
\newcommand{\LLAlg}{L'\text{-}\mathrm{Alg}}
\newcommand{\LLalg}{\L'\text{-}\mathrm{alg}}


\newcommand{\ie}{\text{i.e.\ }}
\newcommand{\eg}{\text{e.g.}}
\newcommand{\resp}{\text{resp.\ }}
\newcommand{\myemph}{\textit} 
\newcommand{\changed}{\todo[noline]{Changed}}




\newtheorem{theorem}{Theorem}[section]
\newtheorem*{theorem*}{Theorem}
\newtheorem{lemma}[theorem]{Lemma} 
\newtheorem{proposition}[theorem]{Proposition} 
\newtheorem{corollary}[theorem]{Corollary}  
\newtheorem{apptheorem}{Theorem}



\theoremstyle{definition}
\newtheorem{definition}[theorem]{Definition}	
\newtheorem*{definition*}{Definition}	



\newtheorem{remark}[theorem]{Remark} 
\newtheorem*{remark*}{Remark} 
\newtheorem{example}[theorem]{Example}
\newtheorem{examples}[theorem]{Examples}
\newtheorem*{example*}{Example}
\newtheorem*{examples*}{Examples}



\newcommand{\defeq}{=_{\mathrm{def}}}
\newcommand{\co}{\colon}
\newcommand{\iso}{\cong} 
\newcommand{\rev}{\mathit{\vee}}
\newcommand{\op}{\mathrm{op}}
\newcommand{\catequiv}{\simeq} 
\newcommand{\cateq}{\simeq} 
\newcommand{\coend}{\int}




\newcommand{\cat}[1]{\mathbb{#1}}
\newcommand{\catA}{\cat{A}}
\newcommand{\catB}{\cat{B}}
\newcommand{\catC}{\cat{C}}
\newcommand{\catD}{\cat{D}}
\newcommand{\catK}{\cat{K}}
\newcommand{\catM}{\cat{M}}


\newcommand{\SSet}{\mathbf{SSet}}
\newcommand{\UU}{\overline{\mathsf{U}}}
\newcommand{\U}{\mathsf{U}}
\newcommand{\Weq}{\mathrm{Eq}}

\author[]{Nicola Gambino}

\address{School of Mathematics, University of Leeds, Leeds LS2 9JT, United Kingdom}
\email{n.gambino@leeds.ac.uk}


\author{Christian Sattler}
\address{School of Mathematics, University of Leeds, Leeds LS2 9JT, United Kingdom}
\email{c.sattler@leeds.ac.uk}




\title[]{Draft Notes on Uniform Kan fibrations}



\date{June 30th, 2015}



\begin{document}

\begin{abstract}
The aim of this short note is to show how Garner's small object argument can be applied to
obtain a weak factorisation system in which the right class of maps are what we
call uniform strong Kan fibrations, i.e.\,Kan fibrations that can be equipped with a choice of lifts for horn
inclusions which satisfy a uniformily condition. 
\end{abstract}


\maketitle
 

 
 
 
\section{Uniform Kan fibrations}

For $k, n$ we write $i_{k,n} \co \Lambda^k_n \to \Delta_n$ for the corresponding horn inclusion. Given 
$m$, we write 
\[
j_{k,n,m} \co \Lambda_{k,n} \times \Delta_m \to \Delta_n \times \Delta_m
\] 
for the 
product of the horn inclusion $i_{k,n}$ with the identity map on $\Delta_m$. We call these maps the
\emph{parametrized horn inclusions}. A morphism of parametrized horn inclusions is a diagram
 \[
\xymatrix@C=1.8cm@R=1.5cm{
 \Lambda_{k',n'} \times \Delta_{m'} \ar[r] \ar[d] &  \Lambda_{k,n} \times \Delta_m \ar[d] \ar[r]^-u & X \ar[d]^{f}  \\
\Delta_{k',n'} \times \Delta_{m'} \ar[r]  \ar@{.>}[urr]^{\phi_{u',v'}} &   \Delta_{n} \times \Delta_{m}  \ar[r]_-v \
 \ar@{.>}[ur]_{\phi_{u,v}}
 & Y \, .}
 \]
 We write $\mathbf{J}$ for the category of parametrized horn inclusions and their morphism.


 
 \begin{definition} A \emph{uniform Kan fibration} is a map of simplicial sets $f \co X \to Y$ 
for which there exists  a function $J$ that provides a filler for every diagram of the form
 \[
\xymatrix@C=1.8cm@R=1.5cm{
 \Lambda_{k,n} \times \Delta_m \ar[r]^-u \ar[d]_{i_{k,n,m}} & X \ar[d]^f \\
 \Delta_{n} \times \Delta_m \ar[r]_-v  \ar@{.>}[ur]^{J_{u,v}}&  Y}
 \]
 such that the uniformity condition  expressed by the following diagram holds:
 \[
\xymatrix@C=1.8cm@R=1.5cm{
 \Lambda_{k',n'} \times \Delta_{m'} \ar[r] \ar[d] &  \Lambda_{k,n} \times \Delta_m \ar[d] \ar[r]^-u & X \ar[d]^{f}  \\
\Delta_{k',n'} \times \Delta_{m'} \ar[r]  \ar@{.>}[urr]^{J_{u',v'}} &   \Delta_{n} \times \Delta_{m}  \ar[r]_-v \
 \ar@{.>}[ur]_{J_{u,v}}
 & Y \, .}
 \]
 \end{definition}

 
 
 
 We wish obtain a weak factorisation system
 in which the right class of maps are exactly the maps of simplicial sets that admit the structure of  a uniform 
 Kan fibrations. In order to do so, we use Garner's small object argument. Thanks to his work, all we have
 to show is the following fact. 
 
 \begin{proposition} The inclusion functor $I \co \mathbf{J} \to \SSet^2$ preserves $\omega$-filtered 
 colimits.
 \end{proposition} 
 
 
\begin{proof} TO BE ADDED.
\end{proof}
 
 The main theorem in Garner's paper then gives us the following result.
 
 \begin{corollary} The category $\SSet$ admits a weak factorisation system whose
 right class consists exactly of the  uniform Kan fibration. 
 \end{corollary}
 
 We brielfy illustrate how the construction of the weak factorisation system $(\mathcal{L},\mathcal{R})$
 works. First of all, we define a functorial factorization on $\SSet$. This is done in four steps. 
 
 \medskip
 
  \noindent
 {\itshape First step.} We form the so-called density comonad $M \co \SSet^2 \to \SSet^2$ 
 associated to the functor $I \co \mathbf{J} \to \SSet$.  This is done by taking the left Kan extension of $I$ along itself, 
 \[
 \xymatrix{
 \mathbf{J} \ar@/_1pc/[dr]_I \ar[r]^I  \ar@{}[dr]|{\quad \Rightarrow}  & \SSet^2 \ar[d]^{M} \\
  & \SSet^2 \, . }
  \]
 For $f \co X \to Y$, we have the coend formula
\[
M(f) = 
\int^{(k,n,m)} \SSet^2( j_{k,m,n}, f )  \cdot j_{k,m,n} \, .
\] 
Here, note that $\SSet^2( j_{k,m,n}, f )$ is  the set of squares of the form
\[
\xymatrix@C=1.2cm@R=1.2cm{
 \Lambda_{k,n} \times \Delta_m \ar[r]^-u \ar[d]_{i_{k,m,n}} & X \ar[d]^f \\
 \Delta_{n} \times \Delta_m \ar[r]_-v   &  Y \, .}
 \]
 For $f \co X \to Y$, the component $\varepsilon_f$ of the counit is given by a diagram of the form
 \[
 \xymatrix@C=1.2cm{
  \bullet \ar[r]  \ar[d]_{M(f)} & X \ar[d]^f \\
  \bullet \ar[r]  & Y }
  \]
 
 
 \medskip
 
 \noindent
 {\itshape Second step.} We define functors $\phi \co
 \SSet \to \SSet$ and $\psi \co \SSet^2 \to \SSet^2$    via the following pushout diagram:
\[
   \xymatrix@C=1.2cm{
 \bullet \ar[r] \ar[d]_{M(f)} & X \ar[d]^{\phi_f} \ar@/^1.5pc/[ddr]^{f} \\
  \bullet \ar[r] \ar@/_1.5pc/[drr]  & FX \ar@{.>}[dr]^-(.3){\psi_f}  \\ 
   & & Y  }
 \]
Here, we used a slight abuse of language and written $FX$ for the common value of $d_1 \phi f$ 
and~$d_0 \psi_f$.
The functor $\phi \co  \SSet^2 \to \SSet^2$ inherits a comonad structure from the 
 comonad~$M$, but we do not spell  the details out.  Also note that the endofunctor
 $\psi \co \SSet  \to \SSet$ acquires the structure of a pointed endofunctor by considering the natural
 transformation with components
 \[
 \xymatrix@C=1.4cm{
 X \ar[d]_f \ar[r]^{\phi_f} \ar[d] & FX \ar[d]^{\psi_f} \\
 Y \ar@{=}[r]  & Y \, .}
 \]
 The third step constructs the free monad on this pointed endofunctor. 
 
 \medskip

\noindent 
 {\itshape Third step.}  For an arrow $f \co X \rightarrow Y$, we wish to define a diagram $X \co  \omega \to \SSet$
 and  maps 
 \[
 f_n \co X_n \to Y \, , \quad a_{n} \co FX_n \to X_{n+1} \, ,
 \] 
 for $n \in \omega$, such that the diagrams
 \[
 \xymatrix{
  X_n   \ar@/_1pc/[dr]_-{f_n}  \ar[rr]^{\sigma_n} &   & X_{n+1} \ar@/^1pc/[dl]^-{f_{n+1}}   \\
   &  Y &  } \qquad  
  \xymatrix@C=1.5cm{
  X_n \ar[r]^{\phi_n}  \ar@/_1pc/[dr]_{\sigma_n} & FX_n \ar[d]^{a_{n}} \\ 
   & X_{n+1} } \qquad 
  \xymatrix@C=1.5cm{
  FX_n \ar[r]^{a_n}  \ar@/_1pc/[dr]_{\psi_n} & FX_n \ar[d]^{f_{n+1}} \\ 
   & Y}    
   \]
   commute, where we write $\phi_n$ instead of $\phi_{f_n}$ for brevity.
Note that the second diagram dictates how to define $\sigma_n$. We proceed recursively, as follows.
 
 \begin{itemize}
 \item Base case. Let $X_0 \defeq X$, $X_1 \defeq FX$ and $a_0 \defeq 1_{FX}$.
 \item Successor step. Let us assume that we have defined the objects $X_n$, $X_{n+1}$, and maps
 $\sigma_n$, $f_n$, $f_{n+1}$ and  $a_{n}$ with the appropriate 
 properties. We consider the pushout
 \[
   \xymatrix@C=1.5cm@R=1.5cm{
 \bullet \ar[r]  \ar[d]_{M(f_{n+1})} & X_{n+1} \ar[d]^{\phi_{n+1}} \ar@/^1.8pc/[ddr]^{f_{n+1}} \\
\bullet \ar[r] \ar@/_1.5pc/[drr]  & FX_{n+1}\ar@{.>}[dr]^-(.3){\psi_{n+1}} \\ 
   & & Y  }
 \]
We then define $X_{n+2}$ and $a_{n+1}$ via  the coequalizer diagram
 \[
 \xymatrix@=3cm{
 FX_n \ar@<1ex>[r]^{\phi_{n+1}  \cdot a_n}  \ar@<-1ex>[r]_{ F(a_n \cdot \phi_n)} & 
 F X_{n+1}   \ar[r]^{a_{n+1}} & X_{n+2} }
 \]
 where we have written $F(\phi_n \cdot a_n)$ for the result of applying the
 functor $F \co \SSet^2 \to \SSet$ to the square
 \[
  \xymatrix{
 X_n \ar[r]^{\phi_n} \ar[d]_{f_n} & FX_n \ar[d]^{\psi_n}  \ar[r]^{a_n} & X_{n+1} \ar[d]^{f_{n+1}}  \\ 
 Y \ar@{=}[r] & Y \ar@{=}[r] & Y }
 \]
 The map $f_{n+2} \co X_{n+2} \to Y$ is then defined via the universal property of $X_{n+2}$,
 so that the following diagram commutes:
 \[
 \xymatrix@C=3cm{
 FX_n \ar@<1ex>[r]^{\phi_{n+1}  \cdot a_n}  \ar@<-1ex>[r]_{ F(a_n \cdot \phi_{n})} & 
 F X_{n+1}   \ar[r]^{a_{n+1}} \ar@/_1pc/[dr]_{\psi_{n+1}} & X_{n+2}  \ar[d]^{f_{n+2}}  \\ 
  & & Y }
 \] 
 \end{itemize}
 
 \medskip
 
 \noindent
 \emph{Fourth step.}  We obtain the required factorisation
\[
\xymatrix{
X \ar[dr]_{f} \ar[rr]^{\lambda_f} & & K_f \ar[dl]^{\rho_f} \\
 & Y & }
 \]
as follows. The object $K_f$ is defined to be the colimit of the diagram $X \co \omega \to \SSet$. We write~$\tau$
for its colimiting cone, that has components of the form
 \[
 \xymatrix{
  X_n   \ar@/_1pc/[dr]_-{\tau_n}  \ar[rr]^{\sigma_n} &   & X_{n+1} \ar@/^1pc/[dl]^-{\tau_{n+1}}   \\
   &  K_f &  }
  \]
Since $f_0 = f$, we define $\lambda_f \co X \to K_f$  to be $\tau_0$.
Exploiting the universal property of $K_f$, the map $\rho_f \co K_f \to Y$ is then defined as 
the unique arrow making the following diagram commutes:
\[
 \xymatrix@=1.5cm{
  X_n   \ar@/_1pc/[dr]^-{\tau_n}  \ar[rr]^{\sigma_n} \ar@/_1pc/[ddr]_{f_n} &   & X_{n+1} \ar@/^1pc/[dl]_-{\tau_{n+1}}  \ar@/^1pc/[ddl]^{f_{n+1}}  \\
   &  K_f \ar[d]_{R_f}  &  \\
    & Y & }
  \]
  This completes the construction of the required factorisation. 
   


\medskip

By Garner's results, the functor mapping $f \co A \to B$ to $\lambda_f \co A \to K_f$ is the underlying
functor of a comonad, whose counit has components given by the 
commutative squares
 \[
 \xymatrix{
 A \ar@{=}[r] \ar[d]_{\lambda_f} & A \ar[d]^f \\
 K_f \ar[r]_{\rho_f} & B \, . }
 \] 
Dually, the functor mapping $f \co A \to B$ to $\rho_f \co K_f \to B$ is part of monad, whose
unit has components given by the commutative squares
 \[
 \xycenter{
 A  \ar[d]_f \ar[r]^{\lambda_f}  & K_f \ar[d]^{\rho_f} \\
 B \ar@{=}[r] & B \, . }
 \]
 We write  $\LAlg$ for the category of coalgebras for the comonad $L$, and $\RAlg$ for the category of algebras for the monad $R$.   We write $\mathcal{L}$ and $\mathcal{R}$ for the
 closures under retracts of the classes of maps admitting an $L$-algebra 
 or an $R$-algebra structure, respectively. Then, $(\mathcal{L}, \mathcal{R})$ is a weak factorisation system. 
 
 \medskip
 
 The presence of algebraic structure provides a canonical way of filling diagrams, as we now recall. 
  Let us consider a commutative square of the form
 \[
 \xymatrix{
 A \ar[r]^u \ar[d]_i & X \ar[d]^f \\
 B \ar[r]_v & Y}
 \]
where $i \in \mathcal{L}$ and $f \in \mathcal{R}$. Without loss of generality, we may assume that
 $i$ is equipped with a structure of $L$-coalgebra and $f$ is equipped with the structure of an
 $R$-algebra. Observe that 
 to give an $L$-coalgebra structure on $i \co A \to B$ amounts
 to giving a diagonal filler in the square
 \[
 \xymatrix{
 A \ar[r]^{\lambda_i}  \ar[d]_{i} & K_i \ar[d]^{\rho_i} \\
 B \ar@{=}[r]  \ar@{.>}[ur]^s & B }
 \]
 Dually, to give an $R$-algebra structure on  $f \co X \to Y$ is to give
 a diagonal filler in the square
 \[
 \xymatrix{
 X  \ar@{=}[r]    \ar[d]_{\lambda_f} & X \ar[d]^f \\
 K_f \ar[r]_{\rho_f}  \ar@{.>}[ur]^t & Y }
 \]
The required diagonal filler  can then be defined as the composite
 \[
 \xymatrix{ B \ar[r]^{s} & K_i \ar[r]^{K_{u,v}} & K_f \ar[r]^{t} & X}
 \]
 where $K_{u,v}$ is obtained by the functoriality of the factorization. 
 
\medskip

Garner's results imply also that the algebraic weak factorisation system
 $(L, R)$  comes equipped with a functor $\eta \co \mathbf{J} \to \LAlg$
 over $\SSet^2$, such that the functor
 \[
 \xymatrix@C=1cm{
  \RAlg \ar[r]^-{\textup{lift}} &
    \big( \LAlg \big)^\pitchfork \ar[r]^-{\eta^\pitchfork} &
   \mathbf{J}^\pitchfork}
  \]
  is an isomorphism. In particular, uniform Kan fibrations are precisely
  the maps for which there exists a $R$-algebra structure. 
 
  

\vfill
 


\end{document}


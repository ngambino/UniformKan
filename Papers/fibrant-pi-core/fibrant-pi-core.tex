\documentclass[reqno,10pt,a4paper,oneside]{amsart}

\usepackage{uniform-kan-prelude}

\title{Dependent Product for Algebraic Fibrations}

\begin{document}

\begin{abstract}
We give a categorical account of algebraic fibrations being preserved under certain dependent product.
This generalizes work by Coquand \etal.
\end{abstract}

\maketitle

\tableofcontents

\section{Lifting against Arrows}

For this section, fix a category $\catE$.
There is a well known Gallois connection $\liftr{\brarghole} \dashv \liftl{\brarghole}$ between the poset of subsets of arrows in $\catE$ and its opposite.
Garner~\cite{garner:small-object-argument} shows how to extend this to an adjunction between categories of arrows:
\begin{equation}
\label{garner-adjunction}
\begin{gathered}
\xymatrix@C+1em{
  \CAT/\catE^{\to}
  \ar@<5pt>[r]^{\liftl{\brarghole}}
  \ar@{}[r]|{\bot}
&
  (\CAT/\catE^{\to})^{\op}
  \ar@<5pt>[l]^{\liftr{\brarghole}}
}
\end{gathered}
\end{equation}

\begin{lemma}
Consider a natural transformation between categories over $\catE^{\to}$:
\[
\xymatrix{
  \cat{A}
  \rrtwocell_G^F{\sigma}
 \ar[dr]_{U}
&&
  \cat{B}
  \ar[dl]^{V}
\\&
  \catE^{\to}
}
\]
Note that this includes the condition $V \sigma = \id_U$.
Then $\liftr{F}$ and $\liftr{G}$ as well as $\liftl{F}$ and $\liftl{G}$ agree:
\begin{mathpar}
\xymatrix{
  \liftr{\cat{A}}
  \ar[dr]_{\liftr{U}}
&&
  \liftr{\cat{B}}
  \lltwocell_{\liftr{F}}^{\liftr{G}}{=}
  \ar[dl]^{\liftr{V}}
\\&
  \catE^{\to}
}
\and
\xymatrix{
  \liftl{\cat{A}}
  \ar[dr]_{\liftl{U}}
&&
  \liftl{\cat{B}}
  \lltwocell_{\liftl{F}}^{\liftl{G}}{=}
  \ar[dl]^{\liftl{V}}
\\&
  \catE^{\to}
}
\end{mathpar}
\end{lemma}

\subsection{Lifts of adjoint functors}

Consider an adjunction as follows:
\[
\xymatrix@C+1em{
  \cat{D}
  \ar@<5pt>[r]^{L}
  \ar@{}[r]|{\bot}
&
  \catE
  \ar@<5pt>[l]^{R}
}
\]
Let $U : \cat{A} \to \cat{D}^{\to}$ and $V : \cat{B} \to \catE^{\to}$ be categories over $\cat{D}^{\to}$ and $\catE^{\to}$, respectively.

\begin{lemma}
\label{lift-of-adjunction}
Lifts of $L^{\to}$ to a functor $\cat{A} \to \liftl{\cat{B}}$ are in natural correspondence with lifts of $R^{\to}$ to a functor $\cat{B} \to \liftr{\cat{A}}$:
\begin{mathpar}
\xymatrix{
  \cat{A}
  \ar@{.>}[r]
  \ar[d]_{U}
&
  \liftl{\cat{B}}
  \ar[d]^{\liftl{V}}
\\
  \cat{D}^{\to}
  \ar[r]^{L^{\to}}
&
  \catE^{\to}
}
\and
\xymatrix{
  \liftr{\cat{A}}
  \ar[d]_{\liftr{U}}
&
  \cat{B}
  \ar@{.>}[l]
  \ar[d]^{V}
\\
  \cat{D}^{\to}
&
  \catE^{\to}
  \ar[l]^{R^{\to}}
}
\end{mathpar}
\end{lemma}

\begin{proof}
Maps from $L^{\to} \cc U$ to $\liftl{V}$ over $\catE^{\to}$ consist of fillers for squares $L^{\to}(U(X)) \to V(Y)$ natural in $X : \cat{A}$ and $Y : \cat{B}$.
Similarly, maps from $R^{\to} \cc V$ to $\liftr{U}$ over $\cat{D}^{\to}$ consist of fillers for squares $U(X) \to R^{\to}(V(Y))$ natural in $X : \cat{A}$ and $Y : \cat{B}$.
Since $L \dashv R$, these situations coincide.
\end{proof}

\subsection{Slicing}

Let $X$ be an object of $\catE$.
Given a category $U : \cat{A} \to \catE^{\to}$ over $\catE^{\to}$, we construct its \emph{slice} over $X$ using (strict) pullback along the map on arrows induced by $\catE/X \to \catE$ forgetting the slicing information:
\[
\xymatrix{
  \cat{A}_{/X}
  \ar[r]
  \ar[d]_{U_{/X}}
  \pullback{dr}
&
  \cat{A}
  \ar[d]^{U}
\\
  (\catE/X)^{\to}
  \ar[r]
&
  \catE^{\to}
}
\]
Explicitly, the category $\cat{A}_{/X}$ has as objects pairs $(Y, h)$ where $Y : \cat{A}$ and $h$ is a map in $\catE$ from the codomain of $U(Y)$ to $X$.

Dually, (strictly) pulling back along the map on arrows induced by $\catE \backslash X \to \catE$ constructs the \emph{coslice} over $X$:
\[
\xymatrix{
  \cat{A}_{\backslash X}
  \ar[r]
  \ar[d]_{U_{\backslash X}}
  \pullback{dr}
&
  \cat{A}
  \ar[d]^{U}
\\
  (\catE \backslash X)^{\to}
  \ar[r]
&
  \catE^{\to}
}
\]

\begin{remark}
\label{pitchfork-slicing}
Just as in the non-algebraic setting, the functor $\liftr{\brarghole}$ and $\liftl{\brarghole}$ commute (up to natural isomorphism) with slicing and coslicing, respectively.
\end{remark}

\subsection{Retract closure}

\newcommand{\ret}{\mathbbm{R}}
\newcommand{\retA}{\mathbbm{a}}
\newcommand{\retB}{\mathbbm{b}}

Let $\ret$ denote the \emph{walking retract}, \ie the category with objects $\retA, \retB$ and morphisms generated by $s : \retA \to \retB$ and $r : \retB \to \retA$ under the relation $r \cc s = \id_{\retA}$.

Given a category $U : \cat{A} \to \catE^{\to}$ over $\catE^{\to}$, we define its retract closure $\overline{U} : \overline{\cat{A}} \to \catE^{\to}$ using (strict) pullback and left composition:
\[
\xymatrix@C+1em{
  \overline{\cat{A}}
  \ar[r]
  \ar[d]
  \ar@/_2em/[dd]_{\overline{U}}
  \pullback{dr}
&
  \cat{A}
  \ar[d]^{U}
\\
  (\catE^{\to})^{\ret}
  \ar[r]^-{(\catE^{\to})^{\retB}}
  \ar[d]^{(\catE^{\to})^{\retA}}
&
  \catE^{\to}
\\
  \catE^{\to}
}
\]
Explicitly, an object of $\overline{\cat{A}}$ consists of $a : \cat{A}$ and a retract of $U(a)$ in $\catE^{\to}$, \ie an arrow $e : \catE^{\to}$ together with maps $s : e \to U(a)$ and $r : U(a) \to e$ such that $r \cc s = \id_e$:
\[
\xymatrix{
  e
  \ar[r]_{s}
  \ar@/^1em/[rr]^{\id}
&
  U(a)
  \ar[r]_{r}
&
  e
}
\]
The action of $\overline{U}$ on this object is $\overline{U}(a, e, s, t) = e$.
A morphism between such objects $(a, e, s, t)$ and $(a', e', s', t')$ consists of a morphism $f : a \to a'$ in $\cat{A}$ and a morphism $h : e \to e'$ in $\catE^{\to}$ such that the evident squares commute:
\[
\xymatrix{
  e
  \ar[r]_{s}
  \ar@/^1em/[rr]^{\id}
  \ar[d]_{h}
&
  U(a)
  \ar[r]_{r}
  \ar[d]^{U(f)}
&
  e
  \ar[d]^{h}
\\
  e'
  \ar[r]^{s'}
  \ar@/_1em/[rr]_{\id}
&
  U(a')
  \ar[r]^{r'}
&
  e'
}
\]
The action of $\overline{U}$ on this morphism is $\overline{U}(f, h) = h$.

We have an evident inclusion $\cat{A} \to \overline{\cat{A}}$ of categories over $\catE^{\to}$ formally induced by $(\catE^{\to})^{\canonical} : \catE^{\to} \to (\catE^{\to})^{\ret}$ being a section to $(\catE^{\to})^{\retB}$.
With the explicit description of objects, it maps $a$ to $(a, U(a), \id_{U(a)}, \id_{U(a)})$.
We also have a map $\overline{\overline{\cat{A}}} \to \overline{\cat{A}}$ of categories over $\catE^{\Box}$ mapping $((a, e, s, r), e', s', r')$ to $(a, e', s \cc s', r' \cc r)$.
This makes retract closure into a monad.

\begin{lemma}
\label{retract-closure}
Recall the adjunction~\eqref{garner-adjunction}.
The functors $\liftr{\brarghole}$ and $\liftl{\brarghole}$ maps unit and multiplication of retract closure to natural isomorphisms.
\end{lemma}

\begin{remark}
\label{retract-closure-slicing}
Taking the retract closure commutes (up to natural isomorphism) with slicing and coslicing.
\end{remark}

\subsection{Kan extension}

Let $U : \cat{A} \to \catE^{\to}$ be a category over $\catE^{\to}$.
Consider a fully faithful functor $F : \cat{A} \to \cat{B}$.
Assume respectively that the pointwise left and right Kan extension of $U$ along $F$ exist.
\begin{mathpar}
\xymatrix{
  \cat{A}
  \ar[dr]_{U}
  \ar[rr]^{F}
&&
  \cat{B}
  \ar[dl]^{\Lan_F U}
\\&
  \catE^{\to}
}
\and
\xymatrix{
  \cat{A}
  \ar[dr]_{U}
  \ar[rr]^{F}
&&
  \cat{B}
  \ar[dl]^{\Ran_F U}
\\&
  \catE^{\to}
}
\end{mathpar}
Note the triangles will be (strictly) commuting since $F$ is fully faithful.

\begin{lemma}
\label{kan-extension-closure}
The following functors induced by the adjunction~\eqref{garner-adjunction} are isomorphisms:
\begin{mathpar}
\xymatrix{
  \liftr{\cat{A}}
  \ar[dr]_{\liftr{U}}
&&
  \liftr{\cat{B}}
  \ar[ll]_{\liftr{F}}^{\simeq}
  \ar[dl]^{\liftr{(\Lan_F U)}}
\\&
  \catE^{\to}
}
\and
\xymatrix{
  \liftl{\cat{A}}
  \ar[dr]_{\liftl{U}}
&&
  \liftl{\cat{B}}
  \ar[ll]_{\liftl{F}}^{\simeq}
  \ar[dl]^{\liftl{(\Ran_F U)}}
\\&
  \catE^{\to}
}
\end{mathpar}
\end{lemma}

\section{Lifting against Squares}

For this section, fix again a category $\catE$.
The \emph{square category} $\catE^{\Box}$ is the double arrow category $(\catE^{\to})^{\to}$.
We have evident inclusions $\squl, \squr, \sqdl, \sqdr$ of the walking object and $\sql, \sqr, \squ, \sqd$ of the walking arrow into the walking square:
\[
\xymatrix{
  \squl
  \ar[r]^{\squ}
  \ar[d]_{\sql}
  \ar@{}[dr]
&
  \squr
  \ar[d]^{\sqr}
\\
  \sqdl
  \ar[r]_{\sqd}
&
  \sqdr
}
\]
We also have projections $\sqhori$ and $\sqvert$ from the walking square to the walking arrow that are sections to $\sql, \sqr$ and $\squ, \sqd$, respectively.

Given a square $X : \catE^{\Box}$, its corners and sides are thus respectively given by functor precomposition as $X\squl, X\squr, X\sqdl, X\sqdr : \catE$ and $X\sql, X\sqr, X\squ, X\sqd : \catE^{\to}$.
Similarly, given a line $f : \catE^{\to}$, then $f\sqhori$ and $f\sqvert$ are the squares that have identities as horizontal and vertical sides, respectively, with the other sides being given by $f$.
Note that if $h : X \to Y$ is a morphism of squares, then \eg $h\sql : X\sql \to Y\sql$ will denote the induced morphism between the left sides, and analogously for the other operations as indicated by the suggestive functor precomposition notation.

Given a square $X : \catE^{\Box}$, we will sometimes equivalently view it as a morphism in $\catE^{\to}$ from $X\sql$ to $X\sqr$.

Recall from~\cite{garner:small-object-argument} that the Gallois connection $\liftl{\brarghole} \dashv \liftr{\brarghole}$ on the class of arrows of $\catE$ may be lifted to an adjunction of categories over $\catE^{\to}$:
\begin{equation}
\label{garner-adjunction}
\begin{gathered}
\xymatrix@C+1em{
  \CAT/\catE^{\to}
  \ar@<5pt>[r]^{\liftl{\brarghole}}
  \ar@{}[r]|{\bot}
&
  (\CAT/\catE^{\to})^{\op}
  \ar@<5pt>[l]^{\liftr{\brarghole}}
}
\end{gathered}
\end{equation}

\begin{lemma}
\label{garner-adjunction-extended}
The adjunction~\eqref{garner-adjunction} can be lifted further to an adjunction as follows:
\begin{equation}
\label{garner-adjunction-extended}
\begin{gathered}
\xymatrix@C+1em{
  \CAT/\catE^{\Box}
  \ar@<5pt>[r]^{\liftl{\brarghole}}
  \ar@{}[r]|{\bot}
&
  (\CAT/\catE^{\Box})^{\op}
  \ar@<5pt>[l]^{\liftr{\brarghole}}
}
\end{gathered}
\end{equation}
\end{lemma}

\begin{proof}[Proof (adapted from \cite{garner:small-object-argument})]
Let us first define the functor $\liftr{\brarghole}$.
It sends $U : \cat{A} \to \catE^{\Box}$ to the category $\liftr{\cat{A}}$ that has as objects pairs $(T, \phi)$ consisting of a square $T : \catE^{\Box}$ with a coherent choice $\phi$ of \emph{compositions}.
For $X : A$ and a ``middle square'' $M : U(X)\sqr \to T\sql$, a composition $\phi(A, M)$ is a lift in the composite square indicated below:
\[
\xymatrix{
  \bullet
  \ar[r]
  \ar[d]
  \ar@{}[dr]|(0.4){U(X)}
&
  \bullet
  \ar[r]
  \ar[d]
  \ar@{}[dr]|(0.3){M}
&
  \bullet
  \ar[r]
  \ar[d]
  \ar@{}[dr]|(0.6){T}
&
  \bullet
  \ar[d]
\\
  \bullet
  \ar[r]
  \ar@{.>}[urrr]
&
  \bullet
  \ar[r]
&
  \bullet
  \ar[r]
&
  \bullet
}
\]
Coherence of composition means that for a morphism $f : X \to X'$ in $\cat{A}$ and a square $M' : U(X)\sqr \to T\sql$, we have $\phi(X, M \cc U(f)\sqr) = \phi(X', M') \cc U(f)\sqdl$.
A morphism in $\liftr{\cat{A}}$ from $(T, \phi)$ to $(T', \phi')$ is a morphism $t : T \to T'$ of squares respecting the choice of liftings $\phi$ and $\phi'$: for $X : A$ and $M : U(X)\sqr \to T$, we must have $t\squr \cc \phi(X, M) = \phi'(X, t\sql \cc M)$.
We have an evident forgetful functor from $\liftr{\cat{A}} \to \catE^{\Box}$.

This concludes the construction of the action of $\liftr{\brarghole}$ on objects.
The action on a morphism $F : \cat{A} \to \cat{B}$ is given by the functor $\liftr{\cat{B}} \to \liftr{\cat{A}}$ over $\catE^{\Box}$ sending $(T, \psi)$ to $(T, \phi)$ where $\phi(X, M) = \psi(F(X), M)$.

The functor $\liftl{\brarghole}$ is defined analogously, but with directions and order of composition swapped.
To see that $\liftl{\brarghole} \dashv \liftr{\brarghole}$, note that given categories $U : \cat{A} \to \catE^{\Box}$ and $V : \cat{B} \to \catE^{\Box}$ over $\catE^{\Box}$, both functors $\cat{A} \to \liftl{\cat{B}}$ and $\cat{B} \to \liftr{\cat{A}}$ may be identified with ``$(\cat{A}, \cat{B})$-lifting operations'': a function $\psi$ that assigns to objects $X : \cat{A}$ and $Y : \cat{B}$ and a ``middle square'' $M : U(X)\sqr \to V(Y)\sql$ a lift for the composite square 
\[
\xymatrix{
  \bullet
  \ar[r]
  \ar[d]
  \ar@{}[dr]|(0.4){U(X)}
&
  \bullet
  \ar[r]
  \ar[d]
  \ar@{}[dr]|(0.3){M}
&
  \bullet
  \ar[r]
  \ar[d]
  \ar@{}[dr]|(0.6){V(Y)}
&
  \bullet
  \ar[d]
\\
  \bullet
  \ar[r]
  \ar@{.>}[urrr]
&
  \bullet
  \ar[r]
&
  \bullet
  \ar[r]
&
  \bullet
}
\]
that is natural in the evident manner in both $X$ and $Y$.
\end{proof}

\begin{remark}
\label{extended-adjunction-gives-normal-one}
Recall the inclusion $E\sqhori : \catE^{\to} \to \catE^{\Box}$ that sends a line $f$ to the square with left and right sides $f$ and up and bottom sides identities.
It related categories over $\catE^{\Box}$ and $\catE^{\to}$ by an adjunction:
\begin{equation}
\label{adjunction-between-E-square-and-E-to}
\begin{gathered}
\xymatrix@C+1em{
  \CAT/\catE^{\Box}
  \ar@<5pt>[r]^{E\sqhori_!}
  \ar@{}[r]|{\bot}
&
  \CAT/\catE^{\to}
  \ar@<5pt>[l]^{(E\sqhori)^*}
}
\end{gathered}
\end{equation}
In fact, as can be confirmed from inspection of the constructions, the adjunction~\eqref{garner-adjunction} arises from the extended adjunction~\eqref{garner-adjunction-extended} by pre- and postcomposition with the adjunction~\eqref{adjunction-between-E-square-and-E-to} and its opposite:
\begin{equation*}
\begin{gathered}
\xymatrix@R+1em@C+1em{
  \CAT/\catE^{\to}
  \ar@<5pt>[r]^{\liftl{\brarghole}}
  \ar@{}[r]|{\bot}
  \ar@<5pt>[d]^{(E\sqhori)^*}
  \ar@{}[d]|{\dashv}
&
  (\CAT/\catE^{\to})^{\op}
  \ar@<5pt>[l]^{\liftr{\brarghole}}
  \ar@<5pt>[d]^{((E\sqhori)^*)^{\op}}
  \ar@{}[d]|{\vdash}
\\
  \CAT/\catE^{\Box}
  \ar@<5pt>[r]^{\liftl{\brarghole}}
  \ar@{}[r]|{\bot}
  \ar@<5pt>[u]^{E\sqhori_!}
&
  (\CAT/\catE^{\Box})^{\op}
  \ar@<5pt>[l]^{\liftr{\brarghole}}
  \ar@<5pt>[u]^{(E\sqhori_!)^{\op}}
}
\end{gathered}
\end{equation*}
If we only choose to either pre- or postcompose, we end up with ``mixed'' versions of the adjunction that we are still going to denote using the same symbols:
\begin{equation}
\label{garner-adjunction-extended-mixed}
\begin{gathered}
\xymatrix@C+1em{
  \CAT/\catE^{\Box}
  \ar@<5pt>[r]^{\liftl{\brarghole}}
  \ar@{}[r]|{\bot}
&
  (\CAT/\catE^{\to})^{\op}
  \ar@<5pt>[l]^{\liftr{\brarghole}}
}
\end{gathered}
\end{equation}
or:
\begin{equation}
\label{garner-adjunction-extended-mixed'}
\begin{gathered}
\xymatrix@C+1em{
  \CAT/\catE^{\to}
  \ar@<5pt>[r]^{\liftl{\brarghole}}
  \ar@{}[r]|{\bot}
&
  (\CAT/\catE^{\Box})^{\op}
  \ar@<5pt>[l]^{\liftr{\brarghole}}
}
\end{gathered}
\end{equation}
\end{remark}

\subsection{Lifts of adjoint functors}

Consider an adjunction as follows:
\[
\xymatrix@C+1em{
  \cat{D}
  \ar@<5pt>[r]^{L}
  \ar@{}[r]|{\bot}
&
  \catE
  \ar@<5pt>[l]^{R}
}
\]
Let $U : \cat{A} \to \cat{D}^{\Box}$ and $V : \cat{B} \to \catE^{\Box}$ be categories over $\cat{D}^{\Box}$ and $\catE^{\Box}$, respectively.
We have a straightforward analogue of \cref{lift-of-adjunction}:

\begin{lemma}
\label{lift-of-adjunction-extended}
Lifts of $L^{\Box}$ to a functor $\cat{A} \to \liftl{\cat{B}}$ are in natural correspondence with lifts of $R^{\Box}$ to a functor $\cat{B} \to \liftr{\cat{A}}$:
\begin{mathpar}
\xymatrix{
  \cat{A}
  \ar@{.>}[r]
  \ar[d]_{U}
&
  \liftl{\cat{B}}
  \ar[d]^{^{\pitchfork}V}
\\
  \cat{D}^{\Box}
  \ar[r]^{L^{\Box}}
&
  \catE^{\Box}
}
\and
\xymatrix{
  \liftr{\cat{A}}
  \ar[d]_{\liftr{U}}
&
  \cat{B}
  \ar@{.>}[l]
  \ar[d]^{V}
\\
  \cat{D}^{\Box}
&
  \catE^{\Box}
  \ar[l]^{R^{\Box}}
}
\end{mathpar}
\end{lemma}

\begin{proof}
Maps from $L^{\Box} \cc U$ to $\liftl{V}$ over $\catE^{\Box}$ consist of composition for squares $L^{\Box}(U(X))\sqr \to V(Y)\sql$ natural in $X : \cat{A}$ and $Y : \cat{B}$.
Similarly, maps from $R^{\Box} \cc V$ to $\liftr{U}$ over $\cat{D}^{\Box}$ consist of composition for squares $U(X)\sqr \to R^{\Box}(V(Y))\sql$ natural in $X : \cat{A}$ and $Y : \cat{B}$.
Since $L \dashv R$, these situations coincide.
\end{proof}

\begin{remark}
\label{lift-of-adjunction-mixed}
\cref{lift-of-adjunction} can be viewed as a special case of \cref{lift-of-adjunction-extended} using the adjunction~\eqref{adjunction-between-E-square-and-E-to}.
In fact, we have analogous statements for the ``mixed'' versions of the lifting adjunction of \cref{extended-adjunction-gives-normal-one}.
\end{remark}

\subsection{Slicing}

Let $X$ be an object of $\catE$.
Given a category $U : \cat{A} \to \catE^{\Box}$ over $\catE^{\Box}$, we construct its \emph{slice} over $X$ using (strict) pullback along the map on arrows induced by $\catE/X \to \catE$ forgetting the slicing information:
\[
\xymatrix{
  \cat{A}_{/X}
  \ar[r]
  \ar[d]_{U_{/X}}
  \pullback{dr}
&
  \cat{A}
  \ar[d]^{U}
\\
  (\catE/X)^{\Box}
  \ar[r]
&
  \catE^{\Box}
}
\]
Explicitly, the category $\cat{A}_{/X}$ has as objects pairs $(Y, h)$ where $Y : \cat{A}$ and $h$ is a map in $\catE$ from the bottom right corner of $U(Y)$ to $X$.

Dually, (strictly) pulling back along the map on arrows induced by $\catE \backslash X \to \catE$ constructs the \emph{coslice} over $X$:
\[
\xymatrix{
  \cat{A}_{\backslash X}
  \ar[r]
  \ar[d]_{U_{\backslash X}}
  \pullback{dr}
&
  \cat{A}
  \ar[d]^{U}
\\
  (\catE \backslash X)^{\Box}
  \ar[r]
&
  \catE^{\Box}
}
\]

\begin{lemma}
\label{pitchfork-slicing-extended}
The functor $\liftr{\brarghole}$ and $\liftl{\brarghole}$ commute (up to natural isomorphism) with slicing and coslicing, respectively.
\end{lemma}

\begin{remark}
\label{slicing-mixed}
The slicing construction for arrows and \cref{pitchfork-slicing} can be viewed as a special case of the slicing construction for squares and \cref{pitchfork-slicing-extended} using the adjunction~\eqref{adjunction-between-E-square-and-E-to}.
In fact, we have analogous constructions and statements for the ``mixed'' versions of the lifting adjunction of \cref{extended-adjunction-gives-normal-one}.
\end{remark}

\subsection{Retract closure}

Let $U : \cat{A} \to \catE^{\Box}$ be a category over $\catE^{\Box}$.
We define its left retract closure $\overline{U}_L : \overline{\cat{A}}_L \to \catE^{\Box}$ as follows.
An object of $\overline{\cat{A}}_L$ is a tuple $(a, e, s, r)$ with $a : \cat{A}$ and $e : \catE^{\Box}$ together with morphisms $s : U(a)\sql \to e\sql$ and $r : e\sqr \to U(a)\sqr$ such that $r \cc U(a) \cc s = e$ where we $U(a)$ and $e$ as morphisms in $\catE^{\to}$:
\[
\xymatrix{
  e\sql
  \ar[r]^{e}
  \ar[d]^{s}
  \ar@/_2em/[dd]_{\id}
&
  e\sqr
  \ar@/^2em/[dd]^{\id}
\\
  U(a)\sql
  \ar[r]^{U(a)}
&
  U(a)\sqr
  \ar[d]_{r}
\\
  e\sql
  \ar[r]^{e}
&
  e\sqr
}
\]
The action of $\overline{U}_L$ on this object is $\overline{U}_L(a, e, s, r) = e$.
A morphism between such objects $(a, e, s, r)$ and $(a', e', s', r')$ consists of a morphism $f : a \to a'$ in $\cat{A}$ and a morphism $h : e \to e'$ in $\catE^{\Box}$ such that the evident squares commute:
\[
\xymatrix{
  e
  \ar[r]_{s}
  \ar@/^1em/[rr]^{\id}
  \ar[d]_{h}
&
  U(a)
  \ar[r]_{r}
  \ar[d]^{U(f)}
&
  e
  \ar[d]^{h}
\\
  e'
  \ar[r]^{s'}
  \ar@/_1em/[rr]_{\id}
&
  U(a')
  \ar[r]^{r'}
&
  e'
}
\]
The action of $\overline{U}_L$ on this morphism is $\overline{U}_L(f, h) = h$.

We have an evident inclusion $\cat{A} \to \overline{\cat{A}}_L$ of categories over $\catE^{\Box}$ mapping $a$ to $(a, U(a), \id_{U(a)\sql}, \id_{U(a)\sqr})$.
We also have a map $\overline{\overline{\cat{A}}_L}_L \to \overline{\cat{A}}_L$ of categories over $\catE^{\Box}$ mapping $((a, e, s, r), e', s', r')$ to $(a, e', s \cc s', r' \cc r)$.
This makes left retract closure into a monad.

\begin{lemma}
\label{retract-closure-extended-left}
Recall the adjunction~\eqref{garner-adjunction-extended}.
The functor $\liftr{\brarghole}$ maps unit and multiplication of left retract closure to natural isomorphisms.
\end{lemma}

\begin{remark}
\label{retract-closure-extended-right}
Dually, we have a right retract closure $\overline{U}_R : \overline{\cat{A}}_R \to \catE^{\Box}$ for a category $U : \cat{A} \to \catE^{\Box}$ over $\catE^{\Box}$.
Right retract closure forms a monad, with the functor $\liftl{\brarghole}$ mapping unit and comultiplication to natural isomorphisms.
\end{remark}

\begin{remark}
\label{retract-closure-extended-slicing}
Taking the left or right retract closure commutes (up to natural isomorphism) with slicing and coslicing.
\end{remark}

\begin{remark}
\label{slicing-mixed}
Retract closure for arrows can be viewed as a special case of left and right retract closure for squares using the adjunction~\eqref{adjunction-between-E-square-and-E-to}.
In fact, we have analogous constructions and statements for the ``mixed'' versions of the lifting adjunction of \cref{extended-adjunction-gives-normal-one}.
\end{remark}

\subsection{Kan extension}

Let $U : \cat{A} \to \catE^{\Box}$ be a category over $\catE^{\Box}$.
Consider a fully faithful functor $F : \cat{A} \to \cat{B}$.
Assume respectively that the pointwise left and right Kan extension of $U$ along $F$ exist.
\begin{mathpar}
\xymatrix{
  \cat{A}
  \ar[dr]_{U}
  \ar[rr]^{F}
&&
  \cat{B}
  \ar[dl]^{\Lan_F U}
\\&
  \catE^{\Box}
}
\and
\xymatrix{
  \cat{A}
  \ar[dr]_{U}
  \ar[rr]^{F}
&&
  \cat{B}
  \ar[dl]^{\Ran_F U}
\\&
  \catE^{\Box}
}
\end{mathpar}
Note the triangles will be (strictly) commuting since $F$ is fully faithful.

\begin{lemma}
\label{kan-extension-closure-extended}
The following functors induced by the adjunction~\eqref{garner-adjunction-extended} are isomorphisms:
\begin{mathpar}
\xymatrix{
  \liftr{\cat{A}}
  \ar[dr]_{\liftr{U}}
&&
  \liftr{\cat{B}}
  \ar[ll]_{\liftr{F}}^{\simeq}
  \ar[dl]^{\liftr{(\Lan_F U)}}
\\&
  \catE^{\Box}
}
\and
\xymatrix{
  \liftl{\cat{A}}
  \ar[dr]_{\liftl{U}}
&&
  \liftl{\cat{B}}
  \ar[ll]_{\liftl{F}}^{\simeq}
  \ar[dl]^{\liftl{(\Ran_F U)}}
\\&
  \catE^{\Box}
}
\end{mathpar}
\end{lemma}

\begin{remark}
\label{kan-extension-closure-mixed}
\cref{lift-of-adjunction} can be viewed as a special case of \cref{lift-of-adjunction-extended} using the adjunction~\eqref{adjunction-between-E-square-and-E-to}.
In fact, we have analogous statements for the ``mixed'' versions of the lifting adjunction of \cref{extended-adjunction-gives-normal-one}.
\end{remark}


\section{Dependent Product for Algebraic Fibrations}

\subsection{The setting}

Let $(\catE, \unit, \otimes)$ be a symmetric monoidal category.
Fix an \emph{interval} object $\interval : \catE$ with a map $\intervalc : \interval \to \unit$ and sections $\intervall, \intervalr : \unit \to \interval$ to $\intervalc$.

\begin{definition}
\label{def:homotopy}
Given maps $f, g : A \to B$ in $\catE$, a \emph{homotopy} $h$ from $f$ to $g$, denoted $h : f \sim g$, is a map $h : \interval \otimes A \to B$ such that $h \cc (\intervall \otimes A) = f$ and $h \cc (\intervalr \otimes A) = g$:
\[
\xymatrix{
  A
  \ar[dr]^{f}
  \ar[d]_{\intervall \otimes A}
\\
  \interval \otimes A
  \ar@{.>}[r]^{h}
&
  B
\\
  A
  \ar[u]^{\intervalr \otimes A}
  \ar[ur]_{g}
}
\]
\end{definition}

Recall the Leibniz construction and its properties \cite[Section 4]{riehl-verity:reedy}.
For a bifunctor $H : \catC \times \catD \to \catE$, we will denote its associated Leibniz construction by $\hat{H} : \catC^{\to} \times \catD^{\to} \to \catE^{\to}$.
Assuming now in addition that $\catE$ is finitely cocomplete, this turns the arrow category of $\catE$ into a symmetric monoidal category $(\catE^{\to}, \hatunit, \hatotimes)$ with unit the canonical map $\hatunit : 0 \to \unit$.

\begin{definition}
\label{def:homotopy-equivalence}
A map $f : A \to B$ is called a \emph{left (right) homotopy equivalence} if it comes with $g : B \to A$ together with homotopies $h : \id_A \sim g \cc f$ and $k : \id_B \sim f \cc g$ (respectively $h : g \cc f \sim \id_A$ and $k : f \cc g \sim \id_B$).
This notion is symmetric, admitting an obvious duality.
Such a left (right) homotopy equivalence is \emph{strong} if $f \cc h = k \cc (\interval \otimes f)$ and \emph{co-strong} if its dual is co-strong, \ie if $g \cc k = h \cc (\interval \otimes g)$.
%
%A \emph{deformation retract} is a homotopy equivalence as above where the homotopy $h$ is trivial (note that this makes $f$ and $g$ into a section-retraction pair).
%Dually, a \emph{co-deformation retract} has the homotopy $k$ trivial (with $g$ and $f$ a section-retraction pair).
\end{definition}

The notion of a left or right strong homotopy equivalence is an obvious generalization of the notion of strong deformation retract, in which in addition the homotopy $h$ is trivial.

Of special importance will be the following ``trivial'' square:
\begin{equation}
\label{trivial-square}
\begin{gathered}
\xymatrix@C+2em{
  0
  \ar[r]^{\hatunit}
  \ar[d]_{\hatunit}
&
  \unit
  \ar[d]^{\intervalr}
\\
  \unit
  \ar[r]_{\intervall}
&
  \interval
}
\end{gathered}
\end{equation}
Read horizontally, it is a map $\thetal : \hatunit \to \intervalr$ in the arrow category.
Read vertically, it is a map $\thetar : \hatunit \to \intervall$ in the arrow category.
The reason for the importance of~\eqref{trivial-square} is the following rather neat characterization of strong homotopy equivalences:
\begin{lemma}
\label{strong-h-equiv-as-section}
Let $f : A \to B$ be a map in $\catE$.
Making $f$ into a strong left homotopy equivalence is equivalent to making $\thetal \hatotimes f : f \to \intervalr \hatotimes f$ into a section.
Dually, making $f$ into a strong right homotopy equivalence is equivalent to making $\thetar \hatotimes f : f \to \intervall \hatotimes f$ into a section.
\end{lemma}

\begin{proof}
By duality, it will suffice to exhibit the first equivalence.

Making $\thetal \hatotimes f : f \to \intervalr \hatotimes f$ into a section means to give a retraction $r$ as follows:
\[
\xymatrix@C+1em{
  f
  \ar[r]^-{\thetal \hatotimes f}
  \ar@/_2em/[rr]_{\id}
&
  \intervalr \hattimes f
  \ar@{.>}[r]^{r}
&
  f
}
\]
To give $r$ from $\intervalr \hattimes f$ to $f$ is to give maps $h : \interval \otimes A \to A$, $g : B \to A$, and $k : \interval \otimes B \to B$ such that $h \cc (\intervalr \otimes A) = g \cc f$, $f \cc h = k \cc (\interval \otimes f)$, and $f \cc g = k \cc (\intervalr \otimes B)$.
The map $r$ constituting a section to $\theta \hattimes f$ means $h \cc (\intervall \otimes A) = \id_A$ and $k \cc (\intervall \otimes B) = \id_B$.
\[
\xymatrix@C+1em{
&&&
  A
  \ar[dr]^{\intervalr \otimes A}
  \ar[dd]^(0.7){f}
\\&
  A
  \ar[dd]^{f}
  \ar[rrr]_(0.4){\intervall \otimes A}
&&&
  \interval \otimes A
  \ar[dd]^{\interval \otimes f}
  \ar@{.>}@/_2em/[lll]_(0.6){h}
\\&&&
  B
  \ar[dr]^{\intervalr \otimes B}
  \ar@{.>}@/^1em/[llu]^{g}
\\&
  B
  \ar[rrr]_{\intervall \otimes B}
&&&
  \interval \otimes B
  \ar@{.>}@/_2em/[lll]^{k}
}
\]
With respect to the requirements of \cref{def:homotopy-equivalence}, the first three equations turn into right endpoint for $h$, co-strength, and right endpoint for $k$, while the two equations for the section constraint turn into left endpoints for $h$ and $k$.
\end{proof}

The utility of \cref{strong-h-equiv-as-section} can immediately be seen in proving the following closure properties, working entirely on the level of arrow categories.

\begin{proposition}
\label{strong-h-equiv-closed-under-monoidal-prod}
If one of the maps $f$ and $g$ is a (left or right) strong homotopy equivalence, then so is $f \hatotimes g$.
\end{proposition}

\begin{proof}
Apply \cref{strong-h-equiv-as-section} and use that functors --- in this case the Leibniz monoidal product in one variable --- preserve sections.
\end{proof}

\begin{proposition}
\label{strong-h-equiv-closed-under-retract}
(Left or right) strong homotopy equivalences are closed under retracts.
\end{proposition}

\begin{proof}
Apply \cref{strong-h-equiv-as-section} and use that functors preserve and sections are closed under retracts.
\end{proof}

\subsection{Trivial Algebraic Fibrations}

Recall the notion of adhesive morphisms~\cite{garner-lack:adhesive}.
Let $\cal{I} : \cat{I} \to \catE$ be a subcategory of adhesive morphisms in $\catE$ with morphisms given by cartesian squares.
Assume the subcategory $\cal{I}$ is closed under the monoidal operations of $\catE^{\to}$, \ie that $(\cat{I}, \hatunit, \hatotimes)$ is itself a monoidal category and $\cal{I}$ preserves the monoidal structure on the nose.

A \emph{trivial algebraic fibration} is a lift of a map in $\catE$ to $\liftr{\cal{I}}$.

\subsection{Algebraic Fibrations}

An \emph{algebraic fibration} is a lift of a map in $\catE$ to $\liftr{(\braces{\intervall, \intervalr} \boxtimes \cal{I})}$.
Here, we have used shorthand notation as follows.
The expression $\braces{l, r}$ denotes the inclusion $1 + 1 \to \catE^{\to}$ valued in $l$ and $r$.
The box product $U \boxtimes V$ of categories $U : \cat{A} \to \catE^{\to}$ and $V : \cat{B} \to \catE^{\to}$ over $\catE^{\to}$ is given by $(\arghole \otimes \arghole) \cc (U \times V) : \cat{A} \times \cat{B} \to \catE^{\to}$.

Note that $\liftr{(\braces{\intervall, \intervalr} \boxtimes \cal{I})} \simeq \liftr{(l \otimes \cal{I}\brarghole)} \times \liftr{(r \otimes \cal{I}\brarghole)}$.

\subsection{Algebraic Fibrations via Composition}

We will write $\intervall$ and $\thetar$ also for the respective corresponding discrete subcategories of $\catE^{\to}$ and $\catE^{\Box}$ on a single object; the same convention applies to other arrows and squares.

Since $\intervall$ is the right side of $\thetar$, we have a functor $F : \thetar \to \overline{\intervall}_L$ of categories over $\catE^{\to}$.
This induces a functor $F \hatotimes \cal{I} : \thetar \hatotimes \cal{I} \to \overline{\intervall}_L \hatotimes \cal{I}$.
Under certain conditions, which we will now examine, the square pitchfork adjunction turns this left functor into an isomorphism.

\begin{lemma}
\label{filling-vs-composition}
Assume
\begin{enumerate}
\item
that $\intervall$ is an object of $\cal{I}$,
\item
that $\thetar : \hatunit \to \intervall$ is a morphism of $\cal{I}$,
\item
that $\intervall$ a right strong homotopy equivalence.
\end{enumerate}
Then $\liftr{(F \hatotimes \cal{I})} : \liftr{(\overline{\intervall}_L \hatotimes \cal{I})} \to \liftr{(\thetar \hatotimes \cal{I})}$ is an isomorphism.
In particular, arrows $\intervall \hatotimes \cal{I}$ and $\thetar \hatotimes \cal{I}$ generate the same right category.
\end{lemma}

\begin{proof}
By the third assumption and \cref{strong-h-equiv-as-section}, we know that $\thetar \hatotimes \intervall$ is a section in $\catE^{\to}$.
This exhibits the domain $\intervall$ of $\thetar \hatotimes \intervall$ as an element of the left retract closure of the latter, giving us a functor $G : \intervall \to \overline{\thetar \hatotimes \intervall}_L$ of categories over $\catE^{\to}$.

By the first assumption and since $\cal{I}$ is closed under Leibniz monoidal product, we have an inclusion $H : \intervall \otimes \cal{I} \to \cal{I}$.

Let us put together all variants of the maps $F, G, H$ that will be used in the following argument in one (not-commuting) diagram of categories over $\cal{E}^{\to}$:
\begin{equation}
\label{filling-vs-composition:0}
\begin{gathered}
\xymatrix{
  \thetar \hatotimes \cal{I}
  \ar[d]
  \ar[dr]^{F \hatotimes \cal{I}}
&
  \intervall \hatotimes \cal{I}
  \ar[d]
  \ar[dr]^{G \hatotimes \cal{I}}
&
  \thetar \hatotimes \intervall \hatotimes \cal{I}
  \ar[d]
  \ar[dr]^{F \hatotimes \interval \hatotimes \cal{I}}
  \ar@/_2em/[ll]_{\thetar \hatotimes H}
&
  \intervall \hatotimes \intervall \hatotimes \cal{I}
  \ar[d]
  \ar@/_2em/[ll]_{\intervall \hatotimes H}
\\
  \overline{\thetar}_L \hatotimes \cal{I}
  \ar[d]
&
  \overline{\intervall}_L \hatotimes \cal{I}
  \ar[d]
&
  \overline{\thetar \hatotimes \intervall}_L \hatotimes \cal{I}
  \ar[d]
&
  \overline{\intervall}_L \hatotimes \intervall \hatotimes \cal{I}
  \ar[d]
\\
  \overline{\thetar \hatotimes \cal{I}}_L
&
  \overline{\intervall \hatotimes \cal{I}}_L
&
  \overline{\thetar \hatotimes \intervall \hatotimes \cal{I}}_L
  \ar@/^2em/[ll]^{\overline{\thetar \hatotimes H}_L}
&
  \overline{\intervall \hatotimes \intervall \hatotimes \cal{I}}_L
  \ar@/^2em/[ll]^{\overline{\intervall \hatotimes H}_L}
}
\end{gathered}
\end{equation}
Recall that left retract closure is a monad, with the right pitchfork functor mapping unit and multiplication to natural isomorphisms.

Let us focus on one half of the diagram~\eqref{filling-vs-composition:0}:
\[
\xymatrix{
  \thetar \hatotimes \cal{I}
  \ar[d]
  \ar[drr]^{d}
&&
  \thetar \hatotimes \intervall \hatotimes \cal{I}
  \ar[d]
  \ar@/_2em/[ll]_{\thetar \hatotimes H}
\\
  \overline{\thetar \hatotimes \cal{I}}
&&
  \overline{\thetar \hatotimes \intervall \hatotimes \cal{I}}_L
  \ar@/^2em/[ll]^{\overline{\thetar \hatotimes H}_L}
}
\]

T (A x B) 

Here, the diagonal map $d$ is given by the Kleisli composition of $F$ and $G$ 

The composition of $d$ with $\overline{\thetar \hatotimes H}_L$ maps $i : \cat{I}$ over the arrow $\thetar \hatotimes i$ to the left retract $(\thetar \hatotimes \intervall \hatotimes i, \intervall \hatotimes i, \thetar \hatotimes i, \mbox{--})$:
\[
\xymatrix@C+1em{
  i
  \ar[r]^-{\thetar \hatotimes i}
  \ar[d]^{\thetar \hatotimes i}
  \ar@/_2em/[dd]_{\id}
&
  \intervall \hatotimes i
  \ar@/^2em/[dd]^{\id}
\\
  \intervall \hatotimes i
  \ar[r]^-{\thetar \hatotimes \intervall \hatotimes i}
&
  \intervall \hatotimes \intervall \hatotimes i
  \ar[d]
\\
  i
  \ar[r]^-{\thetar \hatotimes i}
&
  \intervall \hatotimes i
}
\]


\end{proof}


\newcommand{\stronghe}{\cat{S}}

\begin{definition}
We define a category $\cat{S}_l(\cal{I})$ of strong left homotopy equivalences in $\catE$ relative to some category $\cal{I} : \cat{I} \to \catE^{\to}$ of arrows.
The objects consist of objects $i : \cat{I}$ together with data $(g, h, k)$ making $\cal{I}(i)$ into a strong homotopy equivalence.
A morphism from $i : \cat{I}$ with $\cal{I}(i) : A \to B$ and data $(g, h, k)$ to $i' : \cat{I}$ with $\cal{I}(i') : A' \to B'$ with data $(g', h', k')$ consists of a map $m : i \to i'$ in $\cat{I}$ such that, writing $\cal{I}(m) = (u, v)$ with maps $u : A \to A'$ and $v : B \to B'$, the following diagrams commute:
\begin{mathpar}
\xymatrix{
  B
  \ar[r]^{g}
  \ar[d]^{v}
&
  A
  \ar[d]^{u}
\\
  B'
  \ar[r]^{g'}
&
  A'
}
\and
\xymatrix{
  \interval \otimes A
  \ar[r]^{h}
  \ar[d]^{\interval \otimes u}
&
  A
  \ar[d]^{\interval \otimes u}
\\
  \interval \otimes A'
  \ar[r]^{h'}
&
  A'
}
\and
\xymatrix{
  \interval \otimes B
  \ar[r]^{k}
  \ar[d]^{\interval \otimes v}
&
  B
  \ar[d]^{\interval \otimes v}
\\
  \interval \otimes B'
  \ar[r]^{k'}
&
  B'
}
\end{mathpar}

We have an obvious forgetful functor $\cat{S}_l(\cal{I}) \to \cat{I}$.
Let $\cal{S}_l(\cal{I}) : \cat{S}_l(\cal{I}) \to \catE^{\to}$ be its composition with $\cal{I} : \cat{I} \to \catE^{\to}$.
\end{definition}

\begin{remark}
\label{strong-h-equiv-as-section-algebraic}
Following the proof of \cref{strong-h-equiv-as-section}, the category $\cat{S}_l(\cal{I})$ can isomorphicly be described at the level of the arrow category $\catE^{\to}$ as follows.
An object consists of $i : \cal{I}$ together with a retraction $\rho$ to $\thetal \hatotimes \cal{I}(i)$:
\[
\xymatrix@C+1em{
  \cal{I}(i)
  \ar[r]^-{\thetal \hatotimes \cal{I}(i)}
  \ar[dr]_{\id}
&
  \intervalr \hatotimes \cal{I}(i) \ar[d]^{\rho}
\\&
  \cal{I}(i)
}
\]
A morphism from $(i, \rho)$ to $(i', \rho')$ consists of $m : i \to i'$ that coheres with $\rho$ and $\rho'$ as below:
\[
\xymatrix{
  \intervalr \hatotimes \cal{I}(i)
  \ar[r]^-{\rho}
  \ar[d]^{\intervalr \hatotimes \cal{I}(m)}
&
  f
  \ar[d]^{\cal{I}(m)}
\\
  \intervalr \hatotimes \cal{I}(i')
  \ar[r]^-{\rho'}
&
  \cal{I}(i')
}
\]
With this description, the functor $\cal{S}_l(\cal{I})$ maps $(i, \rho)$ to $i$.
\end{remark}

\begin{lemma}
\label{she-to-retract-closure}
There is functor $\cal{S}_l(\cal{I}) \to \overline{\intervall \hatotimes \cal{I}}$ of categories over $\catE^{\to}$.
\end{lemma}

\begin{proof}
We work with the characterization of \cref{strong-h-equiv-as-section-algebraic}.
That remark already characterizes strong left homotopy equivalences in $\cal{I}$ as particular retracts of $\intervall \hatotimes \cal{I}$, so we could stop here.

Continuing anyway, suppose we are given an object $(i, \rho)$ of $\cat{S}_l(\cal{I})$, lying over the arrow $\cal{I}(i)$.
Observe that it already has the form of a retract $(i, \cal{I}(i), \thetar \hatotimes f, \rho)$:
\[
\xymatrix@C+1em{
  \cal{I}(i)
  \ar[r]^-{\thetal \hatotimes \cal{I}(i)}
  \ar[dr]_{\id}
&
  \intervalr \hatotimes \cal{I}(i) \ar[d]^{\rho}
\\&
  \cal{I}(i)
}
\]

Suppose we are given a morphism $m : (i, \rho) \to (i', \rho')$ in $\cat{S}_l(\cal{I})$.
Observe that it already has the required form of a morphism $(m, \cal{I}(m))$ between retracts:
\[
\xymatrix@C+2em{
  \cal{I}(i)
  \ar[r]_-{\thetar \hatotimes \cal{I}(i)}
  \ar[d]_{\cal{I}(m)}
  \ar@/^2em/[rr]^{\id}
&
  \intervall \hatotimes \cal{I}(i)
  \ar[r]_-{\rho}
  \ar[d]^{\intervall \hatotimes \cal{I}(m)}
&
  \cal{I}(i)
  \ar[d]^{\cal{I}(m)}
\\
  \cal{I}(i')
  \ar[r]^-{\thetar \hatotimes \cal{I}(i')}
  \ar@/_2em/[rr]_{\id}
&
  \intervall \hatotimes \cal{I}(i')
  \ar[r]^-{\rho'}
&
  \cal{I}(i')
}
\]
\end{proof}

\begin{lemma}
\label{horn-times-gen-to-she}
Assume that $\intervall$ is a strong left homotopy equivalence and that $\cal{I}$ is closed under tensoring with $\intervall$.
Then there is a functor $\intervall \hatotimes \cal{I} \to \cat{S}_l(\cal{I})$ of categories over $\catE^{\to}$.
\end{lemma}

\begin{proof}
We work with the characterization of \cref{strong-h-equiv-as-section-algebraic}.

Since $\intervall$ is assumed a strong left homotopy equivalence, we have a retraction $\rho$ as follows:
\[
\xymatrix@C+1em{
  \intervall
  \ar[r]^-{\thetal \hatotimes \intervall}
  \ar[dr]_{\id}
&
  \intervalr \hatotimes \intervall \ar[d]^{\rho}
\\&
  \intervall
}
\]

Suppose we are given an object $i : \cat{I}$, lying over the arrow $\intervall \hatotimes \cal{I}(i)$.
We construct its image $(l \hatotimes i, \rho \hatotimes i)$ by tensoring the previous diagram with $\cal{I}(i)$ and using closure property of $\cal{I}$ from the assumptions:
\[
\xymatrix@C+2em{
  \cal{I}(\intervall \hatotimes i)
  \ar[r]^-{\thetal \hatotimes \cal{I}(\intervall \hatotimes i)}
  \ar[dr]_{\id}
&
  \intervalr \hatotimes \cal{I}(\intervall \hatotimes i \ar[d]^{\rho \hatotimes i})
\\&
  \cal{I}(\intervall \hatotimes i)
}
\]

Suppose we are given a morphism $m : i \to i'$ in $\cat{I}$, lying over map $\intervall \hatotimes \cal{I}(m)$ of arrows.
We send it to the morphism $\intervall \hatotimes m : (\intervall \hatotimes i, \rho \hatotimes i) \to (\intervall \hatotimes i', \rho \hatotimes i')$, again using the closure property of $\cal{I}$ from the assumptions: 
\[
\xymatrix@C+2em{
  l \hatotimes i
  \ar[r]_{\thetar \hatotimes l \hatotimes i}
  \ar[d]_{l \hatotimes \tau}
  \ar@/^2em/[rr]^{\id}
&
  \intervall \hatotimes l \hatotimes i
  \ar@{.>}[r]_{\rho \hatotimes i}
  \ar[d]^{\intervall \hatotimes l \hatotimes \tau}
&
  l \hatotimes i
  \ar[d]^{l \hatotimes \tau}
\\
  l \hatotimes i'
  \ar[r]^{\thetar \hatotimes l \hatotimes i'}
  \ar@/_2em/[rr]_{\id}
&
  \intervall \hatotimes l \hatotimes i'
  \ar[r]^{\rho' \hatotimes i}
&
  l \hatotimes i'
}
\]
Note that the right square commutes by interchange.
\end{proof}

Let $U$ be the comonad given by $\cod$ and its left adjoint.
Recall that $U$ sends an arrow $f : A \to B$ to $\canonical_{0 \to B}$.
We use it to define $\cal{I}_0 : \cat{I} \to \catE^{\to}$ as $\cal{I}_0 = U \cc \cal{I}$.

\begin{lemma}
We consider the following functors over $\catE$:
\[
\xymatrix{
  (\cat{S}_l)_{/ \arghole}
  \ar[r]
  \ar[dr]_{W}
&
  \cat{I}_{/ \arghole}
  \ar[r]
  \ar[d]_(0.4){V}
&
  (\catE/\arghole)^{\to}
  \ar[dl]^{U}
\\&
  \catE 
}
\]
Assume that $V$ is a cartesian fibration and the functor from $V$ to $U$ creates cartesian morphisms.
Then $W$ is a cartesian fibration and the functor from $W$ to $V$ preserves cartesian morphisms.
\end{lemma}

\begin{proof}

\end{proof}


\begin{lemma}
Let $p : Y \to X$ be a right map for $\intervalr \hatotimes \cal{I}_0$. 
Assume that base change along $p$ lifts to a functor $F : \cat{I}_{/X} \to \cat{I}_{/Y}$.
Then it lifts further to a functor $G : (\cat{S}_l)_{/X} \to (\cat{S}_l)_{/Y}$:
\[
\xymatrix{
  (\cat{S}_l)_{/X}
  \ar@{.>}[r]^{G}
  \ar[d]_{(\cal{S}_l)_{/X}}
&
  (\cat{S}_l)_{/Y}
  \ar[d]^{(\cal{S}_l)_{/Y}}
\\
  \cat{I}_{/X}
  \ar@{-->}[r]^{F}
  \ar[d]_{\cal{I}_{/X}}
&
  \cat{I}_{/Y}
  \ar[d]^{\cal{I}_{/Y}}
\\
  (\catE / X)^{\to}
  \ar[r]^{(p^*)^{\to}}
&
  (\catE / Y)^{\to}
}
\]

\end{lemma}

\begin{proof}
Suppose we are given an object $(i, \rho) : (\cat{S}_l)_{/X}$:
\[
\xymatrix@C+1em{
  \cal{I}(i)
  \ar[r]^-{\thetal \hatotimes \cal{I}(i)}
  \ar[dr]_{\id}
&
  \intervalr \hatotimes \cal{I}(i) \ar[d]^{\rho}
\\&
  \cal{I}(i)
}
\]
Applying $F$ gives us $i' : \cat{I}_{/Y}$ and a cartesian square $m : \cal{I}(i') \to \cal{I}(i)$ that lies cartesian over $p$.
We want to find a retraction $\rho'$ to $\thetar \hatotimes \cal{I}(i')$.
\[
\xymatrix@C+2em{
  \cal{I}(i')
  \ar[r]_-{\thetar \hatotimes \cal{I}(i')}
  \ar[d]_{\cal{I}(m)}
  \ar@/^2em/[rr]^{\id}
&
  \intervall \hatotimes \cal{I}(i')
  \ar@{.>}[r]_-{\rho'}
  \ar[d]^{\intervall \hatotimes \cal{I}(m)}
&
  \cal{I}(i')
  \ar[d]^{\cal{I}(m)}
\\
  \cal{I}(i)
  \ar[r]^-{\thetar \hatotimes \cal{I}(i)}
  \ar@/_2em/[rr]_{\id}
&
  \intervall \hatotimes \cal{I}(i)
  \ar[r]^-{\rho}
&
  \cal{I}(i)
}
\]

\end{proof}

Given a generating left category $\cal{A} : \cat{A} \to \catE^{\to}$, the right category $\liftr{\cal{A}} : \liftr{\cat{A}} \to \catE^{\to}$ can be made into a double category with horizontal edge category $\catE$ where vertical composition is given by composition of lifts.
We denote its vertical edge category by $\catE_{\cal{A}}$.
There is an obvious functor $\catE_{\cat{A}} \to \catE$ that forgets the lifting data.


Recall the codomain fibration $\cod : \catE^{\to} \to \catE$.
Pulling back $\cod^{\to}$ along the constant functor $\Delta : \catE \to \catE^{\to}$ results in a fibration $F : \cal{D} \to \catE$ with fibers the arrow categories of the fibers of $\cod$:
\[
\xymatrix{
  \cal{D}
  \ar[r]
  \ar[d]_{F}
  \pullback{dr}
&
  (\catE^{\to})^{\to}
  \ar[d]^{\cod^{\to}}
\\
  \catE
  \ar[r]^{\Delta}
&
  \catE^{\to}
}
\]
Let $F' : \cal{S} \downarrow \catE_{\cal{I}_0} \to \catE_{\cal{I}_0}$ denote the obvious forgetful functor.
Note that the functors $\cat{S} \to \catE^{\to}$ and $\catE_{\cat{I}_0} \to \catE$ extend to a commutative square of functors as follows:
\begin{equation}
\label{spam:0}
\begin{gathered}
\xymatrix{
  \cal{S} \downarrow \catE_{\cat{I}_0}
  \ar[r]^{U}
  \ar[d]_{F'}
&
  \cal{D}
  \ar[d]_{F}
\\
  \catE_{\cat{I}_0}
  \ar[r]^{V}
&
  \catE
}
\end{gathered}
\end{equation}

\begin{lemma}
The functor $F'$ is a cartesian fibration.
The square~\eqref{spam:0} is a morphism of fibrations that creates cartesian morphisms.
\end{lemma}

\begin{proof}
We work with the characterization of $\cal{S} : \cat{S} \to \catE^{\to}$ of \cref{strong-h-equiv-as-section-algebraic}.

Let us first show that $U$ reflects cartesian morphisms.
Fix a map $(p, \phi)$ in $\catE_{\cal{I}_0}$ where $p : X \to Y$ is a map in $\catE$ and $\phi$ is a coherent family of right lifts against $\cal{I}_0$.
Let $\tau : (f, \rho) \to (g, \sigma)$ be a map in $\cal{S} \downarrow \catE_{\cal{I}_0}$ over $(p, \phi)$.
Assume that $U(\tau)$ is cartesian.

objects of $\cat{S}$ over $X$ and $Y$ respectively.


Let us frst define the action of $p^*$ on an object $(f, \rho)$ where $f$ is a map over $X$.


Consider an object $(f, \rho)$ with $f$ over $X$.
We wish to construct an object $(f', \rho')$ with $f'$ over $Y$ the base change of $f$ along $p$.

\[
\xymatrix@C+2em{
  f'
  \ar[r]_{\thetar \hatotimes f'}
  \ar[d]_{u}
  \ar@/^2em/[rr]^{\id}
&
  \intervall \hatotimes f'
  \ar@{.>}[r]_{\rho'}
  \ar[d]^{\intervall \hatotimes u}
&
  f'
  \ar[d]^{u}
\\
  f
  \ar[r]^{\thetar \hatotimes f}
  \ar@/_2em/[rr]_{\id}
&
  \intervall \hatotimes f
  \ar[r]^{\rho}
&
  f
}
\]

slice over X
->
slice over Y


\end{proof}

\begin{lemma}
\begin{enumerate}
Assume:
\item
that $\intervall$ is a strong right homotopy equivalence,
\item
that $\intervall$ is an object of $\cal{I}$.
\end{enumerate}
Let $p : Y \to X$ be a right map for $\cal{I}_0$.
Then base change along $p$ lifts to a functor $(\intervall \hatotimes \cal{I})_{/X} \to \overline{\intervall \hatotimes \cal{I}}_{/Y}$.
\end{lemma}

\begin{proof}
Since $\intervall$ is a strong right homotopy equivalence, we have a retraction $r$ to the map $\thetar \hatotimes \intervall$ in the arrow category.

\end{proof}


\[
\xymatrix@C+2em{
  f'
  \ar[rr]_{\thetar \hatotimes f'}
  \ar[dd]_{u}
  \ar@/^2em/[rrrr]^{\id}
  \ar[dr]
&&
  \intervall \hatotimes f'
  \ar@{.>}[rr]_{s'}
  \ar[dd]^{\intervall \hatotimes u}
  \ar[dr]
&&
  f'
  \ar[dd]^{u}
  \ar[dr]
\\&
  g'
  \ar[rr]_{\thetar \hatotimes g'}
  \ar[dd]_{v}
  \ar@/^2em/[rrrr]^{\id}
&&
  \intervall \hatotimes g'
  \ar@{.>}[rr]_{t'}
  \ar[dd]^{\intervall \hatotimes v}
&&
  g'
  \ar[dd]^{v}
\\
  f
  \ar[rr]^{\thetar \hatotimes f}
  \ar@/_2em/[rrrr]_{\id}
  \ar[dr]
&&
  \intervall \hatotimes f
  \ar[rr]^{s}
  \ar[dr]
&&
  f
  \ar[dr]
\\&
  g
  \ar[rr]^{\thetar \hatotimes g}
  \ar@/_2em/[rrrr]_{\id}
&&
  \intervall \hatotimes g
  \ar[rr]^{t}
&&
  g
}
\]


\paragraph{Composition as restricted lifting problem}

Awfs generated by Leibniz construction of (decidable) subobjects of representables with theta squares.
Also algebraically free on Leibniz construction of (decidable) monomorphisms with theta squares.

\paragraph{Connections induce isomorphism of right categories}

Use characterization of connection in terms of interval endpoint inclusions being strong homotopy equivalences.
This implies theta squares are sections in the arrow category.
Use retract closure from first section.

\subsection{Dependent Product for Algebraic Fibrations}

Let pseudo-fibrations be right maps for representables times interval endpoint inclusions.
Also right map for any presheaf times interval endpoint inclusions.
Show that base change along pseudo-fibrations functorially maps theta squares to left retract closures of theta squares.
By earlier elementary lemmata and previous subsection, it follows that Pi along pseudo-fibrations lifts to functor between right categories.
Verify BC.


\bibliographystyle{plain}
\bibliography{../../common/uniform-kan-bibliography}

\end{document}

\documentclass[reqno,10pt,a4paper,oneside]{amsart}

\usepackage{uniform-kan-prelude}

\title{Dependent Product for Algebraic Fibrations}

\begin{document}

\begin{abstract}
We give a categorical account of algebraic fibrations being preserved under certain dependent product.
This generalizes work by Coquand \etal.
\end{abstract}

\maketitle

\tableofcontents

\section{Lifting against Arrows}

For this section, fix a category $\cat{E}$.
There is a well known Gallois connection $\liftr{\brarghole} \dashv \liftl{\brarghole}$ between the poset of subsets of arrows in $\cat{E}$ and its opposite.
Garner~\cite{garner:small-object-argument} shows how to extend this to an adjunction between categories of arrows:
\begin{equation}
\label{garner-adjunction}
\begin{gathered}
\xymatrix@C+1em{
  \CAT/\cat{E}^{\to}
  \ar@<5pt>[r]^{\liftl{\brarghole}}
  \ar@{}[r]|{\bot}
&
  (\CAT/\cat{E}^{\to})^{\op}
  \ar@<5pt>[l]^{\liftr{\brarghole}}
}
\end{gathered}
\end{equation}

\begin{lemma}
Consider a natural transformation between categories over $\cat{E}^{\to}$:
\[
\xymatrix{
  \cat{A}
  \rrtwocell_G^F{\sigma}
 \ar[dr]_{U}
&&
  \cat{B}
  \ar[dl]^{V}
\\&
  \cat{E}^{\to}
}
\]
Note that this includes the condition $V \sigma = \id_U$.
Then $\liftr{F}$ and $\liftr{G}$ as well as $\liftl{F}$ and $\liftl{G}$ agree:
\begin{mathpar}
\xymatrix{
  \liftr{\cat{A}}
  \ar[dr]_{\liftr{U}}
&&
  \liftr{\cat{B}}
  \lltwocell_{\liftr{F}}^{\liftr{G}}{=}
  \ar[dl]^{\liftr{V}}
\\&
  \cat{E}^{\to}
}
\and
\xymatrix{
  \liftl{\cat{A}}
  \ar[dr]_{\liftl{U}}
&&
  \liftl{\cat{B}}
  \lltwocell_{\liftl{F}}^{\liftl{G}}{=}
  \ar[dl]^{\liftl{V}}
\\&
  \cat{E}^{\to}
}
\end{mathpar}
\end{lemma}

\subsection{Lifts of adjoint functors}

Consider an adjunction as follows:
\[
\xymatrix@C+1em{
  \cat{D}
  \ar@<5pt>[r]^{L}
  \ar@{}[r]|{\bot}
&
  \cat{E}
  \ar@<5pt>[l]^{R}
}
\]
Let $U : \cat{A} \to \cat{D}^{\to}$ and $V : \cat{B} \to \cat{E}^{\to}$ be categories over $\cat{D}^{\to}$ and $\cat{E}^{\to}$, respectively.

\begin{lemma}
\label{lift-of-adjunction}
Lifts of $L^{\to}$ to a functor $\cat{A} \to \liftl{\cat{B}}$ are in natural correspondence with lifts of $R^{\to}$ to a functor $\cat{B} \to \liftr{\cat{A}}$:
\begin{mathpar}
\xymatrix{
  \cat{A}
  \ar@{.>}[r]
  \ar[d]_{U}
&
  \liftl{\cat{B}}
  \ar[d]^{\liftl{V}}
\\
  \cat{D}^{\to}
  \ar[r]^{L^{\to}}
&
  \cat{E}^{\to}
}
\and
\xymatrix{
  \liftr{\cat{A}}
  \ar[d]_{\liftr{U}}
&
  \cat{B}
  \ar@{.>}[l]
  \ar[d]^{V}
\\
  \cat{D}^{\to}
&
  \cat{E}^{\to}
  \ar[l]^{R^{\to}}
}
\end{mathpar}
\end{lemma}

\begin{proof}
Maps from $L^{\to} \cc U$ to $\liftl{V}$ over $\cat{E}^{\to}$ consist of fillers for squares $L^{\to}(U(X)) \to V(Y)$ natural in $X : \cat{A}$ and $Y : \cat{B}$.
Similarly, maps from $R^{\to} \cc V$ to $\liftr{U}$ over $\cat{D}^{\to}$ consist of fillers for squares $U(X) \to R^{\to}(V(Y))$ natural in $X : \cat{A}$ and $Y : \cat{B}$.
Since $L \dashv R$, these situations coincide.
\end{proof}

\subsection{Slicing}

Let $X$ be an object of $\cat{E}$.
Given a category $U : \cat{A} \to \cat{E}^{\to}$ over $\cat{E}^{\to}$, we construct its \emph{slice} over $X$ using (strict) pullback along the map on arrows induced by $\cat{E}/X \to \cat{E}$ forgetting the slicing information:
\[
\xymatrix{
  \cat{A}_{/X}
  \ar[r]
  \ar[d]_{U_{/X}}
  \pullback{dr}
&
  \cat{A}
  \ar[d]^{U}
\\
  (\cat{E}/X)^{\to}
  \ar[r]
&
  \cat{E}^{\to}
}
\]
Explicitly, the category $\cat{A}_{/X}$ has as objects pairs $(Y, h)$ where $Y : \cat{A}$ and $h$ is a map in $\cat{E}$ from the codomain of $U(Y)$ to $X$.

Dually, (strictly) pulling back along the map on arrows induced by $\cat{E} \backslash X \to \cat{E}$ constructs the \emph{coslice} over $X$:
\[
\xymatrix{
  \cat{A}_{\backslash X}
  \ar[r]
  \ar[d]_{U_{\backslash X}}
  \pullback{dr}
&
  \cat{A}
  \ar[d]^{U}
\\
  (\cat{E} \backslash X)^{\to}
  \ar[r]
&
  \cat{E}^{\to}
}
\]

\begin{remark}
\label{pitchfork-slicing}
Just as in the non-algebraic setting, the functor $\liftr{\brarghole}$ and $\liftl{\brarghole}$ commute (up to natural isomorphism) with slicing and coslicing, respectively.
\end{remark}

\subsection{Retract closure}

\newcommand{\ret}{\mathbbm{R}}
\newcommand{\retA}{\mathbbm{a}}
\newcommand{\retB}{\mathbbm{b}}

Let $\ret$ denote the \emph{walking retract}, \ie the category with objects $\retA, \retB$ and morphisms generated by $s : \retA \to \retB$ and $r : \retB \to \retA$ under the relation $r \cc s = \id_{\retA}$.

Given a category $U : \cat{A} \to \cat{E}^{\to}$ over $\cat{E}^{\to}$, we define its retract closure $\overline{U} : \overline{\cat{A}} \to \cat{E}^{\to}$ using (strict) pullback and left composition:
\[
\xymatrix@C+1em{
  \overline{\cat{A}}
  \ar[r]
  \ar[d]
  \ar@/_2em/[dd]_{\overline{U}}
  \pullback{dr}
&
  \cat{A}
  \ar[d]^{U}
\\
  (\cat{E}^{\to})^{\ret}
  \ar[r]^-{(\cat{E}^{\to})^{\retB}}
  \ar[d]^{(\cat{E}^{\to})^{\retA}}
&
  \cat{E}^{\to}
\\
  \cat{E}^{\to}
}
\]
Explicitly, an object of $\overline{\cat{A}}$ consists of $a : \cat{A}$ and a retract of $U(a)$ in $\cat{E}^{\to}$, \ie an arrow $e : \cat{E}^{\to}$ together with maps $s : e \to U(a)$ and $r : U(a) \to e$ such that $r \cc s = \id_e$:
\[
\xymatrix{
  e
  \ar[r]_{s}
  \ar@/^1em/[rr]^{\id}
&
  U(a)
  \ar[r]_{r}
&
  e
}
\]
The action of $\overline{U}$ on this object is $\overline{U}(a, e, s, t) = e$.
A morphism between such objects $(a, e, s, t)$ and $(a', e', s', t')$ consists of a morphism $f : a \to a'$ in $\cat{A}$ and a morphism $h : e \to e'$ in $\cat{E}^{\to}$ such that the evident squares commute:
\[
\xymatrix{
  e
  \ar[r]_{s}
  \ar@/^1em/[rr]^{\id}
  \ar[d]_{h}
&
  U(a)
  \ar[r]_{r}
  \ar[d]^{U(f)}
&
  e
  \ar[d]^{h}
\\
  e'
  \ar[r]^{s'}
  \ar@/_1em/[rr]_{\id}
&
  U(a')
  \ar[r]^{r'}
&
  e'
}
\]
The action of $\overline{U}$ on this morphism is $\overline{U}(f, h) = h$.

We have an evident inclusion $\cat{A} \to \overline{\cat{A}}$ of categories over $\cat{E}^{\to}$ formally induced by $(\cat{E}^{\to})^{\canonical} : \cat{E}^{\to} \to (\cat{E}^{\to})^{\ret}$ being a section to $(\cat{E}^{\to})^{\retB}$.
With the explicit description of objects, it maps $a$ to $(a, U(a), \id_{U(a)}, \id_{U(a)})$.
We also have a map $\overline{\overline{\cat{A}}} \to \overline{\cat{A}}$ of categories over $\cat{E}^{\Box}$ mapping $((a, e, s, r), e', s', r')$ to $(a, e', s \cc s', r' \cc r)$.
This makes retract closure into a monad.

\begin{lemma}
\label{retract-closure}
Recall the adjunction~\eqref{garner-adjunction}.
The functors $\liftr{\brarghole}$ and $\liftl{\brarghole}$ maps unit and multiplication of retract closure to natural isomorphisms.
\end{lemma}

\begin{remark}
\label{retract-closure-slicing}
Taking the retract closure commutes (up to natural isomorphism) with slicing and coslicing.
\end{remark}

\subsection{Kan extension}

Let $U : \cat{A} \to \cat{E}^{\to}$ be a category over $\cat{E}^{\to}$.
Consider a fully faithful functor $F : \cat{A} \to \cat{B}$.
Assume respectively that the pointwise left and right Kan extension of $U$ along $F$ exist.
\begin{mathpar}
\xymatrix{
  \cat{A}
  \ar[dr]_{U}
  \ar[rr]^{F}
&&
  \cat{B}
  \ar[dl]^{\Lan_F U}
\\&
  \cat{E}^{\to}
}
\and
\xymatrix{
  \cat{A}
  \ar[dr]_{U}
  \ar[rr]^{F}
&&
  \cat{B}
  \ar[dl]^{\Ran_F U}
\\&
  \cat{E}^{\to}
}
\end{mathpar}
Note the triangles will be (strictly) commuting since $F$ is fully faithful.

\begin{lemma}
\label{kan-extension-closure}
The following functors induced by the adjunction~\eqref{garner-adjunction} are isomorphisms:
\begin{mathpar}
\xymatrix{
  \liftr{\cat{A}}
  \ar[dr]_{\liftr{U}}
&&
  \liftr{\cat{B}}
  \ar[ll]_{\liftr{F}}^{\simeq}
  \ar[dl]^{\liftr{(\Lan_F U)}}
\\&
  \cat{E}^{\to}
}
\and
\xymatrix{
  \liftl{\cat{A}}
  \ar[dr]_{\liftl{U}}
&&
  \liftl{\cat{B}}
  \ar[ll]_{\liftl{F}}^{\simeq}
  \ar[dl]^{\liftl{(\Ran_F U)}}
\\&
  \cat{E}^{\to}
}
\end{mathpar}
\end{lemma}

\section{Lifting against Squares}

For this section, fix again a category $\cat{E}$.
The \emph{square category} $\cat{E}^{\Box}$ is the double arrow category $(\cat{E}^{\to})^{\to}$.
We have evident inclusions $\squl, \squr, \sqdl, \sqdr$ of the walking object and $\sql, \sqr, \squ, \sqd$ of the walking arrow into the walking square:
\[
\xymatrix{
  \squl
  \ar[r]^{\squ}
  \ar[d]_{\sql}
  \ar@{}[dr]
&
  \squr
  \ar[d]^{\sqr}
\\
  \sqdl
  \ar[r]_{\sqd}
&
  \sqdr
}
\]
We also have projections $\sqhori$ and $\sqvert$ from the walking square to the walking arrow that are sections to $\sql, \sqr$ and $\squ, \sqd$, respectively.

Given a square $X : \cat{E}^{\Box}$, its corners and sides are thus respectively given by functor precomposition as $X\squl, X\squr, X\sqdl, X\sqdr : \cat{E}$ and $X\sql, X\sqr, X\squ, X\sqd : \cat{E}^{\to}$.
Similarly, given a line $f : \cat{E}^{\to}$, then $f\sqhori$ and $f\sqvert$ are the squares that have identities as horizontal and vertical sides, respectively, with the other sides being given by $f$.
Note that if $h : X \to Y$ is a morphism of squares, then \eg $h\sql : X\sql \to Y\sql$ will denote the induced morphism between the left sides, and analogously for the other operations as indicated by the suggestive functor precomposition notation.

Given a square $X : \cat{E}^{\Box}$, we will sometimes equivalently view it as a morphism in $\cat{E}^{\to}$ from $X\sql$ to $X\sqr$.

Recall from~\cite{garner:small-object-argument} that the Gallois connection $\liftl{\brarghole} \dashv \liftr{\brarghole}$ on the class of arrows of $\cat{E}$ may be lifted to an adjunction of categories over $\cat{E}^{\to}$:
\begin{equation}
\label{garner-adjunction}
\begin{gathered}
\xymatrix@C+1em{
  \CAT/\cat{E}^{\to}
  \ar@<5pt>[r]^{\liftl{\brarghole}}
  \ar@{}[r]|{\bot}
&
  (\CAT/\cat{E}^{\to})^{\op}
  \ar@<5pt>[l]^{\liftr{\brarghole}}
}
\end{gathered}
\end{equation}

\begin{lemma}
\label{garner-adjunction-extended}
The adjunction~\eqref{garner-adjunction} can be lifted further to an adjunction as follows:
\begin{equation}
\label{garner-adjunction-extended}
\begin{gathered}
\xymatrix@C+1em{
  \CAT/\cat{E}^{\Box}
  \ar@<5pt>[r]^{\liftl{\brarghole}}
  \ar@{}[r]|{\bot}
&
  (\CAT/\cat{E}^{\Box})^{\op}
  \ar@<5pt>[l]^{\liftr{\brarghole}}
}
\end{gathered}
\end{equation}
\end{lemma}

\begin{proof}[Proof (adapted from \cite{garner:small-object-argument})]
Let us first define the functor $\liftr{\brarghole}$.
It sends $U : \cat{A} \to \cat{E}^{\Box}$ to the category $\liftr{\cat{A}}$ that has as objects pairs $(T, \phi)$ consisting of a square $T : \cat{E}^{\Box}$ with a coherent choice $\phi$ of \emph{compositions}.
For $X : A$ and a ``middle square'' $M : U(X)\sqr \to T\sql$, a composition $\phi(A, M)$ is a lift in the composite square indicated below:
\[
\xymatrix{
  \bullet
  \ar[r]
  \ar[d]
  \ar@{}[dr]|(0.4){U(X)}
&
  \bullet
  \ar[r]
  \ar[d]
  \ar@{}[dr]|(0.3){M}
&
  \bullet
  \ar[r]
  \ar[d]
  \ar@{}[dr]|(0.6){T}
&
  \bullet
  \ar[d]
\\
  \bullet
  \ar[r]
  \ar@{.>}[urrr]
&
  \bullet
  \ar[r]
&
  \bullet
  \ar[r]
&
  \bullet
}
\]
Coherence of composition means that for a morphism $f : X \to X'$ in $\cat{A}$ and a square $M' : U(X)\sqr \to T\sql$, we have $\phi(X, M \cc U(f)\sqr) = \phi(X', M') \cc U(f)\sqdl$.
A morphism in $\liftr{\cat{A}}$ from $(T, \phi)$ to $(T', \phi')$ is a morphism $t : T \to T'$ of squares respecting the choice of liftings $\phi$ and $\phi'$: for $X : A$ and $M : U(X)\sqr \to T$, we must have $t\squr \cc \phi(X, M) = \phi'(X, t\sql \cc M)$.
We have an evident forgetful functor from $\liftr{\cat{A}} \to \cat{E}^{\Box}$.

This concludes the construction of the action of $\liftr{\brarghole}$ on objects.
The action on a morphism $F : \cat{A} \to \cat{B}$ is given by the functor $\liftr{\cat{B}} \to \liftr{\cat{A}}$ over $\cat{E}^{\Box}$ sending $(T, \psi)$ to $(T, \phi)$ where $\phi(X, M) = \psi(F(X), M)$.

The functor $\liftl{\brarghole}$ is defined analogously, but with directions and order of composition swapped.
To see that $\liftl{\brarghole} \dashv \liftr{\brarghole}$, note that given categories $U : \cat{A} \to \cat{E}^{\Box}$ and $V : \cat{B} \to \cat{E}^{\Box}$ over $\cat{E}^{\Box}$, both functors $\cat{A} \to \liftl{\cat{B}}$ and $\cat{B} \to \liftr{\cat{A}}$ may be identified with ``$(\cat{A}, \cat{B})$-lifting operations'': a function $\psi$ that assigns to objects $X : \cat{A}$ and $Y : \cat{B}$ and a ``middle square'' $M : U(X)\sqr \to V(Y)\sql$ a lift for the composite square 
\[
\xymatrix{
  \bullet
  \ar[r]
  \ar[d]
  \ar@{}[dr]|(0.4){U(X)}
&
  \bullet
  \ar[r]
  \ar[d]
  \ar@{}[dr]|(0.3){M}
&
  \bullet
  \ar[r]
  \ar[d]
  \ar@{}[dr]|(0.6){V(Y)}
&
  \bullet
  \ar[d]
\\
  \bullet
  \ar[r]
  \ar@{.>}[urrr]
&
  \bullet
  \ar[r]
&
  \bullet
  \ar[r]
&
  \bullet
}
\]
that is natural in the evident manner in both $X$ and $Y$.
\end{proof}

\begin{remark}
\label{extended-adjunction-gives-normal-one}
Recall the inclusion $E\sqhori : \cat{E}^{\to} \to \cat{E}^{\Box}$ that sends a line $f$ to the square with left and right sides $f$ and up and bottom sides identities.
It related categories over $\cat{E}^{\Box}$ and $\cat{E}^{\to}$ by an adjunction:
\begin{equation}
\label{adjunction-between-E-square-and-E-to}
\begin{gathered}
\xymatrix@C+1em{
  \CAT/\cat{E}^{\Box}
  \ar@<5pt>[r]^{E\sqhori_!}
  \ar@{}[r]|{\bot}
&
  \CAT/\cat{E}^{\to}
  \ar@<5pt>[l]^{(E\sqhori)^*}
}
\end{gathered}
\end{equation}
In fact, as can be confirmed from inspection of the constructions, the adjunction~\eqref{garner-adjunction} arises from the extended adjunction~\eqref{garner-adjunction-extended} by pre- and postcomposition with the adjunction~\eqref{adjunction-between-E-square-and-E-to} and its opposite:
\begin{equation*}
\begin{gathered}
\xymatrix@R+1em@C+1em{
  \CAT/\cat{E}^{\to}
  \ar@<5pt>[r]^{\liftl{\brarghole}}
  \ar@{}[r]|{\bot}
  \ar@<5pt>[d]^{(E\sqhori)^*}
  \ar@{}[d]|{\dashv}
&
  (\CAT/\cat{E}^{\to})^{\op}
  \ar@<5pt>[l]^{\liftr{\brarghole}}
  \ar@<5pt>[d]^{((E\sqhori)^*)^{\op}}
  \ar@{}[d]|{\vdash}
\\
  \CAT/\cat{E}^{\Box}
  \ar@<5pt>[r]^{\liftl{\brarghole}}
  \ar@{}[r]|{\bot}
  \ar@<5pt>[u]^{E\sqhori_!}
&
  (\CAT/\cat{E}^{\Box})^{\op}
  \ar@<5pt>[l]^{\liftr{\brarghole}}
  \ar@<5pt>[u]^{(E\sqhori_!)^{\op}}
}
\end{gathered}
\end{equation*}
If we only choose to either pre- or postcompose, we end up with ``mixed'' versions of the adjunction that we are still going to denote using the same symbols:
\begin{equation}
\label{garner-adjunction-extended-mixed}
\begin{gathered}
\xymatrix@C+1em{
  \CAT/\cat{E}^{\Box}
  \ar@<5pt>[r]^{\liftl{\brarghole}}
  \ar@{}[r]|{\bot}
&
  (\CAT/\cat{E}^{\to})^{\op}
  \ar@<5pt>[l]^{\liftr{\brarghole}}
}
\end{gathered}
\end{equation}
or:
\begin{equation}
\label{garner-adjunction-extended-mixed'}
\begin{gathered}
\xymatrix@C+1em{
  \CAT/\cat{E}^{\to}
  \ar@<5pt>[r]^{\liftl{\brarghole}}
  \ar@{}[r]|{\bot}
&
  (\CAT/\cat{E}^{\Box})^{\op}
  \ar@<5pt>[l]^{\liftr{\brarghole}}
}
\end{gathered}
\end{equation}
\end{remark}

\subsection{Lifts of adjoint functors}

Consider an adjunction as follows:
\[
\xymatrix@C+1em{
  \cat{D}
  \ar@<5pt>[r]^{L}
  \ar@{}[r]|{\bot}
&
  \cat{E}
  \ar@<5pt>[l]^{R}
}
\]
Let $U : \cat{A} \to \cat{D}^{\Box}$ and $V : \cat{B} \to \cat{E}^{\Box}$ be categories over $\cat{D}^{\Box}$ and $\cat{E}^{\Box}$, respectively.
We have a straightforward analogue of \cref{lift-of-adjunction}:

\begin{lemma}
\label{lift-of-adjunction-extended}
Lifts of $L^{\Box}$ to a functor $\cat{A} \to \liftl{\cat{B}}$ are in natural correspondence with lifts of $R^{\Box}$ to a functor $\cat{B} \to \liftr{\cat{A}}$:
\begin{mathpar}
\xymatrix{
  \cat{A}
  \ar@{.>}[r]
  \ar[d]_{U}
&
  \liftl{\cat{B}}
  \ar[d]^{^{\pitchfork}V}
\\
  \cat{D}^{\Box}
  \ar[r]^{L^{\Box}}
&
  \cat{E}^{\Box}
}
\and
\xymatrix{
  \liftr{\cat{A}}
  \ar[d]_{\liftr{U}}
&
  \cat{B}
  \ar@{.>}[l]
  \ar[d]^{V}
\\
  \cat{D}^{\Box}
&
  \cat{E}^{\Box}
  \ar[l]^{R^{\Box}}
}
\end{mathpar}
\end{lemma}

\begin{proof}
Maps from $L^{\Box} \cc U$ to $\liftl{V}$ over $\cat{E}^{\Box}$ consist of composition for squares $L^{\Box}(U(X))\sqr \to V(Y)\sql$ natural in $X : \cat{A}$ and $Y : \cat{B}$.
Similarly, maps from $R^{\Box} \cc V$ to $\liftr{U}$ over $\cat{D}^{\Box}$ consist of composition for squares $U(X)\sqr \to R^{\Box}(V(Y))\sql$ natural in $X : \cat{A}$ and $Y : \cat{B}$.
Since $L \dashv R$, these situations coincide.
\end{proof}

\begin{remark}
\label{lift-of-adjunction-mixed}
\cref{lift-of-adjunction} can be viewed as a special case of \cref{lift-of-adjunction-extended} using the adjunction~\eqref{adjunction-between-E-square-and-E-to}.
In fact, we have analogous statements for the ``mixed'' versions of the lifting adjunction of \cref{extended-adjunction-gives-normal-one}.
\end{remark}

\subsection{Slicing}

Let $X$ be an object of $\cat{E}$.
Given a category $U : \cat{A} \to \cat{E}^{\Box}$ over $\cat{E}^{\Box}$, we construct its \emph{slice} over $X$ using (strict) pullback along the map on arrows induced by $\cat{E}/X \to \cat{E}$ forgetting the slicing information:
\[
\xymatrix{
  \cat{A}_{/X}
  \ar[r]
  \ar[d]_{U_{/X}}
  \pullback{dr}
&
  \cat{A}
  \ar[d]^{U}
\\
  (\cat{E}/X)^{\Box}
  \ar[r]
&
  \cat{E}^{\Box}
}
\]
Explicitly, the category $\cat{A}_{/X}$ has as objects pairs $(Y, h)$ where $Y : \cat{A}$ and $h$ is a map in $\cat{E}$ from the bottom right corner of $U(Y)$ to $X$.

Dually, (strictly) pulling back along the map on arrows induced by $\cat{E} \backslash X \to \cat{E}$ constructs the \emph{coslice} over $X$:
\[
\xymatrix{
  \cat{A}_{\backslash X}
  \ar[r]
  \ar[d]_{U_{\backslash X}}
  \pullback{dr}
&
  \cat{A}
  \ar[d]^{U}
\\
  (\cat{E} \backslash X)^{\Box}
  \ar[r]
&
  \cat{E}^{\Box}
}
\]

\begin{lemma}
\label{pitchfork-slicing-extended}
The functor $\liftr{\brarghole}$ and $\liftl{\brarghole}$ commute (up to natural isomorphism) with slicing and coslicing, respectively.
\end{lemma}

\begin{remark}
\label{slicing-mixed}
The slicing construction for arrows and \cref{pitchfork-slicing} can be viewed as a special case of the slicing construction for squares and \cref{pitchfork-slicing-extended} using the adjunction~\eqref{adjunction-between-E-square-and-E-to}.
In fact, we have analogous constructions and statements for the ``mixed'' versions of the lifting adjunction of \cref{extended-adjunction-gives-normal-one}.
\end{remark}

\subsection{Retract closure}

Let $U : \cat{A} \to \cat{E}^{\Box}$ be a category over $\cat{E}^{\Box}$.
We define its left retract closure $\overline{U}_L : \overline{\cat{A}}_L \to \cat{E}^{\Box}$ as follows.
An object of $\overline{\cat{A}}_L$ is a tuple $(a, e, s, r)$ with $a : \cat{A}$ and $e : \cat{E}^{\Box}$ together with morphisms $s : U(a)\sql \to e\sql$ and $r : e\sqr \to U(a)\sqr$ such that $r \cc U(a) \cc s = e$ where we $U(a)$ and $e$ as morphisms in $\cat{E}^{\to}$:
\[
\xymatrix{
  e\sql
  \ar[r]^{e}
  \ar[d]^{s}
  \ar@/_2em/[dd]_{\id}
&
  e\sqr
  \ar@/^2em/[dd]^{\id}
\\
  U(a)\sql
  \ar[r]^{U(a)}
&
  U(a)\sqr
  \ar[d]_{r}
\\
  e\sql
  \ar[r]^{e}
&
  e\sqr
}
\]
The action of $\overline{U}_L$ on this object is $\overline{U}_L(a, e, s, r) = e$.
A morphism between such objects $(a, e, s, r)$ and $(a', e', s', r')$ consists of a morphism $f : a \to a'$ in $\cat{A}$ and a morphism $h : e \to e'$ in $\cat{E}^{\Box}$ such that the evident squares commute:
\[
\xymatrix{
  e
  \ar[r]_{s}
  \ar@/^1em/[rr]^{\id}
  \ar[d]_{h}
&
  U(a)
  \ar[r]_{r}
  \ar[d]^{U(f)}
&
  e
  \ar[d]^{h}
\\
  e'
  \ar[r]^{s'}
  \ar@/_1em/[rr]_{\id}
&
  U(a')
  \ar[r]^{r'}
&
  e'
}
\]
The action of $\overline{U}_L$ on this morphism is $\overline{U}_L(f, h) = h$.

We have an evident inclusion $\cat{A} \to \overline{\cat{A}}_L$ of categories over $\cat{E}^{\Box}$ mapping $a$ to $(a, U(a), \id_{U(a)\sql}, \id_{U(a)\sqr})$.
We also have a map $\overline{\overline{\cat{A}}_L}_L \to \overline{\cat{A}}_L$ of categories over $\cat{E}^{\Box}$ mapping $((a, e, s, r), e', s', r')$ to $(a, e', s \cc s', r' \cc r)$.
This makes left retract closure into a monad.

\begin{lemma}
\label{retract-closure-extended-left}
Recall the adjunction~\eqref{garner-adjunction-extended}.
The functor $\liftr{\brarghole}$ maps unit and multiplication of left retract closure to natural isomorphisms.
\end{lemma}

\begin{remark}
\label{retract-closure-extended-right}
Dually, we have a right retract closure $\overline{U}_R : \overline{\cat{A}}_R \to \cat{E}^{\Box}$ for a category $U : \cat{A} \to \cat{E}^{\Box}$ over $\cat{E}^{\Box}$.
Right retract closure forms a monad, with the functor $\liftl{\brarghole}$ mapping unit and comultiplication to natural isomorphisms.
\end{remark}

\begin{remark}
\label{retract-closure-extended-slicing}
Taking the left or right retract closure commutes (up to natural isomorphism) with slicing and coslicing.
\end{remark}

\begin{remark}
\label{slicing-mixed}
Retract closure for arrows can be viewed as a special case of left and right retract closure for squares using the adjunction~\eqref{adjunction-between-E-square-and-E-to}.
In fact, we have analogous constructions and statements for the ``mixed'' versions of the lifting adjunction of \cref{extended-adjunction-gives-normal-one}.
\end{remark}

\subsection{Kan extension}

Let $U : \cat{A} \to \cat{E}^{\Box}$ be a category over $\cat{E}^{\Box}$.
Consider a fully faithful functor $F : \cat{A} \to \cat{B}$.
Assume respectively that the pointwise left and right Kan extension of $U$ along $F$ exist.
\begin{mathpar}
\xymatrix{
  \cat{A}
  \ar[dr]_{U}
  \ar[rr]^{F}
&&
  \cat{B}
  \ar[dl]^{\Lan_F U}
\\&
  \cat{E}^{\Box}
}
\and
\xymatrix{
  \cat{A}
  \ar[dr]_{U}
  \ar[rr]^{F}
&&
  \cat{B}
  \ar[dl]^{\Ran_F U}
\\&
  \cat{E}^{\Box}
}
\end{mathpar}
Note the triangles will be (strictly) commuting since $F$ is fully faithful.

\begin{lemma}
\label{kan-extension-closure-extended}
The following functors induced by the adjunction~\eqref{garner-adjunction-extended} are isomorphisms:
\begin{mathpar}
\xymatrix{
  \liftr{\cat{A}}
  \ar[dr]_{\liftr{U}}
&&
  \liftr{\cat{B}}
  \ar[ll]_{\liftr{F}}^{\simeq}
  \ar[dl]^{\liftr{(\Lan_F U)}}
\\&
  \cat{E}^{\Box}
}
\and
\xymatrix{
  \liftl{\cat{A}}
  \ar[dr]_{\liftl{U}}
&&
  \liftl{\cat{B}}
  \ar[ll]_{\liftl{F}}^{\simeq}
  \ar[dl]^{\liftl{(\Ran_F U)}}
\\&
  \cat{E}^{\Box}
}
\end{mathpar}
\end{lemma}

\begin{remark}
\label{kan-extension-closure-mixed}
\cref{lift-of-adjunction} can be viewed as a special case of \cref{lift-of-adjunction-extended} using the adjunction~\eqref{adjunction-between-E-square-and-E-to}.
In fact, we have analogous statements for the ``mixed'' versions of the lifting adjunction of \cref{extended-adjunction-gives-normal-one}.
\end{remark}


\section{Dependent Product for Algebraic Fibrations}

\subsection{The setting}

Let $(\cat{E}, \unit, \otimes)$ be a symmetric monoidal category.
Fix an \emph{interval} object $\interval : \cat{E}$ with a map $\intervalc : \interval \to \unit$ and sections $\intervall, \intervalr : \unit \to \interval$ to $\intervalc$.

\begin{definition}
\label{def:homotopy}
Given maps $f, g : A \to B$ in $\cat{E}$, a \emph{homotopy} $h$ from $f$ to $g$, denoted $h : f \sim g$, is a map $h : \interval \otimes A \to B$ such that $h \cc (\intervall \otimes A) = f$ and $h \cc (\intervalr \otimes A) = g$:
\[
\xymatrix{
  A
  \ar[dr]^{f}
  \ar[d]_{\intervall \otimes A}
\\
  \interval \otimes A
  \ar@{.>}[r]^{h}
&
  B
\\
  A
  \ar[u]^{\intervalr \otimes A}
  \ar[ur]_{g}
}
\]
\end{definition}

Recall the Leibniz construction and its properties \cite[Section 4]{riehl-verity:reedy}.
For a bifunctor $H : \catC \times \catD \to \catE$, we will denote its associated Leibniz construction by $\hat{H} : \catC^{\to} \times \catD^{\to} \to \catE^{\to}$.
Assuming now in addition that $\cat{E}$ is finitely cocomplete, this turns the arrow category of $\cat{E}$ into a symmetric monoidal category $(\cat{E}^{\to}, \hatunit, \hatotimes)$ with unit the canonical map $\hatunit : 0 \to \unit$.

\begin{definition}
\label{def:homotopy-equivalence}
A map $f : A \to B$ is called a \emph{left (right) homotopy equivalence} if it comes with $g : B \to A$ together with homotopies $h : \id_A \sim g \cc f$ and $k : \id_B \sim f \cc g$ (respectively $h : g \cc f \sim \id_A$ and $k : f \cc g \sim \id_B$).
This notion is symmetric, admitting an obvious duality.
Such a left (right) homotopy equivalence is \emph{strong} if $f \cc h = k \cc (\interval \otimes f)$ and \emph{co-strong} if its dual is co-strong, \ie if $g \cc k = h \cc (\interval \otimes g)$.
%
%A \emph{deformation retract} is a homotopy equivalence as above where the homotopy $h$ is trivial (note that this makes $f$ and $g$ into a section-retraction pair).
%Dually, a \emph{co-deformation retract} has the homotopy $k$ trivial (with $g$ and $f$ a section-retraction pair).
\end{definition}

The notion of a left or right strong homotopy equivalence is an obvious generalization of the notion of strong deformation retract, in which in addition the homotopy $h$ is trivial.

Of special importance will be the following ``trivial'' square:
\begin{equation}
\label{trivial-square}
\begin{gathered}
\xymatrix@C+2em{
  0
  \ar[r]^{\hatunit}
  \ar[d]_{\hatunit}
&
  \unit
  \ar[d]^{\intervalr}
\\
  \unit
  \ar[r]_{\intervall}
&
  \interval
}
\end{gathered}
\end{equation}
Read horizontally, it is a map $\thetal : \hatunit \to \intervalr$ in the arrow category.
Read vertically, it is a map $\thetar : \hatunit \to \intervall$ in the arrow category.
The reason for the importance of~\eqref{trivial-square} is the following rather neat characterization of strong homotopy equivalences:
\begin{lemma}
\label{strong-h-equiv-as-section}
Let $f : A \to B$ be a map in $\cat{E}$.
Making $f$ into a strong left homotopy equivalence is equivalent to making $\thetal \hatotimes f : f \to \intervalr \hatotimes f$ into a section.
Dually, making $f$ into a strong right homotopy equivalence is equivalent to making $\thetar \hatotimes f : f \to \intervall \hatotimes f$ into a section.
\end{lemma}

\begin{proof}
By duality, it will suffice to exhibit the first equivalence.

Making $\thetal \hatotimes f : f \to \intervalr \hatotimes f$ into a section means to give a retraction $r$ as follows:
\[
\xymatrix@C+1em{
  f
  \ar[r]^-{\thetal \hatotimes f}
  \ar@/_2em/[rr]_{\id}
&
  \intervalr \hattimes f
  \ar@{.>}[r]^{r}
&
  f
}
\]
To give $r$ from $\intervalr \hattimes f$ to $f$ is to give maps $h : \interval \otimes A \to A$, $g : B \to A$, and $k : \interval \otimes B \to B$ such that $h \cc (\intervalr \otimes A) = g \cc f$, $f \cc h = k \cc (\interval \otimes f)$, and $f \cc g = k \cc (\intervalr \otimes B)$.
The map $r$ constituting a section to $\theta \hattimes f$ means $h \cc (\intervall \otimes A) = \id_A$ and $k \cc (\intervall \otimes B) = \id_B$.
\[
\xymatrix@C+1em{
&&&
  A
  \ar[dr]^{\intervalr \otimes A}
  \ar[dd]^(0.7){f}
\\&
  A
  \ar[dd]^{f}
  \ar[rrr]_(0.4){\intervall \otimes A}
&&&
  \interval \otimes A
  \ar[dd]^{\interval \otimes f}
  \ar@{.>}@/_2em/[lll]_(0.6){h}
\\&&&
  B
  \ar[dr]^{\intervalr \otimes B}
  \ar@{.>}@/^1em/[llu]^{g}
\\&
  B
  \ar[rrr]_{\intervall \otimes B}
&&&
  \interval \otimes B
  \ar@{.>}@/_2em/[lll]^{k}
}
\]
With respect to the requirements of \cref{def:homotopy-equivalence}, the first three equations turn into right endpoint for $h$, co-strength, and right endpoint for $k$, while the two equations for the section constraint turn into left endpoints for $h$ and $k$.
\end{proof}

The utility of \cref{strong-h-equiv-as-section} can immediately be seen in proving the following closure properties, working entirely on the level of arrow categories.

\begin{proposition}
\label{strong-h-equiv-closed-under-monoidal-prod}
If one of the maps $f$ and $g$ is a (left or right) strong homotopy equivalence, then so is $f \hatotimes g$.
\end{proposition}

\begin{proof}
Apply \cref{strong-h-equiv-as-section} and use that functors --- in this case the Leibniz monoidal product in one variable --- preserve sections.
\end{proof}

\begin{proposition}
\label{strong-h-equiv-closed-under-retract}
(Left or right) strong homotopy equivalences are closed under retracts.
\end{proposition}

\begin{proof}
Apply \cref{strong-h-equiv-as-section} and use that functors preserve and sections are closed under retracts.
\end{proof}

\subsection{Trivial Algebraic Fibrations}

Recall the notion of adhesive morphisms~\cite{garner-lack:adhesive}.
Let $\cal{I} : \cat{I} \to \cat{E}$ be a subcategory of adhesive morphisms in $\cat{E}$ with morphisms given by cartesian squares.
Assume the subcategory $\cal{I}$ is closed under the monoidal operations of $\cat{E}^{\to}$, \ie that $(\cat{I}, \hatunit, \hatotimes)$ is itself a monoidal category and $\cal{I}$ preserves the monoidal structure on the nose.

A \emph{trivial algebraic fibration} is a lift of a map in $\cat{E}$ to $\liftr{\cal{I}}$.

\subsection{Algebraic Fibrations}

An \emph{algebraic fibration} is a lift of a map in $\cat{E}$ to $\liftr{(\braces{\intervall, \intervalr} \boxtimes \cal{I})}$.
Here, we have used shorthand notation as follows.
The expression $\braces{l, r}$ denotes the inclusion $1 + 1 \to \cat{E}^{\to}$ valued in $l$ and $r$.
The box product $U \boxtimes V$ of categories $U : \cat{A} \to \cat{E}^{\to}$ and $V : \cat{B} \to \cat{E}^{\to}$ over $\cat{E}^{\to}$ is given by $(\arghole \otimes \arghole) \cc (U \times V) : \cat{A} \times \cat{B} \to \cat{E}^{\to}$.

Note that $\liftr{(\braces{\intervall, \intervalr} \boxtimes \cal{I})} \simeq \liftr{(l \otimes \cal{I}\brarghole)} \times \liftr{(r \otimes \cal{I}\brarghole)}$.


 $\braces{l, r} \otimes \cal{I}\brarghole$ for $l \otimes \cal{I}$


\section{Dependent Product for Algebraic Fibrations}

\subsection{Trivial Algebraic Fibrations}

Awfs generated by (decidable) subobjects of representables.
Also algebraically free on (decidable) monomorphisms.
TODO: Terminology implies we have a model structure.
Haven't talked about that yet.

\subsection{Algebraic Fibrations}

Awfs generated by Leibniz construction of (decidable) subobjects of representables with interval endpoint inclusions.
Also algebraically free on Leibniz construction of (decidable) monomorphisms with interval endpoint inclusions.

\subsection{Algebraic Fibrations via Composition}

Since $\intervall$ is the right side of $\thetal$, we have a map $F : \thetar \to \overline{\intervall}_L$ of categories over $\cat{E}^{\to}$.

\begin{lemma}
Assume that $\intervall$ a right strong homotopy equivalence and that $\cal{I}$ is closed under Leibniz monoidal product with $\intervall$.
Then $\liftr{F} : \liftr{\overline{\intervall}_L} \to \liftr{\thetar}$ is an isomorphism.
\end{lemma}

\begin{proof}
By \cref{strong-h-equiv-as-section}, we have $\thetar \otimes \intervall$ a section.
This gives us a map $G : \intervall \to \overline{\thetar \otimes \intervall}_L$ of categories over $\cat{E}^{\to}$.
We have $\cal{I}$ closed under Leibniz monoidal product with $\intervall$, meaning we have an inclusion $H : \intervall \otimes \cal{I} \to \cal{I}$.
Altogether, we have the following (not necessarily commuting) maps of categories over $\cat{E}^{\to}$:
\[
\xymatrix{
  \thetar \otimes \cal{I}
  \ar[d]
  \ar[dr]^{F \otimes \cal{I}}
&
  \intervall \otimes \cal{I}
  \ar[d]
  \ar[dr]^{G \otimes \cal{I}}
&
  \thetar \otimes \intervall \otimes \cal{I}
  \ar[d]
  \ar[dr]^{F \otimes \interval \otimes \cal{I}}
  \ar@/_2em/[ll]_{\thetar \otimes H}
&
  \intervall \otimes \intervall \otimes \cal{I}
  \ar[d]
  \ar@/_2em/[ll]_{\intervall \otimes H}
\\
  \overline{\thetar \otimes \cal{I}}
&
  \overline{\intervall \otimes \cal{I}}_L
&
  \overline{\thetar \otimes \intervall \otimes \cal{I}}_L
  \ar@/^2em/[ll]^{\overline{\thetar \otimes H}_L}
&
  \overline{\intervall \otimes \intervall \otimes \cal{I}}_L
  \ar@/^2em/[ll]^{\overline{\intervall \otimes H}_L}
}
\]
This gives two ways to go from $\intervall \otimes \overline{ \otimes \cal{I}}_L$ to ??

\end{proof}



\paragraph{Composition as restricted lifting problem}

Awfs generated by Leibniz construction of (decidable) subobjects of representables with theta squares.
Also algebraically free on Leibniz construction of (decidable) monomorphisms with theta squares.

\paragraph{Connections induce isomorphism of right categories}

Use characterization of connection in terms of interval endpoint inclusions being strong homotopy equivalences.
This implies theta squares are sections in the arrow category.
Use retract closure from first section.

\subsection{Dependent Product for Algebraic Fibrations}

Let pseudo-fibrations be right maps for representables times interval endpoint inclusions.
Also right map for any presheaf times interval endpoint inclusions.
Show that base change along pseudo-fibrations functorially maps theta squares to left retract closures of theta squares.
By earlier elementary lemmata and previous subsection, it follows that Pi along pseudo-fibrations lifts to functor between right categories.
Verify BC.

\section{Test}

Let $U$ be left arrows for filling.
Let $V$ be left squares for composition.

We have an inclusion $U \to \overline{V}$.


\bibliographystyle{plain}
\bibliography{../../common/uniform-kan-bibliography}

\end{document}

\documentclass[reqno,10pt,a4paper,oneside,draft]{amsart}

\setcounter{tocdepth}{1}
%\usepackage[parfill]{parskip}

\usepackage{uniform-kan-prelude}


\newcommand{\LL}{\mathsf{L}}
\newcommand{\RR}{\mathsf{R}}


\title{The Frobenius condition, right properness \linebreak and uniform fibrations}

\begin{document}

\begin{abstract}
We introduce and study the notion of a uniform fibration in categories with a functorial cylinder.
We show that in a wide class of presheaf categories, including simplicial sets and cubical sets with connections, uniform fibrations are the right class of an algebraic weak factorization system that satisfies the functorial Frobenius condition.
This implies that pushforward along a uniform fibration preserves uniform fibrations.
When instantiated in simplicial sets, this result solves the open problem of finding a constructive counterpart of one of the key facts underpinning Voevodsky's simplicial model of univalent foundations and gives a new proof of the right properness of the Quillen model structure for Kan complexes.
In cubical sets, it extends results by Coquand and others.
\end{abstract}

\author{Nicola Gambino}
\address{School of Mathematics, University of Leeds, Leeds LS2 9JT, UK}
\email{n.gambino@leeds.ac.uk}

\author{Christian Sattler}
\address{School of Mathematics, University of Leeds, Leeds LS2 9JT, UK}
\email{c.sattler@leeds.ac.uk}

% \date{\today}

\maketitle

\tableofcontents

% \newpage

\section{Introduction}

Bousfield's weak factorization systems~\cite{bousfield-wfs} are a cornerstone of modern homotopical algebra~\cite{quillen-homotopical}, not least because they support a very elegant treatment of Quillen model categories~\cite{joyal-tierney-notes}.
Yet, they have some unsatisfactory aspects, such as the failure of left and right maps to be closed under all colimits.
These issues can be remedied by considering algebraic weak factorisation systems~\cite{garner:small-object-argument,grandis-tholen-nwfs}, which involve maps equipped with a suitably natural family of solutions for lifting problems, rather than maps merely satisfying a lifting property.
The presence of this additional structure on maps has proved to be crucial for applications of algebraic weak factorization systems in higher-dimensional category theory~\cite{batanin-cisinski-weber,garner:globular-operator-awfs,garner-homomorphisms} and in Voevodsky's univalent foundations programme~\cite{awodey-cubical,coquand-cubical-sets,cohen-et-al:cubicaltt,pitts-cubical-nominal,swan-awfs}.

Our general aim here is to develop further the theory of algebraic weak factorization systems and provide some new applications.
In particular, we introduce the notion of uniform (trivial) fibration in a category with a functorial cylinder and use it to construct algebraic weak factorization systems on presheaf categories, obtaining generalizations of some important results of Cisinski on Quillen model structures on presheaf categories~\cite{cisinski-asterisque}.
As applications of our results, we solve an open problem in the univalent foundations programme, namely that of proving a constructive version of the fact that pushforward along Kan fibrations preserves Kan fibrations ~\cite{coquand-non-constructivity-kan,voevodsky-simplicial-model}.
We also obtain a new proof of the right properness of the Quillen model structure for Kan complexes~\cite{quillen-homotopical}, which avoids entirely the use of topological realization~\cite{hovey-model-categories} and the theory of minimal fibrations~\cite{joyal-tierney:simplicial-homotopy-theory}.

We are particularly interested in the property of weak factorization systems known as the Frobenius condition~\cite[Definition~3.2.1]{garner:types-omega-groupoids}.\footnote{See~\cite{clementino:frobenius} for an explanation of the connection with Lawvere's Frobenius reciprocity condition~\cite{lawvere-equality}.} One reason for the interest in this condition, which asserts that pullback along a right map preserves left maps, is that it is closely related to the right properness condition for a Quillen model structure, which asserts that pullback along a fibration preserves weak equivalences.
Indeed, for a Quillen model structure in which the cofibrations are the monomorphisms, such as the Cisinski model structures~\cite{cisinski-asterisque}, right properness is equivalent to the Frobenius condition for the weak factorization system of acyclic cofibrations and fibrations.
Furthermore, when pushforward along right maps exists, \ie pullback along right maps has a right adjoint, the Frobenius condition is equivalent to the preservation of right maps by pushforward along a right map.
In the category of simplicial sets, this amounts exactly to the preservation of Kan fibrations by pushforward along Kan fibrations mentioned above, a result that plays an important role in the definition of the simplicial model of univalent foundations~\cite[Lemma~2.3.1]{voevodsky-simplicial-model}.
Furthermore, the Frobenius condition is intimately related to the homotopy-theoretic aspects of Martin-L\"of's rules for identity types~\cite{gambino-garner:idtypewfs}.
It was also studied for (strict) factorization systems in~\cite{rosicky-tholem-factorization} (see Proposition~3.5 and Remark~3.6 therein).

When working with algebraic weak factorization systems, it is natural to consider the functorial Frobenius condition introduced by Garner and van den Berg in~\cite{garner:topological-simplicial}.\footnote{In~\cite{garner:topological-simplicial}, the functorial Frobenius condition was actually formulated for cloven weak factorization systems, which are slighly more general then algebraic weak factorization systems.} Since algebraically-free algebraic weak factorization systems, such as those produced by Garner's small object argument~\cite{garner:small-object-argument}, play an important role in our development, we isolate conditions on the generating category of an algebraically-free algebraic weak factorization system that are necessary and sufficient for the algebraic weak factorization system to satisfy the functorial Frobenius condition.
In order to do so, we introduce what we call the uniform Frobenius condition, which involves both a version of the usual Frobenius condition for maps, but also a Beck-Chevalley condition~\cite{benabou-descente,lawvere-equality} for morphisms of maps (\ie commutative squares), and provides a convenient notion to organize our analysis.

The development in this paper leads up to the proof of the existence of algebraic weak factorization systems with the functorial Frobenius condition on presheaf categories equipped with a functorial cylinder (\cref{frobenius-uniform-presheaf}).
The right maps of these algebraic weak factorization systems are uniform (trivial) fibrations, which we define in the general setting of a category equipped with a functorial cylinder, building on ideas of Cisinski~\cite{cisinski-asterisque}: uniform trivial fibrations are defined by algebraic orthogonality (\ie by requiring a natural choice of diagonal fillers) with respect to a category~$\mathcal{M}$ of monomorphisms and pullback squares, while uniform fibrations are defined by algebraic orthogonality with respect to a category obtained by applying a Leibniz construction (in the sense of~\cite{riehl-verity:reedy}) to the endpoint inclusions of the functorial cylinder and~$\cal{M}$.

Our first main result (\cref{thm:sset-cset-nwfs}) is that uniform trivial fibrations and uniform fibrations in presheaf categories equipped with a functorial cylinder are the right maps of two algebraic weak factorization systems.
This is proved using Garner's small object argument~\cite{garner:small-object-argument}, after having shown that uniform (trivial) fibrations can be be defined equivalently by algebraic orthogonality with respect to a small category of maps (\cref{small-gen-triv-kan}).
Working with algebraic notions of fibrations is essential to prove this fact, since it allows us to rephrase a general lifting problem as a colimit of a functorial family of lifting problems, providing a further example of the good interaction between colimits and lifting problems in algebraic weak factorization systems.

Our second main result (\cref{thm:ac-kan-is-uniform}) shows in what sense the notion of a uniform (trivial) fibration subsumes the standard notion of a (trivial) fibration, defined by a weak orthogonality property.
In particular, we show that, in a presheaf category over an elegant Reedy category~\cite{bergner-rezk-elegant}, a map can be equipped with the structure of a uniform (trivial) fibration if and only if it satisfies the right lifting property (without any uniformity condition) with respect to the maps involved in the definition of a uniform (trivial) fibration.
In order to obtain this result, we develop a careful decomposition of the lifting problems involved in the uniformity condition.

For our third main result, we return to consider the general setting of categories equipped with a functorial cylinder with connections and show that, under mild assumptions, uniform fibrations satisfy the uniform Frobenius condition (\cref{uniform-fibrations-uniform-frobenius}).
As a direct application, we obtain that the algebraic weak factorization systems on presheaf categories with uniform fibrations as right maps constructed earlier in the paper satisfy the uniform Frobenius condition (\cref{frobenius-uniform-presheaf}).

When instantiated to simplicial sets, this result implies that the algebraic weak factorization system with uniform Kan fibrations as right maps satisfies the Frobenius property, and therefore pushforward along a uniform Kan fibration preserves uniform Kan fibrations.
Our proof of this fact is constructive, \ie avoids the use of the law of excluded middle and the axiom of choice, and therefore solves the open problem of obtaining a constructive counterpart of the fact, mentioned above, that pushforward along a Kan fibration preserves Kan fibrations.
This latter fact was shown in~\cite{coquand-non-constructivity-kan} to be unprovable constructively without modifications.
Because of our results, it seems natural to regard that independence result simply as an indication that the standard notion of a Kan fibration is not suitable for developing simplicial homotopy theory constructively, rather than as evidence that the simplicial setting is inherently non-constructive. 

When instantiated in the category of cubical sets with connections, instead, \cref{frobenius-uniform-presheaf} gives a new proof of the fact that pushforward along a uniform Kan fibration preserves uniform Kan fibrations~\cite{cohen-et-al:cubicaltt}.
This proof avoids entirely complex calculations with cubical sets (\cf~\cite{huber-thesis}).
Indeed, one of the initial motivations for this work was to explore whether the theory of uniform fibrations in cubical sets could be developed at a greater level of generality, so as to make it applicable also to simplicial sets.

In the setting of simplicial sets, \cref{frobenius-uniform-presheaf} provides also a new proof of the right properness of the Quillen model structure for Kan complexes that avoids the use of geometric realization and minimal fibrations (see \cref{rem:rightpropernessnew}).

\subsection*{Remarks}

For brevity, we do not always discuss the special cases of our results on algebraic weak factorization systems for standard weak factorization systems.
Such results can be readily obtained by replacing the existence of structure in the algebraic setting with the satisfaction of properties, functors between categories with inclusions between classes, pullbacks with intersections, and so on.
This idea is illustrated in a few examples in the first sections of the paper and then highlighted only when it is of particular interest.

In the hope of making the paper more widely accessible, we also confine comments on constructivity issues to a series of remarks and footnotes.
However, let us point out here for the interested readers that our development does not rely on the law of the excluded middle or the axiom of choice, apart from~\cref{thm:sset-cset-nwfs} and the results in~\cref{sec:non-alg}.
For~\cref{thm:sset-cset-nwfs}, we argue that the result holds constructively for simplicial sets and cubical sets separately, in~\cref{rem:constructive-small-object}.%
\footnote{A constructive proof of~\cref{thm:sset-cset-nwfs} in full generality would require an analysis of aspects of the theory of locally presentable categories and of Garner's small object argument which we leave for future investigation.}
For the results in~\cref{sec:non-alg}, where we show how the classical notion of a fibration is equivalent to the uniform one, the use of the axiom of choice seems unavoidable.

\subsection*{Organization of the paper}

Section~\ref{sec:ortf} establishes basic facts about orthogonality functors.
In Section~\ref{sec:frobc} we introduce the Frobenius, Beck-Chevalley, and uniform Frobenius conditions and relate them to the functorial Frobenius condition.
Section~\ref{sec:unif} introduces uniform fibrations, giving examples of these notions in presheaf categories in general and in simplicial and cubical sets in particular, and establishing the existence of algebraic weak factorization systems with uniform (trivial) fibrations as right maps.
The relationship between uniform fibrations and their standard counterparts is studied in \cref{sec:non-alg}.
After establishing some auxiliary results in \cref{sec:unifshe} on strong homotopy equivalences, we prove the uniform Frobenius property for uniform fibrations in~\cref{sec:frocuf}.
Appendix~\ref{app:tecp} contains the proof of a technical lemma.

\subsection*{Acknowledgements}

We are grateful to Steve Awodey, Simon Huber, Andrew Pitts, Emily Riehl and Andrew Swan for helpful discussions and comments on earlier versions of the paper; and to Richard Garner and Fosco Loregian for pointing us to useful references.

This material is based on research sponsored by the Air Force Research Laboratory, under agreement number FA8655-13-1-3038, by a grant from the John Templeton Foundation, and by an EPSRC grant (EP/M01729X/1).

% \newpage

\section{Preliminaries}
\label{sec:fib-and-frob}


Throughout the paper, we work with a category $\cal{E}$ that we assume to be locally presentable and locally cartesian closed. In particular,
$\catE$ is complete and cocomplete and pullback preserves all limits and colimits, since it has both a left 
adjoint and  a right adjoint. For an arrow $f \co X \to Y$, we write $f^* \co \cal{E}_{/Y} \to \cal{E}_{/X}$ for the pullback along $f$, 
$f_{!} \co \cal{E}_{/X} \to \cal{E}_{/Y}$ for its left adjoint, which is given by composition with $f$, and $f_* \co \cal{E}_{/X} \to \cal{E}_{/Y}$
for its right adjoint, which we call \emph{pushforward} along $f$. We will write $\cal{E}^\to$ for the category of arrows of $\cal{E}$
and $\cal{E}^\to_{\mathrm{cart}}$ for the wide subcategory of~$\cal{E}^\to$ with cartesian squares as maps. 

Our definitions and results are illustrated  in two running examples. The first the is the category~$\SSet$ of simplicial sets, defined as the category of 
presheaves over the simplex category~$\Delta$. The second is the category $\CSet$ of cubical sets of~\cite{cohen-et-al:cubicaltt}. This is defined as the category of presheaves over 
the category $\Box$ with objects of the form $\Box^A$ with $|A| \in \mathbb{N}$ and morphisms~$\Box^A \to \Box^B$ given by functions from $B$ to the free de Morgan algebra on $A$. 
Thus,~$\CSet$ has symmetries, diagonals, connections, and involutions.\footnote{Note that the presence of symmetries precludes $\Box$ from being a Reedy category.} We stress that the only feature of the category $\Box$ relevant for the main results  are connections and symmetries.
Thus, our results apply equally to many other variations of cubical sets, excluding however those considered in~\cite{coquand-cubical-sets,huber-thesis}, which do not have connections.





Recall from~\cite{kamps-porter:homotopy} that a \emph{functorial cylinder} in $\calE$ is an endofunctor $\interval \otimes (-) \co \catE \to \catE$  equipped with natural transformations 
\[
\lcyl \otimes (-) \co   \Id_\catE \to \interval \otimes (-) \, , \quad \rcyl \otimes (-) \co \Id_\catE \to \interval \otimes (-) \, , 
\] 
called the left and right \emph{endpoint inclusions}, respectively. Such a functorial cylinder is said to have \emph{contractions} if $\lcyl \otimes (-)$ and $\rcyl \otimes (-)$ have a common retraction $\ccyl \otimes (-) \co \interval \otimes (-) \to \Id_\catE$ making the following diagrams commute:
\[
\xymatrix@C+2em{
  \Id_\catE
  \ar[r]^-{\lcyl \otimes (-)}
  \ar@{=}[dr]
&
  \interval \otimes (-)
  \ar[d]^(0.4){\ccyl \otimes (-)}
&
  \Id_\catE
  \ar[l]_-{\rcyl \otimes (-)}
  \ar@{=}[dl]
\\&
  \Id_\catE
\rlap{.}}
\]
Further, a functorial cylinder with contractions as above has \emph{connections} if, for $k \in \braces{0, 1}$, there is a natural transformations $c^k \otimes (-) \co \interval \otimes \interval \otimes (-) \to \interval \otimes (-)$ such that the diagrams
\begin{equation} \label{connections:0}
\begin{gathered}
\xymatrix@C+3em{
  \interval \otimes (-)
  \ar[r]^-{\kcyl \otimes \interval \otimes (-)}
  \ar[d]_{\ccyl \otimes (-)}
&
  \interval \otimes \interval \otimes (-)
  \ar[d]^{c^k \otimes (-)}
&
  \interval \otimes (-)
  \ar[l]_-{\interval \otimes \kcyl \otimes (-)}
  \ar[d]^{\ccyl \otimes (-)}
\\
  \Id_\catE
  \ar[r]_{\kcyl \otimes (-)}
&
  \interval \otimes (-)
&
  \Id_\catE
  \ar[l]^{\kcyl \otimes (-)}
}
\end{gathered}
\end{equation}
and
\begin{equation} \label{connections:1}
\begin{gathered}
\xymatrix@C+3em{
  \interval \otimes (-)
  \ar[r]^-{\kcylinv \otimes \interval \otimes (-)}
  \ar@{=}[dr]
&
  \interval \otimes \interval \otimes (-)
  \ar[d]^(0.4){c^k \otimes (-)}
&
  \interval \otimes (-)
  \ar[l]_-{\interval \otimes \kcylinv \otimes (-)}
  \ar@{=}[dl]
\\&
  \interval \otimes (-)
}
\end{gathered}
\end{equation}
commute. We adopt the convention of associating the tensor product notation to the right. If the functor $\interval\otimes (-) \co \catE \to \catE$ has a right adjoint
$\exp(I, -) \co \catE \to \catE$, then we obtain a \emph{functorial cocylinder} 
\[
(\exp(\interval, -), \exp(\bar{\delta}^0, -), \exp(\bar{\delta}^1, -)) \, ,
\]
\ie a functorial cylinder in the opposite category $\catE^{\op}$~\cite{kamps-porter:homotopy}. Contractions and (effective) connection carry over as well. In this
case, note also that $\interval\otimes (-) \co \catE \to \catE$ preserves colimits. 



Our main examples of functorial cylinders with  contractions and connections, discussed below, arise from a monoidal
structure on $\catE$, written $(\catE, \otimes, \top)$ and the presence of an \emph{interval object}, \ie an object~$\interval \in \catE$ 
equipped with maps~$\lcyl, \rcyl \co \top \to \interval$, called the left 
and right \emph{endpoint inclusions}, respectively. In such a situation, a functorial cylinder is given by tensoring with $\interval$. 
It is then immediate to isolate sufficient structure on the interval object that guarantees the presence of contractions and connections on the associated functorial cylinder: 
contractions can be given by a map $\ccyl \co \interval \to \top$, connections by maps for $k \in \braces{0, 1}$, $c^k \co \interval \otimes \interval \to \interval$,
all satisfying suitable axioms. Note that if the unit $\top$ of the monoidal structure is a terminal object, then an interval object  has contractions in a canonical way.





\begin{example}[Simplicial sets] \label{exa:cyl-in-sset}
In $\SSet$, considered as a monoidal category with respect to its cartesian structure, 
an interval object is given by $\Delta^1$ with endpoint inclusions~$\kcyl  \co \braces{k} \to \Delta^1$ given 
by special cases of horn inclusions, $\kcyl = h^1_k$, for $k \in \braces{ 0, 1}$. We have contractions, since the monoidal structure is cartesian,
and uniquely determined connections $c^k \co \Delta^1 \times \Delta^1 \to \Delta^1$, with $k \in \braces{0, 1}$, given on points by 
$c^0(x, y) = \min(x, y)$ and $c^1(x, y) = \max(x, y)$.
For this, note that~$\Delta^1$ and~$\Delta^1 \times \Delta^1$ are nerves of posets and that $\min$ and $\max$ are monotone.
\end{example}

\begin{example}[Cubical sets] \label{exa:cyl-in-cuset}
The category $\Box$ has a symmetric monoidal structure, which makes cubical sets into a symmetric monoidal closed category.
In $\CSet$, an interval object with contractions is given by $\Box^1$ with endpoint inclusions $\kcyl \co \braces{k} \to \Box^1$  given by the maps from a singleton set to the free de Morgan algebra 
on an empty set that pick the bottom element for $k = 0$ and the top element for $k =1$.
We have contractions since the unit of the monoidal structure  is a terminal object. 
Connection operations $c^k \co \Box^1 \otimes \Box^1 \iso \Box^2 \to \Box^1$ for $k \in \braces{0, 1}$ are 
 given by the maps from a singleton set to the free de Morgan algebra on a two-element set $\braces{x, y}$ that pick $x \wedge y$ and $x \vee y$, respectively).
\end{example}





Let us recall that the definition of the Frobenius property.
  
  \begin{definition} We say that a weak factorisation system $(L, R)$ has the \emph{Frobenius property} if the pullback of an $L$-map along  an $R$-map is again an $L$-map. 
  \end{definition} 
  
  As mentioned in the introduction, a model structure~$(\mathsf{Weq}, \Fib, \Cof)$ in which the cofibrations are the monomorphisms,
is right proper, \ie the pullback of a weak equivalence along a fibration is again a weak equivalence, 
if and only if the weak factorisation system~$(\TrivCof, \Fib)$ of trivial cofibrations and fibrations has the Frobenius property. In particular, the right properness of the model structure on $\SSet$ for Kan complexes is equivalent to the Frobenius property
for the weak factorisation system of trivial cofibrations and Kan fibrations, which asserts 
that the pullback of a trivial cofibration along a Kan fibration
is again a trivial cofibration. 

The next proposition is a straightforward observation, which follows by standard
properties of the interaction between lifting properties and adjoint functors. We include it since it is useful
in the proof of~\cref{thm:non-alg-frobenius}.



\begin{proposition} 
\label{thm:frobenius-equivalence}
Let $(L, R)$ be a weak factorisation system and $I \subseteq L$ be a class of maps such that~$I^\pitchfork = R$. Then the following are equivalent:
\begin{enumerate}[(i)] 
\item $(L,R)$ has the Frobenius property. 
\item For every $p \co Y \to X$, the pushforward $p^* \co \cal{E}_{/Y} \to \cal{E}_{/X}$ preserves $R$-maps.
\item The pullback of an $I$-map along 
an $R$-map is an $L$-map.
\end{enumerate}
\end{proposition} 



\section{Fibrations} 

Let us now assume that $\catE$ is equipped with a functorial cylinder $\interval\otimes (-) \co \catE \to \catE$ with contractions and connections, and that 
$\interval\otimes (-) \co \catE \to \catE$ has a right adjoint. 
The notion of a fibration that we introduced and study below will be relative not only to $\cal{E}$ and its functorial cylinder,
but also to a weak factorisation system on $\cal{E}$ satisfying some assumptions that we encapsulate in the notion of a \emph{suitable
weak factorisation system}, introduced in \cref{thm:suitable-wfs} below. In order to state that definition, we introduce some notation.  
For $k \in \{ 0, 1\}$ and a map $i \co A \to B$ in $\catE$, we define the \emph{Leibniz product} of~$\kcyl \otimes (-)$ and~$i$ to be the map 
\[
\kcyl \hatotimes i \co (\interval \otimes A) +_A B \to \interval \otimes B
\] 
determined uniquely by the following pushout diagram:
\begin{gather*}
\xymatrix@C=1.2cm{
  A
  \ar[r]^{i}
  \ar[d]_{\kcyl \otimes A}
&
  B
  \ar@/^2pc/[ddr]^{\kcyl \otimes B}
  \ar[d]
&\\
  \interval \otimes A
  \ar@/_1pc/[drr]_{\interval \otimes i}
  \ar[r]
&
  (\interval \otimes A) +_A B
  \ar[dr]^-{\delta^k \hatotimes i}
&\\&&
  \interval \otimes B
\rlap{.}}
\end{gather*}
We shall use Leibniz products with endpoint inclusions throughout the paper. In the next definition and onwards, we write
$(\Cof, \TrivFib)$ to denote a suitable weak factorisation system, and refer to maps in $\Cof$ as \emph{cofibrations} and maps in 
$\TrivFib$ as \emph{trivial fibrations}, since this helps to convey some of the basic intuition motivating our development. 


\begin{definition} \label{thm:suitable-wfs} A weak factorisation system $(\Cof, \TrivFib)$ is said to be \emph{suitable} if the following hold:
\begin{enumerate}[(i)]
 \item $(\Cof, \TrivFib)$  is cofibrantly generated,
\item Every object is cofibrant, \ie for every $X \in \cal{E}$, the unique map $\bot_X \co 0 \to X$ is a cofibration, 
 \item Cofibrations are closed under pullbacks, \ie for every pullback square
\[
\xymatrix{
B \ar[d]_j \ar[r]^q & A \ar[d]^i \\
Y \ar[r]_p & X }
\]
if $i \in \Cof$ then $j \in \Cof$, 
\item Cofibrations are closed under Leibniz product with the endpoint inclusions, \ie for every $i \in \Cof$ we 
have~$\kcyl \hatotimes i \in \Cof$.
\end{enumerate}
\end{definition}


\begin{example} \label{thm:generation-presheaf-cisinski} By a result of Cisinski, every presheaf category admits a suitable weak factorisation
system whose cofibrations are the mononomorphisms. 
\end{example} 


Let us now fix a suitable weak factorisation system $(\Cof,\TrivFib)$ in $\catE$.
We wish to define a new weak factorisation system, to be thought of as consisting of trivial cofibrations and fibrations. For this, we begin by defining the class of maps, called \emph{cylinder inclusions}, that determines the notion of a fibration. This class is defined by letting
\begin{equation}
\label{equ:cylinc}
\Cof_\otimes \defeq \{ \kcyl \hatotimes i \ | \ k \in \{ 0, 1 \} \, , \; i \in \Cof \}
\end{equation}

\begin{definition} \label{non-alg-fib} A \emph{fibration} is a map with the right lifting property with respect to all cylinder inclusions.
\end{definition} 


We then define a of \emph{trivial cofibration} to be a map with the left lifting property with respect to all fibrations.
We write $\Fib$ for the class of fibrations and $\TrivCof$ for the class of trivial cofibrations, so that 
$\Fib = \liftr{(\Cof_\otimes)}$ and $\TrivCof = \liftl{(\Fib)}$.  We illustrate these notions in our running examples.

\begin{example}[Simplicial sets] \label{thm:fib-is-kan}
In $\SSet$, let~$(\Cof, \TrivFib)$  be the weak factorisation system of monomorphisms and trivial Kan fibrations, which is cofibrantly generated
by the boundary inclusions. Then, a fibration in our sense  is the same thing as a Kan fibration in the usual sense. 
Indeed, a fibration in our sense is a map with the right lifting property with respect to all maps of the form
\[
h_k^1 \hatotimes i \co   (\Delta^1 \times A) \cup (\braces{k} \times B) \to  \Delta^1 \times B \, ,
\]
given by the Leibniz product of the endopoint inclusion $h_k^1 \co \braces{k} \to \Delta^1$ with a monomorphism $i \co A \to B$. The equivalence between this notion and the standard notion of a Kan fibration is well-known and follows from the fact that an higher dimensional horn $h_i^n$ is a retract of the Leibniz product $\kcyl \hatotimes h_i^n$~\cite[Chap.~IV, Sec.~2]{gabriel-zisman:calculus-of-fractions}.
\end{example}

\begin{example}[Cubical sets]
In $\CSet$, let $(\Cof, \TrivFib)$ be the weak factorisation system in which~$L$  consists of all monomorphisms.
A  fibration in $\CSet$ will be called a \emph{Kan fibration}.
To illustrate the relationship with the standard Kan filling condition for cubes, let $i^n \co \partial \Box^n \to \Box^n$ for $n \in \mathbb{N}$ denote the boundary inclusion for~$\Box^n$, which is given by the $n$-fold Leibniz tensor of~$[\lcyl, \rcyl] \co \braces{0, 1} \to \Box^1$.
%\footnote{Note that for our choice of cube category, which is not Reedy, this will not be a maximal non-trivial subobject.}
Then $\lcyl \hatotimes i^n$ and $\rcyl \hatotimes i^n$ are the inclusions $\sqcup_1^{1+n} \to \Box^{1+n}$ and $\sqcap_1^{1+n} \to \Box^{1+n}$, respectively.
Since the cube category has symmetries, a Kan fibration in $\CSet$  has the left lifting property also with respect to open box inclusions, which are counterparts in cubical sets of the horn inclusions in simplicial sets.
%\footnote{Without symmetries, one should consider two-sided Leibniz tensors such as $i^a \hatotimes \lcyl \hatotimes i^b \co \sqcup_a^{a+1+b} \to \Box^{a+1+b}$.}
The fibrant cubical types defined in~\cite{cohen-et-al:cubicaltt} arise by considering the  $L'$-fibrations where~$L'$ is the
subclass of $L$ spanned by monomorphisms\footnote{Or decidable monomorphisms, if working constructively} classified by their face lattice~$\Phi$, \ie those arising as a pullback of $\braces{\top} \to \Phi$, excluding for example diagonals in cubes.
%Since the presence of symmetries, diagonals, and connections precludes the cube category from being Reedy, this may turn out to be a %natural restriction. 
Note that  $L'$ is the left class of a suitable weak factorisation system and is thus covered by our development.
\end{example}

 
\begin{proposition} \label{thm:wfstimes} $(\TrivCof, \Fib)$ is weak factorisation system in $\cal{E}$.
\end{proposition}

\begin{proof} Let $I$ be a generating set for the suitable weak factorisation system $(\Cof, \TrivFib)$.
We  define $I_\otimes \defeq \{ \delta^k \hatotimes i \ | \ i \in I \}$. By the small object argument, 
there is a weak factorisation system $(L, R)$ cofibrantly generated by $I_\otimes$, where
$L$ is the saturation of $I_\otimes$ (\ie its closure under isomorphisms, composition, retracts, pushouts and transfinite composition) 
and $R = \liftr{(I_\otimes)}$. 
We claim that $R$ is precisely the class of
fibrations, \ie that $\liftr{I_\otimes} = \liftr{(\Cof_\otimes)}$. For this, it suffices to observe that, since~$\Cof$ is the saturation of~$I$, the class
$\Cof_\otimes$ and the set~$I_\otimes$ have the same saturation (since the Leibniz product $\kcyl \hatotimes (-)$ preserves all the operations involved in saturation), and therefore they have the same right orthogonal class.
\end{proof}

Note that, since $\Cof$ is  closed under Leibniz products with the endpoint inclusions, we have that~$\TrivCof \subseteq \Cof$, \ie
that every trivial cofibration is a cofibration. 
\begin{remark} \label{cisinski-remark}
We also relate the notion of a fibration introduced here with the notion of a naive fibration introduced by Cisinski~\cite{cisinski-asterisque}.
For this, let $m \otimes (-)  \co \Id_\catE + \Id_\catE \to \interval \otimes (-)$ be given by
\[
  m \otimes (-) \defeq [\lcyl \otimes (-), \rcyl \otimes (-)] \, .
\]
A \emph{naive Cisinski fibration} would be defined as a map with the right lifting property with respect to the class
\[
L_\otimes' \defeq \{ (m \hatotimes (-))^n ( \kcyl \hatotimes i ) \ | \ n \in \mathbb{N} \, , \;  i \in L \} 
\]
Informally, one could think of this set as the closure of $J$ under Leibniz product with $m$. 

Our reasons for working with~$L_\otimes$ instead of this class are twofold. First, if the functorial cylinder is induced by tensoring with an interval object in a symmetric monoidal category and the class $L$ is closed under Leibniz product with the boundary inclusion $[\lcyl, \rcyl] \co 1 + 1 \to \interval$, then by permuting the Leibniz product one sees that every fibration also has canonically the structure of a  naive Cisinski fibration.
This is the case in our main examples, where $L$ consists of all monomorphisms, the map $[\lcyl, \rcyl]$ is a monomorphism, and Leibniz product preserves monomorphisms. For our second point, we use some notions introduced in~\cref{sec:frobprop}.
If we assume generalized connections (in the sense of $(1+n+1)$-ary operations that behave as connections for fixed middle~$n$ arguments), and that $L$ is closed under Leibniz product with $[\lcyl, \rcyl]$ (for example, this is the case if $[\lcyl, \rcyl]$ is a monomorphism and the maps in~$L$ are the monomorphisms in a presheaf category), then the maps in $L_\otimes$ can still be shown to be strong homotopy equivalences. This gives rise to an inclusion $L_\otimes' \to L_\otimes \cap S$  (where $S$ is the class of strong homotopy equivalences that we define in \cref{sec:frobprop}). By \cref{thm:main-sheretract}, it then follows that every naive Cisinski fibration is a fibration.
\end{remark}


\section{The Frobenius property for fibrations}
\label{sec:frobprop}


Our aim in this section is to show that
the weak factorisation system $(\TrivCof, \Fib)$ of  \cref{thm:wfstimes} satisfies the Frobenius property. By the above and~\cref{thm:fib-is-kan}, the application of this result in $\SSet$ gives a new proof of the right properness of the model structure for Kan complexes. 
Our proof of the Frobenius property uses the notion of a \emph{strong homotopy equivalence}, which we introduce in~\cref{def:strhe}, and consists of three main steps. First, in~\cref{strong-h-equiv-as-section-non-alg}, we give a characterization of strong homotopy equivalences as cetain retracts. Secondly, in~\cref{thm:main-sheretract}, we show that the maps in $\Cof$ which are strong homotopy
equivalences are retracts of maps in $\Cof_\otimes$ and that every map in $\Cof_\otimes$ is both in $\Cof$ and a strong homotopy equivalence. This implies that a map is
a fibration if and only if it has the right lifting property with respect to the strong homotopy equivalences in $\Cof$. The third and final step of the proof, in~\cref{thm:non-alg-frobenius-she},  consists in showing that the pullback
of a strong homotopy equivalence along a fibration is again an strong homotopy equivalence, which easily implies 
the Frobenius condition for $(\TrivCof, \Fib)$.


Let $f, g \co X \to Y$ be maps in $\catE$.
Recall that a \emph{homotopy from $f$ to $g$}, denoted $\phi \co f \sim g$, is a morphism $\phi \co \interval \otimes X \to Y$ such that the following diagram commutes:
\begin{equation} \label{equ:homotopy}
\begin{gathered}
\xymatrix@C=1.2cm{
  X
  \ar[r]^-{\lcyl \otimes X}
  \ar[dr]_{f}
&
  \interval \otimes X
  \ar[d]^(0.4){\phi}
&
  X
  \ar[dl]^{g}
  \ar[l]_-{\rcyl \otimes X}
\\&
  Y
\rlap{.}}
\end{gathered}
\end{equation}
We say that a map $f \co X \to Y$ is called a \emph{left} (or {\emph{$0$-oriented}) \emph{homotopy equivalence} if there exist $g \co Y \to X$ and homotopies $\phi \co g \cc f \sim \id_X $, $\psi \co f \cc g \sim \id_Y$.
Dually, we have a \emph{right} (or {\emph{ $1$-oriented}) \emph{homotopy equivalence} if there exist $g \co Y \to X$ and homotopies  $\phi \co \id_X \sim g \cc f$, $\psi \co \id_Y \sim f \cc g$.  The notion of a strong
homotopy equivalence, defined below, is obtained by requiring an additional condition on the homotopies $\phi$ and $\psi$. 



\begin{definition} \label{def:strhe}
A map $f \co X \to Y$ is called a \emph{strong $k$-oriented homotopy equivalence} if there is a map $g$ and homotopies $\phi$ and $\psi$ making $f$ into a
$k$-oriented homotopy equivalence such that the following diagram commutes:
\[
\xymatrix{
  \interval \otimes X
  \ar[r]^{\interval \otimes f}
  \ar[d]_{\phi}
&
  \interval \otimes Y
  \ar[d]^{\psi}
\\
  X
  \ar[r]_{f}
&
  Y \, . 
}
\]
\end{definition}

For $k \in \braces{0, 1}$, we write $S_k$ for the class of strong $k$-oriented homotopy equivalences and let $S \defeq S_0 \cup S_1$.
Elements of $S$ will be called \emph{strong homotopy equivalences}. 

\begin{remark}  The components of the endpoint 
inclusion $\kcyl \otimes (-) \co \Id_\catE \to \interval \otimes (-)$
are strong $k$-oriented homotopy equivalences. In fact, they are 
strong $k$-oriented deformation retracts, \ie strong $k$-oriented homotopy equivalences for which 
the homotopy~$\phi$ is trivial, \ie  $\phi = \varepsilon \otimes X$, where $\varepsilon$ is the contraction of the functorial cylinder. 
To show this, one exploits crucially the assumption that the functorial cylinder has
connections. For $k = 0$, the retraction is given by~$\varepsilon \otimes X \co \interval \otimes X \to X$ and the homotopy 
$\psi \co (\lcyl \otimes X) \circ (\varepsilon \otimes X)  \sim 
1_{\interval \otimes X}$ is given by the connection~$c^0 \otimes X$. 
The axioms for the left and right endpoint of $\psi$ follow from the left sides of~\eqref{connections:0} and~\eqref{connections:1}, respectively.
The axiom for strength follows from the right part of~\eqref{connections:0}. 
\end{remark}




Our first step is to give an alternative characterization of strong $k$-oriented homotopy equivalences.
For this, observe that we have a commutative diagram 
\begin{equation}
\label{equ:thetak}
\begin{gathered}
\xymatrix@C+2em{
  X
  \ar[r]^-{\kcylinv \otimes X}
  \ar[d]_{f}
&
 \interval \otimes X 
 \ar[r]^-{\iota_1} 
 &
  (\interval \otimes X) +_X Y \ar[d]^{\kcyl \hatotimes f} 
  \\
  Y
  \ar[rr]_-{\kcylinv \otimes Y}
  & 
&
  \interval \otimes Y
}
\end{gathered}
\end{equation}
for $k \in \braces{0, 1}$. This diagram gives us an arrow~$\thetak \hatotimes f \co f \to \kcyl \hatotimes f$ in the category of maps 
in~$\catE$, which we use in the next lemma to provide the following characterization of strong homotopy equivalences as retractions. 

\begin{lemma} \label{strong-h-equiv-as-section-non-alg}
A map $f \co X \to Y$ is a strong $k$-oriented equivalence 
if and only if  the map $\thetak \hatotimes f  \co f \to \kcyl \hatotimes f$ exhibits $f$ as a retract of $\kcyl \hatotimes f$, \ie there are
dotted arrows as follows:
\[
\xymatrix@C+4em{
  X
  \ar[r]^-{\iota_1 \cc (\kcylinv \otimes X)}
  \ar[d]_f
&
  (\interval \otimes X) +_X Y
  \ar[d]^{\kcyl \hatotimes f}
  \ar@{.>}[r]
&
  X
  \ar[d]^f
\\
  Y
  \ar[r]_-{\kcylinv \otimes Y}
&
  \interval \otimes Y
  \ar@{.>}[r]
&
  Y
\rlap{,}}
\]
such that the two horizontal composites are identities.
\end{lemma}

\begin{proof}
First, by a standard diagram-chasing argument, giving the square on the right is equivalent to giving maps $\phi \co \interval \otimes X \to X$, $g \co Y \to X$, $\psi \co \interval \otimes Y \to Y$ such that the following diagrams commute:
\begin{align} \label{equ:first-three}
\begin{gathered}
\xymatrix{
  X
  \ar[r]^-{\kcyl \otimes X}
  \ar[d]_f
&
  \interval \otimes X
  \ar[d]^{\phi}
\\
  Y \ar[r]_-{g}
&
  X
\rlap{,}}
\end{gathered}
&&
\begin{gathered}
\xymatrix{
  Y
  \ar[r]^-g
  \ar[d]_{\kcyl \otimes Y}
&
  X
  \ar[d]^f
\\
  \interval \otimes Y
  \ar[r]_-{\psi}
&
  Y
\rlap{,}}
\end{gathered}
&&
\begin{gathered}
\xymatrix{
  \interval \otimes X
  \ar[r]^-\phi
  \ar[d]_{I \otimes f}
&
  X
  \ar[d]^{f}
\\
  \interval \otimes Y
  \ar[r]_-{\psi}
&
  Y
\rlap{.}}
\end{gathered}
\end{align}
Second, requiring that the two horizontal composites are identities means that the diagrams
\begin{align} \label{equ:second-two}
\begin{gathered}
\xymatrix@C+2em{
  X
  \ar[r]^-{\kcylinv \otimes X}
  \ar@{=}[dr]
&
  \interval \otimes X
  \ar[d]^\phi
\\&
  X
\rlap{,}}
\end{gathered}
&&
\begin{gathered}
\xymatrix@C+2em{
  Y
  \ar[r]^-{\kcylinv \otimes Y}
  \ar@{=}[dr]
&
  \interval \otimes Y
  \ar[d]^{\psi}
\\&
  Y
}
\end{gathered}
\end{align}
commute.
With reference to the diagram in~\eqref{equ:homotopy}, the equations in~\eqref{equ:first-three} provide endpoint~$k$ for~$\phi$ and~$\psi$, and strength, respectively; while the equations in~\eqref{equ:second-two} provide endpoints~$1-k$ for~$\phi$ and~$\psi$, respectively.
\end{proof}


Note that \cref{strong-h-equiv-as-section} implies that strong $k$-oriented homotopy equivalences are closed under retracts.
Indeed, let $f$ be a strong homotopy equivalence and $g$ be a retract of $f$. Then, $\thetak \hatotimes g$ is a retract of 
$\thetak \hatotimes f$, since $\thetak \hatotimes (-) \co \catE^\to \to \catE^\to$ is a functor and functors preserve retracts.
Since~$f$ is a strong homotopy equivalence, $\thetak \hatotimes f$ is a section. But then $\thetak \hatotimes g$ is a also
a section (since
sections are closed under retracts), and so $g$ is a strong $k$-oriented homotopy equivalence. 


\begin{remark} \label{thm:kcylf-is-she}
For every $f$, we have that $\kcyl \hatotimes f$ is a strong $k$-oriented homotopy equivalence. GIVE PROOF.
\end{remark} 

For our second step, we return to consider a suitable weak factorisation system $(\Cof, \TrivFib)$ and the induced weak factorisation system
$(\TrivCof, \Fib)$. we focus on the class $\Cof \cap S$, \ie the class of cofibrations that are strong homotopy equivalences. We begin by relating them to the cylinder inclusions, as defined in~\eqref{equ:cylinc}.

\begin{lemma}  \label{thm:main-sheretract} \hfill 
\begin{enumerate}[(i)] 
\item $\Cof \cap S \subseteq \TrivCof$,
\item $\Cof_\otimes \subseteq \Cof \cap S$. 
\end{enumerate}
\end{lemma}

\begin{proof} For part (i), let $f \in  \Cof \cap S$. Since $f \in S$, \cref{strong-h-equiv-as-section-non-alg} implies that $f$ is a retract of~$\kcyl \hatotimes f$, for some $k \in \braces{0 \, , 1 }$. But since $f \in \Cof$, we have $\kcyl \hatotimes f \in \Cof_\otimes$. But $\Cof_\otimes \subseteq \TrivCof$ and $\TrivCof$ is closed under retracts. For part~(ii), recall 
that, since cofibrations are closed under Leibniz product with~$\kcyl$ (condition (iv) in \cref{thm:suitable-wfs}), we have~$\Cof_\otimes  \subseteq \Cof$.
Then, observe that the assumption that the functorial cylinder has connections implies that, 
for every $f \co X \to Y$, the map $\kcyl \hatotimes f$ is a strong $k$-oriented homotopy equivalence, as observed 
in~\cref{thm:kcylf-is-she}.
\end{proof}


Note that, by \cref{thm:main-sheretract}, we have $\Fib = \liftr{(\Cof \cap S)}$, \ie a map is a fibration if and only if it has the right lifting property with respect to the cofibrations that are strong homotopy equivalences. Indeed, 
\[
  \Fib =  \liftr{\TrivCof}  \subseteq \liftr{(\Cof \cap S)} \, , \quad
  \liftr{(\Cof \cap S)}  \subseteq \liftr{(\Cof_\otimes)} = \Fib \rlap{.} 
\]
The next lemma is the third main step of the proof of the Frobenius property for $(\TrivCof, \Fib)$. 

\begin{lemma} 
\label{thm:non-alg-frobenius-she}
The pullback of a strong $k$-oriented homotopy equivalence along a fibration is again a strong $k$-oriented 
homotopy equivalence.
\end{lemma} 

\begin{proof} Consider a pullback diagram
\begin{equation}
\label{non-alg-strong-h-equiv-uniform-base-change:0}
\begin{gathered}
\xymatrix{
 B \ar[d]_j \ar[r]^q & A \ar[d]^i \\ 
Y \ar[r]_p & X }
\end{gathered}
\end{equation}
where $i  \in S_k$ and $p \in \Fib$. We wish to show that $j \in  S_k$. Since $i \in S_k$, $\thetak \hatotimes i \co i \to \kcyl \hatotimes i$ has a retraction $\rho \co \kcyl \hatotimes i \to i$.  We show that $\thetak \hatotimes j \co j \to \kcyl \hatotimes j$ has a retraction 
$\tau \co  \kcyl \hatotimes j \to j$. We define the retraction $\tau$ so that the diagram below, where $\sigma \co j \to i$ is 
the pullback in~\eqref{non-alg-strong-h-equiv-uniform-base-change:0}, commutes:
\[
\xymatrix@C+2em{
j 
  \ar[r]^-{\thetak \hatotimes j}
  \ar[d]_{\sigma}
&
  \kcyl \hatotimes j 
  \ar@{.>}[r]^-{\tau}
  \ar[d]_{\kcyl \hatotimes \sigma}
&
 j 
  \ar[d]^{\sigma}
\\
  i
  \ar[r]_-{\thetak \hatotimes i}
&
  \kcyl \hatotimes i
  \ar[r]_-{\rho}
&
  i
\rlap{.}}
\]
Since the pullback square $\sigma$ is a Cartesian arrow with respect to the codomain fibration, it suffices to solve this problem on codomains, again omitting horizontal composite identities:
\[
\xymatrix@C+2em{
  Y
  \ar[r]^-{\kcylinv \otimes Y}
  \ar[d]_{p}
&
  \interval \otimes Y
  \ar@{.>}[r]^-{\cod(\tau)}
  \ar[d]^{\interval \otimes p}
&
  Y
  \ar[d]^{p}
\\
  X
  \ar[r]_-{\kcylinv \otimes X}
&
  \interval \otimes X
  \ar[r]_-{\cod(\rho)}
&
  X
\rlap{.}}
\]
To find the dotted arrow is to construct a diagonal filler in the following square:
\[
\xymatrix@C+2em{
  Y
  \ar@{=}[rr]
  \ar[d]_{\kcylinv \otimes Y}
&&
  Y
  \ar[d]^{p}
\\
  \interval \otimes Y
  \ar[r]_-{I \otimes p}
  \ar@{.>}[urr]^(0.4){\cod(\tau)}
&
  \interval \otimes X
  \ar[r]_-{\cod(\rho)}
&
  X
\rlap{.}}
\]
But now $\kcyl \otimes X \iso 
\kcyl \hatotimes \bot_X$, where $\bot_X \co 0 \to X$ is the unique map from the initial object of $\catE$ into $X$, which is 
a cofibration by the assumption that $(\Cof, \TrivFib)$ is suitable (condition~(ii) of \cref{thm:suitable-wfs}). The required
filler then exists since $\kcyl \otimes X$
is isomorphic to a map in $\Cof_\otimes$ and~$p \in \Fib = \liftr{(\Cof_\otimes)}$.
\end{proof} 


\begin{theorem} \label{thm:non-alg-frobenius}
The weak factorisation system $(\TrivCof, \Fib)$ has the Frobenius property. 
\end{theorem}

\begin{proof} By~\cref{thm:frobenius-equivalence}, it suffices to show that, for a pullback of the form
\begin{equation*}
\begin{gathered}
\xymatrix{
 B \ar[d]_j \ar[r]^q & A \ar[d]^i \\ 
Y \ar[r]_p & X }
\end{gathered}
\end{equation*}
where $i  \in \TrivCof$ and  $p \in \Fib$, we have $j \in \TrivCof$. 
By part (ii) of~\cref{thm:main-sheretract}, $i \in \Cof \cap S$. By the closure of $\Cof$ under pullbacks (condition~(iii) of \cref{thm:suitable-wfs}) and~\cref{thm:non-alg-frobenius-she},
it follows that~$j \in \Cof \cap S$. But $\Cof \cap S \subseteq \TrivCof$ by part~(i) of~\cref{thm:main-sheretract} and 
the closure of $\TrivCof$ under retracts.
\end{proof}



When applied to the category of simplicial sets, \cref{thm:non-alg-frobenius} gives a new proof of the right properness of the Quillen model structure for Kan complexes. This proof avoids both the use of topological realization, which is used in~\cite[Theorem~13.1.13]{hirschhorn-model-localizations} to deduce the desired result from the right properness of the model structure on topological spaces in which the fibrations are the Serre fibrations. It also avoids the use of the theory of minimal fibrations, which is used in~\cite[Theorem~1.7.1]{joyal-tierney-notes} to establish the result working purely combinatorially.




% \newpage

\section{Categories of orthogonal maps}
\label{sec:ortf}

This section starts the second part of the paper, in which we are interested in generalizations of the results obtained earlier in the setting
of algebraic weak factorisation systems, in which one has maps that do not just satisfy orthogonality properties, but are rather equipped
with additional structure, providing diagonal fillers for diagrams. The second part of the paper follows closely the first one, but in this section
we will need to recall and establish some facts about the algebraic setting that are useful for our development. 

A key difference between the non-algebraic and the algebraic setting is that in the algebraic setting we do not to consider just classes of arrows in $\catE$, but rather categories~$\cal{I}$, to be thought of as an indexing category, and functors~$u \co \cal{I} \to \catE^\to$. In the following, we write $u_i \co A_i \to B_i$ for the effect of such a functor on~$i \in \cal{I}$.  Let us begin by recalling the following definition from~\cite{garner:small-object-argument}.

\begin{definition} \label{def:right-map}
Let $u \co \cal{I} \to \catE^\to$.
\begin{enumerate}[(i)]
\item A \emph{right $\cal{I}$-map} $(f, \phi) \co X \to Y$ consists of a map $f \co X \to Y$ in $\catE$ and a function~$\phi$ that assigns to each $i \in \cal{I}$ and commuting square
\[
\xymatrix@C=2cm{
  A_i
  \ar[r]^{s}
  \ar[d]_{u_i}
&
  X
  \ar[d]^f
\\
  B_i
  \ar[r]_{t}
&
  Y
}
\]
a diagonal filler $\phi(i,s, t) \co B_i \to X$, satisfying the following naturality condition: for every diagram of the form
\[
\xymatrix{
  A_i
  \ar[r]^{a}
  \ar[d]_{u_i}
&
  A_j
  \ar[r]^{s}
  \ar[d]_{u_j}
&
  X
  \ar[d]^f
\\
  B_i
  \ar[r]_{b}
&
  B_j
  \ar[r]_{t}
&
  Y
\rlap{,}}
\]
where the left square is the image of $\sigma \co i \to j$ in $\cal{I}$ under $u$, we have that
\[
  \phi(j, s, t) \cc b = \phi(i, s \cc a, t \cc b) \, .
\]
\item A \emph{right $\cal{I}$-map morphism} $\alpha \co (f, \phi) \to (f', \phi')$ is a square $\alpha \co f \to f'$ in~$\catE$ satisfying an evident compatibility condition with respect to the choices of diagonal fillers, which we omit.
\end{enumerate}
\end{definition}

For  $u \co \cal{I} \to \catE^\to$, we write $\liftr{\cal{I}}$ for the category of right $\cal{I}$-maps and their morphisms, and 
$\liftr{u} \co \liftr{\cal{I}} \to \catE^\to$ for the evident forgetful functor, which we call the \emph{right orthogonal} of $u$. Recall from~\cite{garner:small-object-argument} that the function mapping $u \co \cal{I} \to \catE^\to$ to its right orthogonal $\liftr{u} \co \liftr{\cal{I}} \to \catE^\to$ defines the action on objects of a functor of categories over $\catE^\to$,
\[
  \liftr{\brarghole} \co \CAT_{/\catE^\to}^{\op} \to \CAT_{/\catE^\to} \rlap{.}
\]
The action of this functor on maps, which we will use in~\cref{thm:orth-nat} below, is evident.
As shown in~\cite[Proposition~3.8]{garner:small-object-argument}, the orthogonality functors form an adjunction
\begin{equation} \label{garner-adjunction}
\begin{gathered}
\xymatrix@C+2em{
  \CAT_{/\catE^\to}
  \ar@<5pt>[r]^-{\liftl{\brarghole}}
  \ar@{}[r]|-{\bot}
&
  \CAT_{/\catE^\to}^{\op} \rlap{.}
  \ar@<5pt>[l]^-{\liftr{\brarghole}}
}
\end{gathered}
\end{equation}
Even if objects of $\CAT_{/\catE^\to}$ are pairs of the form $(\cal{I}, u)$, where $\cal{I}$ is a category $u \co
\cal{I} \to \catE^\to$ is a functor, in the following we shall often refer to them simply by their domains. A similar convention applies
to maps in $\CAT_{/\catE^\to}$. For example, we write the components of the unit and counit simply as 
\[
\eta_\cal{I} \co \cal{I} \to \liftl{(\liftr{\cal{I})}} \, \qquad 
\varepsilon_\cal{I} \co \cal{I} \to \liftr{(\liftl{\cal{I}})} \rlap{.}
\]






The rest of this section is devoted to establishing some facts regarding categories of orthogonal maps in general, describing the interplay between orthogonality functors and retract closure, slicing, adjunctions, Leibniz adjunctions and Kan extensions. Most of these facts are expected generalizations of well-known statements for classes of weakly orthogonal classes in the standard setting.
They are probably known to experts, but we could not find them in the literature and hence we include them since they are used in the remainder of the paper. We omit  most proofs, which are straightforward. 

\begin{proposition} \label{thm:orth-nat}
Consider a natural transformation between categories over $\catE^\to$ as below,
\[
\xymatrix{
  \cal{I}
  \rrtwocell_G^F{\sigma}
 \ar[dr]_{u}
&&
  \cal{J}
  \ar[dl]^{v}
\\&
  \catE^\to
\rlap{,}}
\]
satisfying in particular the condition that $v \sigma = \id_u$.
Then $\liftr{F} = \liftr{G}$ and $\liftl{F} = \liftl{G}$.
\end{proposition}


Given a functor $u \co \cal{I} \to \calE^\to$, we define its~\emph{retract closure} $\overline{u} \co \overline{\cal{I}} \to \catE^\to$ as follows.
An object of $\overline{\cal{I}}$ is a tuple~$(i, e, \sigma, \tau)$ consisting of an object $i \in \cal{I}$, an object $e \in \catE^\to$, and maps $\sigma \co e \rightarrow u_i$ $\rho \co u_i \rightarrow e$ in $\catE^\to$ which exhibit $e$ as a retract of $u_i$ in $\catE^\to$, \ie such that
\[
\xymatrix{
  e
  \ar[r]^{\sigma}
  \ar@{=}[dr]
&
  u_i
  \ar[d]^{\rho}
\\&
  e
\rlap{.}}
\]
A map $(f, \kappa) \co (i, e, \sigma, \tau) \to (i', e', \sigma', \tau')$ in $\overline{\cal{I}}$ consists of a map $f \co i \to i'$ in $\cal{I}$ and a map $\kappa \co e \rightarrow e'$ in $\catE^\to$ such that the following diagram in $\catE^\to$ commutes:
\[
\xymatrix{
  e
  \ar[r]^{\sigma}
  \ar[d]_{\kappa}
&
  u_i
  \ar[r]^{\rho}
  \ar[d]^{u_f}
&
  e
  \ar[d]^{\kappa}
\\
  e'
  \ar[r]_{\sigma'}
&
  u_{i'}
  \ar[r]_{\rho'}
&
  e'
\rlap{.}}
\]
The functor $\overline{u} \co \overline{\cal{I}} \to \catE^\to$ is then defined on objects by letting $\overline{u}(i, e, \sigma, \tau) \defeq e$, and on maps by letting $\overline{u}(f, \kappa) \defeq \kappa$.
The operation of retract closure gives a monad in an evident way. 

\begin{proposition} \label{retract-closure}
The orthogonality functors send the components of the unit and multiplication of the retract closure monad to natural isomorphisms, and so for every $u \co \cal{I} \to \catE^\to$, we have canonical isomorphisms of categories
\[
\begin{aligned}
  \liftr{(\overline{\cal{I}})} &\iso \liftr{\cal{I}}
\, ,&
  \liftr{(\overline{\overline{\cal{I}}})} &\iso \liftr{\overline{\cal{I}}}
\, ,&
  \liftl{(\overline{\cal{I}})} &\iso \liftl{\cal{I}}
\, ,&
 \liftl{(\overline{\overline{\cal{I}}})} &\iso \liftl{\overline{\cal{I}}}
\end{aligned}
\]
over $\catE^\to$.
\qed
\end{proposition}



Given a functor $u \co \cal{I} \to \catE^\to$ and $X \in \catE$, we define a slice category $\cal{I}_{/X}$ and functor $u_{/X} \co \cal{I}_{/X} \to \catE_{/X}^\to$ as follows.
The category $\cal{I}_{/X}$ has as objects pairs consisting of an object $i \in \cal{I}$ and a commutative triangle of the form
\[
\xymatrix@C-1em{
  A_i
  \ar[dr]
  \ar[rr]^{u_i}
&&
  B_i
  \ar[dl]
\\&
  X
\rlap{.}}
\]
The functor $u_{/X} \co \cal{I}_{/X} \to \catE_{/X}^\to$ sends such a pair to $u_i \co A_i \to B_i$, viewed as a morphism in~$\catE_{/X}$.
There is a also a \emph{coslice} category over $X$, described dually,  which we denote $u_{\backslash X} \co  \cal{I}_{\backslash X} 
\to \catE^\to$. With these definitions in place, the commutation between slicing and weak orthogonality functors in the setting of algebraic weak factorization systems can be stated as follows.

\begin{proposition} \label{pitchfork-slicing}
Let $u \co \cal{I} \to \catE^\to$ and $X \in \cal{E}$.
\begin{enumerate}[(i)]
\item The right orthogonality functor commutes with slicing, \ie we have $\liftr{(\cal{I}_{/X})} = (\liftr{\cal{I}})_{/X}$ 
as categories over $\catE^\to$.
\item The left orthogonality functor commutes with coslicing, \ie we have $\liftl{(\cal{I}_{\backslash X})} = (\liftl{\cal{I}})_{\backslash X}$
as categories over $\catE^\to$.
\qed
\end{enumerate}
\end{proposition}

In contrast to the right orthogonality functor, the left orthogonality functor does not commute with slicing in general.
However, it does under certain assumptions, as described in \cref{pitchfork-slicing-grothendieck} below. 
%Note that \cref{pitchfork-slicing-grothendieck} has an evident dual, which we do not state explicitly.

\begin{proposition} \label{pitchfork-slicing-grothendieck}
Let $u \co \cal{I} \to \cal{E}^\to$ and assume that
\[
\xymatrix@C-1em{
  \cal{I}
  \ar[rr]^{u}
  \ar[dr]_{\cod_{\cal{E}} \cc u}
&&
  \cal{E}^\to
  \ar[dl]^{\cod_{\cal{E}}}
\\&
  \cal{E}
}
\]
is a morphism of Grothendieck fibrations.
Then the left orthogonality functor commutes with slicing on $u$, \ie for $X \in \cal{E}$ we have
$\liftl{(\cal{I}_{/X})} = (\liftl{\cal{I}})_{/X}$.
\end{proposition}

\begin{proof}
First note that the composite $\cod_{\cal{E}_{/X}} \cc u_{/X}$ is also a Grothendieck fibration.
When constructing the category of left maps for $\cal{I}$ or $\cal{I}_{/X}$, an application of base change and naturality of diagonal fillers shows that it is sufficient to consider lifting problems with the bottom arrow an identity.
But for such lifting problems the order of slicing makes no difference.
\end{proof}



\begin{corollary} \label{pitchfork-slicing-monad}
\leavevmode
\begin{enumerate}[(i)]
\item The monad $\liftl{(\liftr{(-)})}$ commutes with slicing.
\item The monad $\liftr{(\liftl{(-)})}$ commutes with coslicing.
\end{enumerate}
\end{corollary}

\begin{proof}
Use \cref{pitchfork-slicing} and note, for the first statement, that categories of right maps satisfy the assumptions of \cref{pitchfork-slicing-grothendieck}.
The second statement follows dually.
\end{proof}

\begin{proposition}
The retract closure commutes with slicing and coslicing, in the sense that for every $u \co \cal{I} \to \catE^\to$ we have
$\overline{\cal{I}_{/X}} = \overline{\cal{I}}_{/X}$ as categories over $\catE^\to$.
\qed
\end{proposition}




\begin{proposition} \label{lift-of-adjunction}
Let u $u \co \cal{I} \to \catE^\to$ and $v \co \cal{J} \to \cal{F}^\to$.
For an adjunction $F \dashv G$ between $\cal{E}$ and $\cal{F}$, the following are equivalent:
\begin{enumerate}[(i)]
\item the functor $F \co \catE^\to \to \cal{F}^\to$ lifts to a functor $F \co \cal{I} \to \liftl{\cal{J}}$ making the following diagram commute:
\[
\xymatrix@C=1.2cm{
  \cal{I}
  \ar[r]^{F}
  \ar[d]_{u}
&
  \liftl{\cal{J}}
  \ar[d]^{\liftl{v}}
\\
  \catE^\to
  \ar[r]_-{F}
&
  \cal{F}^\to
\rlap{,}}
\]
\item the functor $G \co \cal{F}^\to \to \catE^\to$ lifts to a functor $G \co \cal{J} \to \liftr{\cal{I}}$ making the following diagram commute:
\[
\xymatrix@C=1.2cm{
  \cal{J}
    \ar[d]_{v}
\ar[r]^{G}
&
  \liftr{\cal{I}}
  \ar[d]^{\liftr{u}}
\\
  \cal{F}^\to
   \ar[r]_{G}
&
  \catE^\to
\rlap{.}  }
\]
\end{enumerate}
\qed
\end{proposition}

\begin{proposition} \label{lift-of-mates}
Let u $u \co \cal{I} \to \catE^\to$ and $v \co \cal{J} \to \cal{F}^\to$.
Let $F_1 \dashv G_1$ and $F_2 \dashv G_2$ be adjunctions between $\cal{E}$ and $\cal{F}$ satisfying the equivalent conditions of \cref{lift-of-adjunction}.
Let $m \co F_1 \to F_2$ and $n \co G_2 \to G_1$ be natural transformations forming mates.
Then the following are equivalent:
\begin{enumerate}[(i)]
\item
The natural transformation $m$ lifts as follows:
\[
\xymatrix@C=1.2cm{
  \cal{I}
  \rtwocell_{F_2}^{F_1}{m}
  \ar[d]_{u}
&
  \liftl{\cal{J}}
  \ar[d]^{\liftl{v}}
\\
  \catE^\to
  \rtwocell_{F_2}^{F_1}{m}
&
  \cal{F}^\to
\rlap{,}}
\]
\item
The natural transformation $n$ lifts as follows:
\[
\xymatrix@C=1.2cm{
  \cal{J}
  \rtwocell_{G_1}^{G_2}{n}
  \ar[d]_{u}
&
  \liftr{\cal{I}}
  \ar[d]^{\liftl{v}}
\\
  \cal{F}^\to
  \rtwocell_{G_1}^{G_2}{n}
&
  \cal{E}^\to
\rlap{.}}
\]
\end{enumerate}
\qed
\end{proposition}



We generalize \cref{lift-of-adjunction} to Leibniz adjunctions~\cite{riehl-verity:reedy}.
Let us fix bifunctors $F \co \cal{K} \times \catE \to \cal{F}$ and $G \co \cal{K}^{\op} \times \cal{F} \to \catE$ related pointwise, for $k \in \cal{K}$, by an adjunction:
\[
\xymatrix@C+1em{
  \catE
  \ar@<5pt>[r]^{F(k, \arghole)}
  \ar@{}[r]|{\bot}
&
  \cal{F}
  \ar@<5pt>[l]^{G(k, \arghole)}
\rlap{.}}
\]
Let $ \widehat{F} \co \cal{K}^\to \times \catE^\to \to \cal{F}^\to$, 
$ \widehat{G} \co (\cal{K}^{\op})^\to \times \cal{F}^\to \to \catE^\to$
denote the Leibniz constructions for~$F$ and~$G^{\op}$, using pullback instead of pushout for~$\widehat{G}$.
Here and below we assume that the categories under consideration have sufficient structure to carry out
the relevant Leibniz constructions. 

\begin{proposition} \label{lift-of-leibniz-adjunction}
Let $u \co \cal{I} \to \catE^\to$, $v \co \cal{J} \to \cal{F}^\to$ be functors.
Then the following are equivalent for $h \co X \to Y$ in $\cal{K}$:
\begin{enumerate}[(i)]
\item liftings $F' \co \cal{I} \to \liftl{\cal{J}}$ of $\widehat{F}(h, \arghole) \co \catE^\to \to \cal{F}^\to$ making the following diagram commute:
\[
\xymatrix@C=1.2cm{
  \cal{I}
  \ar[r]^{F'}
  \ar[d]_{u}
&
  \liftl{\cal{J}}
  \ar[d]^{\liftl{v}}
\\
  \catE^\to
  \ar[r]_-{\widehat{F}(h, \arghole)}
&
  \cal{F}^\to
\rlap{,}}
\]
\item liftings $G' \co \cal{J} \to \liftr{\cal{I}}$ of $\widehat{G}(h, \arghole) \co \cal{F}^\to \to \catE^\to$ making the following diagram commute:
\[
\xymatrix@C+2em{
  \cal{J}
  \ar[d]_{v}
  \ar[r]^{G'}
&
  \liftr{\cal{I}}
  \ar[d]^{\liftr{u}}
\\
  \cal{F}^\to
  \ar[r]_-{\widehat{G}(h, \arghole)}
&
  \catE^\to
\rlap{.}}
\]
\end{enumerate}
\end{proposition}



% \cref{lift-of-adjunction} is the special case of \cref{lift-of-leibniz-adjunction} where $\cal{K}$ is the terminal category.

\begin{remark} \label{pitchfork-leibniz-most-general-example} We will apply~\cref{lift-of-leibniz-adjunction} in \cref{sec:unif}
as follows. We will have categories~$\cal{E}$ and $\cal{F}$, and take $\cal{K}$ to be the category of adjunctions 
$U \dashv V$ with $U \co \catE \to \cal{F}$ and $V \co \cal{F} \to \catE$, with morphisms from $U_1 \dashv V_1$ to $U_2 \dashv V_2$ 
given by natural transformations $u \co U_1 \to U_2$ and $v \co V_2 \to V_1$ forming mates. We can then take 
$F \co \cal{K} \times \catE \to \cal{F}$ and~$G \co \cal{K}^{\op} \times \cal{F} \to \catE$ to be the 
application of left and right adjoint, respectively.
% Note that we have fully faithful forgetful functors $\cal{K} \to [\catE, \cal{F}]$ and $\cal{K} \to [\cal{F}, \catE]^{\op}$.
 \end{remark}


Next, we establish some facts about the interaction between orthogonality functors and Kan extensions along fully faithful functors. 

\begin{proposition} \label{kan-extension-closure}
Let $F \co \cal{I} \to \cal{J}$ be a fully faithful functor.
\begin{enumerate}[(i)]
\item Assume that the pointwise left Kan extension of $u \co \cal{I} \to \catE^\to$ along $F$ exists:
\[
\xymatrix@!C@C-1em{
  \cal{I}
  \ar[dr]_{u}
  \ar[rr]^{F}
&&
  \cal{J}
  \ar[dl]^{\Lan_F u}
\\&
  \catE^\to
\rlap{.}}
\]
Then the functor $\liftr{F} \co \liftr{\cal{J}} \to \liftr{\cal{I}}$, fitting in the diagram
\[
\xymatrix@!C@C-1em{
  \liftr{\cal{I}}
  \ar[dr]_{\liftr{u}}
&&
  \liftr{\cal{J}}
  \ar[ll]_{\liftr{F}}
  \ar[dl]^{\liftr{(\Lan_F u)}}
\\&
  \catE^\to
\rlap{,}}
\]
is an isomorphism.
\item Assume that the pointwise right Kan extension of $u \co \cal{I} \to \catE^\to$ along $F$ exists:
\[
\xymatrix@!C@C-1em{
  \cal{I}
  \ar[dr]_{u}
  \ar[rr]^{F}
&&
  \cal{J}
  \ar[dl]^{\Ran_F u}
\\&
  \catE^\to
\rlap{.}}
\]
Then the functor $\liftl{F} \co \liftl{\cal{J}} \to \liftl{\cal{I}}$, fitting in the diagram
\[
\xymatrix@!C@C-1em{
  \liftl{\cal{I}}
  \ar[dr]_{\liftl{u}}
&&
  \liftl{\cal{J}}
  \ar[ll]_{\liftl{F}}
  \ar[dl]^{\liftl{(\Ran_F u)}}
\\&
  \catE^\to
\rlap{,}}
\]
is an isomorphism.
\qed
\end{enumerate}
\end{proposition}



Finally, we prove two results in the special case that $\catE$ is a presheaf category. Recall that
we write $\catE_{\cart}^\to$ for the category of arrows and pullback squares in~$\cal{E}$.

\begin{lemma} \label{left-kan-extension-of-representables}
Let $\cal{J}$ be a full subcategory of $\catE_{\cart}^\to$ closed under base change to representables, \ie closed under pullbacks along morphisms with domain a representable presheaf.
Let $\cal{I}$ denote its restriction to arrows into representables.
\[
\xymatrix@C-1em{
  \cal{I}
  \ar[rr]
  \ar[dr]
&&
  \cal{J}
  \ar[dl]
\\&
  \catE^\to
}
\]
Then the inclusion $\cal{J} \to \catE^\to$ is the left Kan extension of $\cal{I} \to \catE^\to$ along $\cal{I} \to \cal{J}$.
\end{lemma}

\begin{proof}
Since $\catE^\to$ is cocomplete, we can verify the claim using the colimit formula for left Kan extensions.
All of the following will be functorial in an object $j \co A \to B$ of $\cal{J}$.
We consider the diagram in $\catE^\to$ indexed by pullback squares of the form
\[
\xymatrix@C=1.2cm{
  A'
  \ar[r]
  \ar[d]_{i}
  \pullback{dr}
&
  A
  \ar[d]^{j}
\\
  \yon(x)
  \ar[r]_-{b}
&
  B
}
\]
with $i \co A' \to \yon(x)$ in $\cal{I}$ and valued $i$.
Our goal is to show that its colimit is $j$.
Using the assumption that $\cal{J}$ is closed under base change to representables, the given diagram can be described equivalently as the the diagram indexed by maps $b \co \yon(x) \to B$ and valued $b^*(j)$.
The claim can then be restated as $\colim_{b \co \yon(x) \to B} b^*(j) \iso j$, which holds since pullback commutes with colimits in presheaf categories, and $\colim_{b \co \yon(x) \to B} \yon(x) \iso B$.
\end{proof}

\begin{proposition} \label{awfs-on-arrows-into-representables}
Let~$\cal{J}$ be a full subcategory of $\catE_{\cart}^\to$ closed under base change to representables.
Let $\cal{I}$ denote its restriction to arrows into representables,
\[
\xymatrix@C-1em{
  \cal{I}
  \ar[rr]
  \ar[dr]
&&
  \cal{J}
  \ar[dl]
\\&
  \catE^\to
\rlap{.}}
\]
Then $\liftr{\cal{I}} = \liftr{\cal{J}}$.
\end{proposition}

\begin{proof}
The result follows by combining part~(i) of \cref{kan-extension-closure} and \cref{left-kan-extension-of-representables}.
\end{proof}

% \newpage




\section{The functorial Frobenius condition}
\label{sec:frobc}


Recall from~\cite{garner:small-object-argument,grandis-tholen-nwfs} that an \emph{algebraic weak factorization} system~$(\LL, \RR)$ on a category $\cal{E}$ consist of a pair of functors $\LL, \RR \co \cal{E}^\to \to \cal{E}^\to$ providing factorizations~$f = \RR f \cc \LL f$ functorially in~$f \in \cal{E}^\to$ and an extension of the 
evident structure of (co)pointed endofunctors of $\LL$ and $\RR$ to that of a (co)monad, such that the canonical natural transformation $\LL \RR \to \RR \LL$ is a distributive law. 
We write $\pCoalg{\LL}$ and $\pAlg{\RR}$ for the categories of (co)monad algebras and $\pcoalg{\LL}$ and $\palg{\RR}$ for the categories of (co)pointed endofunctor (co)algebras over $\LL$ and $\RR$, respectively.
These come equipped with forgetful functors into $\cal{E}^\to$ and fulfill 
\[
\pcoalg{\LL} = \liftl{\pAlg{\RR}} \, , \quad \palg{\RR} = \liftr{\pCoalg{\LL}} \, .
\]
The an algebraic weak factorization system $(\LL, \RR)$ has an underlying ordinary weak factorisation system, in which the left (right) class 
consists of the maps that admit the structure making them into an element of $\pcoalg{\LL}$ (of $\palg{\RR}$, respectively). Recall also that an algebraic weak factorisation system 
$(\LL, \RR)$ is said to be \emph{algebraically-free} on $u \co \cal{I} \to \cal{E}^\to$ of arrows if $\pAlg{\RR} = \liftr{\cal{I}}$. By Garner's small object argument~\cite{garner:small-object-argument}, every $u \co \cal{I} \to \cal{E}^\to$ with $\cal{I}$ small determines an algebraic weak factorisation system $(\LL, \RR)$ that is algebraically-free on $\cal{I}$. We call such algebraic
weak factorization systems \emph{cofibrantly generated}.


We now state the analogue for a algebraic weak factorisation systems of the the Frobenius condition for a weak factorisation system.
This condition, called the functorial Frobenius condition, was stated for cloven weak factorisation systems in~\cite{garner:topological-simplicial},
but applies without change to algebraic weak factorisation systems. For a category $\cal{E}$, we write $\cal{E}^\to \times_\cal{E} \cal{E}^\to$
for the pullback of $\cod \co \cal{E}^\to \to \cal{E}$ with itself. This is the category of cospans in
$\cal{E}$, \ie pairs of arrows with common codomain. Pullback gives a functor
$P \co  \cal{E}^\to \times_{\cal{E}} \cal{E}^\to \to \cal{E}^\to$
sending a cospan $(g, h)$ to $h^*(g)$.

\begin{definition} \label{functorial-frobenius}
An algebraic weak factorization system $(\LL, \RR)$ satisfies the \emph{functorial Frobenius condition} if pullback along $\palg{\RR}$
preserves $\pcoalg{\LL}$, in the sense of a lift $\tilde{P}$ of $P$, as follows:
\[
\xymatrix@C+2em{
  \pcoalg{\LL} \times_{\cal{E}} \palg{\RR}
  \ar@{.>}[r]^{\tilde{P}}
  \ar[d]
&
  \pcoalg{\LL}
  \ar[d]
\\
  \cal{E}^\to \times_{\cal{E}} \cal{E}^\to
  \ar[r]_{P}
&
  \cal{E}^\to
\rlap{.}}
\]
\end{definition}

In this section we obtain an analogue of~\cref{thm:frobenius-equivalence} for algebraic weak factorisation systems, \ie 
a necessary and sufficient condition for a cofibrantly generated awfs  to satisfy the functorial Frobenius condition, expressed 
purely in terms of a generating category. In the next section, we apply this characterization  to prove that the algebraic weak 
factorization system~$(\mathsf{TrivCof}, \mathsf{Fib})$ of~\cref{thm:sset-cset-nwfs} has the functorial Frobenius property. 
In order to prove the desired characterization, it is convenient to work with a variant of the functorial Frobenius condition,
which we introduce in~\cref{uniform-frob-nice} below. 

For this, we  introduce some notation. Given $u \co \cal{I} \to \cal{E}^\to$, we write $\cal{I}_{/\cal{E}}^{\to}$ for the category with objects
given by an object $X \in \cal{E}$ and a map  $u_i \co A_i \to B_i$ in $\cal{E}_{/X}$. We write $s \co \cal{I}_{/\cal{E}}^{\to} \to \cal{E}$ for the first projection. Then, for $v \co \cal{J} \to \cal{E}^\to$, we can form a pullback
\[
\xymatrix{
 \cal{I}_{/\cal{E}} \times_{\cal{E}} \cal{J} \ar[r] \ar[d] & 
 \cal{J} \ar[d]^v \\
  \cal{I}_{/\cal{E}} \ar[r]_s &
  \cal{E}^\to}
  \]
When $u$ is the identity on $\cal{E}^\to$ and $v$ is the codomain functor $\cod \co \cal{E}^\to \to \cal{E}$, we obtain
\[
\xymatrix{
 \cal{E}_{/\cal{E}} \times_{\cal{E}} \cal{E}^\to \ar[r] \ar[d] & 
 \cal{E} \ar[d]^{\cod} \\
  \cal{E}^\to_{/\cal{E}} \ar[r]_s &
  \cal{E}^\to}
  \]
Pullback gives a functor $Q \co  \cal{E}^\to_{/\cal{E}} \times_{\cal{E}} \cal{E}^\to \to \cal{E}_{/\cal{E}}^\to$
sending $(f, h)$ to $h^*(f)$. 



\begin{definition} \label{uniform-frob-nice} Let $u \co \cal{I} \to \cal{E}^\to$, $v \co \cal{J} \to \catE^\to$
We say that $(u, v)$  satisfies the \emph{uniform Frobenius condition} if
pullback along $\cal{J}$-maps sends $\cal{I}$-maps to ${}^\pitchfork(\cal{I}^\pitchfork)$-maps, \ie the functor $Q$ admits a lift $\tilde{Q}$ as below:
\begin{equation} \label{uniform-frob-nice:lift}
\begin{gathered}
\xymatrix@C+2em{
  \cal{I}_{/\cal{E}} \times_{\cal{E}} \cal{J}
  \ar@{.>}[r]^{\tilde{Q}}
  \ar[d]_{u_{/\cal{E}} \times_{\cal{E}} v}
&
  \liftl{(\liftr{\cal{I}})}_{/\cal{E}}
  \ar[d]^{\liftl{(\liftr{u})}_{/\cal{E}}}
\\
  \cal{E}^\to_{/\cal{E}} \times_{\cal{E}} \cal{E}^\to
  \ar[r]_{Q}
&
  \cal{E}_{/\cal{E}}^\to
\rlap{.}}
\end{gathered}
\end{equation}
\end{definition}

As the next proposition shows, the uniform Frobenius condition of \cref{uniform-frob-nice} can be simplified under some assumptions.

\begin{proposition}
\label{uniform-frob-even-nicer}
Let $v \co \cal{J} \to \cal{E}^\to$ be a functor such that
\begin{equation} \label{uniform-frob:comprehension}
\begin{gathered}
\xymatrix@C-1em{
  \cal{J}
  \ar[rr]^{v}
  \ar[dr]_{\cod_{\cal{E}} \cc v}
&&
  \cal{E}^\to
  \ar[dl]^{\cod_{\cal{E}}}
\\&
  \cal{E}
}
\end{gathered}
\end{equation}
is a morphism of Grothendieck fibrations. Then the following are equivalent:
\begin{enumerate}[(i)] 
\item the pair $(u, v)$ satisfies the  uniform Frobenius condition,
\item there is a lift $\tilde{P}$ as below:
\begin{equation} \label{uniform-frob-even-nicer:lift}
\begin{gathered}
\xymatrix@C+2em{
  \cal{I} \times_{\cal{E}} \cal{J}
  \ar@{.>}[r]^{\tilde{P}}
  \ar[d]_{u \times_{\cal{E}} v}
&
  \liftl{(\liftr{\cal{I}})}
  \ar[d]^{\liftl{(\liftr{u})}}
\\
  \cal{E}^\to \times_{\cal{E}} \cal{E}^\to
  \ar[r]_{P}
&
  \cal{E}^\to
\rlap{.}}
\end{gathered}
\end{equation}
\end{enumerate}
\end{proposition}

\begin{proof}
We will show that giving lifts $\tilde{Q}$ and $\tilde{P}$ as in~\eqref{uniform-frob-nice:lift} and \eqref{uniform-frob-even-nicer:lift}, respectively,
is equivalent.
The situation \eqref{uniform-frob-nice:lift} a priori represents a more general scenario: we are pulling back an arrow $u_i \co A \to B$ with $i \in \cal{I}$ over an object $Y$ along a map $v_j \co X \to Y$ with $j \in \cal{J}$.
In contrast, in \eqref{uniform-frob-even-nicer:lift} we restrict to the case that the map $B \to Y$ is the identity.
It follows that any lift $\tilde{Q}$ induces a lift $\tilde{P}$.
However, under the stated assumption, the lift $\tilde{Q}$ can be reconstructed from the lift $\tilde{P}$ by first pulling back $v_j$ along $B \to Y$ to a map $v_{j'} \co B \to D$ with $j' \to j$ a map in $\cal{J}$ in the above situation and then setting $\tilde{Q}(i, j) \defeq \tilde{P}(i, j')$.
Furthermore, this choice of $\tilde{Q}$ is forced upon us (up to isomorphism) by functoriality of $Q$ and the fact that $\liftl{(\liftr{u})}$ reflects isomorphisms.
\end{proof}

We can now restate the functorial Frobenius condition of~\cref{functorial-frobenius} equivalently in terms of
the uniform Frobenius condition of~\cref{uniform-frob-nice}. For this, we apply \cref{uniform-frob-even-nicer}
when $\cal{J}$ is a category of $\RR$-algebras of an algebraic weak factorization system $(\LL, \RR)$. 

\begin{proposition}
Let $(\mathsf{L}, \mathsf{R})$ be an algebraic weak factorization system on $\cal{E}$. Then the following are
equivalent:
\begin{enumerate}[(i)] 
\item $(\LL, \RR)$ satisfies the functorial Frobenius condition,
\item $(\pcoalg{\LL}, \palg{\RR})$ satisfies the uniform Frobenius condition.
\end{enumerate}
\end{proposition}

\begin{proof} The claim follows combining \cref{uniform-frob-even-nicer} and the fact that, $\pcoalg{\LL}$ is a category of left maps,
there are functors over $\cal{E}^\to$ back and forth between $\pcoalg{\LL}$ and $\liftl{(\liftr{(\pcoalg{\LL})})}$
\end{proof}


The reformulation of the functorial Frobenius condition as a uniform Frobenius condition is that the latter admits an 
equivalent rephrasing in terms of pushforward, rather than pullback, functors, which will be essential to our characterization of algebraically-free
algebraic weak factorisation systems satisfying the functorial Frobenius condition. As a first step towards the pushforward formulation,
we decompose the uniform Frobenius condition into an object part and a morphism part. For this, recall that, for an arrow $f \co X \to Y$, we have an adjunction~$f_{!} \dashv f^*$ between composition and pullback. 
Given a functor $u \co \cal{I} \to \catE^\to$, it is immediate to check that the left composition functor lifts as follows:
\[
\xymatrix@C+1em{
  \cal{I}_{/X}
  \ar[r]^-{f_!}
  \ar[d]_{u_{/X}}
&
  \cal{I}_{/Y}
  \ar[d]^{u_{/Y}}
\\
  \calE_{/X}^\to
  \ar[r]_-{f_!}
&
  \calE_{/Y}^\to
\rlap{.}}
\]
By \cref{lift-of-adjunction}, the pullback functor $f^* \co \catE_{/Y} \to \catE_{/X}$ then lifts to slices of the right orthogonality categories,
\[
\xymatrix@C=1.5cm{
  \liftr{\cal{I}}_{/Y}
  \ar[d]_{{\liftr{u}}_{/Y}}
  \ar[r]^{f^*}
&
  \liftr{\cal{I}}_{/X}
  \ar[d]^{{\liftr{u}}_{/X}}
\\
  {\catE}_{/Y}^\to
  \ar[r]_{f^*}
&
  \catE_{/X}^\to
\rlap{.}}
\]






\begin{proposition}  \label{def:uniFrobcond}
 Let $u \co \cal{I} \to \catE^\to$ and $v \co \cal{J} \to \catE^\to$. Then $(u,v)$ satisfies the
uniform Frobenius property  if and only if the following hold:
\begin{enumerate}[(i)]
\item for every  $j \in \cal{J}$, pullback along $v_j \co C_j \to D_j$ lifts to a functor
\[
\xymatrix@C+3em{
  \cal{I}_{/Y}
  \ar[r]^-{\widetilde{v_j^*}}
  \ar[d]_-{u_{/D_j}}
&
  \liftl{(\liftr{\cal{I}})}_{/C_j}
  \ar[d]^-{\liftl{(\liftr{u})}_{/C_j}}
\\
  \catE_{/D_j}^\to \ar[r]_-{v_j^*}
&
  \catE_{/C_j}^\to
\rlap{.}}
\]
\item for every $\tau \co j \to k$ in $\cal{J}$, the square $v_\tau \co v_j \to v_k$ in 
\begin{equation} 
\label{def:beck-chevalley:0}
\begin{gathered}
\xymatrix@C+3em{
  C_j
  \ar[r]^{v_j}
  \ar[d]_{s}
&
  D_j
  \ar[d]^{t}
\\
  C_{k}
  \ar[r]_-{v_{k}}
&
  D_{k}
\rlap{,}}
\end{gathered}
\end{equation}
is such that the canonical natural transformation
\[
\xymatrix@C+3em{
  \catE^\to_{/D_j}
  \ar[d]_{t_!}
  \ar[r]^{v_j^*}
  \ar@{}[dr]|{\textstyle\Downarrow \rlap{$\labelstyle\phi$}}
&
  \catE^\to_{/C_j}
  \ar[d]^{s_!}
\\
  \catE^\to_{/D_k}
  \ar[r]_{v_k^*}
&
  \catE^\to_{/C_k}
}
\]
lifts to a natural transformation
\[
\xymatrix@C+3em{
  \cal{I}_{/D_j}
  \ar[r]^{\widetilde{v_j^*}}
  \ar[d]_{t_!}
  \ar@{}[dr]|{\textstyle\Downarrow \rlap{$\labelstyle\tilde{\phi}$}}
&
  \liftl{(\liftr{\cal{I}})}_{/C_j}
  \ar[d]^{s_!}
\\
  \cal{I}_{/D_k}
  \ar[r]_{\widetilde{v_k^*}}
&
  \liftl{(\liftr{\cal{I}})}_{/C_k}
\rlap{.}}
\]
\end{enumerate}
\end{proposition}

\begin{proof}
Since the functor $\liftl{(\liftr{u})}$ is faithful, a lift $\widetilde{v_j^*}$ consists just of a lift of the action of $v_j^*$ on objects that is coherent with respect to the action of $v_j^*$ on morphisms, in the sense that the action of $v_j^*$ on morphisms determines uniquely the action of its
lift on morphisms.
Coherence in morphisms in $\cal{I}_{/\cal{E}} \times_{\cal{E}} \cal{J}$ of the action on objects of $\tilde{v^*}$ separates into two parts:
\begin{enumerate}[(1)] 
\item coherence in morphisms in $\cal{I}_{/D_j}$ for fixed $v_j \co C_j \to D_j$ with $j \in \cal{J}$,
\item coherence in morphisms in $\cal{J}$.
\end{enumerate}
The action of $\tilde{v^*}$ on objects together with coherence (1) constitutes part (i). Coherence (2) constitutes the  part (ii).
\end{proof}


The statement of~\cref{def:uniFrobcond} makes it clear that the uniform Frobenius condition requires the specification of additional structure, rather than the mere satisfaction of a property.
Explicitly, it requires that $v_j^*(u_i) \co v_j^* X_i \to v_j^* Y_i$ is equipped with the structure of a left $\liftr{\cal{I}}$-map for $i \in \cal{I}$ and $u_i \co X_i \to Y_i$ over $D_j$, such that morphisms in $\cal{I}_{/D_j}$ induce morphisms of left $\liftr{\cal{I}}$-maps. However, at the level of squares, 
 the  condition in part (ii) of \cref{def:uniFrobcond} is just a property, not additional structure.
Explicitly, it requires the components of the natural transformation $\phi$ to be morphisms of left $\liftr{\cal{I}}$-maps.
Also note that if the commutative square~\eqref{def:beck-chevalley:0} is a pullback, then the canonical natural transformation~$\phi$ is an isomorphism by the usual Beck-Chevalley condition, and so is~$\tilde{\phi}$ since $\liftl{(\liftr{u})}$ reflects isomorphisms.
\cref{def:uniFrobcond} has an immediate dual, expressed in terms of pushforward rather than pullback, which we state next.


\begin{proposition} \label{lift-pushforward} \label{lift-pushforward-BC}
 Let $u \co \cal{I} \to \catE^\to$ and $v \co \cal{J} \to \catE^\to$. Then $(u, v)$ satisfies the
uniform Frobenius property if and only if the following hold:
\begin{enumerate}[(i)] 
\item for every  $j \in \cal{J}$, pushforward along $v_j \co C_j \to D_j$ lifts to a functor
\[
\xymatrix@C=1.5cm{
  {\liftr{\cal{I}}}_{/X}
  \ar[r]^{(v_j)_*}
  \ar[d]_{u_{/C_j}}
&
  {\liftr{\cal{I}}}_{/D_j}
  \ar[d]^{{\liftr{u}}_{/D_j}}
\\
  \catE_{/C_j}^\to
  \ar[r]_{(v_j)_*}
&
  \catE_{/D_j}^\to
\rlap{.}}
\]
\item for every $\tau \co j \to k$, the square $v_\tau \co v_j \to v_k$, as in~\eqref{def:beck-chevalley:0}, is such that the canonical natural transformation
\[
\xymatrix@C+2em{
  \catE^\to_{/C_k}
  \ar[r]^{(v_k)_*}
  \ar[d]_{s^*}
  \ar@{}[dr]|{\textstyle\Downarrow \rlap{$\labelstyle\psi$}}
&
  \catE^\to_{/D_k}
  \ar[d]^{t^*}
\\
  \catE^\to_{/C_j}
  \ar[r]_{(v_j)_*}
&
  \catE^\to_{/D_j}
}
\]
lifts to a natural transformation
\[
\xymatrix@C+2em{
  {\liftr{\cal{I}}}_{/C_k}
  \ar[r]^{(v_k)_*}
  \ar[d]_{s^*}
  \ar@{}[dr]|{\textstyle\Downarrow \rlap{$\labelstyle\tilde{\psi}$}}
&
  {\liftr{\cal{I}}}_{/D_k}
  \ar[d]^{t^*}
\\
  {\liftr{\cal{I}}}_{/C_j}
  \ar[r]_{(v_j)_*}
&
  {\liftr{\cal{I}}}_{/D_j}
\rlap{.}}
\]
\end{enumerate}
\end{proposition}

\begin{proof} For (i), apply \cref{lift-of-adjunction} to the adjunction $v_j^* \dashv (v_j)_*$ with $u = u_{/X}$ and $v = {\liftr{u}}_{/Y}$, using \cref{pitchfork-slicing-grothendieck} to permute slicing and the left orthogonality functor. For (ii), apply \cref{lift-of-mates} to  
the adjunctions $v_j^* s_! \dashv (v_j)_* s^*$ and $v_k^* t_! \dashv (v_k)_* t^*$ recalling that $\phi$ and $\psi$ are mates.
\end{proof}




We now give the desired characterization of the functorial Frobenius condition for algebraically-free algebraic weak factorization systems.




\begin{theorem} \label{thm:frobenius-comparison}
Let $(\LL, \RR)$ be a algebraic weak factorisation system, algebraically free on $u \co \cal{I} \to \cal{E}^\to$. Then the following
are equivalent:
\begin{enumerate}[(i)] 
\item  $(\LL, \RR)$ satisfies the functorial Frobenius condition,
\item $(\cal{I}, \liftr{\cal{I}})$ satisfies the uniform Frobenius condition.
\end{enumerate}
\end{theorem}

\begin{proof} First observe that there are functors over $\cal{E}^\to$ back and forth between $\pcoalg{\LL} = \liftl{(\liftr{\cal{I}})}$ and $\overline{\pCoalg{\LL}}$ (this is the case for (co)pointed endofunctor and (co)monad algebras over any (co)monad).
Applying $\liftr{(-)}$ and using \cref{retract-closure}, it follows that there are maps in $\CAT_{/\cal{E}^\to}$ back and forth between $\liftr{\cal{I}}$ and $\palg{\RR}$, and hence also between $\liftl{(\liftr{\cal{I}})}$ and $\pcoalg{\LL}$.

The functorial Frobenius condition for $(\LL, \RR)$ can then be restated equivalently as the existence of a lift
\begin{equation} 
\label{frobenius-comparison:0}
\begin{gathered}
\xymatrix@C+2em{
  \liftl{(\liftr{\cal{I}})} \times_{\cal{E}} \liftr{\cal{I}}
  \ar@{.>}[r]
  \ar[d]
&
  \liftl{(\liftr{\cal{I}})}
  \ar[d]
\\
  \cal{E}^\to \times_{\cal{E}} \cal{E}^\to
  \ar[r]_{P'}
&
  \cal{E}^\to
}
\end{gathered}
\end{equation}
By  \cref{uniform-frob-even-nicer}, the uniform Frobenius condition for $(\cal{I}, \liftr{\cal{I}})$, instead, requires a lift 
\begin{equation}
\label{equ:spellout}
\begin{gathered}
\xymatrix@C+2em{
  \cal{I} \times_{\cal{E}} \liftr{\cal{I}}
  \ar@{.>}[r]^{\tilde{P}}
  \ar[d] 
&
  \liftl{(\liftr{\cal{I}})}
  \ar[d]
\\
  \cal{E}^\to \times_{\cal{E}} \cal{E}^\to
  \ar[r]_{P}
&
  \cal{E}^\to
\rlap{.}}
\end{gathered}
\end{equation}
So we need to show that a lift as in~\eqref{frobenius-comparison:0} is equivalent to  one as in~\eqref{equ:spellout}. Note the subtle difference between the two diagrams:  we are concerned with pullbacks along maps coming from~$\liftl{(\liftr{\cal{I}})}$ in~\eqref{frobenius-comparison:0} and coming from $\cal{I}$ in~\eqref{equ:spellout}. 


In one direction, given a lift as in~\eqref{frobenius-comparison:0}, composition with the unit map from $\cal{I}$ to $\liftl{(\liftr{\cal{I}})}$ gives
a lift as in~\eqref{equ:spellout}. In the other direction, one first shows that \eqref{frobenius-comparison:0} implies a relativized version of itself, with arrows as in~\eqref{uniform-frob-nice:lift} and with $\liftl{(\liftr{\cal{I}})}$ instead of $\cal{I}$ on the left. We then use an analogue of \cref{lift-pushforward} to write it separately as two components, first lifts of pushforward
\[
\xymatrix@C=1.5cm{
  {\liftr{\cal{I}}}_{/X}
  \ar[r]^{f_*}
  \ar[d]_{u_{/X}}
&
  \liftr{(\liftl{({\liftr{\cal{I}}})})}_{/Y}
  \ar[d]^{{\liftr{u}}_{/Y}}
\\
  \catE_{/X}^\to
  \ar[r]_{f_*}
&
  \catE_{/Y}^\to
}
\]
along any fixed right $\cal{I}$-map $f \co X \to Y$, and second lifts of the canonical map $t^* g_* p \to f_* s^* p$ to a right $\liftl{(\liftr{\cal{I}})}$-map for $(s, t) \co f \to g$ a map between right $\cal{I}$-maps and $p$ a right $\cal{I}$-map.
These are provided by the uniform Frobenius condition in the form of part~(ii) of \cref{lift-pushforward} after postcomposing the lifts in there with the unit  $\liftr{\cal{I}} \to \liftr{(\liftl{(\liftr{\cal{I}})})}$ of the adjunction~\eqref{garner-adjunction}.
\end{proof}

We conclude the section with some results on the functorial Frobenius condition which will be used in \cref{sec:frocuf}.

\begin{proposition} \label{uniform-frobenius-functorial}
Consider categories $u_t \co \cal{I} \to \cal{E}^\to$ and $v_t \co \cal{J} \to \cal{E}^\to$, for $t \in \braces{1, 2}$, related as follows:
\begin{align*}
\xymatrix@!C@C-3em{
  \cal{I}_1
  \ar[rr]^{F}
  \ar[dr]_{u_1}
&&
  \liftl{(\liftr{\cal{I}_2})}
  \ar[dl]^{\liftl{(\liftr{u_2})}}
\\&
  \catE^\to
\rlap{,}}
&&
\xymatrix@C-1.5em{
  \cal{I}_2
  \ar[rr]^-{G}
  \ar[dr]_{u_2}
&&
  \cal{I}_1
  \ar[dl]^{u_1}
\\&
  \catE^\to
\rlap{,}}
&&
\xymatrix@C-1.5em{
  \cal{J}_2
  \ar[rr]^-{H}
  \ar[dr]_{v_2}
&&
  \cal{J}_1
  \ar[dl]^{v_1}
\\&
  \catE^\to
\rlap{.}}
\end{align*}
If $(u_1, v_1)$ satisfies the uniform Frobenius condition, then so does $(u_2, v_2)$.
\end{proposition}

\begin{proof}
We work with the formulation of the Frobenius condition in \cref{uniform-frob-nice}.
In the diagram $\cal{I}$, the premiss of the desired implication corresponds to a functor $Q_1$ making the inner square commute while the conclusion corresponds to a functor $Q_2$ making the lower trapezoid commute:
\[
\xymatrix{
&
  {\cal{I}_2}_{/\cal{E}} \times_{\cal{E}} \cal{J}_2
  \ar[rr]^{Q_1}
  \ar[dd]^(0.25){{u_1}_{/\cal{E}} \times_{\cal{E}} v_1}
&&
  \liftl{(\liftr{\cal{I}_2})}_{/\cal{E}}
  \ar[dr]
  \ar[dd]_(0.25){\liftl{(\liftr{u_1})}_{/\cal{E}}}
\\
  {\cal{I}_1}_{/\cal{E}} \times_{\cal{E}} \cal{J}_1
  \ar[ur]
  \ar[dr]_{{u_2}_{/\cal{E}} \times_{\cal{E}} v_2}
  \ar[rrrr]^{Q_2}
&&&&
  \liftl{(\liftr{\cal{I}_1})}_{/\cal{E}}
  \ar[dl]^{\liftl{(\liftr{u_2})}_{/\cal{E}}}
\\&
  \cal{E}^\to_{/\cal{E}} \times_{\cal{E}} \cal{E}^\to
  \ar[rr]_{P}
&&
  \cal{E}^\to_{/\cal{E}}
\rlap{.}}
\]
We may construct $Q_2$ from~$Q_1$ by pasting the inner square with the left triangle, given by $G$ and~$H$, and the right triangle, given by $F$ and the multiplication of the monad $\liftl{(\liftr{(-)})}$.
\end{proof}


\begin{remark}
In some applications, we need functoriality of the uniform Frobenius condition of~$\cal{J}$ with respect to~$\cal{I}$ in one of the parameters.
If only functoriality in $\cal{I}$ is desired, the map $H$ will default to an identity.
If only functoriality in $\cal{J}$ is desired, the maps $F$ and $G$ will default to a component of the unit of the monad $\liftl{(\liftr{(-)})}$ and an identity, respectively.
\end{remark}

\begin{proposition} \label{uniform-frobenius-product-u} Let $u_1 \co \cal{I}_1 \to \cal{E}^\to$, $u_2 \co \cal{I}_2 \to \cal{E}^\to$ and
$v \co \cal{J} \to \cal{E}^\to$. If $(u_1, v)$ and $(u_2, v)$ satisfy the uniform Frobenius property, then so does $(u, v)$, where 
$u \co \cal{I}_1 \times_{\cal{E}^\to} \cal{I}_2 \to \cal{E}^\to$ is obtained by the pullback of $u_1$ and $u_2$.
\end{proposition}

\begin{proof}
Use the universal property of products in the category $\CAT_{/\cal{E}^\to}$, noting that we have a canonical map
\[
\liftl{(\liftr{\cal{I}_1})} \times_{\cal{E}^\to} \liftl{(\liftr{\cal{I}_2})} \to \liftl{(\liftr{(\cal{I}_1 \times_{\cal{E}^\to} \cal{I}_2)})}
\]
over $\cal{E}^\to$ given by the adjunction~\eqref{garner-adjunction}.
\end{proof}




% \newpage

\section{Uniform fibrations}
\label{sec:unif}




\newcommand{\CC}{\mathsf{C}}
\newcommand{\TF}{\mathsf{F_t}}

 This section is devoted to introducing the notion of a uniform fibration, which is the algebraic counterpart of the notion of a fibration
introduced in~\cref{sec:fib-and-frob}. For this, we first introduce suitable algebraic weak factorisation systems, in complete analogy with
the way we defined suitable weak factorisation systems in~\cref{thm:suitable-wfs}. When introducing and discussing suitable algebraic
weak factorisation systems, we use the notation $(\CC, \TF)$ and let
\[
  \Cof \defeq \pcoalg{\CC} \, , \quad \TrivFib \defeq \palg{\TF}
  \]
We then refer to objects in $\Cof$ as (uniform) \emph{cofibrations} and objects in $\TrivFib$ as (uniform) \emph{trivial fibrations}.
Again, we hope that this helps to convey some of the basic intuition motivating our development.  

\begin{definition} \label{thm:suitable-awfs}
An algebraic weak factorisation system $(\CC, \TF)$ is said to be \emph{suitable} if the following conditions hold.
\begin{enumerate}[(i)]
\item $(\CC, \TF)$ is cofibrantly generated.
\item The functor $\bot \co \cal{E} \to \cal{E}^\to$ mapping $X \in \cal{E}$ to $\bot_X \co 0_{\cal{E}} \to X$ factors through
$\Cof$, as in the diagram
\[
\xymatrix@C+3em{
\catE \ar[r]^{\tilde{\bot}} \ar@/_1pc/[dr]_{\bot} & \Cof \ar[d] \\
  & \catE^\to}
\]
\item For every $f \co Y \to X$, pullback lifts to $\Cof$, 
\[
\xymatrix@C+3em{
  \Cof
  \ar@{.>}[r]^{\tilde{f^*}}
  \ar[d]_{u}
&
  \Cof
  \ar[d]^{u}
\\
  \catE^\to_{/Y}
  \ar[r]_{f^*}
&
  \catE^\to_{/X}
\rlap{.}}
\]
\item $\Cof$ is closed under Leibniz product with endpoint inclusions, in the sense that the Leibniz product functor
lifts as follows:
\[
\xymatrix@C+3em{
  \Cof 
  \ar@{.>}[r]^{\widetilde{\kcyl \hatotimes (-)}}
  \ar[d]_{u}
&
 \Cof
  \ar[d]^{u}
\\
  \catE^\to
  \ar[r]_{\kcyl \hatotimes (-)}
&
  \catE^\to
\rlap{.}}
\]
\end{enumerate}
\end{definition}



The main difference between the notion of a suitable weak factorisation system~(\cref{thm:suitable-wfs}) and that of a suitable algebraic
weak factorisation system (\cref{thm:suitable-awfs}) is that the conditions in the former express the closure of the class of left maps under certain functions, while the conditions in the latter express the closure of the category of left maps under certain functors. 

\begin{example} \label{unif-triv-fib-sset}
The categories $\SSet$ and $\CSet$ admit suitable algebraic weak factorisation systems $(\CC, \TF)$ that are algebraically-free on the
categories $\cal{M}$ of all monomorphisms and pullback squares, \ie such that $\TrivFib = \liftr{\cal{M}}$. The right maps of these algebraic weak factorisation systems may
be seen as a natural algebraic counterpart of trivial Kan fibrations and so they will be called \emph{uniform trivial Kan fibrations}. See \cref{rem-lift-suitable} and 
\cref{justify-sset-cset-examples} for details. 
\end{example} 


Let us now fix a suitable algebraic weak factorisation system $(\CC, \TF)$. 
We need some notation. For $u \co \cal{I} \to \catE^\to$ and $k \in \braces{0, 1}$, we define a functor $\kcyl \hatotimes u \co \cal{I} \to \catE^\to$ by letting
\[
  (\kcyl \hatotimes u)_i \defeq \kcyl \hatotimes u_i  \defeq \hateval(\kcyl, u_i) \, .
\]
We adopt similar conventions for other natural transformations (or maps of such) written using the tensor notation.
Now let $\cal{I}_\otimes \defeq \cal{I} + \cal{I}$ and define $u_\otimes \co \cal{I}_\otimes \to \catE^\to$ via the coproduct diagram
\begin{equation} \label{equ:u-tensor}
\begin{gathered}
\xymatrix@C+2em{
  \cal{I}
  \ar[r]^{\iota_0}
  \ar[dr]_-{\lcyl \hatotimes u}
&
  \cal{I}_\otimes
  \ar[d]^(.4){u_\otimes}
&
  \cal{I}
  \ar[dl]^-{\rcyl \hatotimes u}
  \ar[l]_{\iota_1}
\\&
  \catE^\to
\rlap{.}}
\end{gathered}
\end{equation}
Applying these definitions to $u \co \Cof \co \cal{E}^\to$, we obtain a functor $u_\otimes \co \Cof_\otimes \to \cal{E}^\to$.
The image of this functor are maps of the form $\kcyl \hatotimes i$, where $k \in \braces{ 0 \, , 1}$ and $i$ is a uniform
cofibration.


\begin{definition} \label{def:I-fibration} \hfill 
\begin{enumerate}[(i)]
\item A \emph{uniform fibration} is a right $\Cof_\otimes$-map.
\item A \emph{morphism of uniform fibrations} is a morphism of right $\Cof_\otimes$-maps.
\end{enumerate}
\end{definition}


We write $\Fib$ for the category of uniform fibrations and their morphisms, which is given by~$\Fib  \defeq \liftr{(\Cof_\otimes)}$. We now characterize uniform fibrations in terms of right $\Cof$-maps. For this, we use use the cocylinder structure on the right adjoint $\exp(\interval, -)$ to $\interval \otimes (-)$. We obtain a functor $\hatexp(\bar{\delta}^k, -)$, defined in terms of a pullback, dual to $\kcyl \hatotimes (-)$, which was defined in terms of a pushout, for~$k \in \braces{0, 1}$. These form an adjunction as follows:
\[
\xymatrix@C+4em{
  \catE^\to
  \ar@<1ex>[r]^{\kcyl \hatotimes (-)}
  \ar@{}[r]|{\bot}
&
  \catE^\to
  \ar@<1ex>[l]^{\hatexp(\bar{\delta}^k, -)}
\rlap{.}}
\]
The desired characterization of uniform fibration now follows from the general results in \cref{sec:ortf}.

\begin{proposition} \label{prod-exp-general}
For a map $p \co X \to Y$ in $\catE$, the following are equivalent:
\begin{enumerate}[(i)]
\item $p$ admits the structure of a uniform fibration,
\item $\hatexp(\bar{\delta}^0, p)$ and $\hatexp(\bar{\delta}^1, p)$ admit the structure of a uniform trivial fibrations.
\end{enumerate}
\end{proposition}

\begin{proof}
First recall that the right orthogonality functor is contravariant and part of the adjunction~\eqref{garner-adjunction}, hence sends coproducts to products of categories over $\catE^\to$.
The remainder of the claim follows from \cref{lift-of-leibniz-adjunction} as applied in \cref{pitchfork-leibniz-most-general-example} and the preceeding discussion.
\end{proof}



We show that uniform fibrations are the right maps of an algebraic weak factorization system, using Garner's small object argument~\cite{garner:small-object-argument} (see also~\cite[Proposition~16]{bourke-garner-I}). For this, it is sufficient to establish
that uniform fibrations can be defined equivalently by right uniform orthogonality with respect to a small category over $\catE$.




\newcommand{\TC}{\mathsf{C_t}}
\newcommand{\FF}{\mathsf{F}}



\begin{theorem} \label{thm:sset-cset-nwfs} The category $\cal{E}$ admits an algebraic weak factorization systems $(\TC, \FF)$ such that  the category of $\mathsf{F}$-maps is the category of uniform fibrations, \ie $\palg{\mathsf{F}} = \Fib$.
\end{theorem}

\begin{proof} Let $\cal{I}$ be a small generating category for $(\CC, \TF)$, which since 
$(\CC, \TF)$ is suitable (condition~(i) of \cref{thm:suitable-awfs}), and consider $\cal{I}_\otimes$ and $u_\otimes$, defined as in~\eqref{equ:u-tensor}.
By \cref{prod-exp-general}, there is an isomorphism $\Fib  = \liftr{(\cal{I}_\otimes)}$. The claim now follows 
from the smallness of  $\cal{I}_\otimes$, via Garner's small object argument.
\end{proof}


We let $\TrivCof \defeq \pcoalg{\TC}$ and refer to its elements as (uniform) \emph{trivial cofibrations}. 




\begin{example} \label{unif-fib-sset} If we start from the algebraic weak factorisation systems $(\CC, \TF)$ in~$\SSet$ and~$\CSet$ of \cref{unif-triv-fib-sset}
and apply~\cref{thm:sset-cset-nwfs}, we obtain algebraic weak factorisation systems $(\TC, \FF)$ whose right maps are algebraic counterparts of Kan fibrations,
and hence will be called \emph{uniform Kan fibrations}. In the case of $\CSet$, these are  exactly the uniform Kan fibrations considered in~\cite{cohen-et-al:cubicaltt}. 
Thus,  as a special case of~\cref{thm:sset-cset-nwfs}, we recover the existence in $\CSet$ of an algebraic weak factorisation system with uniform Kan fibrations as the right 
maps~\cite{cohen-et-al:cubicaltt,swan-awfs}.
\end{example} 




\section{The functorial Frobenius property for uniform fibrations}
\label{sec:frocuf}


We now continue to consider the fixed suitable algebraic weak factorisation system $(\CC, \TF)$ and the induced 
algebraic weak factorisation system $(\TC, \FF)$, in which the $\FF$-maps are exactly the
uniform fibrations. Our aim in this section is to show that $(\TC, \FF)$ satisfies the functorial
Frobenius property. 

For this, we follow a strategy analogous to the one we used in \cref{sec:frobprop}. In particular, we begin
by organizing strong $k$-oriented homotopy equivalences into a category $\cal{S}_k$. Its objects are
4-tuples $(f, g, \phi, \psi)$ consisting of an arrow 
$f \co A \to B$ together with data $g \co B \to A$, $\phi \co \interval \otimes A \to A$, $\psi \co \interval \otimes B \to B$ making $f$ into a strong $k$-oriented homotopy equivalence in the sense of~\cref{def:strhe}.  A morphism $m \co (f, g, \phi, \psi) \to (f', g', \phi', \psi')$ consists of maps $s \co A \to A', t \co B \to B'$ such that the following diagrams commute:
\begin{align*}
\xymatrix{
  A
  \ar[r]^{s}
  \ar[d]_{f}
&
  A'
  \ar[d]^{f'}
\\
  B
  \ar[r]_{t}
&
  B'
\rlap{,}}
&&
\xymatrix{
  B
  \ar[r]^{t}
  \ar[d]_{g}
&
  B'
  \ar[d]^{g'}
\\
  A
  \ar[r]_{s}
&
  A'
\rlap{,}}
&&
\xymatrix{
  \interval \otimes A
  \ar[d]_{\phi}
  \ar[r]^{\interval \otimes s}
&
  I \otimes A'
  \ar[d]^{\phi'}
\\
  A
  \ar[r]_{s}
&
  A'
\rlap{,}}
&&
\xymatrix{
  \interval \otimes B
  \ar[d]_{\psi}
  \ar[r]^{\interval \otimes t}
&
  I \otimes B'
  \ar[d]^{\psi'}
\\
  B
  \ar[r]_{t}
&
  B'
\rlap{.}}
\end{align*}
There is an obvious first projection functor $p_k \co \cal{S}_k \to \catE^\to$. \cref{strong-h-equiv-as-section} below extends
the logical equivalence of \cref{strong-h-equiv-as-section-non-alg}  to an isomorphism of categories. In its statement, we refer
to the maps~$\thetak \hatotimes f \co f \to \kcyl \hatotimes f$ in~\eqref{equ:thetak}. 




\begin{lemma} \label{strong-h-equiv-as-section}
For $k \in \braces{0, 1}$, the category $\cal{S}_k$ of strong $k$-oriented homotopy equivalences in $\catE$ can be described isomorphically as the category of arrows $f \in \catE^{\to}$ with a retraction $\rho$ of $\thetak \hatotimes f$.
In detail,
\begin{enumerate}[(i)]
\item objects are pairs $(f, \rho)$ consisting of $f \in \catE^\to$ and a retraction $\rho$ of $\thetak \hatotimes f$, as below:
\[
\xymatrix@C+2em{
  f
  \ar[r]^-{\thetak \hatotimes f}
  \ar@{=}[dr]
&
  \kcyl \hatotimes f \ar[d]^{\rho}
\\&
  f
\rlap{,}}
\]
\item morphisms $\tau \co (f, \rho) \to (f', \rho')$ are maps $\tau \co f \to f'$ such that the below diagram commutes:
\[
\xymatrix@C+2em{
  \kcyl \hatotimes f
  \ar[d]_-{\rho}
  \ar[r]^{\kcyl \hatotimes \tau}
&
 \kcyl \hatotimes f'
  \ar[d]^-{\rho'}
\\
  u_i
  \ar[r]_{\tau}
&
  u_{i'}
\rlap{.}}
\]
\end{enumerate}
\end{lemma}

\begin{proof} The object part of the correspondence is essentially \cref{strong-h-equiv-as-section-non-alg}. The morphism part is
straightforward. 
\end{proof}

\begin{remark} \label{thm:kcylf-is-she-alg}
For every $f$, we have that $\kcyl \hatotimes f$ is a strong $k$-oriented homotopy equivalence. GIVE PROOF.
\end{remark} 


We need some notation. Let $u \co \cal{I} \to \cal{E}^\to$. For $k \in \braces{0,1}$, we define $\cal{I}(\cal{S}_k)$ via the pullback
\begin{equation}
\label{rel-strong-h-equiv-def}
\begin{gathered}
\xymatrix{
 \cal{I}(\cal{S}_k) \ar[r] \ar[d] & \cal{I} \ar[d]^u \\
\cal{S}_k \ar[r]_{p_k} & \cal{E}^\to
}
\end{gathered}
\end{equation}
This means that an object of $\cal{I}(\cal{S}_k)$ consist of an object $i \in \cal{I}$ together with data making the map $u_i \co
A_i \to B_i$ into a 
strong $k$-oriented homotopy equivalence. The common composite of the arrows in the pullback above will be written
 $u(\cal{S}_k) \co \cal{I}(\cal{S}_k) \to \cal{E}^\to$. In particular, we have
\[
\Cof ( \cal{S}_k ) \to \cal{E}^\to
\]
The next lemma is the analogue of~\cref{thm:main-sheretract}. Define
\[
\Cof(\cal{S}) \defeq  \Cof(\cal{S}_0) + \Cof(\cal{S}_1)
\]


\begin{proposition} \label{thm:strong-hequiv} \hfill 
\begin{enumerate}[(i)]
\item   \label{thm:onedir} There is a functor
\[
\xymatrix@!C@C-2.5em{
\Cof(\cal{S})
  \ar[dr]
  \ar@<4pt>[rr]^F
&&
  \TrivCof
  \ar[dl]
\\&
  \catE^\to \, .
}
\]
\item  \label{thm:twodir} There is a functor
\[
\xymatrix@!C@C-2.5em{
    \Cof_\otimes
  \ar[dr]
  \ar[rr]^G
&&
\Cof(\cal{S})
  \ar[dl]
\\&
  \catE^\to
}
\]
\end{enumerate}
\end{proposition}

\begin{proof} For (i), we first fix $k \in \braces{0 \, , 1 }$ and show that there is a functor
\begin{equation}
\label{lem:from-strong-hequiv}
\begin{gathered}
\xymatrix@!C@C-2em{
 \Cof( \cal{S}_k)
  \ar[dr]_{u(\cal{S}_k)}
  \ar[rr]^{M_k}
&&
  \overline{\Cof}
  \ar[dl]^-{\overline{\kcyl \hatotimes u}}
\\&
  \catE^\to
\rlap{.}}
\end{gathered}
\end{equation}
We only describe the action of the functor $M_k$ on an object $(f, \rho)$, leaving the evident definition of the action on arrows to the reader.
Recall that $\rho$ is a retraction of $\thetak \hatotimes f \co f \to \kcyl \hatotimes f$, exhibiting~$f$ as a retract of~$\kcyl \hatotimes f$.
Thus, we may define $M_k(f, \rho) \defeq (f, f, \thetak \hatotimes f, \rho)$.
Observe that this definition makes the diagram for $M_k$ commute. The claim in (i) now follows by combining 
the cases $k = 0$ and $k = 1$ and composing with a component of the unit of the monad $\liftl{(\liftr{(-)})}$.
Note that the retract closure in~\eqref{lem:from-strong-hequiv} vanishes because of \cref{retract-closure}.


For (ii), we first show that there is a functor
\[
\xymatrix@!C@C-2em{
  \Cof
  \ar[dr]_{\kcyl \hatotimes u} \ar[rr]^{N_k}
&&
  \Cof(\cal{S}_k)
  \ar[dl]^{\cal{S}_k(u)}
\\&
   \catE^\to
\rlap{.}}
\]
For this, recall from~\eqref{rel-strong-h-equiv-def} that $S_k(\Cof)$ was defined as the product of $\cal{S}_k$ and $\Cof$  
in $\CAT_{/\cal{E}^\to}$. The map to the first factor is the identity, and the map to the second factor is given by the assumption that $\Cof$ is closed under Leibniz product with the endpoint inclusions, which is part of the hypotheses of a suitable algebraic weak factorisation system.
The claim in (ii) then follows by combining the cases $k = 0$ and $k = 1$.
\end{proof}



\begin{remark} \label{relating-strong-hequiv-and-uniform-fib}
We also have a pair of functors
\[
\xymatrix@!C@C-2.5em{
  \liftr{\Cof(\cal{S})}
  \ar[dr]
  \ar@<4pt>[rr]^{\liftr{(N_0 + N_1)}}
&&
   \Fib
  \ar[dl]
  \ar@<4pt>[ll]^{\liftr{(M_0 + M_1)}}
\\&
  \catE^\to
}
\]
relating uniform fibrations with right maps with respect to $\LL$-maps equipped with the structure of 
a strong homotopy equivalences. These are not, in general, inverses to each other in either way.
\end{remark}

We are now ready to show that the algebraic weak factorization system in which the right maps are the uniform fibrations (constructed in~\cref{thm:sset-cset-nwfs}) satisfies the functorial Frobenius property.
For this, we use our characterisation of the functorial Frobenius property in algebraically-free algebraic weak factorization systems
stated in \cref{thm:frobenius-comparison}. As a step, we show that that  strong homotopy equivalences and uniform fibrations satisfy the uniform Frobenius condition.






\begin{lemma} \label{technical} The pair $(\cal{S}, \Fib)$ satisfies the uniform Frobenius condition.
\end{lemma}




\begin{proof} The core of the argument is to obtain, for $k \in \braces{0, 1}$, a lift $\tilde{P}$ in 
\begin{equation} \label{strong-h-equiv-uniform-base-change:goal}
\begin{gathered}
\xymatrix@C+2em{
  \cal{S}_k \times_{\cal{E}} \Fib
  \ar@{.>}[r]^{\tilde{P}}
  \ar[d]
&
  \cal{S}_k
  \ar[d]
\\
  \cal{E}^\to \times_{\cal{E}} \cal{E}^\to
  \ar[r]_{P}
&
  \cal{E}^\to
\rlap{.}}
\end{gathered}
\end{equation}
where $P \co \cal{E}^\to \times_{\cal{E}} \cal{E}^\to \to \cal{E}^\to$ be the pullback functor, sending a cospan $(g, h)$ to $h^* g$. Indeed,
composing $\tilde{P}$ and a component of the unit of the monad $\liftl{(\liftr{(-)})}$ will allow us to apply \cref{uniform-frob-even-nicer} and
establish the required uniform Frobenius property. 

So it remains to show how to obtain the lift in in \eqref{strong-h-equiv-uniform-base-change:goal}. Its action on objects is described
in the proof of~\cref{thm:non-alg-frobenius-she}. For the lift of the action on morphisms, suppose we are given a map $j \to j'$ in $\Fib$ forming a square
\[
\xymatrix{
  X
  \ar[r]^{v_j}
  \ar[d]_{s}
&
  Y
  \ar[d]^{t}
\\
  X'
  \ar[r]_{v_{j'}}
&
  Y'
}
\]
and a map $(g, \rho) \to (g', \rho')$ in $\cal{S}_k$ with $\tau \co g \to g'$ forming a square
\[
\xymatrix{
  B
  \ar[r]
  \ar[d]_{g}
&
  B'
  \ar[d]^{g'}
\\
  Y
  \ar[r]_-{t}
&
  Y'
}
\]
such that $\tau$ commutes with the retractions $\rho$ and $\rho'$ as follows:
\[
\xymatrix{
  \kcyl \hatotimes g
  \ar[r]^-{\rho}
  \ar[d]_{\kcyl \hatotimes \tau}
&
  g
  \ar[d]^{\tau}
\\
  \kcyl \hatotimes g'
  \ar[r]_-{\rho'}
&
  g'
\rlap{.}}
\]
Let $(\bar{g}, \bar{\rho})$ and $(\bar{g}', \bar{\rho}')$ denote the respective action of $P$ on the objects $(g, \rho, j)$ and $(g', \rho', j')$ as constructed in the first part of the proof.
Recall that this includes pullback squares $\sigma \co \bar{g} \to g$ and $\sigma' \co \bar{g}' \to g'$ with bottom side $v_j \co X \to Y$ and $v_{j'} \co X' \to Y'$, respectively, as in~\eqref{non-alg-strong-h-equiv-uniform-base-change:0}.
The square $\tau \co g \to g'$ pulls back to a square $\bar{\tau} \co \bar{g} \to \bar{g'}$ with bottom side $v_{j'}$.
We want to show that $\bar{\tau}$ in addition forms a morphism of strong $k$-oriented homotopy equivalences from $(\bar{g}, \bar{\rho})$ to $(\bar{g}', \bar{\rho}')$.
For this, we have to verify commutativity of the following diagram:
\[
\xymatrix{
  \kcyl \hatotimes \bar{g}'
  \ar[r]^-{\bar{\rho}}
  \ar[d]_{\kcyl \hatotimes \bar{\tau}}
&
  \bar{g}
  \ar[d]^{\bar{\tau}}
\\
  \kcyl \hatotimes \bar{g}'
  \ar[r]_-{\bar{\rho}'}
&
  \bar{g}'
\rlap{.}}
\]
Recall the construction of $\bar{\rho}$ and $\bar{\rho}'$, omitting horizontal composite identities for readability:
\[
\xymatrix@!C{
  \bar{g}
  \ar[rr]^-{\thetak \hatotimes \bar{g}}
  \ar[dd]_{\sigma}
  \ar[dr]^{\bar{\tau}}
&&
  \kcyl \hatotimes \bar{g}
  \ar@{.>}[rr]^-{\bar{\rho}}
  \ar[dd]^(0.3){\kcyl \hatotimes \sigma}|!{[dl];[dr]}{\hole}
  \ar[dr]^{\kcyl \hatotimes \bar{\tau}}
&&
  \bar{g}
  \ar[dd]^(0.3){\sigma}|!{[dl];[dr]}{\hole}
  \ar[dr]^{\bar{\tau}}
\\&
  \bar{g}'
  \ar[rr]^-(0.3){\thetak \hatotimes \bar{g}'}
  \ar[dd]_(0.3){\sigma'}
&&
  \kcyl \hatotimes \bar{g}'
  \ar@{.>}[rr]^-(0.3){\bar{\rho}'}
  \ar[dd]^(0.3){\kcyl \hatotimes \sigma'}
&&
  \bar{g}'
  \ar[dd]^{\sigma'}
\\
  g
  \ar[rr]^-(0.25){\thetak \hatotimes g}|!{[ur];[dr]}{\hole}
  \ar[dr]^{\tau}
&&
  \kcyl \hatotimes g
  \ar[rr]^-(0.3){\rho}|!{[ur];[dr]}{\hole}
  \ar[dr]^{\kcyl \hatotimes \tau}
&&
  g
  \ar[dr]^{\tau}
\\&
  g'
  \ar[rr]^-{\thetak \hatotimes g'}
&&
  \kcyl \hatotimes g
  \ar[rr]^-{\rho'}
&&
  g
\rlap{.}}
\]
Our goal is to show that the top right square commutes.
Since that square commutes after composing it with the pullback square $\sigma'$, it suffices to show that the square commutes when projected to codomains, again omitting horizontal composite identities:
\[
\xymatrix@C+2em{
  X
  \ar[rr]^-{\kcylinv \otimes X}
  \ar[dd]_{v_j}
  \ar[dr]^{s}
&&
  \interval \otimes X
  \ar@{.>}[rr]^-{\cod(\bar{\rho})}
  \ar[dd]^(0.35){\interval \otimes v_j}|!{[dl];[dr]}{\hole}
  \ar[dr]^{\interval \otimes s}
&&
  X
  \ar[dd]^(0.35){v_j}|!{[dl];[dr]}{\hole}
  \ar[dr]^{s}
\\&
  X'
  \ar[rr]^-(0.3){\kcylinv \otimes X'}
  \ar[dd]^(0.3){v_{j'}}
&&
  \interval \otimes X'
  \ar@{.>}[rr]^-(0.3){\cod(\bar{\rho}')}
  \ar[dd]^(0.3){\interval \otimes v_{j'}}
&&
  X'
  \ar[dd]^{v_{j'}}
\\
  Y
  \ar[rr]^-(0.25){\kcylinv \otimes Y}|!{[ur];[dr]}{\hole}
  \ar[dr]^{t}
&&
  \interval \otimes Y
  \ar[rr]^-(0.3){\cod(\rho)}|!{[ur];[dr]}{\hole}
  \ar[dr]^{\interval \otimes t}
&&
  Y
  \ar[dr]^{t}
\\&
  Y'
  \ar[rr]^-{\kcylinv \otimes Y'}
&&
  \interval \otimes Y'
  \ar[rr]^-{\cod(\rho')}
&&
  Y'
\rlap{.}}
\]
But this follows from coherence of lifts in the following morphism of lifting problems:
\[
\xymatrix@C+1em{
  X
  \ar@{=}[rrrr]
  \ar[dd]_{\kcylinv \otimes X}
  \ar[dr]^{s}
&&&&
  X
  \ar[dd]_(0.3){v_j}|!{[dlll];[dr]}{\hole}|!{[dddlll];[dr]}{\hole}
  \ar[dr]^{s}
\\&
  X'
  \ar@{=}[rrrr]
  \ar[dd]^(0.66){\kcylinv \otimes X'}
&&&&
  X'
  \ar[dd]^{v_{j'}}
\\
  \interval \otimes X
  \ar[rr]^(0.7){I \otimes v_j}|!{[ur];[dr]}{\hole}
  \ar@{.>}[uurrrr]^(0.67){\cod(\bar{\rho})}|!{[ur];[dr]}{\hole}|!{[ur];[urrrrr]}{\hole} % bug: doesn't work?
  \ar[dr]_{I \otimes s}
&&
  \interval \otimes Y
  \ar[rr]_(0.7){\cod(\rho)}|!{[dl];[urrr]}{\hole}
  \ar[dr]^(0.7){\interval \otimes t}|!{[dl];[urrr]}{\hole}
&&
  Y
  \ar[dr]^{t}
\\&
  \interval \otimes X'
  \ar[rr]_{I \otimes v_{j'}}
  \ar@{.>}[uurrrr]^(0.67){\cod(\bar{\rho}')}
&&
  \interval \otimes Y'
  \ar[rr]_{\cod(\rho')}
&&
  Y'
\rlap{.}}
\]
Here, the left and right faces form morphisms in $\TrivCof$ and $\Fib$, respectively, making the lifts cohere as needed.

By combining the cases $k = 0$ and $k = 1$, we obtain that $\cal{S}_0 + \cal{S}_1$ and uniform fibrations satisfy the uniform Frobenius
condition. The claim now follows by \cref{uniform-frobenius-functorial}.
\end{proof}





\begin{theorem} \label{uniform-fibrations-uniform-frobenius}
The algebraic weak factorisation system $(\mathsf{C_t}, \mathsf{F})$  satisfies the functorial Frobenius condition.
\end{theorem}


\begin{proof} First, we show that $(  \Fib, \Cof(\cal{S}))$
 satisfies the uniform Frobenius condition. Recall  that~$\Cof(\cal{S}_k)$ is the pullback of $u \co \Cof \to \cal{E}$ and~$p \co \cal{S}_k  \to \cal{E}$.
Using \cref{uniform-frobenius-product-u}, we deduce the required claim from \cref{technical} and hypothesis (iii) in the
definition of a suitable algebraic weak factorization system (\cref{thm:suitable-awfs}). 

The fact that uniform fibrations and strong homotopy equivalences relative to $\Cof$ satisfy the uniform Frobenius condition 
implies the desired claim, by applying functoriality of uniform Frobenius in the form of \cref{uniform-frobenius-functorial} with~$F$ 
the functor in part~(\ref{thm:onedir}) of~\cref{thm:strong-hequiv} and~$G$ the functor in part~(\ref{thm:twodir}) of ~\cref{thm:strong-hequiv}. 
\end{proof}


As special cases, we obtain the pushforward versions of the Frobenius and Beck-Chevalley condition for uniform fibrations.
First, pushforward lifts to slices of the category of uniform $\cal{I}$-fibrations.

\begin{corollary} \label{uniform-fibrations-frobenius-pushforward} \label{uniform-fibrations-BC-pushforward}
\hfill
\begin{enumerate}[(i)] 
\item For every uniform fibration
$p \co X \to Y$, pushforward along $p$ lifts to a functor
\[
\xymatrix@C+2em{
  \Fib_{/X}
  \ar[r]^{p_*}
  \ar[d]
&
  \Fib_{/Y}
  \ar[d]
\\
  \catE_{/X}^\to
  \ar[r]_{p_*}
&
  \catE_{/Y}^\to
\rlap{.}}
\]
\item For every map of uniform $\cal{I}$-fibrations $(s, t) \co p \to q$ , where $p \co X \to Y$ and $q \co U \to V$, the canonical natural transformation $\psi \co t^* q_* \to p_* s^*$ lifts to a natural transformation
\[
\xymatrix@C+2em{
  \Fib_{/U}
  \ar[r]^{q_*}
  \ar[d]_{s^*}
  \ar@{}[dr]|{\textstyle\Downarrow \rlap{$\labelstyle \psi'$}}
&
  \Fib_{/V}
  \ar[d]^{t^*}
\\
  \Fib_{/X}
  \ar[r]_{p_*}
&
  \Fib_{/Y}
\rlap{.}}
\]
If $(s, t) \co p \to q$ forms a pullback square, then $\psi'$ is a natural isomorphism.
\end{enumerate}
\end{corollary}


\begin{proof}
For (i), the claim follows from~\cref{uniform-fibrations-uniform-frobenius}. For (ii), the claim follows from \cref{lift-pushforward-BC}.
and \cref{uniform-fibrations-uniform-frobenius}.
\end{proof}


\begin{example} The application of \cref{uniform-fibrations-frobenius-pushforward} in $\SSet$
and $\CSet$ shows that pushforward along a uniform Kan fibration preserves uniform Kan fibrations. 
Since exponentiation is a special case of pushforward, this result shows also that the exponential of two uniform Kan complexes (defined in the evident way) is again a uniform Kan complex.
In fact, as usual, only the base needs to be assumed uniform Kan. This can be seen as follows.
Let $p \co Y \to X$ be a uniform Kan fibration and $i \co A \to B$ be a monomorphism.
Recall that~$\cal{M}(\cal{S}_k)$ is closed under Leibniz product with~$i$, for $k \in \braces{0, 1}$ (it is true separately for $\cal{M}$ and $\cal{S}_k$).
By \cref{lift-of-leibniz-adjunction}, we have that~$\hatexp(i, p)$ is again a uniform Kan fibration.
The above statement is then the special case~$X \defeq 1$ and $A \defeq 0$.
\end{example} 


\section{Uniform fibrations in presheaf categories}
\label{sec:fib-psh}

The aim of this final section is to study in more detail the notion of a uniform fibration in the case when $\cal{E}$ is a presheaf category.  Let us begin by fixing the setting in which we shall we
be working.  First, we assume that~$\cal{E}$ is a presheaf category and that the functorial cylinder~$\interval \otimes (-) \co \cal{E} \to \cal{E}$ satisfies not only
our standing assumptions of having contractions and connections and of possessing a right adjoint, but also the following two conditions: 
\begin{enumerate}[({C}1)] 
\item $\interval \otimes (-) \co \cal{E} \to \cal{E}$ preserves pullback squares,
\item the natural transformations $\kcyl \otimes (-) \co \id_\cal{E} \to \interval \otimes (-)$, for $k \in \braces{ 0, 1}$, are cartesian.
\end{enumerate} 
Secondly, we let $\cal{M}$ be a full subcategory of $\cal{E}^\to_\cart$ satisfying the following assumptions:
\begin{enumerate}[({M}1)] 
\item the elements of $\cal{M}$ are monomorphisms,
\item for every $X \in \cal{E}$, the unique map $\bot_X \co 0 \to X$ is in $\cal{M}$, 
\item the elements of $\cal{M}$ are closed under pullbacks,
\item the elements of $\cal{M}$ are closed under Leibniz product with the endpoint inclusions.
\end{enumerate} 


The next result not only provides us with a wide class of examples of suitable algebraic weak factorisation systems, but also shows that there are situations in which object-wise assertions,
as in (M2)-(M5) above, can be strengthened to functoriality properties, as required to obtain a suitable weak algebraic weak factorisation system (\cref{thm:suitable-awfs}). 

\begin{proposition}  \label{rem-lift-suitable} 
There exists a suitable algebraic weak factorisation system $(\CC, \TF)$ on $\cal{E}$ that is algebraically free on $\cal{M}$, \ie such that $\TrivFib = \liftr{\cal{M}}$.
\end{proposition}

\begin{proof}
Let $\cal{I}$ be the full subcategory of $\cal{M}$ spanned by maps with a representable presheaf as codomain. 
Since $\cal{I}$ is small, by Garner's small object argument~\cite{garner:small-object-argument} it generates a
cofibrantly generated algebraic weak factorisation system $(\CC, \TF)$ with $\TrivCof =  \liftr{\cal{I}}$. Furthermore, we have~$\liftr{\cal{M}} = \liftr{\cal{I}}$.
This follows from \cref{awfs-on-arrows-into-representables} using that $\cal{M}$ is a full subcategory of~$\cal{E}^\to_\cart$. Indeed, for a map $f \co X \to Y$ in $\catE$, to give a natural choice of fillers for all diagrams with an arbitrary element of $\cal{M}$ on the left is the same as to give a natural choice of fillers for only those diagrams which are the form
\[
\xymatrix{
  A
  \ar[r]
  \ar[d]
&
  X
  \ar[d]^f
\\
  \yon(x)
  \ar[r]
&
  Y
\rlap{,}}
\]
where $\yon(x)$ the Yoneda embedding of some $x \in \catC$.

It remains to check that $(\CC, \TF)$ satisfies conditions~(ii)-(iv) in~\cref{thm:suitable-awfs}.
These follow from the corresponding object-wise assumptions~(iv)-(vi) via standard diagram-chasing arguments, which we omit.
For condition~(iv), the crucial property being used in that proof is that elements of $\cal{M}$, being monomorphisms in a topos, are adhesive maps~\cite{garner-lack:adhesive}. 
\end{proof}


Let us point out that the  reduction to the lifting problems against maps in $\cal{M}$ to those with a representable codomain in the proof of \cref{rem-lift-suitable} exploits the good behaviour of the right orthogonality functor with respect to colimits as described in~\cref{awfs-on-arrows-into-representables}, which is not available in the setting of standard weak factorisation systems. 



\begin{remark} \label{justify-sset-cset-examples}
Since assumptions (C1)-(C2) and (M1)-(M5) are fulfilled in both $\SSet$ and~$\CSet$, with $\cal{M}$ the category of all monomorphisms and pullback squares, \cref{rem-lift-suitable} establishes 
the existence of  the algebraic weak factorisations system  of  \cref{unif-triv-fib-sset}. 
\end{remark}


\begin{remark} \label{rem:constructive-small-object}  We discuss how, in the case of $\SSet$ and~$\CSet$, the existence of the algebraic weak factorisation systems $(\CC, \TF)$,
having uniform trivial Kan fibrations as right maps,  and $(\TC, \FF)$, having uniform Kan fibrations as right maps, and the Frobenius property for $(\TC, \FF)$  can can be proved constructively, \ie without using the law of excluded middle or the axiom of choice.

We begin by observing that \cref{rem-lift-suitable} can be proved constructively in $\SSet$ and $\CSet$. 
In order to see why this is the case, first note that in these examples every subobject of a representable is finitely presentable and that the functorial cylinder $\interval \otimes (-) \co \cal{E} \to \cal{E}$ preserves finitely presentable objects.
Since subobjects of representables are finitely presentable, the values of the inclusion $u \co \cal{I} \to \cal{E}^\to$ are finitely presentable objects of $\cal{E}^\to$.
An inspection of the proof of \cite[Theorem~4.4]{garner:small-object-argument} shows that this suffices to construct the algebraically-free algebraic weak factorization system $(\CC, \TF)$ on $u \co \cal{I} \to \cal{E}^\to$, and in fact the sequence constructing the appropriate free monad converges after $\omega$ steps. 
Next, note that also \cref{thm:sset-cset-nwfs} can be proved constructively for $\SSet$ and $\CSet$, via a reasoning that is analogous to the one above, noting that also the values of $u_\otimes \co \Cof_\otimes \co \cal{E}^\to$ are finitely presentable by the assumption of the functorial cylinder and the fact that finitely presentable objects are closed under pushout. Finally, the general proof of
\cref{uniform-fibrations-uniform-frobenius} and  \cref{uniform-fibrations-frobenius-pushforward}, establishing the functorial Frobenius condition for  $(\TC, \FF)$ and its pushforward analogue, are constructive. 

We have therefore obtained a constructive proof that the pushforward of a uniform Kan fibration preserves uniform Kan fibrations in $\SSet$ and $\CSet$. For $\SSet$,  this
result may be considered as a constructive counterpart of the fact that the pushforward along a Kan fibration preserves Kan fibrations, which cannot be proved constructively~\cite{coquand-non-constructivity-kan}. 

Finally, let us point out that if one adds the assumption that elements of $\cal{M}$ are \emph{decidable} monomorphisms, then the argument above carries over without relying on the Power Set axiom to establish the smallness of $\cal{I}$. We believe that this restriction is important also to treat constructively further aspects of the theory, such as the existence
of fibrant universes~\cite{cohen-et-al:cubicaltt}.
\end{remark}  



The rest of this section is devoted to give a characterization of uniform trivial fibrations in terms of the partial map classifier, and then use this characterization to show that the notions of a uniform (trivial) fibration and of a (trivial) fibration are somehow equivalent, in the sense that a map is a (trivial) fibration if and only it it admits the structure of a uniform (trivial) fibration.  
We begin with a simple lemma.




\begin{lemma} \label{identities-in-M}
For all $X \in \cal{E}$,  $\id_X \in \cal{M}$ .
\end{lemma}

\begin{proof}  The claim follows by inspection of the following diagram:
\[
\xymatrix@C+3em{
  X
  \ar[r]
  \ar[d]_{\id_X}
  \pullback{dr}
&
  \Id \otimes X
  \ar[d]^{\delta^0 \hatotimes \bot_X}
\\
  \Id \otimes X
  \ar[r]_{\delta^0 \hatotimes \bot_X}
&
  I \otimes X
\rlap{.}}
\]
By conditions~(M2) and~(M4), the right map is in $\cal{M}$.
By condition~(C2), the square is a pullback.
With condition~(M3), the left map $\id_X$ is then in $\cal{M}$.
\end{proof}

 Let $\cal{I}$ be the full subcategory of $\cal{M}$ spanned by maps with a representable presheaf as codomain. 



\begin{lemma} \label{partial-map-classifier}
There exists a subobject
\begin{equation} \label{partial-map-classifier:0}
\begin{gathered}
\xymatrix{
  1
  \ar[r]
  \ar[d]
  \pullback{dr}
&
  1
  \ar[d]
\\
  K
  \ar@{>->}[r]
&
  \Omega
}
\end{gathered}
\end{equation}
of the subobject classifier that classifies $\cal{I}$.
\end{lemma} 

\begin{proof}
Concretely, define $K(A)$, for $A \in \cat{C}$, as the set of subobjects of~$y A$ that are elements of~$\cal{M}$.
There is an evident inclusion $K \to \Omega$ giving rise to the pullback square~\eqref{partial-map-classifier:0} since identities are contained in $\cal{M}$ by \cref{identities-in-M}.
Any element $X \to Y$ of $\cal{M}$ arises as base change of $1 \to K$ along a unique map $Y \to K$, sending an element of $y A \to Y$ for $A \in \cat{C}$ to the subobject of $y A$ given by the base change of $X \to Y$ along $y A \to Y$.
Conversely, any base change of $1 \to K$ along an element $\sigma \co y A \to K$ results in the subobject of $y A$ selected by $\sigma$, an element of $\cal{I}$.
\end{proof} 

Fix $Y \in \cal{E}$ and let us work in the slice category $\cal{E}_{/Y}$.
Let $t \co Y \to K \times Y$ denote the classifier for $\cal{I}_{/Y}$.
Let $\pi \co K \times Y \to Y$ denote the map to the terminal object.
The \emph{partial object classifier} $P_Y$ relative to $Y$ associated to~$\cal{I}$ is the polynomial endofunctor on~$\cal{E}_{/Y}$ given by the data
\[
\xymatrix{
  Y
&
  Y
  \ar[l]_-{\id_K}
  \ar[r]^-{t}
&
  K \times Y
  \ar[r]^-{\pi}
&
  Y
}
\]
as $P_Y = \pi_! \cc t_*$.
Note that we have $t^* t_* = \Id$ since $t$ is monic, giving rise to a pullback square
\begin{equation} \label{partial-map-classifier:1}
\begin{gathered}
\xymatrix{
  X
  \ar[r]^-{\eta_X}
  \ar[d]
  \pullback{dr}
&
  P_Y X
  \ar[d]
\\
  Y
  \ar[r]_-{t}
&
  K \times Y
}
\end{gathered}
\end{equation}
for every $X \in \cal{E}_{/Y}$.


\begin{theorem} Giving the structure of a uniform trivial fibration on a map $f \co X \to Y$ is equivalent to giving a diagonal filler for the diagram
\begin{equation*}
\begin{gathered}
\xymatrix{
  X
  \ar@{=}[r]
  \ar[d]_{\eta_X}
&
  X
  \ar[d]^{f}
\\
  P_Y X
  \ar[r]
  \ar@{.>}[ur]
&
  Y
\rlap{,}}
\end{gathered}
\end{equation*}
\ie a retraction of $\eta_X$ in $\cal{E}_{/Y}$.
\end{theorem}

\begin{proof} 
Recall from \cref{rem-lift-suitable} that we have $\liftr{\cal{M}} = \liftr{\cal{I}}$.
Let $u \co \cal{I} \to \cal{E}^\to$ denote the inclusion.
Following the first step of Garner's algebraic small object argument, a lift of a map $f \co X \to Y$ to an element of $\liftr{I}$ is given by a diagonal filler in the canonical square $(\Lan_u u)(f) \to f$.
Developing the left Kan extension, we have
\begin{align*}
(\Lan_u u)(f)
&=
\colim_{i \in \cal{I}, u_i \to f} u_i
\\&=
\colim_{\sigma : y C \to K \times Y,\ \cal{E}_{/Y}(t^* \sigma, X)} \sigma^* t
\\&=
\colim_{\sigma : y C \to P_Y(X)} \sigma^* \eta_X
\\&=
\eta_X
.
\end{align*}
Here, the penultimate step uses the pullback square~\eqref{partial-map-classifier:1} and the ultimate step uses the Yoneda lemma and preservation of colimits by pullbacks.
\end{proof} 


\begin{corollary} \label{unif-vs-non-unif} \hfill 
\begin{enumerate}[(i)] 
\item  Every trivial fibration can be equipped with the structure of a uniform trivial fibration, \ie
the underlying maps of uniform trivial fibrations are exactly the maps having right lifting property with respect to 
the maps in $\cal{M}$.
\item Every fibration can be equipped with the structure of a uniform  fibration.
\end{enumerate}
\end{corollary} 

\begin{proof} For part (i), we first prove that that the $\cal{I}$-classifier $1 \to K$ is  in $\cal{M}$.
This assertion is equivalent to the statement that $\cal{M} \to \cal{E}^\to_{\cart}$ lifts colimits.
This is the case since $\cal{E}$ is a topos.
Since  $1 \to K$ is  in $\cal{M}$.  then also $\eta$ has components in $\cal{M}$ by~\eqref{partial-map-classifier:1}.
Part (ii) follows from part (i) and \cref{prod-exp-general}.
\end{proof} 


\begin{remark} \label{elegant-reedy}
If $\cal{E}$ be the category of presheaves over an elegant Reedy category~\cite{bergner-rezk-elegant}.
In this context, it is natural to say that a map in $\cal{E}$ is called a \emph{trivial Kan fibration} if it has the right lifting property with respect to boundary inclusions.
Since, by skeletal decomposition, we may write every monomorphism in $\cal{E}$  as an $\omega$-composition of cobase changes of coproducts of boundary inclusions of 
simplices (this is a non-constructive argument), trivial Kan fibrations are precisely the maps having the right lifting property with respect to all monomorphisms.
If we now let $\cal{M}$ be the full subcategory of $\cal{E}^\to_\cart$ spanned by all monomorphisms, \cref{unif-vs-non-unif} implies that every trivial fibration is a uniform trivial fibration.
\end{remark}


\bibliographystyle{plain}
\bibliography{../../common/uniform-kan-bibliography}

\end{document}

\documentclass[reqno,10pt,a4paper,oneside]{amsart}

\setcounter{tocdepth}{1}

\usepackage{uniform-kan-prelude}

\title{On the pushforward of uniform Kan fibrations}

\begin{document}

\begin{abstract}
We give a categorical account of algebraic fibrations being preserved under certain dependent product.
This generalizes work by Coquand \etal.
\end{abstract}

\maketitle

\tableofcontents


\section{Introduction}

This paper contributes to the ongoing efforts to define a constructive counterpart of Voevodsky's 
simplicial model of Martin-L\"of's dependent type theory extended with the Univalence Axiom. 

\newpage


\section{Elements of abstract homotopy theory} 

Let $(\catE, \otimes, \unit, \sigma)$ be a symmetric monoidal category, which we consider fixed throughout this section. 



\begin{definition} An \emph{interval} in $\catE$ is a tuple $\interval = (\interval, \intervall, \intervalr, \varepsilon)$ consisting of  an object $\interval \in \catE$, 
and morphisms $\intervall, \intervalr \co \unit \to \interval$ and $\intervalc \co \interval \to \unit$,   such that the following diagrams commute:
\[
\xymatrix{
\unit \ar[r]^\intervall \ar[dr]_{\id_\unit} & \interval \ar[d]^-{\varepsilon} & \unit \ar[dl]^{\id_\unit} \ar[l]_{\intervalr}  \\
 & \unit & }
 \]
\end{definition}

Let us now fix an interval $\interval = (\interval,  \intervall, \intervalr, \varepsilon)$ in $\catE$. 


\begin{definition}
\label{def:homotopy}
Let $f, g \co X \to Y$ be maps in $\catE$. A \emph{homotopy} $h$ from $f$ to $g$, denoted $h \co f \sim g$, is a morphism $h \co \interval \otimes X \to Y$ such that the following diagrams commute:
\[
\xymatrix@C=1.2cm{
X \ar[r]^-{\intervall \otimes X} \ar[dr]_{f} & \interval \otimes X \ar[d]^{h} & X \ar[dl]^{g} \ar[l]_-{\intervalr \otimes X}  \\
 & Y & }
 \]
\end{definition}




\begin{definition}
\label{def:homotopy-equivalence}
A map $f \co X \to Y$ is called a \emph{left (right) homotopy equivalence} if there exist $g \co Y \to X$ and homotopies $h \co \id_X \sim g \cc f$ and $k \co
\id_Y \sim f \cc g$ (respectively $h \co g \cc f \sim \id_X$ and $k \co f \cc g \sim \id_Y$). Such a left (right) homotopy equivalence is said to be \emph{strong} if the
following diagram commutes:
\[
\xymatrix{
\interval \otimes X \ar[r]^{\interval \otimes f } \ar[d]_{h} & \interval \otimes Y \ar[d]^{k} \\
X \ar[r]_{f} & Y}
\]
and it is said to be \emph{co-strong} if  the following diagram commutes:
\[
\xymatrix{
\interval \otimes Y \ar[r]^{\interval \otimes g } \ar[d]_{k} & \interval \otimes X \ar[d]^{h} \\
Y \ar[r]_{g} & X \, .}
\]
%
%A \emph{deformation retract} is a homotopy equivalence as above where the homotopy $h$ is trivial (note that this makes $f$ and $g$ into a section-retraction pair).
%Dually, a \emph{co-deformation retract} has the homotopy $k$ trivial (with $g$ and $f$ a section-retraction pair).
\end{definition}

\begin{remark} The notion of homotopy equivalence is symmetric and admits an evident duality, and a homotopy equivalence is strong if 
and only if its dual is co-strong. 
\end{remark}


\begin{remark}
The notion of a left or right strong homotopy equivalence is a generalization of the notion of a strong deformation retract, which is obtained by requiring also
that the homotopy $h$ is trivial.
\end{remark} 


We wish to give an alternative characterisation of strong homotopy equivalences, which will be useful to establish some of their closure properties. In order to do this,  let us assume also that $\catE$ is finitely cocomplete. Recall from~\cite[Section 4]{riehl-verity:reedy} that, given $f \co X \to Y$ and $g \co U \to V$, their \emph{pushout product} is the arrow $f \hatotimes g$ fitting in the following pushout diagram:
\[
\xymatrix@R=1.2cm{
X \otimes U \ar[r]^{X \otimes g}  \ar[d]_{f \otimes U} & X \otimes V \ar@/^2pc/[ddr]^{f \otimes V} \ar[d] & \\ 
Y \otimes U \ar@/_2pc/[drr]_{V \otimes g} \ar[r] & (Y \otimes U) +_{X \otimes U} (X \otimes V) \ar[dr]^-(.35){f \hatotimes g}  & \\ 
 & & Y \otimes V} 
 \]
This operation extends to a bifunctor  $\catE^{\to} \times \catE^{\to} \to \catE^{\to}$, which equips the arrow category of $\catE$ with a symmetric monoidal structure, with unit the unique map $\hatunit \co \initial \to \unit$.  If the symmetric
monoidal structure of~$\catE$ is closed, then so is that of $\catE^\to$. Writing $[X,Y]$ for the internal hom in~$\catE$,
the internal hom $\hatexp(f,g)$ in $\catE^\to$ of $f \co X \to Y$ and $g \co U \to V$, is obtained by the following 
pullback diagram
\[
\xymatrix@R=1.2cm{
[Y, U] \ar@/^2pc/[drr]^{[f,U]} \ar@/_2pc/[ddr]_{[Y,g]}  \ar[dr]^{\hatexp(f,g)} & & \\ 
 & [Y,V] \times_{[X,V]} [X,U]  \ar[d] \ar[r] & [X,U] \ar[d]^{[X,g]} \\
 & [Y,V] \ar[r]_{[f,V]} & [X,V] }
 \]

\medskip

We write $\theta$ for the  commutative square
\begin{equation}
\label{trivial-square}
\begin{gathered}
\xymatrix@C+2em{
  \initial
  \ar[r]^{\hatunit}
  \ar[d]_{\hatunit}
&
  \unit
  \ar[d]^{\intervalr}
\\
  \unit
  \ar[r]_{\intervall}
&
  \interval
}
\end{gathered}
\end{equation}
This square,  which will play an important role in  our development, gives us two maps in the arrow category: 
\[
\thetal \co \hatunit \Rightarrow \intervalr  \, , \quad \thetar \co \hatunit \Rightarrow \intervall \,. 
\]
Using the square $\theta$, we provide the following  characterization of strong homotopy equivalences.

\begin{lemma}
\label{strong-h-equiv-as-section}
Let $f \co X  \to Y$ be a morphism in $\catE$.
\begin{enumerate}[(i)]
\item $f$ is a strong left homotopy equivalence if and only if $\thetal \hatotimes f \co f \Rightarrow \intervalr \hatotimes f$ is a section.
\item $f$ is a strong right homotopy equivalence if and only if $\thetar \hatotimes f \co f \Rightarrow  \intervall \hatotimes f$ is a section.
\end{enumerate}
\end{lemma}

\begin{proof}
By duality, it suffices to exhibit the equivalence in (i). To say that $\thetal \hatotimes f \co f \to \intervalr \hatotimes f$ is a section means that
there is retraction $\rho$, as follows:
\[
\xymatrix@C+1em{
  f
  \ar[r]^-{\thetal \hatotimes f}
  \ar[dr]_{\id_f} &   \intervalr \hattimes f \ar[d]^{\rho} \\
&   f
}
\]
First, standard diagram-chasing shows that giving $\rho \co \intervalr \hattimes f \Rightarrow f$ is equivalent  to giving maps $h \co \interval \otimes X \to X$, $g \co Y \to X$, and $k \co \interval \otimes Y \to Y$ such that the following diagrams commute:
\begin{equation}
\label{equ:first-three}
\xycenter{
X \ar[r]^-{\intervalr \otimes X}  \ar[d]_f & \interval \otimes X \ar[d]^{h} \\
Y \ar[r]_{g} & X}  \qquad
\xycenter{
Y \ar[r]^-{\intervalr \otimes Y} \ar[d]_g & \interval \otimes Y \ar[d]^{k} \\
X \ar[r]_f & B} \qquad
\xycenter{ 
\interval \otimes X \ar[d]_h \ar[r]^{I \otimes f} & \interval \otimes Y \ar[d]^k \\
X \ar[r]_{f} & Y }
\end{equation}
Secondly, requiring that $\rho$ is a section to $\theta \hattimes f$ means that the diagrams
\begin{equation}
\label{equ:second-two}
\xycenter{
X \ar[r]^-{\intervall \otimes X} \ar[dr]_{\id_X} & \interval \otimes X \ar[d]^h \\ 
 & X } \qquad
 \xycenter{
 Y \ar[r]^-{\intervall \otimes Y}  \ar[dr]_{\id_Y} & \interval \otimes Y \ar[d]^{k} \\
  & Y} 
\end{equation}
commute. With reference to \cref{def:homotopy-equivalence}, the equations in~\eqref{equ:first-three} provide right endpoint for $h$, 
right endpoint for $k$, and strength for $h$, respectively; while the equations in~\eqref{equ:second-two} provide left endpoints for~$h$ and~$k$, respectively.
\end{proof}

\cref{strong-h-equiv-as-section} entails the following closure properties of strong homotopy equivalences, which are obtained working entirely at the level of arrow categories.

\begin{proposition}
\label{strong-h-equiv-closed-under-monoidal-prod}
If either $f$ or $g$ is a left (respectively, right) strong homotopy equivalence, then so is $f \hatotimes g$.
\end{proposition}

\begin{proof}
Apply \cref{strong-h-equiv-as-section} and use that functors (in this case the Leibniz monoidal product in one variable) preserve sections.
\end{proof}

\begin{proposition}
\label{strong-h-equiv-closed-under-retract}
Left or right strong homotopy equivalences are closed under retracts.
\end{proposition}

\begin{proof}
Use \cref{strong-h-equiv-as-section},  that functors preserve sections, and that  sections are closed under retracts.
\end{proof}



\section{Orthogonality functors}
\label{sec:ortf}

For this section, fix a category $\catE$. Given a set of morphisms $\cal{I} \subseteq \catE^\to$, we 
write $\liftr{\cal{I}}$ to be the class of morphisms of $\catE$ that have 
the right lifting property with respect to all the morphisms in~$\cal{I}$. Dually, we write $\liftl{\cal{I}}$ for the class of morphisms of $\catE$ that have the left lifting property with respect to all the morphisms of $\cal{I}$. 
Here, we shall be interested in algebraic counterparts of these notions. Instead of starting from a subset of $\catE^\to$, we consider a category $\cal{I}$, to be thought of an indexing category, and a functor $u \co \cal{I} \to \catE^\to$, which assigns an arrow $u_i \co A_i \to B_i$ in $\catE$ to each index $i \in \cal{I}$.


 \begin{definition} Let $u \co \cal{I} \to \catE^\to$ be a functor. 
 \begin{enumerate}[(i)] 
 \item  An \emph{$\cal{I}$-injective map}
 is a pair $(f, \phi)$ consisting of a map $f \co X \to Y$ equipped with a right  $\cal{I}$-lifting function, \ie 
 a function  $\phi$ that assigns to each $i \in \cal{I}$ and commuting square
\[
\xymatrix@C=2cm{
A_i \ar[r]^{s}   \ar[d]_{u_i} & X \ar[d]^f \\
B_i \ar[r]_{t} & Y}
\]
a diagonal filler $\phi(i,s, t) \co B_i \to X$, satisfying the following naturality 
condition: for every diagram of the form
\[
\xymatrix{
A_i \ar[r]^a \ar[d]_{u_i} & A_j \ar[r]^{s}  \ar[d]_{u_j} & X \ar[d]^f   \\
B_i \ar[r]_{b}  & B_j  \ar[r]_{t}  & Y }
\]
where the right-hand side square is the image of $\sigma \co i \to j$ in $\cal{I}$, 
we have that 
\[
\phi(j, s, t) \, b = \phi(i, s  a, t  b) \, .
\]
\item A \emph{morphism} $\cal{I}$-injective maps $\alpha \co (f, \phi) \to (f', \phi')$ is a 
square $\alpha \co f \Rightarrow f'$ in~$\catE$ satisfying an evident compatibility condition 
with respect to the ifting functions, which we omit. 
\end{enumerate}
\end{definition}

For a functor $u \co \cal{I} \to \catE^\to$, we write $\liftr{\cal{I}}$ for the category  of 
$\cal{I}$-injective maps and their morphisms. There is a forgetful functor~$\liftr{u} \co \liftr{\cal{I}} \to \catE^\to$
mapping $(f, \phi)$ to $f$.

\medskip

The extra generality obtained by allowing $u \co \cal{I} \to \catE^\to$ to be a functor rather than a mere 
inclusion of a  subcategory will be very important for our purposes, as the next construction and definitions
show. Given an interval $\interval = (\interval, \intervall, \intervalr, \varepsilon)$ and a functor $u \co \cal{I} \to \catE^\to$,
we define a category~$\cal{I}_\interval$ and a functor $u_\interval \co \cal{I}_\interval \to \catE^\to$ as follows. First of all, we define $\cal{I}_\interval$ as the coproduct
 \[
 \cal{I}_\interval  \defeq \cal{I} + \cal{I} \, .
 \] 
 Next, let $u_{0} \co \cal{I} \to \catE^\to \, , \; u_{1} \co \cal{I} \to \catE^\to$ be given by
 \[
u_{0}(i) \defeq  \intervall \hatotimes u_i \; , \quad
u_{1}(i) \defeq  \intervalr \hatotimes u_i \; , 
\]
for $i \in \cal{I}$, where we used the pushout product on $\catE^\to$. The functor $u_\interval \co \cal{I}_\interval \to \catE^\to$ is then given by the coproduct diagram
\begin{equation}
\label{equ:u-interval}
\vcenter{\hbox{\xymatrix@C=1.2cm{
\cal{I} \ar[r] \ar[dr]_-{u_0} & \cal{I}_\interval \ar[d]^(.4){u_\interval} & \cal{I} \ar[dl]^-{u_1} \ar[l] \\ 
 & \catE^\to }}}
\end{equation}
Note that, even if $u \co \cal{I} \to \catE^\to$ is an inclusion, $u_\interval \co \cal{I}_\interval \to \catE^\to$ is not.


\begin{definition} \label{A-fibration} Let  $\interval = (\interval, \intervall, \intervalr,\varepsilon)$ be an interval and
 $u \co \cal{I} \to \catE^\to$ a functor. A \emph{uniform $\cal{I}$-fibration} is a $\cal{I}_\interval$-injective map,
 \ie a pair $(f, \phi)$ consisting of a map $f \co X \to Y$ and a function $\phi$ that assigns diagonal fillers to all diagrams of the form
\[
\xymatrix{
\bullet \ar[r] \ar[d]_{\intervall \hatotimes u_i} & X \ar[d]^p \\
B_i \otimes \interval \ar[r] & Y} \qquad \xymatrix{
\bullet \ar[r] \ar[d]_{\intervalr \hatotimes u_i} & X \ar[d]^p \\
B_i \otimes \interval \ar[r] & Y}
\]
subject to the naturality condition.
\end{definition}

The notion of a uniform Kan fibration will be defined as special cases of the notions of
$\cal{I}$-fibration. However, since $\cal{I}$-fibrations are defined as $\cal{I}_\interval$-injective 
maps, the study of categories of injective maps and of orthogonality functors allows us to
establish also properties of $\cal{I}$-fibrations.  For this, we work with a fixed category $\catE$, without assuming the presence of an interval. First of all, recall from~\cite{garner:small-object-argument} that the function mapping $u \co \cal{I} \to \catE^\to$ to its right orthogonal $\liftr{u} \co \liftr{\cal{I}} \to \cal{E}^\to$ defines the action on objects of the \emph{right orthogonality functor}
\[
\liftr{\brarghole} \co  (\CAT/\catE^{\to})^{\op} \to \CAT/\catE^{\to} \, .
\]
The action  on arrows is defined as follows. Given a commutative triangle of the form
\[
\xymatrix{
\cal{I} \ar[dr]_u \ar[rr]^F & & \cal{J} \ar[dl]^{v} \\
 & \catE^\to }
 \]
we define 
\[
\xymatrix{
\liftr{\cal{J}} \ar[dr]_{\liftr{u}} \ar[rr]^{\liftr{F}} & & \liftr{\cal{J}} \ar[dl]^{\liftr{v}} \\
 & \catE^\to }
\]
as follows: for $(f, \phi) \in \liftr{J}$, we let $\liftr{F}(f,\phi) \defeq (f, \phi_F)$, where $\phi_F(i, s, t) \defeq \phi(Fi, s, t)$. 
Just as the standard orthogonality operations determine a Galois connection between the poset of subsets of arrows in~$\catE$ and its opposite, the orthogonality functors form an adjunction 
\begin{equation}
\label{garner-adjunction}
\begin{gathered}
\xymatrix@C+2em{
  \CAT/\catE^{\to}
  \ar@<5pt>[r]^-{\liftl{\brarghole}}
  \ar@{}[r]|-{\bot}
&
  (\CAT/\catE^{\to})^{\op} \, .
  \ar@<5pt>[l]^-{\liftr{\brarghole}}
}
\end{gathered}
\end{equation}
In the remainder of this section, we extend some useful facts about orthogonality operations to orthogonality functors.






\begin{proposition}
Consider a natural transformation between categories over $\catE^{\to}$:
\[
\xymatrix{
  \cal{I}
  \rrtwocell_G^F{\sigma}
 \ar[dr]_{u}
&&
  \cal{J}
  \ar[dl]^{v}
\\&
  \catE^{\to}
}
\]
Note that this includes the condition $v \sigma = \id_u$.
Then $\liftr{F} = \liftr{G}$ and $\liftl{F} = \liftl{G}$, 
\begin{mathpar}
\xymatrix{
  \liftr{\cal{I}}
  \ar[dr]_{\liftr{u}}
&&
  \liftr{\cal{J}}
  \lltwocell_{\liftr{F}}^{\liftr{G}}{=}
  \ar[dl]^{\liftr{v}}
\\&
  \catE^{\to}
}
\and
\xymatrix{
  \liftl{\cal{I}}
  \ar[dr]_{\liftl{u}}
&&
  \liftl{\cal{J}}
  \lltwocell_{\liftl{F}}^{\liftl{G}}{=}
  \ar[dl]^{\liftl{v}}
\\&
  \catE^{\to}
}
\end{mathpar}
\end{proposition}

\begin{proof} For $(f, \phi) \in \liftr{\cal{J}}$, we have $\liftr{F}(f, \phi) = (f, \phi_F)$ and $\liftr{G}(f, \phi) = (f, \phi_G)$.
We claim that the functions $\phi_F$ and $\phi_G$ coincide. Observe that 
for every $i \in \cal{I}$, we have that $\sigma_i \co v_{Fi}  \Rightarrow v_{Gi}$ is the identity square on $u_i
\co A_i \to B_i$. Hence, by the naturality condition for $\phi$, applied to the diagram 
\[
\xymatrix{
A_i \ar[r]^{\id_{A_i}} \ar[d]_{v_{Fi}}  & A_i \ar[d]^{v_{Gi}} \ar[r]^{s}  & X \ar[d]^{f} \\
B_i \ar[r]_{\id_{B_i}} & B_i \ar[r]_{t} & Y \, ,}
\]
we have  that $\phi_F(i, s, t) = \phi_G(i, s, t)$, as required.
\end{proof} 

\medskip

We now consider the effect of adjoint functors on orthogonality. In the standard setting, it is well known that if 
we have classes of maps $\cal{I} \subseteq \cal{E}^\to$ and $\cal{J} \subseteq \cal{F}^\to$ and an adjunction
\[
\xymatrix@C+1em{
  \cal{E}
  \ar@<5pt>[r]^{F}
  \ar@{}[r]|{\bot}
&
  \cal{F}
  \ar@<5pt>[l]^{G}
}
\]
then $F(\cal{I}) \subseteq \liftl{\cal{J}}$ if and only if $\cal{I} \subseteq G(\cal{J})$. Our next lemma provides the counterpart of this fact in our setting.




\begin{proposition} \label{lift-of-adjunction} 
Let $u \co \cal{I} \to \cal{E}^{\to}$ and $v \co \cal{J} \to \cal{F}^{\to}$ be functors and consider an adjunction
\[
\xymatrix@C+1em{
  \cal{E}
  \ar@<5pt>[r]^{F}
  \ar@{}[r]|{\bot}
&
  \cal{F}
  \ar@<5pt>[l]^{G}
}
\]
Then, the following are equivalent:
\begin{enumerate}[(i)] 
\item the  functor $F \co \cal{E}^\to \to \cal{F}^\to$ extends to a functor $F' \co \cal{I} \to \liftl{\cal{J}}$ making the following diagram commute:
\[
\xymatrix@C=1.2cm{
  \cal{I}
  \ar[r]^{F'}
  \ar[d]_{u}
&
  \liftl{\cal{J}}
  \ar[d]^{\liftl{v}}
\\
  \cal{E}^{\to}
  \ar[r]_-{F}
&
  \cal{F}^{\to}\, ,}
\]
\item the functor $G \co \cal{F}^\to \to \cal{E}^\to$ extends to a functor $G' \co \cal{J} \to \liftr{\cal{I}}$, making the following diagram commute:
\[
\xymatrix{
  \liftr{\cal{I}}
  \ar[d]_{\liftr{u}}
&
  \cal{J}
  \ar[l]_{G'} 
  \ar[d]^{v}
\\
  \cal{E}^{\to}
&
  \cal{F}^{\to}
  \ar[l]^{G}
}
\]
\end{enumerate}
\end{proposition}

\begin{proof} Giving a functor $F' \co \cal{I} \to \liftl{\cal{J}}$ as above is the same thing as giving fillers for squares of the form
\[
\xymatrix{
FA \ar[d]_{F u_i} \ar[r] & C \ar[d]^{v_j} \\
FB \ar[r] & D }
\]
natural in $i  \in \cal{I}$ and $j \in \cal{J}$. Similarly, giving a functor $G' \co \cal{J} \to \liftl{\cal{I}}$ as above is the same thing as giving fillers for squares 
of the form
\[
\xymatrix{
A \ar[d]_{u_i} \ar[r] & GC \ar[d]^{Gv_j} \\
B \ar[r] & GD }
\]
 natural in $i \in \cal{I}$ and $j \in \cal{J}$. Since $F \dashv G$, these situations coincide.
\end{proof}

We can apply \cref{lift-of-adjunction} to relate $\cal{I}$-injective maps and 
 $\cal{I}_\interval$-fibrations, which were introduced in \cref{A-fibration}.


\begin{corollary} \label{prod-exp-general}
Let $\interval = (\interval, \intervall, \intervalr, \varepsilon)$ be an interval in $\catE$ and
 $u \co \cal{I} \to \catE^\to$ be a functor. For every map $f \co X \to Y$ in $\cal{E}$ 
 the following are equivalent: 
\begin{enumerate}[(i)]
\item $f$ admits the structure of a $\cal{I}_\interval$-fibration. 
\item $\hatexp(\intervall, f)$ and $\hatexp(\intervalr, f)$ admit the structure of $\cal{I}$-injective maps.
\end{enumerate} 
\end{corollary}


\medskip


We now consider the interaction between the orthogonality functors and the slicing operation. In the classical setting it is well-known that the right orthogonality operation commutes with slicing, while the left orthogonality operation commutes with coslicing.  In order to provide a counterpart of this fact in our setting, we need some auxiliary definitions. Given a functor $u \co \cal{I} \to \catE^{\to}$ and $X \in \catE$, we define the category $\cal{I}/X$
and a functor $u/X \co \cal{I}/X \to (\cal{E}/X)^\to$ as follows. The category $\cal{I}/X$ has as objects pairs consisting of an object $a \in \cal{I}$ and a commutative triangle of the form
\[
\xymatrix{
A_i \ar[dr] \ar[rr]^{u_i} & & B_i \ar[dl] \\
 & X }
 \]
The functor $u/X \co \cal{I}/X \to (\cal{E}/X)^\to$ sends such a pair to $u_i \co A_i \to B_i$, viewed as a morphism in $\cal{E}/X$. This category fits into the
following pullback diagram:
\[
\xymatrix{
  \cal{I}/X
  \ar[r]
  \ar[d]_{u/X}
  \pullback{dr}
&
  \cal{I}
  \ar[d]^{u}
\\
  (\catE/X)^{\to}
  \ar[r]
&
  \catE^{\to}
}
\]
where we used the functor on arrow categories induced by the forgetful functor $\operatorname{dom} \co \catE/X \to \catE$.  Dually, taking the strict pullback along the map on arrows induced by the forgetful functor 
$\operatorname{cod} \co X/\cal{E} \to \catE$ constructs the \emph{coslice} over $X$:
\[
\xymatrix{
  X/\cal{I}
  \ar[r]
  \ar[d]_{X/u}
  \pullback{dr}
&
  \cal{I}
  \ar[d]^{u}
\\
  (X/\catE)^{\to}
  \ar[r]
&
  \catE^{\to}
}
\]
which also admits an explicit description, dual to the one given above for $\cal{I}/X$. With these definitions in place, we can now state the counterpart in our setting of the familiar commutation between slicing and orthogonality operations. 



\begin{proposition} \hfill 
\label{pitchfork-slicing}
\begin{enumerate}[(i)]
\item The right orthogonality functor commutes with slicing, \ie for every $u \co \cal{I} \to \cal{E}$, we have
\[
  \liftr{\cal{I}}/X = \liftr{(\cal{I}/X)}
\]
as categories over $\cal{E}^\to$.
\item The left orthogonality functor commutes with coslicing, \ie for every $u \co \cal{I} \to \cal{E}$, we have
\[
 \liftl{\cal{I}} \backslash X = \liftl{\cal{I} \backslash X}
\]
as categories over $\cal{E}^\to$.
\end{enumerate}
\end{proposition}

\begin{proof} We only consider (i). The claim follows by unfolding definitions, but we describe the objects of the category explicitly for clarity. They are given by 
tuples consisting of an arrow in~$\cal{E}/X$, 
\[
\xymatrix{
X \ar[dr] \ar[rr]^f  &  & Y \ar[dl] \\
 & X & }
 \]
and a function $\phi$ that assigns a diagonal filler to every diagram in $\cal{E}$ of the form
\[
\xymatrix{
A_i \ar[r] \ar[d]_{u_i} & X \ar[d]^{f} \\
B_i \ar[r] & Y}
\]
where $i \in \cal{I}$, subject to a uniformity condition. 
\end{proof}

The next corollary combines \cref{lift-of-adjunction} and \cref{pitchfork-slicing} to obtain 
a fact about the interaction of orthogonality functors with the pullback and pushforward
functors. This will be useful to establish our main result.

\begin{corollary}
\label{lift-dependent-product}
Let $f \co X \to Y$ be a map in $\catE$ admitting pushforward and pullback:
\[
\xymatrix@C+1em{
  \catE/X
  \ar@<5pt>[r]^{f_*}
  \ar@{}[r]|{\top}
&
  \catE/Y
  \ar@<5pt>[l]^{f^*}
}
\]
Let $u \co \cal{I} \to \catE^{\to}$ be a functor. The following are
equivalent:
\begin{enumerate}[(i)]
\item lifts of pushforward
\[
\xymatrix@C=1.5cm{
\liftr{\cal{I}}/X
\ar[r]^{f_*}
  \ar[d]_{u/X}
&
  \liftr{\cal{I}}/Y
  \ar[d]^{u/Y}
\\
  (\catE/X)^{\to}
   \ar[r]_{f_*}
&
  (\catE/Y)^{\to}
 }
\]
\item lifts of pullback
\[
\xymatrix@C=1.5cm{
  \cal{I}/Y
   \ar[r]^{f^*}
  \ar[d]_{u/Y} 
  &
  \liftl{ ( \liftr{\cal{I}}/X ) }
  \ar[d]^{\liftl{(\liftr{u}/X)}}
     \\
     (\catE/Y)^{\to} \ar[r]_{f^*} &
   (\catE/X)^{\to} 
}
\]
\item functors $F$ making the following diagram commute:
\[
\xymatrix@C=1.2cm@R=1.5cm{
\liftr{\cal{I}}/X \ar[rr]^F \ar[dr]_{\liftr{u}/X} & &  \liftr{\cal{I}}/ Y \ar[dl]^(.4){\ \liftr{( (u/Y) \cc f^*)}}  \\
 & (\cal{E}/X)^\to & }
\]
\end{enumerate}
\end{corollary}

\begin{proof}
Recall from \cref{pitchfork-slicing} that slicing commutes with right orthogonality functor.
For the first correspondence, apply \cref{lift-of-adjunction} to the adjunction $p^* \dashv p_*$ with  $v = \liftr{u}$.
The last statement is simply the adjunction~\eqref{garner-adjunction}.
\end{proof}

\medskip

Next, we consider the interaction between the orthogonality functors and closure under retracts. In the ordinary setting, it is well-known that
applying the left (or right) orthogonality operation to a class of morphisms produces the same result as applying it to its retract closure. 
In order to establish a counterpart of this fact, we need again some definitions. 
Given a  functor $u \co \cal{I} \to \catE^{\to}$, we define its retract closure $\overline{u} \co \overline{\cal{I}} \to \catE^{\to}$ as follows. 
An object of $\overline{\cal{I}}$ is a tuple~$(i, e, \sigma, \tau)$ consisting of an object $i \in \cal{I}$, an arrow $e \in \cal{E}^\to$ together with squares $\sigma \co e \Rightarrow u_i$ and $\rho \co u_i \Rightarrow e$,
which exhibit $e$ as a retract of $u_i$ in  $\catE^{\to}$,  \ie such that $\sigma \cc \rho = \id_e$. 
A morphism $(f, \kappa) \co (i, e, \sigma, \tau) \to (i', e', \sigma', \tau')$ of $\overline{\cal{I}}$  consists of a morphism $f \co i \to i'$ in $\cal{I}$ and a square $\kappa \co e \Rightarrow e'$  such that the following diagram in $\cal{E}^\to$ commutes:
\[
\xymatrix{
  e
  \ar[r]^{\sigma}
    \ar[d]_{\kappa}
&
  u_i
  \ar[r]^{\rho}
  \ar[d]^{u_f}
&
  e
  \ar[d]^{\kappa}
\\
  e'
  \ar[r]_{\sigma'}
&
  u_{i'}
  \ar[r]_{\rho'}
&
  e' \, .
}
\]
The functor $\overline{u} \co \overline{\cal{I}} \to \catE^\to$ is then defined  on objects  by letting 
$\overline{u}(i, e, \sigma, \tau) \defeq e$,
and on morphisms by letting $\overline{u}(f, \kappa) \defeq \kappa$. The operation of retract closure gives a monad: for $u \co \cal{I} \to \catE^{\to}$,
the components of the multiplication and the unit, 
\[
\mu_\cal{I} \co \overline{\overline{\cal{I}}} \to \overline{\cal{I}} \, , \quad
\eta_\cal{I} \co \cal{I} \to \overline{\cal{I}} \, ,
\]
are defined by letting
\[
\mu_\cal{I}((i, e, \sigma,  \rho), e', \sigma', \rho') \defeq (i, e', \sigma \cc \sigma', \rho' \cc \rho) \, , \quad
\eta_\cal{I}(i) \defeq (i, u_i, \id_{u_i}, \id_{u_i}) \, .
\]




\begin{proposition}
\label{retract-closure}
The orthogonality functors send the components of the unit and multiplication of the retract closure monad into natural
isomorphisms, and so for every $u \co \cal{I} \to \catE^\to$, we have isomorphisms of categories
\begin{gather*} 
 \liftr{(\overline{\cal{I}})} \iso \liftr{\cal{I}} \, , \quad
 \liftr{(\overline{\overline{\cal{I}}})} \iso \liftr{\overline{\cal{I}}}  \qquad
 \liftl{(\overline{\cal{I}})} \iso \liftl{\cal{I}} \, , \quad
 \liftl{(\overline{\overline{\cal{I}}})} \iso \liftl{\overline{\cal{I}}}
\end{gather*} 
over $\catE^\to$. \qed
\end{proposition}




\begin{remark} Let $\ret$ denote the \emph{walking retract}, \ie the category with objects $\retA, \retB$ and morphisms generated by $s \co \retA \to \retB$ and $r \co \retB \to \retA$ under the relation $r \cc s = \id_{\retA}$. The retract closure of $u \co \cal{I} \to \catE^\to$ fits into the following diagram, involving strict pullback and left composition:
\[
\xymatrix@C+1em{
  \overline{\cal{I}}
  \ar[r]
  \ar[d]
  \ar@/_2em/[dd]_{\overline{u}}
  \pullback{dr}
&
  \cal{I}
  \ar[d]^{u}
\\
  (\catE^{\to})^{\ret}
  \ar[r]^-{(\catE^{\to})^{\retB}}
  \ar[d]^{(\catE^{\to})^{\retA}}
&
  \catE^{\to}
\\
  \catE^{\to}
}
\]
The unit of the monad is formally induced by $(\catE^{\to})^{\canonical} \co \catE^{\to} \to (\catE^{\to})^{\ret}$ being a section to $(\catE^{\to})^{\retB}$.
\end{remark}


\begin{remark}
\label{retract-closure-slicing}
Taking the retract closure commutes with slicing and coslicing.
\end{remark}

\medskip

We conclude this section by considering the interaction between the orthogonality functors and  Kan extensions.




\begin{proposition} Let $F \co \cal{I} \to \cal{J}$ be a fully faithful functor. 
\label{kan-extension-closure}
\begin{enumerate}[(i)]
\item Assuming that the pointwise left Kan extension of 
$u \co \cal{I} \to \catE^{\to}$ along $F$ exists
\[
\xymatrix{
  \cal{I}
  \ar[dr]_{u}
  \ar[rr]^{F}
&&
  \cal{J}
  \ar[dl]^{\Lan_F u}
\\&
  \catE^{\to}
}
\]
then the functor $\liftr{F} \co \liftr{\cal{J}} \to \liftr{\cal{I}}$,  fitting in the diagram
\[
\xymatrix{
  \liftr{\cal{I}}
  \ar[dr]_{\liftr{u}}
&&
  \liftr{\cal{J}}
  \ar[ll]_{\liftr{F}}
  \ar[dl]^{\liftr{(\Lan_F u)}}
\\&
  \catE^{\to}
}
\]
is an isomorphism.
\item Assuming that the pointwise right Kan extension of 
$u \co \cal{I} \to \catE^{\to}$ along $F$ exists
\[
\xymatrix{
  \cal{I}
  \ar[dr]_{u}
  \ar[rr]^{F}
&&
  \cal{J}
  \ar[dl]^{\Ran_F u}
\\&
  \catE^{\to}
}
\]
then the functor $\liftl{F} \co \liftl{\cal{J}} \to \liftl{\cal{I}}$, fitting in the diagram
\[
\xymatrix{
  \liftl{\cal{I}}
  \ar[dr]_{\liftl{u}}
&&
  \liftl{\cal{J}}
  \ar[ll]_{\liftl{F}}
  \ar[dl]^{\liftl{(\Ran_F u)}}
\\&
  \catE^{\to}
}
\]
is an isomorphism. \qed
\end{enumerate}
\end{proposition}








\section{Uniform trivial Kan fibrations}

Recall that classically a trivial Kan fibration, which is defined by right orthogonality with respect to arbitrary monomorphisms. We now introduce our counterpart of this notion. We say that a monomorphism 
$i \co A \to B$ is  \emph{decidable} if for every $n$, the function $i_n \co A_n \to B_n$ has a decidable image. 
Note that the class of decidable monomorphisms is closed under base change. We define $\cal{M}$ as the full subcategory of $\SSet_\cart$ spanned by decidable monomorphisms. 

\begin{definition} A \emph{uniform trivial Kan fibration}  is a $\cal{M}$-injective map,
\ie a map $f \co X \to Y$ of simplicial sets equipped with a function $\phi$
that assigns to every decidable monomorphism $i \co A \to B$ and commuting square 
 \[
 \xymatrix{
 A \ar[r] \ar[d]_i & X \ar[d]^f \\
 B \ar[r] & Y}
 \]
a diagonal filler $\phi(i, s, t) \co B \to X$, subject to the following naturality condition: for every 
diagram 
\[
\xymatrix{
A \ar[r]^{h} \ar[d]_{i} & C \ar[d]^{j}  \ar[r]^s & X \ar[d]^f \\
B \ar[r]_{k} & D \ar[r]_t & Y }
\]
where the left-hand side square is a pullback, we have that $\phi(j, s, t) \, k = \phi(i, s  h, t  k)$.
 \end{definition} 
 
 
 
 
 
 
Classically, a map is a  trivial Kan fibration if and only if it has the right lifting property with respect to the
set of boundary inclusions $i^n \co \partial \Delta^n \to \Delta^n$. We wish to establish a counterpart of 
this fact in our setting, by showing that the category of uniform trivial Kan fibrations can also be 
characterized as the right orthogonal category of  a small category.






\medskip

\newcommand{\yon}{\mathrm{y}} 

In order to do this, let us briefly return to consider the setting of~\cref{sec:ortf} and prove two useful lemmas,
for which we make the further assumption that $\catE$ is a presheaf category, \ie $\catE = \hat{\cat{C}}$, where $\catC$ is some small category. We write $\yon \co \cat{C} \to \catE$ for the Yoneda embedding.

\begin{lemma}
\label{left-kan-extension-of-representables}
Let $\cal{J}$ be a full subcategory of $\catE_{\cart}^{\to}$ closed under base change to representables.
Let $\cal{I}$ denote its restriction to arrows into representables.
\[
\xymatrix{
  \cal{I}
  \ar[rr]
  \ar[dr]
&&
  \cal{J}
  \ar[dl]
\\&
  \catE^{\to}
}
\]
Then, the inclusion $\cal{J} \to \catE^{\to}$ is the left Kan extension of $\cal{I} \to \catE^{\to}$ along $\cal{I} \to \cal{J}$.
\end{lemma}



\begin{proof}
Since $\catE^{\to}$ is cocomplete, we can verify the claim using  the colimit formula for left Kan extensions.
All of the following will be functorial in an object $j \co A \to B$ of $\cal{J}$.
We consider the diagram indexed by cartesian squares of the form
\[
\xymatrix@C=1.2cm{
  A'
  \ar[r]
  \ar[d]_{i}
  \pullback{dr}
&
  A
  \ar[d]^{j}
\\
  \yon(c) 
  \ar[r]_-b 
&
  B
}
\]
with $i \co A' \to \yon(c)$ in $\cal{I}$ and valued $i$.
Our goal is to show that its colimit of this diagram in $\catE^{\to}$ is $j$.
Using the assumption that $\cal{J}$ is closed under pullback to representables, the given diagram
can be described equivalently as the the diagram indexed by maps $b \co \yon(c) \to B$ and valued $b^*(j)$. The claim can then be restated as  $\colim_{b : \yon(c) \to B} b^*(j) \iso j$, which 
holds since pullback commutes with colimits in presheaf categories, and  $\colim_{b : \yon(c) \to B} \yon(c) \iso B$.
\end{proof}


\begin{remark} It would be of interest to prove \cref{left-kan-extension-of-representables} by combining 
the codomain fibration and the corresponding left Kan extension claim for the codomain part
\[
\xymatrix{
  \cat{C}
  \ar[rr]^{y}
  \ar[dr]_{y}
&&
  \hat{\catC}
  \ar[dl]^{\id}
\\&
  \hat{\cat{C}}
}
\]
which holds by the co-Yoneda lemma.
\end{remark}



\begin{lemma}
\label{awfs-on-arrows-into-representables}
Let $\cal{J}$ be a full subcategory of $\catE_{\cart}^{\to}$ closed under base change to representables.
Let $\cal{I}$ denote its restriction to arrows into representables.
\[
\xymatrix{
  \cal{I}
  \ar[rr]
  \ar[dr]
&&
  \cal{J}
  \ar[dl]
\\&
  \catE^{\to}
}
\]
Then $\liftr{\cal{I}} = \liftr{\cal{J}}$.
\end{lemma}

\begin{proof} The result follows by combining \cref{left-kan-extension-of-representables} and part~(i) of \cref{kan-extension-closure}. 
\end{proof}


\begin{theorem} \label{small-gen-triv-kan}
The category of uniform trivial Kan fibrations is isomorphic to the right orthogonality 
category of the following full subcategories of $\cal{M}$, the category of decidable
monomorphisms and cartesian squares:
\begin{align*}
\cal{M}_1 & = \braces{i \co A \rightarrow \Delta^{n} \ | \ i \text{ is a  decidable monomorphism} } \\ 
\cal{M}_2  & = \braces{i \co A \rightarrow \Delta^{n_1} \times \ldots \times \Delta^{n_k} 
\ | \ i \text{ is a  decidable monomorphism} }  \\
\cal{M}_3  & = \braces{ i \co A \rightarrow B \mid \ i \text{ is a decidable monomorphism and 
$B$ is finite and finite-dimensional}} 
\end{align*}
\end{theorem}

\begin{proof}    \cref{awfs-on-arrows-into-representables} implies that we have that $\liftr{\cal{M}_1}  = \liftr{\cal{M}}$.
For the other equalities, observe that for every full subcategory $\cal{S} \subseteq \cal{M}$ containing $\cal{M}_1$ we have that $\liftr{\cal{M}} = \liftr{\cal{S}}$, since $\cal{M}_1$ is the restriction to maps into representables  of $\cal{M}$. 
\end{proof}

The classes of maps considered in~\cref{small-gen-triv-kan} have different advantages. For example, 
the category $\cal{M}_1$ is small, while other classes have better closure properties. 




\begin{proposition} There exists an algebraic weak factorisation system $(\mathsf{L}, \mathsf{R})$ on
$\SSet$ such that the category of $\mathsf{R}$-algebras is the category of uniform trivial Kan fibrations. 
In particular, there is a functorial factorisation of maps of simplicial sets which sends
a map $f \co X \to Y$ to a diagram of the form
\[
\xymatrix{ 
X \ar[rr]^f \ar[dr]_{i_f}  & & Y \\
 & C_f \ar[ur]_{p_f} }
 \]
 where $p_f$ admits the structure of  a uniform trivial Kan fibration and 
 $i_f$ admits the structure of a $\mathsf{L}$-coalgebra.
\end{proposition}

\begin{proof} Since the category $\cal{M}_1$ defined in~\cref{small-gen-triv-kan} is small,  
it is possible to apply Garner's small object argument to
obtain an algebraic weak factorisation system $(\cal{L}, \cal{R})$.
The fact that $\cal{R}$ is the category of uniform trivial Kan fibrations
 follows from \cref{small-gen-triv-kan}.
 \end{proof} 

Note that ~\cref{awfs-on-arrows-into-representables} cannot be used to show 
a map $p \co X \to Y$ of simplicial sets admits the structure of a uniform trivial Kan fibration if and only if it is a trivial Kan fibration in the usual sense since  the class of boundary inclusions  is contained strictly  in~$\cal{I}$. 




\section{Uniform Kan fibrations}


We now introduce the notion of a uniform Kan fibration, which will be our constructive counterpart
of the standard notion of a Kan fibration. For this, recall that in $\SSet$ we have an interval, given by
$\Delta_1$, the horn inclusions $h^0_1 \co 1 \to \Delta_1$, $h^1_1 \co 1 \to \Delta_1$ and $! \co
\Delta_1 \to 1$. For a horn inclusion $h^{k}_n \co 1 \to \Delta_1$ and  a monomorphism $i \co A \to B$
their pushout product $h^k_n \hattimes i$ is given by the following pushout diagram
\[
\xymatrix{
 A \ar[r]^{i}  \ar[d]_{h^k_n \times A} &  B \ar@/^1em/[ddr]^{h^k_1 \times B} \ar[d] & \\ 
\Delta_1 \times A \ar@/_1em/[drr]_{\Delta_1 \times i} \ar[r] & \bullet \ar[dr]^-(.35){h^k_1 \hattimes i}  & \\ 
 & & \Delta_1 \times B} 
 \]
 Thus, given the inclusion $u \co \cal{M} \to \SSet^\to$ of decidable morphisms and pullback squares, 
 we define the functor $u_{\Delta_1} \co \cal{M}_{\Delta_1} \co \SSet^\to$ as in~\eqref{equ:u-interval}. 
 Explicitly, $ \cal{M}_{\Delta_1} = \cal{M} + \cal{M}$ and
 \[
  u_{\Delta_1} (\iota_k(i)) = h^k_1 \hattimes i \, ,
  \]
  for $k \in \{ 0, 1 \}$. 
   

\begin{definition} A \emph{uniform Kan fibration} is a $\cal{M}_{\Delta_1}$-fibration, \ie 
a map  $p \co X \to Y$ of simplicial sets equipped with a function $\phi$ that assigns
to every decidable monomorphism $i \co A \to B$, $k \in \{0, 1\}$  and commuting
square a diagram of the form
\[
\xymatrix{
\bullet \ar[r] \ar[d]_{h^k_1 \hattimes i} & X \ar[d]^p \\
\Delta_1 \times B \ar[r] & Y }
\]
a diagonal filler, subject to a naturality condition. 
\end{definition} 


Here, higher-dimensional horns are omitted since they are included indirectly as retracts of one-dimensional horns 
by Leibniz product with certain subobjects of representables. We now show that, working classically, a map $p \co X \to Y$ admits the structure of a uniform (trivial) Kan fibration if and only if it is a (trivial) Kan fibration in the usual sense. 
The next lemma is a special case of \cref{prod-exp-general}; we state it explicitly for later reference.
  
 \begin{lemma} \label{prod-exp-for-Kan}  For every  map $p \co X \to Y$ of simplicial sets the following are equivalent: 
\begin{enumerate}[(i)]
\item the map $p$ admits the structure of a uniform Kan fibration,
\item the maps $\hatexp(\intervall, p)$ and $\hatexp(\intervalr, p)$ admit the structure of
uniform trivial Kan fibrations. 
\end{enumerate} 
\end{lemma}

\begin{proposition} There exists an algebraic weak factorisation system $(\mathsf{L}, \mathsf{R})$
such that the category of $\mathsf{R}$-algebras is the category of uniform Kan fibrations. 
In particular, there is a functorial factorisation of maps of simplicial sets which sends
a map $f \co X \to Y$ to a diagram of the form
\[
\xymatrix{ 
X \ar[rr]^f \ar[dr]_{i_f}  & & Y \\
 & P_f \ar[ur]_{p_f} }
 \]
 where $p_f$ admits the structure of  a uniform Kan fibration and 
 $i_f$ admits the structure of a $\mathsf{L}$-coalgebra.
\end{proposition} 

\begin{proof} The claim follows from Garner's small object argument, once we find a 
functor $u \co \cal{I} \rightarrow \catE^\to$ such that $\cal{I}$ is small and the
category of uniform Kan fibrations is isomorphic to $\liftr{\cal{I}}$. If we consider
the inclusion $u \co \mathcal{M}_1 \to \cal{E}^\to$ and perform the construction
$u_\interval \co (\mathcal{M}_1)_\interval \to \catE^\to$, the result follows 
by \cref{small-gen-triv-kan} and \cref{prod-exp-for-Kan}. 
\end{proof}


\medskip
 
Let us define a subcategory $\cal{I} \subseteq \SSet^\to$. The objects are the boundary inclusions
$i_n \co \partial \Delta_n \to \Delta_n$ and the identity maps $\id_{\Delta_n} \co \Delta_n \to \Delta_n$; the
maps are the identity squares and those of the form
 \[
\xymatrix@C=1.2cm{
  \partial \Delta_n
  \ar[r]
  \ar[d]_{i_n}
&
  \Delta_{n-1}
  \ar[d]^{\id_{\Delta_{n-1}}}
\\
  \Delta_n
  \ar[r]_-{s^k_{n-1}}
&
  \Delta_{n-1}
}
\] 


\begin{definition} A \emph{regular trivial Kan fibration} is an $\cal{I}$-injective map, \ie a map $p \co X \to Y$ 
equipped with a function that assigns fillers to all squares of the form
\begin{equation}
\label{equ:boundary-filler}
\xycenter{
\partial \Delta_n \ar[d]_{i_n} \ar[r] & X \ar[d]^{p} \\
\Delta_n \ar[r] & Y } 
\end{equation}
subject to the following naturality condition: for every diagram of the form
\begin{equation}
\label{equ:factor-via-id}
{\vcenter{\hbox{\xymatrix@C=1.2cm{
  \partial \Delta_n
  \ar[r]
  \ar[d]_{i^n}
&
  \Delta_{n-1}
  \ar[r]
  \ar[d]^{\id_{\Delta_{n-1}}}
&
  X
  \ar[d]^{p}
\\
  \Delta_n
  \ar[r]_-{s_k^{n-1}}
&
  \Delta_{n-1}
  \ar[r]
&
  Y
}}}}
\end{equation}
the composite filler is coherent with respect to the trivial filler in the right square. 
\end{definition}


\begin{lemma}[ZFC] \label{triv-Kan-is-regular}
Every trivial Kan fibration admits the structure of a regular trivial Kan fibration.
\end{lemma}

\begin{proof} By the axiom of choice, we can choose  designated fillers for squares as in~\eqref{equ:boundary-filler}
 based on (using excluded middle) whether that square factors as in~\eqref{equ:factor-via-id}. Note that it does not matter which degeneracy we choose if multiple are available, since the resulting diagonal filler will be coherent with 
 all possible choices.
\end{proof} 


\begin{lemma} \label{reg-triv-is-unif-Kan}
Every regular trivial Kan fibration admits the structure of a uniform trivial Kan fibration.
\end{lemma}

\begin{proof} Let us consider a map $p \co X \to Y$ equipped with the structure of a 
regular trivial fibration. By \cref{small-gen-triv-kan}, it is sufficient to show that $p$
can be equipped with the structure of a $\cal{M}_1$-injective map, where $\cal{M}_1$
is the full subcategory of $\SSet^{\mathbf{2}}_\cart$  spanned by monomorphisms into representables.
So, let us consider a square of the form
\[
\xymatrix{
A \ar[d]_i \ar[r] & X \ar[d]^p \\
\Delta_n \ar[r]  & Y }
\]
where $i$ is a decidable monomorphism. 
We define a diagonal filler by decomposing $i$ into a finite composition of cobase changes of boundary inclusions, filling each of these using~\cref{triv-Kan-is-regular}.
Crucially, this process is independent of the actual order of the boundary fillings (note that this is not true for the analogous situation of horn fillings). In order to prove the naturality condition of uniform trivial Kan fibrations, 
let us consider a diagram of the form
\[
\xymatrix{
  A
  \ar[r]
  \ar[d]_i
  \pullback{dr}
&
  B
  \ar[d]_j 
  \ar[r]
&
  X
  \ar[d]^p 
\\
  \Delta_{n}
  \ar[r]
&
  \Delta_{m}
  \ar[r]
&
  Y
}
\]
where the left-hand side square is a pullback. 
By ``vertical'' induction and the remark on order invariance of boundary fillings, it will suffice to study the case where the middle vertical map is a boundary inclusion $i_n \co \partial \Delta_n \to \Delta_n$.
Working ``horizontally'', it suffices to study the situation where the map $\Delta_{m} \to \Delta_n$ is a face or degeneracy map as $\Delta$ is generated by these.

Let us first examine the case of a face operation.
\[
\xymatrix{
  \Delta_n
  \ar[r]
  \ar[d]
  \pullback{dr}
&
  \partial \Delta_{n+1}
  \ar[d]
  \ar[r]
&
  X
  \ar[d]
\\
  \Delta_n
  \ar[r]_{d^k_{n+1}}
&
  \Delta^{n+1}
  \ar[r]
&
  Y
}
\]
Since the left vertical map is necessarily the identity, the filler for the composite square is uniquely determined, so there is no coherence to be verified.

Let us now examine the case of a degeneracy operation.
\[
\xymatrix{
  2 \times \partial \Delta_n
  \ar[r]
  \ar[d]
  \ar@/^2em/[rr]^(0.3){\pi_2}
  \pullback{dr}
&
  \bigcup_{i \neq k, k+1} \Delta_{[n+1] - i}
  \ar[r]
  \ar[d]
  \pullback{dr}
&
  \partial \Delta_n
  \ar[d]
  \ar[r]
&
  X
  \ar[dd]
\\
  2 \times \Delta_n
  \ar[r]
  \ar@/_2em/[rr]_(0.3){\pi_2}
&
  \partial \Delta_{n+1}
  \ar[r]
  \ar[d]
  \ar@{.>}[urr]
  \pullback{ul}
&
  \Delta_n
  \ar[d]
  \ar@{.>}[ur]
\\&
  \Delta_{n+1}
  \ar[r]_{s_k^n}
  \ar@{.>}[uurr]
&
  \Delta_n
  \ar[r]
  \ar@{.>}[uur]
&
  Y
}
\]
The pullback of the boundary inclusion $\partial \Delta_n \to \Delta_n$ along $s^k_n$ decomposes as a cobase change of two parallel boundary inclusions of dimension $n$ followed by a boundary inclusion of dimension $n+1$,
as indicated.
The two parallel boundary fillings are identical copies of the original right square boundary filling, so they cohere as indicated.
Finally, the filling for the boundary inclusion~$\partial \Delta_{n+1} \to \Delta_{n+1}$ coheres as indicated by how boundary filling was originally defined for degenerate squares.
\end{proof}











\begin{theorem}[ZFC]  \hfill 
\begin{enumerate}[(i)]
\item  Every trivial Kan fibration admits the structure of a uniform trivial Kan fibration.
\item Every Kan fibration admits the structure of a uniform  Kan fibration.
\end{enumerate} 
\end{theorem}

\begin{proof} The claim in (i) follows by \cref{triv-Kan-is-regular}  and \cref{reg-triv-is-unif-Kan}. For (ii), let
$p \co X \to Y$ be a Kan fibration. By the non-algebraic counterpart of \cref{prod-exp-for-Kan}, it follows 
that $\hatexp(h_k^1, p)$ is a trivial Kan fibration for $k = 0, 1$. The claim now follows  by \cref{prod-exp-for-Kan}. 
\end{proof}


\section{Strong homotopy equivalences}

\medskip


Recall the notion of an adhesive morphism~\cite{garner-lack:adhesive}.  Let $u \co \cal{I} \to \catE^{\to}$ be a subcategory of adhesive morphisms in $\catE$ with morphisms given by cartesian squares. Assume that the subcategory $\cal{I}$ is closed under the monoidal operations of $\catE^{\to}$, \ie that $(\cal{I}, \hatunit, \hatotimes)$ is itself a monoidal category and $u$ preserves the monoidal structure on the nose. 

\newpage









\bibliographystyle{plain}
\bibliography{../../common/uniform-kan-bibliography}

\end{document}

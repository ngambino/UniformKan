\documentclass[reqno,10pt,a4paper,oneside]{amsart}

\setcounter{tocdepth}{1}

\usepackage{uniform-kan-prelude}

\title{On uniform Kan fibrations and pushforward functors}

\begin{document}

\begin{abstract}
We give a categorical account of algebraic fibrations being preserved under certain dependent product.
This generalizes work by Coquand \etal.
\end{abstract}

\maketitle

\tableofcontents





\section{Elements of abstract homotopy theory} 

Let $(\catE, \otimes, \unit)$ be a symmetric monoidal category, which we consider fixed throughout this section. 

\begin{definition} An \emph{interval} in $\catE$ is a tuple $\interval = (\interval, c, \intervall, \intervalr)$ consisting of  an object $\interval \in \catE$, 
and morphisms $\intervall, \intervalr : \unit \to \interval$ and $\intervalc : \interval \to \unit$,   such that the following diagrams commute:
\[
\xymatrix{
\unit \ar[r]^\intervall \ar[dr]_{\id_\unit} & \interval \ar[d]^-{c} & \unit \ar[dl]^{\id_\unit} \ar[l]_{\intervalr}  \\
 & \unit & }
 \]
\end{definition}

Let us now fix an interval $\interval = (\interval, c, \intervall, \intervalr)$ in $\catE$. 


\begin{definition}
\label{def:homotopy}
Let $f, g \co A \to B$ be morphisms in $\catE$. A \emph{homotopy} $h$ from $f$ to $g$, denoted $h : f \sim g$, is a morphism $h : \interval \otimes A \to B$ such that the following diagrams commute:
\[
\xymatrix@C=1.2cm{
A \ar[r]^-{\intervall \otimes A} \ar[dr]_{f} & \interval \otimes A \ar[d]^{h} & A \ar[dl]^{g} \ar[l]_-{r \otimes A}  \\
 & B & }
 \]
\end{definition}




\begin{definition}
\label{def:homotopy-equivalence}
A map $f : A \to B$ is called a \emph{left (right) homotopy equivalence} if there exist $g \co B \to A$ and homotopies $h \co \id_A \sim g \cc f$ and $k \co
\id_B \sim f \cc g$ (respectively $h \co g \cc f \sim \id_A$ and $k \co f \cc g \sim \id_B$). Such a left (right) homotopy equivalence is said to be \emph{strong} if the
following diagram commutes:
\[
\xymatrix{
\interval \otimes A \ar[r]^{\interval \otimes f } \ar[d]_{h} & \interval \otimes B \ar[d]^{k} \\
A \ar[r]_{f} & B}
\]
and it is said to be \emph{co-strong} if  the following diagram commutes:
\[
\xymatrix{
\interval \otimes B \ar[r]^{\interval \otimes g } \ar[d]_{k} & \interval \otimes A \ar[d]^{h} \\
B \ar[r]_{g} & A \, .}
\]
%
%A \emph{deformation retract} is a homotopy equivalence as above where the homotopy $h$ is trivial (note that this makes $f$ and $g$ into a section-retraction pair).
%Dually, a \emph{co-deformation retract} has the homotopy $k$ trivial (with $g$ and $f$ a section-retraction pair).
\end{definition}

\begin{remark} The notion of homotopy equivalence is symmetric and admits an evident duality, and a homotopy equivalence is strong if 
and only if its dual is co-strong. 
\end{remark}


\begin{remark}
The notion of a left or right strong homotopy equivalence is a generalization of the notion of a strong deformation retract, which is obtained by requiring also
that the homotopy $h$ is trivial.
\end{remark} 


We wish to give an alternative characterisation of strong homotopy equivalences, which will be useful to establish some of their closure properties. 
For this, let us assume also that $\catE$ is finitely cocomplete. Recall from~\cite[Section 4]{riehl-verity:reedy} that, given $f \co A \to B$ and $g \co C \to D$, their \emph{pushout product} is the arrow $f \hatotimes g$ fitting in the following pushout diagram:
\[
\xymatrix{
A \otimes C \ar[r]^{A \otimes g}  \ar[d]_{f \otimes C} & A \otimes D \ar@/^1em/[ddr]^{f \otimes D} \ar[d] & \\ 
B \otimes C \ar@/_1em/[drr]_{B \otimes g} \ar[r] & \bullet \ar[dr]^-(.35){f \hatotimes g}  & \\ 
 & & B \otimes D} 
 \]
This operation extends to a bifunctor  $\catE^{\to} \times \catE^{\to} \to \catE^{\to}$, which equips the arrow category of $\catE$ with a symmetric monoidal structure, with unit the canonical map $\hatunit : 0 \to \unit$. We write $\theta$ for the following commutative square, which will play an important role in  our development:
\begin{equation}
\label{trivial-square}
\begin{gathered}
\xymatrix@C+2em{
  0
  \ar[r]^{\hatunit}
  \ar[d]_{\hatunit}
&
  \unit
  \ar[d]^{\intervalr}
\\
  \unit
  \ar[r]_{\intervall}
&
  \interval
}
\end{gathered}
\end{equation}
This square gives us two maps in the arrow category: 
\[
\thetal : \hatunit \Rightarrow \intervalr  \, , \quad \thetar : \hatunit \Rightarrow \intervall \,. 
\]
Using the square $\theta$, we provide the following  characterization of strong homotopy equivalences.

\begin{lemma}
\label{strong-h-equiv-as-section}
Let $f \co A \to B$ be a morphism in $\catE$.
\begin{enumerate}[(i)]
\item $f$ is a strong left homotopy equivalence if and only if $\thetal \hatotimes f \co f \Rightarrow \intervalr \hatotimes f$ is a section.
\item $f$ is a strong right homotopy equivalence if and only if $\thetar \hatotimes f \co f \Rightarrow  \intervall \hatotimes f$ is a section.
\end{enumerate}
\end{lemma}

\begin{proof}
By duality, it suffices to exhibit the equivalence in (i). To say that $\thetal \hatotimes f : f \to \intervalr \hatotimes f$ is a section means that
there is retraction $\rho$, as follows:
\[
\xymatrix@C+1em{
  f
  \ar[r]^-{\thetal \hatotimes f}
  \ar[dr]_{\id_f} &   \intervalr \hattimes f \ar[d]^{\rho} \\
&   f
}
\]
First, standard diagram-chasing shows that giving $\rho \co \intervalr \hattimes f \Rightarrow f$ is equivalent  to giving maps $h : \interval \otimes A \to A$, $g : B \to A$, and $k : \interval \otimes B \to B$ such that the following diagrams commute:
\begin{equation}
\label{equ:first-three}
\xycenter{
A \ar[r]^-{\intervalr \otimes A}  \ar[d]_f & \interval \otimes A \ar[d]^{h} \\
B \ar[r]_{g} & A}  \qquad
\xycenter{
B \ar[r]^-{\intervalr \otimes B} \ar[d]_g & \interval \otimes B \ar[d]^{k} \\
A \ar[r]_f & B} \qquad
\xycenter{ 
\interval \otimes A \ar[d]_h \ar[r]^{I \otimes f} & \interval \otimes B \ar[d]^k \\
A \ar[r]_{f} & B }
\end{equation}
Secondly, requiring that $\rho$ is a section to $\theta \hattimes f$ means that the diagrams
\begin{equation}
\label{equ:second-two}
\xycenter{
A \ar[r]^-{\intervall \otimes A} \ar[dr]_{\id_A} & \interval \otimes A \ar[d]^h \\ 
 & A } \qquad
 \xycenter{
 B \ar[r]^-{\intervall \otimes B}  \ar[dr]_{\id_B} & \interval \otimes B \ar[d]^{k} \\
  & B} 
\end{equation}
commute. With reference to \cref{def:homotopy-equivalence}, the equations in~\eqref{equ:first-three} provide right endpoint for $h$, 
right endpoint for $k$, and strength for $h$, respectively; while the equations in~\eqref{equ:second-two} provide left endpoints for~$h$ and~$k$, respectively.
\end{proof}

\cref{strong-h-equiv-as-section} entails the following closure properties of strong homotopy equivalences, which are obtained working entirely at the level of arrow categories.

\begin{proposition}
\label{strong-h-equiv-closed-under-monoidal-prod}
If either $f$ or $g$ is a left (respectively, right) strong homotopy equivalence, then so is $f \hatotimes g$.
\end{proposition}

\begin{proof}
Apply \cref{strong-h-equiv-as-section} and use that functors (in this case the Leibniz monoidal product in one variable) preserve sections.
\end{proof}

\begin{proposition}
\label{strong-h-equiv-closed-under-retract}
Left or right strong homotopy equivalences are closed under retracts.
\end{proposition}

\begin{proof}
Use \cref{strong-h-equiv-as-section},  that functors preserve sections, and that  sections are closed under retracts.
\end{proof}



\section{Orthogonality functors}
\label{sec:ortf}

For this section, fix a category $\catE$. Given a set of morphisms $\cal{A} \subseteq \catE^\to$, we 
write $\liftr{\cal{A}}$ to be the class of morphisms of $\catE$ that have 
the right lifting property with respect to all the morphisms in~$\cal{A}$. Dually, we write $\liftl{\cal{A}}$ for the class of morphisms of $\catE$ that have the left lifting property with respect to all the morphisms of $\cal{A}$. 
Here, we shall be interested in algebraic counterparts of these notions. Instead of starting from a subset of $\catE^\to$, we consider a category $\cal{A}$, to be thought of an indexing category, and a functor $U \co \cal{A} \to \catE^\to$, which assigns an arrow $U a$ in $\catE$ to each index $a \in \cal{A}$.


 \begin{definition} Let $U \co \cal{A} \to \catE^\to$ be a functor. An \emph{$\cal{A}$-injective map}
 is a map $p \co X \to Y$ equipped with a right  $\cal{A}$-lifting structure, \ie 
 a function  $\phi$ that assigns to each $a \in \cal{A}$ and commuting square
\[
\xymatrix@C=2cm{
A \ar[r]^{h}   \ar[d]_{U a} & X \ar[d]^p \\
B \ar[r]_{k} & Y}
\]
a diagonal filler $\phi(a, h, k) \co B \to D$, satisfying the following condition: for every $\sigma \co a \to a'$ in $\cal{A}$, 
with image under $U$ given by
\[
\xymatrix{
A \ar[r]^s \ar[d]_{Ua} & A' \ar[d]^{U a'}  \\
B \ar[r]_{t} & B'}
\]
we have that $\phi(a', h, k) \cc t = \phi(a, hs, kt)$, \ie the inner triangle in the diagram
\[
\xymatrix@C=2cm@R=1.5cm{
A \ar[r]^s \ar[d]_{Ua} & A' \ar[r]^{h}  & X \ar[d]^p   \\
B \ar[r]_{t}  \ar[urr]^{\phi(a, hs, k t) \quad } & B'  \ar[r]_k  \ar[ur]_{\phi(a', h, k)} & Y }
\]
commutes. 
\end{definition}


There is a dual notion of left lifting structure, which we do not spell out. Given $U \co \cal{A} \to \catE^\to$, we define the category $\liftr{\cal{A}}$ and the functor   $\liftr{U} \co \liftr{\cal{A}} \to \catE^\to$ as follows. The objects of
$ \liftr{\cal{A}}$ are pairs $(g, \phi)$ consisting of an arrow $g \co C \to D$ and a right lifting structure for $g$.
The arrows $\alpha \co (g, \phi) \to (g', \phi')$ are squares $\alpha \co g \Rightarrow g'$ in $\cal{A}$ satisfying an appropriate compatibility condition with respect to the right lifting structures. The functor $\liftr{U} \co \liftr{\cal{A}} \to \catE^\to$ in then given by the evident forgetful functor.  Just as the standard orthogonality operations determine a Galois connection between the poset of subsets of arrows in $\catE$ and its opposite, the functors orthogonality functors form an adjunction 
\begin{equation}
\label{garner-adjunction}
\begin{gathered}
\xymatrix@C+2em{
  \CAT/\catE^{\to}
  \ar@<5pt>[r]^-{\liftl{\brarghole}}
  \ar@{}[r]|-{\bot}
&
  (\CAT/\catE^{\to})^{\op}
  \ar@<5pt>[l]^-{\liftr{\brarghole}}
}
\end{gathered}
\end{equation}

The extra generality obtained by allowing $U \co \cal{A} \to \catE^\to$ to be a functor rather than merely the 
inclusion of a  subcategory will be very important for our purposes, as the next construction and definitions
show. Given an interval $\interval = (\interval, \intervall, \intervalr)$ and a functor $U \co \cal{A} \to \catE^\to$,
we define a category~$\cal{A}_\interval$ and a functor $U_\interval \co \cal{A}_\interval \to \catE^\to$ as follows. First of all, we define $\cal{A}_\interval$ as the coproduct
 \[
 \cal{A}_\interval  \defeq \cal{A} + \cal{A} \, .
 \] 
 Next, let $U_\intervall \co \cal{A} \to \catE^\to \, , \; U_\intervalr \co \cal{A} \to \catE^\to$ be given by
 \[
U_\intervall(a) \defeq  \intervall \hatotimes U(a) \, , \quad
U_\intervalr(a) \defeq  \intervalr \hatotimes U(a) \, , 
\]
for $a \in \cal{A}$, where we used the pushout product on $\catE^\to$. The functor $U_\interval \co \cal{A}_\interval \to \catE^\to$ is then given by the coproduct diagram
\[
\xymatrix@C=1.2cm{
\cal{A} \ar[r] \ar[dr]_-{U_\intervall} & \cal{A}_\interval \ar[d]^(.4){U_\interval} & \cal{A} \ar[dl]^-{U_\intervalr} \ar[l] \\ 
 & \catE^\to }
\]
Note that, even if $U \co \cal{A} \to \catE^\to$ is an inclusion, $U_\interval \co \cal{A}_\interval \to \catE^\to$ is not.


\begin{definition} \label{A-fibration} Let  $\interval = (\interval, \intervall, \intervalr)$ be an interval and
 $U \co \cal{A} \to \catE^\to$ a functor.
A \emph{uniform $\cal{A}$-fibration} is a map $p \co X \to Y$ equipped with a right $\cal{A}_\interval$-lifting structure, \ie a function that assigns diagonal fillers to all diagrams of the form
\[
\xymatrix{
\bullet \ar[r] \ar[d]_{U(a) \otimes \intervall} & X \ar[d]^p \\
B \otimes I \ar[r] & Y} \qquad \xymatrix{
\bullet \ar[r] \ar[d]_{U(a) \otimes \intervalr} & X \ar[d]^p \\
B \otimes I \ar[r] & Y}
\]
where $a \in \cal{A}$ is mapped to $Ua \co A \to B$, subject to the naturality condition.
\end{definition}

The notion of a trivial Kan fibration and of a Kan fibrations will be defined as special cases of the notions of
$\cal{A}$-injective map and $\cal{A}$-fibration, respectively. Some of their properties can be established 
simoultaneously, since $\cal{A}$-fibrations are defined as  $\cal{A}_\interval$-injective maps.
In the remainder of this section, we extend some useful facts about orthogonality operations to orthogonality functors. For this, we work with a fixed category $\catE$, without assuming the presence of an interval. 


\begin{proposition}
Consider a natural transformation between categories over $\catE^{\to}$:
\[
\xymatrix{
  \cal{A}
  \rrtwocell_G^F{\sigma}
 \ar[dr]_{U}
&&
  \cal{B}
  \ar[dl]^{V}
\\&
  \catE^{\to}
}
\]
Note that this includes the condition $V \sigma = \id_U$.
Then $\liftr{F} = \liftr{G}$ and $\liftl{F} = \liftl{G}$, 
\begin{mathpar}
\xymatrix{
  \liftr{\cal{A}}
  \ar[dr]_{\liftr{U}}
&&
  \liftr{\cal{B}}
  \lltwocell_{\liftr{F}}^{\liftr{G}}{=}
  \ar[dl]^{\liftr{V}}
\\&
  \catE^{\to}
}
\and
\xymatrix{
  \liftl{\cal{A}}
  \ar[dr]_{\liftl{U}}
&&
  \liftl{\cal{B}}
  \lltwocell_{\liftl{F}}^{\liftl{G}}{=}
  \ar[dl]^{\liftl{V}}
\\&
  \catE^{\to}
}
\end{mathpar}
\end{proposition}

\begin{proof} The image of $(g, \psi) \in \liftr{\cal{B}}$ under $\liftr{F}$ is $(g, \psi_F)$ and its
image under $\liftr{G}$ is~$(g, \psi_G)$. But the functions $\psi_F$ and $\psi_G$ coincide since, for
every $a \in \cal{A}$, we have that $\sigma_a \co VFa \Rightarrow VGa$ is the identity square on $Ua$. 
By the naturality condition for $\psi$, applied to the diagram 
\[
\xymatrix{
A \ar[r]^{\id_A} \ar[d]_{VFa}  & A \ar[d]^{VGa} \ar[r]^{h}  & C \ar[d]^g \\
B \ar[r]_{\id_B} & B \ar[r]_{k} & D \, ,}
\]
we have  that $\psi_F(a, h, k) = \psi_G(a, h, k)$, as required.
\end{proof} 

\medskip

We now consider the effect of adjoint functors on orthogonality. In the standard setting, it is well known that if 
we have classes of maps $\cal{A} \subseteq \cal{E}^\to$ and $\cal{B} \subseteq \cal{F}^\to$ and an adjunction
\[
\xymatrix@C+1em{
  \cal{E}
  \ar@<5pt>[r]^{F}
  \ar@{}[r]|{\bot}
&
  \cal{F}
  \ar@<5pt>[l]^{G}
}
\]
then $F(\cal{A}) \subseteq \liftl{\cal{B}}$ if and only if $\cal{A} \subseteq G(\cal{B})$. Our next lemma provides the counterpart of this fact in our setting.




\begin{proposition} \label{lift-of-adjunction} 
Let $U : \cal{A} \to \cal{E}^{\to}$ and $V : \cal{B} \to \cal{F}^{\to}$ be functors and consider an adjunction
\[
\xymatrix@C+1em{
  \cal{E}
  \ar@<5pt>[r]^{F}
  \ar@{}[r]|{\bot}
&
  \cal{F}
  \ar@<5pt>[l]^{G}
}
\]
Then, the following are equivalent:
\begin{enumerate}[(i)] 
\item the  functor $F \co \cal{E}^\to \to \cal{F}^\to$ extends to a functor $F' \co \cal{A} \to \liftl{\cal{B}}$ making the following diagram commute:
\[
\xymatrix@C=1.2cm{
  \cal{A}
  \ar[r]^{F'}
  \ar[d]_{U}
&
  \liftl{\cal{B}}
  \ar[d]^{\liftl{V}}
\\
  \cal{E}^{\to}
  \ar[r]_-{F}
&
  \cal{F}^{\to}\, ,}
\]
\item the functor $G \co \cal{F}^\to \to \cal{E}^\to$ extends to a functor $G' \co \cal{B} \to \liftr{\cal{A}}$, making the following diagram commute:
\[
\xymatrix{
  \liftr{\cal{A}}
  \ar[d]_{\liftr{U}}
&
  \cal{B}
  \ar[l]_{G'} 
  \ar[d]^{V}
\\
  \cal{E}^{\to}
&
  \cal{F}^{\to}
  \ar[l]^{G}
}
\]
\end{enumerate}
\end{proposition}

\begin{proof} Giving a functor $F' \co \cal{A} \to \liftl{\cal{B}}$ as above is the same thing as giving fillers for squares of the form
\[
\xymatrix{
FA \ar[d]_{FUa} \ar[r] & C \ar[d]^{Vb} \\
FB \ar[r] & D }
\]
natural in $a  \in \cal{A}$ and $b \in \cal{B}$. Similarly, giving a functor $G' \co \cal{B} \to \liftl{\cal{A}}$ as above is the same thing as giving fillers for squares 
of the form
\[
\xymatrix{
A \ar[d]_{Ua} \ar[r] & GC \ar[d]^{GVb} \\
B \ar[r] & GD }
\]
 natural in $a \in \cal{A}$ and $b \in \cal{B}$. Since $F \dashv G$, these situations coincide.
\end{proof}

The next corollary applies \cref{lift-of-adjunction} to the notion of an $\cal{A}$-fibration, introduced
in \cref{A-fibration}.


\begin{corollary} Let $\interval = (\interval, \intervall, \intervalr)$ be an interval in $\catE$ and
 $U \co \cal{A} \to \catE^\to$ be a functor. For every map $p \co X \to Y$ in $\cal{E}$ 
 the following are equivalent: 
\begin{enumerate}[(i)]
\item the map $p$ admits a right $\cal{A}_I$-lifting structure. 
\item the maps $\hatexp(\intervall, p)$ and $\hatexp(\intervalr, p)$ admit right $\cal{A}$-lifting structures.
\end{enumerate} 
Thus, a map $p$ is an $\cal{A}$-fibration exactly if $\hatexp(\intervall, p)$ and $\hatexp(\intervalr, p)$ are 
$\cal{A}$-injective maps.
\end{corollary}


\medskip


We now consider slicing. In the classical setting it is well-known that the right orthogonality operation commutes with slicing, while the left orthogonality operation commutes with coslicing.  In order to provide a counterpart of this fact in our setting, we need some auxiliary definitions. Given a functor $U : \cal{A} \to \catE^{\to}$ and $X \in \catE$.  we define the category $\cal{A}/X$
and a functor $U/X \co \cal{A}/X \to (\cal{E}/X)^\to$ as follows. The category $\cal{A}/X$ has as objects pairs consisting of an object $a \in \cal{A}$ and a commutative triangle of the form
\[
\xymatrix{
A \ar[dr] \ar[rr]^{Ua} & & B  \ar[dl] \\
 & X }
 \]
The functor $U/X$ sends such a pair to $Ua \co A \to B$, viewed as a morphism in $\cal{E}/X$. This category fits into the
following pullback diagram:
\[
\xymatrix{
  \cal{A}/X
  \ar[r]
  \ar[d]_{U/X}
  \pullback{dr}
&
  \cal{A}
  \ar[d]^{U}
\\
  (\catE/X)^{\to}
  \ar[r]
&
  \catE^{\to}
}
\]
where we used the functor on arrow categories induced by $\catE/X \to \catE$ forgetting the slicing information.  Dually, taking the strict pullback along the map on arrows induced by the forgetful functor $X/\cal{E} \to \catE$ constructs the \emph{coslice} over $X$:
\[
\xymatrix{
  X/\cal{A}
  \ar[r]
  \ar[d]_{X/U}
  \pullback{dr}
&
  \cal{A}
  \ar[d]^{U}
\\
  (X/\catE)^{\to}
  \ar[r]
&
  \catE^{\to}
}
\]
which also admits an explicit description dual to the one given above for $\cal{A}/X$. With these definitions in place, we can now state the counterpart in our setting of the familiar commutation between slicing and orthogonality operations. 



\begin{proposition} \hfill 
\label{pitchfork-slicing}
\begin{enumerate}[(i)]
\item The right orthogonality functor commutes with slicing, \ie for every $U \co \cal{A} \to \cal{E}$, we have
\[
  (\liftr{\cal{A}})/X = \liftr{(\cal{A}/X)}
\]
as categories over $\cal{E}^\to$.
\item The left orthogonality functor commutes with coslicing, \ie for every $U \co \cal{A} \to \cal{E}$, we have
\[
 \liftl{\cal{A}} \backslash X = \liftl{\cal{A} \backslash X}
\]
as categories over $\cal{E}^\to$.
\end{enumerate}
\end{proposition}

\begin{proof} We only consider (i). The claim follows by unfolding definitions, but we describe the objects of the category explicitly for clarity. They are given by 
tuples consisting of an arrow in~$\cal{E}/X$, 
\[
\xymatrix{
C \ar[dr] \ar[rr]^g  &  & D \ar[dl] \\
 & X & }
 \]
and a function $\phi$ that assigns a diagonal filler to every diagram in $\cal{E}$ of the form
\[
\xymatrix{
A \ar[r] \ar[d]_{Ua} & C \ar[d]^g \\
B \ar[r] & D}
\]
where $a \in \cal{A}$, subject to a uniformity condition. 
\end{proof}

The next corollary combines \cref{lift-of-adjunction} and \cref{pitchfork-slicing} to obtain 
a fact about the interaction of orthogonality functors with the pullback and pushforward
functors. This will be useful to establish our main result.

\begin{corollary}
\label{lift-dependent-product}
Let $p : Y \to X$ be a map in $\catE$ admitting pullback and pushforward:
\[
\xymatrix@C+1em{
  \catE/Y
  \ar@<5pt>[r]^{p_*}
  \ar@{}[r]|{\top}
&
  \catE/X
  \ar@<5pt>[l]^{p^*}
}
\]
Let $U : \cal{A} \to \catE^{\to}$ be a category over $\catE^{\to}$. The following are
equivalent:
\begin{enumerate}[(i)]
\item lifts of pushforward
\[
\xymatrix@C=1.5cm{
\liftr{\cal{A}}/Y
\ar[r]^{p_*}
  \ar[d]_{U/Y}
&
  \liftr{\cal{A}}/X
  \ar[d]^{U/X}
\\
  (\catE/Y)^{\to}
   \ar[r]_{p_*}
&
  (\catE/X)^{\to}
 }
\]
\item lifts of pullback
\[
\xymatrix@C=1.5cm{
  \cal{A}/X
   \ar[r]^{p^*}
  \ar[d]_{U/X} 
  &
  \liftl{ ( \liftr{\cal{A}}/Y ) }
  \ar[d]^{\liftl{(\liftr{U}/Y)}}
     \\
     (\catE/X)^{\to} \ar[r]_{p^*} &
   (\catE/Y)^{\to} 
}
\]
\item functors $F$ making the following diagram commute:
\[
\xymatrix@C=1.2cm@R=1.5cm{
\liftr{\cal{A}}/Y \ar[rr]^F \ar[dr]_{\liftr{U}/Y} & &  \liftr{\cal{A}}/ X \ar[dl]^(.4){\ \liftr{( (U/X) \cc p^*)}}  \\
 & (\cal{E}/Y)^\to & }
\]
\end{enumerate}
\end{corollary}

\begin{proof}
Recall from \cref{pitchfork-slicing} that slicing commutes with right orthogonality functor.
For the first correspondence, apply \cref{lift-of-adjunction} to the adjunction $p^* \dashv p_*$ with  $V = \liftr{U}$.
The last statement is simply the adjunction~\eqref{garner-adjunction}.
\end{proof}

\medskip

Next, we consider the interaction between the orthogonality functors and closure under retracts. In the ordinary setting, it is well-known that
applying the left (or right) orthogonality operation to a class of morphisms produces the same result as applying it to its retract closure. 
In order to establish a counterpart of this fact, we need again some definitions. 
Given a  functor $U : \cal{A} \to \catE^{\to}$, we define its retract closure $\overline{U} : \overline{\cal{A}} \to \catE^{\to}$ as follows. 
An object of $\overline{\cal{A}}$ is a tuple~$(a, e, \sigma, \tau)$ consisting of an object $a \in \cal{A}$, an arrow $e \in \cal{E}^\to$ together with squares $\sigma \co e \Rightarrow Ua$ and $\rho \co Ua \Rightarrow e$,
which exhibit $e$ as a retract of $U a$ in  $\catE^{\to}$,  \ie such that $\sigma \cc \rho = \id_e$. 
A morphism $(f, \kappa) \co (a, e, \sigma, \tau) \to (a', e', \sigma', \tau')$ of $\overline{\cal{A}}$  consists of a morphism $f \co a \to a'$ in $\cal{A}$ and a square $\kappa \co e \Rightarrow e'$  such that the following diagram in $\cal{E}^\to$ commutes:
\[
\xymatrix{
  e
  \ar[r]^{\sigma}
    \ar[d]_{\kappa}
&
  U a
  \ar[r]^{\rho}
  \ar[d]^{U f}
&
  e
  \ar[d]^{\kappa}
\\
  e'
  \ar[r]_{\sigma'}
&
  Ua'
  \ar[r]_{\rho'}
&
  e' \, .
}
\]
The functor $\overline{U} \co \overline{\cal{A}} \to \catE^\to$ is then defined  on objects  by letting $\overline{U}(a, e, \sigma, \tau) \defeq e$,
and on morphisms by letting $\overline{U}(f, \kappa) \defeq \kappa$. The operation of retract closure gives a monad: for $U : \cal{A} \to \catE^{\to}$,
the components of the multiplication and the unit, 
\[
\mu_\cal{A} \co \overline{\overline{\cal{A}}} \to \overline{\cal{A}} \, , \quad
\eta_\cal{A} \co \cal{A} \to \overline{\cal{A}} \, ,
\]
are defined by letting
\[
\mu_\cal{A}((a, e, \sigma,  \rho), e', \sigma', \rho') \defeq (a, e', \sigma \cc \sigma', \rho' \cc \rho) \, , \quad
\eta_\cal{A}(a) \defeq (a, Ua, \id_{Ua}, \id_{Ua}) \, .
\]




\begin{proposition}
\label{retract-closure}
The orthogonality functors send the components of the unit and multiplication of the retract closure monad into natural
isomorphisms, and so for every $U \co \cal{A} \to \catE^\to$, we have isomorphisms of categories
\begin{gather*} 
 \liftr{(\overline{\cal{A}})} \iso \liftr{\cal{A}} \, , \quad
 \liftr{(\overline{\overline{\cal{A}}})} \iso \liftr{\overline{\cal{A}}}  \qquad
 \liftl{(\overline{\cal{A}})} \iso \liftl{\cal{A}} \, , \quad
 \liftl{(\overline{\overline{\cal{A}}})} \iso \liftl{\overline{\cal{A}}}
\end{gather*} 
over $\catE^\to$. \qed
\end{proposition}




\begin{remark} Let $\ret$ denote the \emph{walking retract}, \ie the category with objects $\retA, \retB$ and morphisms generated by $s : \retA \to \retB$ and $r : \retB \to \retA$ under the relation $r \cc s = \id_{\retA}$. The retract closure of $U \co \cal{A} \to \catE^\to$ fits into the following diagram, involving strict pullback and left composition:
\[
\xymatrix@C+1em{
  \overline{\cal{A}}
  \ar[r]
  \ar[d]
  \ar@/_2em/[dd]_{\overline{U}}
  \pullback{dr}
&
  \cal{A}
  \ar[d]^{U}
\\
  (\catE^{\to})^{\ret}
  \ar[r]^-{(\catE^{\to})^{\retB}}
  \ar[d]^{(\catE^{\to})^{\retA}}
&
  \catE^{\to}
\\
  \catE^{\to}
}
\]
The unit of the monad is formally induced by $(\catE^{\to})^{\canonical} : \catE^{\to} \to (\catE^{\to})^{\ret}$ being a section to $(\catE^{\to})^{\retB}$.
\end{remark}


\begin{remark}
\label{retract-closure-slicing}
Taking the retract closure commutes with slicing and coslicing.
\end{remark}

\medskip

We conclude this section by considering the interaction between the orthogonality functors and  Kan extensions.




\begin{proposition} Let $F \co \cal{A} \to \cal{B}$ be a fully faithful functor. 
\label{kan-extension-closure}
\begin{enumerate}[(i)]
\item Assuming that the pointwise left Kan extension of 
$U \co \cal{A} \to \catE^{\to}$ along $F$ exists
\[
\xymatrix{
  \cal{A}
  \ar[dr]_{U}
  \ar[rr]^{F}
&&
  \cal{B}
  \ar[dl]^{\Lan_F U}
\\&
  \catE^{\to}
}
\]
then the functor $\liftr{F} \co \liftr{\cal{B}} \to \liftr{\cal{A}}$,  fitting in the diagram
\[
\xymatrix{
  \liftr{\cal{A}}
  \ar[dr]_{\liftr{U}}
&&
  \liftr{\cal{B}}
  \ar[ll]_{\liftr{F}}
  \ar[dl]^{\liftr{(\Lan_F U)}}
\\&
  \catE^{\to}
}
\]
is an isomorphism.
\item Assuming that the pointwise right Kan extension of 
$U \co \cal{A} \to \catE^{\to}$ along $F$ exists
\[
\xymatrix{
  \cal{A}
  \ar[dr]_{U}
  \ar[rr]^{F}
&&
  \cal{B}
  \ar[dl]^{\Ran_F U}
\\&
  \catE^{\to}
}
\]
then the functor $\liftl{F} \co \liftl{\cal{B}} \to \liftl{\cal{A}}$, fitting in the diagram
\[
\xymatrix{
  \liftl{\cal{A}}
  \ar[dr]_{\liftl{U}}
&&
  \liftl{\cal{B}}
  \ar[ll]_{\liftl{F}}
  \ar[dl]^{\liftl{(\Ran_F U)}}
\\&
  \catE^{\to}
}
\]
is an isomorphism. \qed
\end{enumerate}
\end{proposition}








\section{Trivial Kan fibrations}

Recall that classically a trivial Kan fibration, which is defined by right orthogonality with respect to arbitrary monomorphisms. We now introduce our counterpart of this notion. We say that a monomorphism 
$i \co A \to B$ is  \emph{decidable} if for every $n$, the function $i_n : A_n \to B_n$ has a decidable image. 
Note that the class of decidable monomorphisms is closed under base change. We define $\cal{M}$ as the full subcategory of $\SSet_\cart$ spanned by decidable monomorphisms. 

\begin{definition} We say that a map $p \co X \to Y$ of simplicial sets is a \emph{uniform trivial Kan fibration}
 if it can be equipped with a right lifting structure with respect to $\cal{M}$, \ie there exist a function that 
 assigns a diagonal filler to every diagram of the form
 \[
 \xymatrix{
 A \ar[r] \ar[d]_i & X \ar[d]^p \\
 B \ar[r] & Y}
 \]
 where $i$ is a decidable monomorphism, satisfying the naturality condition. 
 \end{definition} 
 
 Classically, the class of trivial Kan fibrations can be characterized also as being the right orthogonal class to the
set of boundary inclusions $i^n : \partial \Delta^n \to \Delta^n$. Analogously, in our setting, the category of uniform trivial uniform
Kan fibrations can also be characterized as the right orthogonal category of  a small category.
In order to do this, we return to consider the setting of Section~\ref{sec:ortf}, but making the
further assumption that $\catE$ is a presheaf category, \ie $\catE = \hat{\cat{C}}$, where $\catC$ is some small category. Our intention is to apply \cref{kan-extension-closure} to full subcategories of $\catE_{\cart}^{\to}$, the arrow category $\catE$ with cartesian squares as morphisms. For this, we need two general lemmas.

\begin{lemma}
\label{left-kan-extension-of-representables}
Let $\cal{B}$ be a full subcategory of $\catE_{\cart}^{\to}$ closed under base change to representables.
Let $\cal{A}$ denote its restriction to arrows into representables.
\[
\xymatrix{
  \cal{A}
  \ar[rr]
  \ar[dr]
&&
  \cal{B}
  \ar[dl]
\\&
  \catE^{\to}
}
\]
Then, the inclusion $\cal{B} \to \catE^{\to}$ is the left Kan extension of $\cal{A} \to \catE^{\to}$ along $\cal{A} \to \cal{B}$.
\end{lemma}


\begin{proof}
Since $\catE^{\to}$ is cocomplete, we can use the colimit formula for left Kan extensions to verify the claim.
All of the following will be functorial in an object $f \co X \to Y$ of $\cal{B}$.
We consider the diagram indexed by cartesian squares
\[
\xymatrix{
  U
  \ar[r]
  \ar[d]_{h}
  \pullback{dr}
&
  X
  \ar[d]^{f}
\\
  \hom(\arghole, M)
  \ar[r]
&
  Y
}
\]
with $h \in \cal{A}$ and valued $h$.
Our goal is to show that its colimit in $\catE^{\to}$ is $f$.
Using the assumption that $\cal{B}$ is closed under base change to representables, this is equivalently the diagram indexed by representables $s : \hom(\arghole, M) \to Y$ and valued $s^* i$.
Generalized to this level, the statement $\colim_{s : \hom(\arghole, M) \to Y} s^* f \simeq f$ holds since base change commutes with colimits and 
$\colim_{s : \hom(\arghole, M) \to Y} \hom(\arghole, M) \simeq Y$.
\end{proof}


\begin{remark} It would be of interest to prove \cref{left-kan-extension-of-representables} combining 
the codomain fibration and the corresponding left Kan extension claim for the codomain part
\[
\xymatrix{
  \cat{C}
  \ar[rr]^{y}
  \ar[dr]_{y}
&&
  \hat{\catC}
  \ar[dl]^{\id}
\\&
  \hat{\cat{C}}
}
\]
which holds by the co-Yoneda lemma.
\end{remark}



\begin{lemma}
\label{awfs-on-arrows-into-representables}
Let $\cal{B}$ be a full subcategory of $\catE_{\cart}^{\to}$ closed under base change to representables.
Let $\cal{A}$ denote its restriction to arrows into representables.
\[
\xymatrix{
  \cal{A}
  \ar[rr]
  \ar[dr]
&&
  \cal{B}
  \ar[dl]
\\&
  \catE^{\to}
}
\]
Then $\liftr{\cal{A}} = \liftr{\cal{B}}$.
\end{lemma}

\begin{proof} The result follows by combining \cref{left-kan-extension-of-representables} and part~(i) of \cref{kan-extension-closure}. 
\end{proof}


\begin{theorem} \label{small-gen-triv-kan}
The category of uniform trivial Kan fibrations is isomorphic to the right orthogonality 
category of the following full subcategories of $\cal{M}$, the category of decidable
monomorphisms and cartesian squares:
\begin{align*}
\cal{M}_1 & = \braces{i \co A \rightarrow \Delta^{n} \ | \ i \text{ is a  decidable monomorphism} } \\ 
\cal{M}_2  & = \braces{i \co A \rightarrow \Delta^{n_1} \times \ldots \times \Delta^{n_k} 
\ | \ i \text{ is a  decidable monomorphism} }  \\
\cal{M}_3  & = \braces{A \hookrightarrow B \mid \text{$B$ finite and finite-dim.}} \\
\end{align*}
\end{theorem}

\begin{proof}    \cref{awfs-on-arrows-into-representables} implies that we have that $\liftr{\cal{M}_1}  = \liftr{\cal{M}}$.
For the other equalities, observe that for every full subcategory $\cal{S} \subseteq \cal{M}$ containing $\cal{M}_1$ we have that $\liftr{\cal{M}} = \liftr{\cal{S}}$, since $\cal{M}_1$ is the restriction to maps into representables  of $\cal{M}$. 
\end{proof}

The classes of maps considered in~\cref{small-gen-triv-kan} have different advantages. For example, 
the class $\cal{A}_1$ is small, while other classes have better closure properties. 




\begin{corollary} There exists an algebraic weak factorisation system $(\cal{L}, \cal{R})$ in which
$\cal{R}$ is the category of uniform trivial Kan fibrations.
\end{corollary}

\begin{proof} Since the category $\cal{M}_1$ defined in~\cref{small-gen-triv-kan} is small,  
it is possible to apply Garner's small object argument to
obtain an algebraic weak factorisation system $(\cal{L}, \cal{R})$.
The fact that $\cal{R}$ is the category of uniform trivial Kan fibrations
 follows from \cref{small-gen-triv-kan}.
 \end{proof} 

Note that ~\cref{awfs-on-arrows-into-representables} cannot be used to show 
a map $p \co X \to Y$ of simplicial sets admits the structure of a uniform trivial Kan fibration if and only if it is a trivial Kan fibration in the usual sense since  the class of boundary inclusions  is contained strictly  in~$\cal{A}$. 




\section{Uniform Kan fibrations}




 In $\SSet$, for a horn inclusion $h^{k}_n \co 1 \to \Delta_1$ and  a monomorphism $i \co A \to B$
their pushout product $h^k_n \hattimes i$ is given by the following pushout diagram
\[
\xymatrix{
 A \ar[r]^{i}  \ar[d]_{h^k_n \times A} &  B \ar@/^1em/[ddr]^{h^k_1 \times B} \ar[d] & \\ 
\Delta_1 \times A \ar@/_1em/[drr]_{\Delta_1 \times i} \ar[r] & \bullet \ar[dr]^-(.35){h^k_1 \hattimes i}  & \\ 
 & & \Delta_1 \times B} 
 \]
 
 

\begin{definition} A map of simplicial sets $p \co X \to Y$ is said to be a \emph{uniform Kan fibration} if there exists
a function that provides filles for all diagrams of the form
\[
\xymatrix{
\bullet \ar[r] \ar[d]_{h^k_1 \hattimes i} & X \ar[d]^p \\
\Delta_1 \times B \ar[r] & Y}
\]
where $i \co A \to B$ is a decidable monomorphism and $k \in \{0, 1\}$. 

\end{definition} 




We now show that, working classically, a map $p \co X \to Y$ is a uniform (trivial) Kan fibration if and only
if it is a (trivial) Kan fibration in the usual sense. 
 The rationale for omitting higher dimensional horns is that those (at least in bare form) are indirectly included as retracts of one dimensional horns by Leibniz product with certain subobjects of representables. 





We compare with the classical notions. For this, we introduce an auxiliary definition. A \emph{regular trivial fibration} is a map $p : X \to Y$ with designated fillers $d$ for squares $i^n \to p$ such that whenever the square factors as shown below,
\[
\xymatrix{
  \partial \Delta^n
  \ar[r]
  \ar[d]^{i^n}
&
  \Delta^{n-1}
  \ar[r]
  \ar[d]^{\id}
&
  X
  \ar[d]^{p}
\\
  \Delta^n
  \ar[r]_{s_k^{n-1}}
  \ar@{.>}[urr]^(0.3){d}
&
  \Delta^{n-1}
  \ar[r]
&
  Y
}
\]
the composite filler is coherent with respect to the trivial filler in the right square. Regular trivial fibrations are the right class for the algebraic weak factorization system with generating left category having objects $\cal{I}_1 \cup \braces{\id : \Delta^n \to \Delta^n}$ and non-trivial morphisms as above:
\[
\xymatrix{
  \partial \Delta^n
  \ar[r]
  \ar[d]^{i^n}
&
  \Delta^{n-1}
  \ar[d]^{\id}
\\
  \Delta^n
  \ar[r]_{s_k^{n-1}}
&
  \Delta^{n-1}
}
\]

Classically, every ordinary trivial fibration is a regular trivial fibration, simply because we can choose (using the axiom of choice) designated fillers for $i^n \to p$ based on (using excluded middle) whether that square factors through some $s_k^{n-1} : i^n \to \id$.
Note that it does not matter which degeneracy we choose if multiple are available --- the resulting diagonal filler will cohere with all possible choices.

\begin{lemma}
Every regular trivial fibration is an $\cal{I}_2$-trivial fibration.
\end{lemma}

\begin{proof}
For the rest of the proof, fix a regular trivial fibration $p : X \to Y$.
Consider a square $i \to p$ with $i : \cal{I}_2$.
We define a diagonal filler by decomposing $i$ into a finite composition of cobase changes of boundary inclusions, filling each of these according to the previous paragraph.
Crucially, this process is independent of the actual order of the boundary fillings (note that this is not true for the analogous situation of horn fillings).
For each morphism in $\cal{I}_2$ and commuting triangle of squares
\[
\xymatrix{
  A'
  \ar[r]
  \ar[d]
  \pullback{dr}
&
  A
  \ar[d]
  \ar[r]
&
  X
  \ar[d]
\\
  \Delta^{n'}
  \ar[r]
  \ar@{.>}[urr]
&
  \Delta^n
  \ar[r]
  \ar@{.>}[ur]
&
  Y
}
\]
we need to exhibit coherence of fillers as indicated.
By ``vertical'' induction and the remark on order invariance of boundary fillings, it will suffice to study the case where the middle vertical map is a boundary inclusion $i^n : \partial \Delta^n \to \Delta^n$.
Working ``horizontally'', it suffices to study the situation where the map $\Delta^{n'} \to \Delta^n$ is a face or degeneracy map as $\Delta$ is generated by these.

Let us first examine the case of a face operation.
\[
\xymatrix{
  \Delta^n
  \ar[r]
  \ar[d]
  \pullback{dr}
&
  \partial \Delta^{n+1}
  \ar[d]
  \ar[r]
&
  X
  \ar[d]
\\
  \Delta^n
  \ar[r]_{d_k^{n+1}}
  \ar@{.>}[urr]
&
  \Delta^{n+1}
  \ar[r]
  \ar@{.>}[ur]
&
  Y
}
\]
Since the left vertical map is necessarily the identity, the filler for the composite square is uniquely determined, so there is no coherence to be verified.

Let us now examine the case of a degeneracy operation.
\[
\xymatrix{
  2 \times \partial \Delta^n
  \ar[r]
  \ar[d]
  \ar@/^2em/[rr]^(0.3){\pi_2}
  \pullback{dr}
&
  \bigcup_{i \neq k, k+1} \Delta^{[n+1] - i}
  \ar[r]
  \ar[d]
  \pullback{dr}
&
  \partial \Delta^n
  \ar[d]
  \ar[r]
&
  X
  \ar[dd]
\\
  2 \times \Delta^n
  \ar[r]
  \ar@/_2em/[rr]_(0.3){\pi_2}
&
  \partial \Delta^{n+1}
  \ar[r]
  \ar[d]
  \ar@{.>}[urr]
  \pullback{ul}
&
  \Delta^n
  \ar[d]
  \ar@{.>}[ur]
\\&
  \Delta^{n+1}
  \ar[r]_{s_k^n}
  \ar@{.>}[uurr]
&
  \Delta^n
  \ar[r]
  \ar@{.>}[uur]
&
  Y
}
\]
The pullback of the boundary inclusion $\partial \Delta^n \to \Delta^n$ along $s_k^n$ decomposes as a cobase change of two parallel boundary inclusions of dimension $n$ followed by a boundary inclusion of dimension $n+1$ as indicated.
The two parallel boundary fillings are identical copies of the original right square boundary filling, so they cohere as indicated.
Finally, the filling for the boundary inclusion $\partial \Delta^{n+1} \to \Delta^{n+1}$ coheres as indicated by how boundary filling was originally defined for degenerate squares.
\end{proof}











\begin{lemma}
Classically, every ordinary fibration is an $\cal{I}_5$-fibration.
\end{lemma}

\begin{proof}
Let $p : X \to Y$ be an ordinary fibration.
Then $\hatexp(h_k^1, p)$ is an ordinary trivial fibration for $k = 0, 1$.
Classically, it can be made into regular trivial fibration.
By previous lemmata, we then have $\hatexp(h_k^1, p)$ an $\cal{I}_2$-trivial and $\cal{I}_5$-trivial fibration.
But that means $p$ is an $\cal{I}_5$-fibration.
\end{proof}


\section{Strong homotopy equivalences}

\medskip


Recall the notion of an adhesive morphism~\cite{garner-lack:adhesive}.  Let $U : \cal{A} \to \catE^{\to}$ be a subcategory of adhesive morphisms in $\catE$ with morphisms given by cartesian squares. Assume that the subcategory $\cal{A}$ is closed under the monoidal operations of $\catE^{\to}$, \ie that $(\cal{A}, \hatunit, \hatotimes)$ is itself a monoidal category and $U$ preserves the monoidal structure on the nose. 

\newpage









\bibliographystyle{plain}
\bibliography{../../common/uniform-kan-bibliography}

\end{document}

\documentclass[reqno,10pt,a4paper,oneside]{amsart}

\usepackage{uniform-kan-prelude}

\title{Dependent Product for Algebraic Fibrations}

\begin{document}

\begin{abstract}
We give a categorical account of algebraic fibrations being preserved under certain dependent product.
This generalizes work by Coquand \etal.
\end{abstract}

\maketitle

\tableofcontents

\section{Algebraic Weak Factorization Systems}

\subsection{Review}

Recall the notion of \emph{algebraic weak factorization system} from~\cite{garner:small-object-argument} (there called \emph{natural weak factorization system}).

\subsection{Elementary Properties}

\paragraph{Slicing}

\paragraph{Retract closure}

\paragraph{Closure under left Kan extension}

\paragraph{Lifts of adjoint functors}

\section{Generalizing to Restricted Lifting Problems}

\subsection{Definitions}

\paragraph{Box category}

\paragraph{Extended Pitchfork Adjunction}

\subsection{Elementary Properties}

\paragraph{Slicing}

\paragraph{Left Retract closure}

\paragraph{Closure under left Kan extension}

\paragraph{Lifts of adjoint functors}

\section{General Facts about Dependent Product}

\subsection{For Lifting Problems}

\subsection{For Restricted Lifting Problems}

\section{Dependent Product for Algebraic Fibrations}

\subsection{The Setting}

\subsection{Trivial Algebraic Fibrations}

Awfs generated by (decidable) subobjects of representables.
Also algebraically free on (decidable) monomorphisms.
TODO: Terminology implies we have a model structure.
Haven't talked about that yet.

\subsection{Algebraic Fibrations}

Awfs generated by Leibniz construction of (decidable) subobjects of representables with interval endpoint inclusions.
Also algebraically free on Leibniz construction of (decidable) monomorphisms with interval endpoint inclusions.

\subsection{Algebraic Fibrations via Composition}

\paragraph{Composition as restricted lifting problem}

Awfs generated by Leibniz construction of (decidable) subobjects of representables with theta squares.
Also algebraically free on Leibniz construction of (decidable) monomorphisms with theta squares.

\paragraph{Connections induce isomorphism of right categories}

Use characterization of connection in terms of interval endpoint inclusions being strong homotopy equivalences.
This implies theta squares are sections in the arrow category.
Use retract closure from first section.

\subsection{Dependent Product for Algebraic Fibrations}

Let pseudo-fibrations be right maps for representables times interval endpoint inclusions.
Also right map for any presheaf times interval endpoint inclusions.
Show that base change along pseudo-fibrations functorially maps theta squares to left retract closures of theta squares.
By earlier elementary lemmata and previous subsection, it follows that Pi along pseudo-fibrations lifts to functor between right categories.
Verify BC.

\bibliographystyle{plain}
\bibliography{../../common/uniform-kan-bibliography}

\end{document}

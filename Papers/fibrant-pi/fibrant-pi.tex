\documentclass[reqno,10pt,a4paper,oneside]{amsart}

\setcounter{tocdepth}{1}

\usepackage{uniform-kan-prelude}




\title{On the pushforward of uniform Kan fibrations in simplicial and cubical sets}

\begin{document}

\begin{abstract}
We prove a  constructive counterpart of the fact that,
in the category of simplicial sets, the pushforward functor along a Kan fibration preserve Kan fibrations. 
In order to do so, we develop further the theory of algebraic weak factorisation systems and 
 introduce the notion of a uniform Kan fibration in simplicial sets. Our
general approach subsumes also the existing work on pushforward along
a Kan fibration in cubical sets.
\end{abstract}

\author{Nicola Gambino}
\address{School of Mathematics, University of Leeds, Leeds LS2 9JT, UK}
\email{n.gambino@leeds.ac.uk}

\author{Christian Sattler}
\address{School of Mathematics, University of Leeds, Leeds LS2 9JT, UK}
\email{c.sattler@leeds.ac.uk}

\date{\today}

\maketitle

\tableofcontents


\section{Introduction}

This paper contributes to the ongoing efforts to define a constructive version of Voevodsky's simplicial model of univalent foundations~\cite{voevodsky-simplicial-model}. We focus on the classical result that, in the category of simplicial sets, the pushforward functors (\ie the right adjoints to the pullback functors) along a Kan fibrations preserve Kan fibrations,
which is necessary to define the interpretation of $\Pi$-types in the simplicial model. Remarkably,
this result cannot be proved constructively~\cite{coquand-non-constructivity-kan}, even though it follows from apparently basic facts of simplicial homotopy theory. In order to improve on this situation, Bezem, Coquand and Huber are developing a new model of 
univalent foundation using cubical, rather than simplicial, sets~\cite{coquand-cubical-sets}. In particular, they introduced the notion of a \emph{uniform Kan fibration} and shown that, in the category of cubical sets,  
the pushforward functor along a uniform Kan fibrations preserves uniform Kan fibrations. Because of these results, one may get the impression that it is necessary to switch from simplicial sets to
cubical sets in order to obtain constructive results.


Our aim here is to show that this is not the case: we will introduce a a notion of uniform Kan fibration in simplicial sets and prove that the pushforward functor along a uniform Kan fibrations preserves uniform Kan fibrations, working constructively. Furthermore, we will show that, working in ZFC, a map of simplicial sets is a uniform Kan fibration if and only if it is a Kan fibration in the usual sense, thus showing that our result is classically equivalent to the standard one. Our approach is inspired by some aspects of the development of cubical sets and by the theory of natural weak factorisation systems~\cite{grandis-tholen-nwfs} and algebraic weak factorisation systems~\cite{garner:small-object-argument}. In fact, we will work at a level of generality that allows us to obtain as a special case also the corresponding statement for cubical sets proved in~\cite{coquand-cubical-sets}. \medskip


In order to discuss our proof in more detail, let us briefly recall the classical proof of the fact that the pushforward functor along a Kan fibration preserves Kan fibrations. First of all, recalling the category of simplicial sets admits
a weak factorisation system $(\mathcal{L}, \mathcal{R})$, in which $\mathcal{L}$ is the class of trivial cofibrations
 (\ie maps of simplicial sets that are both weak equivalences and monomorphisms) and $\mathcal{R}$ is the class
 of Kan fibrations. By standard facts on orthogonality classes, for a Kan fibration $p \co X \to Y$, the 
 pushforward functor $p_* \co \SSet/X \to \SSet/Y$ preserves Kan fibrations if and only if its left adjoint,
 the pullback functor $p^* \co \SSet/Y \to \SSet/X$,  preserves trivial cofibrations.
 
This latter fact is then proved in two ways. One way is to combine the fact that the model structure on simplicial sets is right proper (\ie that pullbacks along Kan fibrations preserve weak equivalences) with the fact that pullbacks preserve monomorphisms.
Alternatively, one can show that the pullback of one of the generating arrows for trivial cofibrations




\newpage


\section{Right maps and fibrations}
\label{sec-orthog-functors}


 Given a set of morphisms $\cal{C}$ of $\catE$, we 
write $\liftr{\cal{C}}$ to be the class of morphisms of $\catE$ that have 
the right lifting property with respect to all the morphisms in~$\cal{C}$. Dually, we write $\liftl{\cal{C}}$ for the class of morphisms of $\catE$ that have the left lifting property with respect to all the morphisms of $\cal{C}$. 
Here, we shall be interested in algebraic counterparts of these notions. Instead of starting from a subset of $\catE^\to$, we consider a category $\cal{C}$, to be thought of an indexing category, and a functor $u \co \cal{C} \to \catE^\to$, which assigns an arrow $u_i \co A_i \to B_i$ in $\catE$ to each index $i \in \cal{C}$.


 \begin{definition} Let $u \co \cal{C} \to \catE^\to$ be a functor. 
 \begin{enumerate}[(i)] 
 \item  A \emph{right $\cal{C}$-map}
 is a pair $(f, \phi)$ consisting of a map $f \co X \to Y$ in $\cal{E}$ and a right lifting function~$\phi$ for $\cal{C}$, \ie 
 a function that assigns to each $i \in \cal{C}$ and commuting square
\[
\xymatrix@C=2cm{
A_i \ar[r]^{s}   \ar[d]_{u_i} & X \ar[d]^f \\
B_i \ar[r]_{t} & Y}
\]
a diagonal filler $\phi(i,s, t) \co B_i \to X$, satisfying the following naturality 
condition: for every diagram of the form
\[
\xymatrix{
A_i \ar[r]^a \ar[d]_{u_i} & A_j \ar[r]^{s}  \ar[d]_{u_j} & X \ar[d]^f   \\
B_i \ar[r]_{b}  & B_j  \ar[r]_{t}  & Y }
\]
where the left-hand side square is the image of $\sigma \co i \to j$ in $\cal{C}$ under $u$, 
we have that 
\[
\phi(j, s, t) \, b = \phi(i, s  a, t  b) \, .
\]
\item A \emph{morphism} of right $\cal{C}$-maps $\alpha \co (f, \phi) \to (f', \phi')$ is a 
square $\alpha \co f \to f'$ in~$\catE$ satisfying an evident compatibility condition 
with respect to the right lifting functions, which we omit. 
\end{enumerate}
\end{definition}

For a functor $u \co \cal{C} \to \catE^\to$, we write $\liftr{\cal{C}}$ for the category  of 
right $\cal{C}$-maps and their morphisms. There is a forgetful functor~$\liftr{u} \co \liftr{\cal{C}} \to \catE^\to$
mapping $(f, \phi)$ to $f$. Note that the naturality condition becomes vacouos when $\cal{C}$ is a discrete category. Even in this case, however, in order to have a right $\cal{C}$-map it is necessary to define a function $\phi$ providing 
diagonal fillers, and this is sometimes a non-trivial problem without assuming the axiom of choice. 

\begin{example}  \label{exa-triv-kan-fib}
Let $\cal{M}$ be the full subcategory of $\SSet^\to$ spanned by decidable mo\-no\-mor\-phisms, \ie
mo\-no\-mor\-phisms $i \co A \to B$ such that, for all $n$, the function $i_n \co A_n \to B_n$ has decidable
image. A \emph{uniform trivial Kan fibration} is defined to be a right $\cal{M}$-map, \ie 
a map $f \co X \to Y$ of simplicial sets equipped with a function $\phi$
that assigns to every decidable monomorphism $i \co A \to B$ and commuting square 
 \[
 \xymatrix{
 A \ar[r]^s \ar[d]_i & X \ar[d]^f \\
 B \ar[r]_t & Y}
 \]
a diagonal filler $\phi(i, s, t) \co B \to X$, subject to the following naturality condition: for every 
diagram 
\[
\xymatrix{
A \ar[r]^{h} \ar[d]_{i} & C \ar[d]^{j}  \ar[r]^s & X \ar[d]^f \\
B \ar[r]_{k} & D \ar[r]_t & Y }
\]
where the left-hand side square is a pullback, we have that $\phi(j, s, t) \, k = \phi(i, s  h, t  k)$.
\end{example}

As we explain below, the additional generality obtained by allowing $u \co \cal{C} \to \catE^\to$ to be a functor rather than a mere inclusion of a  subcategory will be very useful. 

\medskip



Let us now assume that $\catE$ is equipped with symmetric monoidal structure $(\catE, \otimes, \unit, \sigma)$.
Note that we write $\unit$ for the unit of the tensor product. The initial and terminal objects of $\catE$, when assumed to exist, will be denoted by $\initial$ and $\terminal$, respectively.

\begin{definition} An \emph{interval} in $\catE$ is a tuple $\interval = (\interval, \intervall, \intervalr, \varepsilon)$ consisting of  an object $\interval \in \catE$, 
and morphisms $\intervall, \intervalr \co \unit \to \interval$, called the \emph{left endpoint} and
\emph{right endoint} of $\interval$, respectively, 
 and $\intervalc \co \interval \to \unit$,   such that the following diagrams commute:
\[
\xymatrix{
\unit \ar[r]^\intervall \ar[dr]_{\id_\unit} & \interval \ar[d]^-{\varepsilon} & \unit \ar[dl]^{\id_\unit} \ar[l]_{\intervalr}  \\
 & \unit & }
 \]
\end{definition}

Let us now fix an interval $\interval = (\interval,  \intervall, \intervalr, \varepsilon)$ in $\catE$ assume that $\catE$ is finitely cocomplete. Recall from~\cite[Section 4]{riehl-verity:reedy} that, given $f \co X \to Y$ and $g \co U \to V$, their \emph{pushout product} is the arrow $f \hatotimes g$ fitting in the following pushout diagram:
\[
\xymatrix@R=1.2cm{
X \otimes U \ar[r]^{X \otimes g}  \ar[d]_{f \otimes U} & X \otimes V \ar@/^2pc/[ddr]^{f \otimes V} \ar[d] & \\ 
Y \otimes U \ar@/_2pc/[drr]_{V \otimes g} \ar[r] & (Y \otimes U) +_{X \otimes U} (X \otimes V) \ar[dr]^-(.35){f \hatotimes g}  & \\ 
 & & Y \otimes V} 
 \]
This operation extends to a bifunctor  $\catE^{\to} \times \catE^{\to} \to \catE^{\to}$, which equips the arrow category of $\catE$ with a symmetric monoidal structure, with unit the unique map $\hatunit \co \initial \to \unit$.
Given a functor $u \co \cal{C} \to \catE^\to$, we wish to define a category~$\cal{C}_\interval$ and a functor $u_\interval \co \cal{C}_\interval \to \catE^\to$. First of all, let $\cal{C}_\interval$ be the coproduct $\cal{C}_\interval  \defeq \cal{C} + \cal{C}$.
 Next, let $u_{\intervall} \co \cal{C} \to \catE^\to \, , \; u_{\intervalr} \co \cal{C} \to \catE^\to$ be given by
 \[
u_{\intervall}(i) \defeq  \intervall \hatotimes u_i \; , \quad
u_{\intervalr}(i) \defeq  \intervalr \hatotimes u_i \; , 
\]
for $i \in \cal{C}$, where we used the pushout product on $\catE^\to$. The functor $u_\interval \co \cal{C}_\interval \to \catE^\to$ is then given by the coproduct diagram
\begin{equation}
\label{equ:u-interval}
\vcenter{\hbox{\xymatrix@C=1.2cm{
\cal{C} \ar[r] \ar[dr]_-{u_\intervall} & \cal{C}_\interval \ar[d]^(.4){u_\interval} & \cal{C} \ar[dl]^-{u_\intervalr} \ar[l] \\ 
 & \catE^\to }}}
\end{equation}
Note that, even if $u \co \cal{C} \to \catE^\to$ is an inclusion, $u_\interval \co \cal{C}_\interval \to \catE^\to$ is not.



\begin{definition} A \emph{$(\cal{C}, \interval)$-fibration} is a right $\cal{C}_\interval$-map, \ie pair $(f, \phi)$ consisting of a map $f \co X \to Y$ and a function $\phi$ that assigns diagonal fillers to all diagrams of the form
\[
\xymatrix{
\bullet \ar[r] \ar[d]_{\intervall \hatotimes u_i} & X \ar[d]^p \\
B_i \otimes \interval \ar[r] & Y} \qquad \xymatrix{
\bullet \ar[r] \ar[d]_{\intervalr \hatotimes u_i} & X \ar[d]^p \\
B_i \otimes \interval \ar[r] & Y}
\]
subject to the naturality condition. 
\end{definition}

\begin{example} Consider the interval in $\SSet$ given by $\Delta_1$, with the horn inclusions 
$h^k_1 \co 1 \to \Delta_1$ ($k \in \{ 0, 1 \}$) as endpoints. Recall from \cref{exa-triv-kan-fib} that we
have the inclusion $u \co \cal{M} \to \SSet^\to$. A right $\cal{M}_{\Delta_1}$-map will be called
a \emph{uniform Kan fibration}. We unfold this definition. For a horn inclusion $h^{k}_n \co 1 \to \Delta_1$ and  a monomorphism $i \co A \to B$ their pushout product $h^k_n \hattimes i$ is given by the following pushout diagram
\[
\xymatrix{
 A \ar[r]^{i}  \ar[d]_{h^k_n \times A} &  B \ar@/^1em/[ddr]^{h^k_1 \times B} \ar[d] & \\ 
\Delta_1 \times A \ar@/_1em/[drr]_{\Delta_1 \times i} \ar[r] & \bullet \ar[dr]^-(.35){h^k_1 \hattimes i}  & \\ 
 & & \Delta_1 \times B} 
 \]
 Thus, given the inclusion $u \co \cal{M} \to \SSet^\to$ of decidable morphisms and pullback squares, 
 we define the functor $u_{\Delta_1} \co \cal{M}_{\Delta_1} \co \SSet^\to$ as in~\eqref{equ:u-interval}. 
 Explicitly, we let $\cal{M}_{\Delta_1} = \cal{M} + \cal{M}$ and define $u_{\Delta_1}$ via the universal
 property of coproducts:
  \[
 \xymatrix@R=1.2cm@C=1.2cm{
 \cal{M} \ar[dr]_{u^0}  \ar[r]^{\iota_0} &  \cal{M}_{\Delta_1} \ar[d]^{u_{\Delta_1}} & \cal{M} \ar[dl]^{u^1} \ar[l]_{\iota_1} \\
  & \SSet^\to & }
  \]
  where $u^k(i) = h^k_1 \hattimes i$,   for $k \in \{ 0, 1 \}$ and $i \co A \to B$ in $\cal{M}$. 
A  uniform Kan fibration is a right $\cal{M}_{\Delta_1}$-map, \ie 
a map  $p \co X \to Y$ of simplicial sets equipped with a function $\phi$ that assigns
to every decidable monomorphism $i \co A \to B$, $k \in \{0, 1\}$  and commuting
square a diagram of the form
\[
\xymatrix{
\bullet \ar[r] \ar[d]_{h^k_1 \hattimes i} & X \ar[d]^p \\
\Delta_1 \times B \ar[r] & Y }
\]
a diagonal filler, subject to a naturality condition.  Here, higher-dimensional horns are omitted since they are included indirectly as retracts of one-dimensional horns by Leibniz product with certain subobjects of representables.
 \end{example} 
  
  \begin{remark}  If the symmetric
monoidal structure of~$\catE$ is closed, then so is that of $\catE^\to$. Writing $[X,Y]$ for the internal hom in~$\catE$,
the internal hom $\hatexp(f,g)$ in $\catE^\to$ of $f \co X \to Y$ and $g \co U \to V$, is obtained by the following 
pullback diagram
\[
\xymatrix@R=1.2cm{
[Y, U] \ar@/^2pc/[drr]^{[f,U]} \ar@/_2pc/[ddr]_{[Y,g]}  \ar[dr]^{\hatexp(f,g)} & & \\ 
 & [Y,V] \times_{[X,V]} [X,U]  \ar[d] \ar[r] & [X,U] \ar[d]^{[X,g]} \\
 & [Y,V] \ar[r]_{[f,V]} & [X,V] }
 \]
 \end{remark}


\section{Orthogonality functors} 

We now study categories of left and right maps. For this, we work with a fixed category~$\catE$, without assuming the presence of an interval. First of all, recall from~\cite{garner:small-object-argument} that the function mapping $u \co \cal{C} \to \catE^\to$ to its right orthogonal $\liftr{u} \co \liftr{\cal{C}} \to \cal{E}^\to$ defines the action on objects of the \emph{right orthogonality functor}
\[
\liftr{\brarghole} \co  (\CAT/\catE^{\to})^{\op} \to \CAT/\catE^{\to} \, .
\]
The action  on arrows is defined as follows. Given a commutative triangle of the form
\[
\xymatrix{
\cal{C} \ar[dr]_u \ar[rr]^F & & \cal{D} \ar[dl]^{v} \\
 & \catE^\to }
 \]
we define 
\[
\xymatrix{
\liftr{\cal{D}} \ar[dr]_{\liftr{u}} \ar[rr]^{\liftr{F}} & & \liftr{\cal{D}} \ar[dl]^{\liftr{v}} \\
 & \catE^\to }
\]
as follows: for $(f, \phi) \in \liftr{J}$, we let $\liftr{F}(f,\phi) \defeq (f, \phi_F)$, where $\phi_F(i, s, t) \defeq \phi(Fi, s, t)$. 
Just as the standard orthogonality operations determine a Galois connection between the poset of subsets of arrows in~$\catE$ and its opposite, the orthogonality functors form an adjunction 
\begin{equation}
\label{garner-adjunction}
\begin{gathered}
\xymatrix@C+2em{
  \CAT/\catE^{\to}
  \ar@<5pt>[r]^-{\liftl{\brarghole}}
  \ar@{}[r]|-{\bot}
&
  (\CAT/\catE^{\to})^{\op} \, .
  \ar@<5pt>[l]^-{\liftr{\brarghole}}
}
\end{gathered}
\end{equation}
In the remainder of this section, we extend some useful facts about orthogonality operations to orthogonality functors.






\begin{proposition}
Consider a natural transformation between categories over $\catE^{\to}$:
\[
\xymatrix{
  \cal{C}
  \rrtwocell_G^F{\sigma}
 \ar[dr]_{u}
&&
  \cal{D}
  \ar[dl]^{v}
\\&
  \catE^{\to}
}
\]
Note that this includes the condition $v \sigma = \id_u$.
Then $\liftr{F} = \liftr{G}$ and $\liftl{F} = \liftl{G}$, 
\begin{mathpar}
\xymatrix{
  \liftr{\cal{C}}
  \ar[dr]_{\liftr{u}}
&&
  \liftr{\cal{D}}
  \lltwocell_{\liftr{F}}^{\liftr{G}}{=}
  \ar[dl]^{\liftr{v}}
\\&
  \catE^{\to}
}
\and
\xymatrix{
  \liftl{\cal{C}}
  \ar[dr]_{\liftl{u}}
&&
  \liftl{\cal{D}}
  \lltwocell_{\liftl{F}}^{\liftl{G}}{=}
  \ar[dl]^{\liftl{v}}
\\&
  \catE^{\to}
}
\end{mathpar}
\end{proposition}

\begin{proof} For $(f, \phi) \in \liftr{\cal{D}}$, we have $\liftr{F}(f, \phi) = (f, \phi_F)$ and $\liftr{G}(f, \phi) = (f, \phi_G)$.
We claim that the functions $\phi_F$ and $\phi_G$ coincide. Observe that 
for every $i \in \cal{C}$, we have that $\sigma_i \co v_{Fi}  \Rightarrow v_{Gi}$ is the identity square on $u_i
\co A_i \to B_i$. Hence, by the naturality condition for $\phi$, applied to the diagram 
\[
\xymatrix{
A_i \ar[r]^{\id_{A_i}} \ar[d]_{v_{Fi}}  & A_i \ar[d]^{v_{Gi}} \ar[r]^{s}  & X \ar[d]^{f} \\
B_i \ar[r]_{\id_{B_i}} & B_i \ar[r]_{t} & Y \, ,}
\]
we have  that $\phi_F(i, s, t) = \phi_G(i, s, t)$, as required.
\end{proof} 

\subsection*{Orthogonality and adjoints}
We now consider the effect of adjoint functors on orthogonality. In the standard setting, it is well known that if 
we have classes of maps $\cal{C} \subseteq \cal{E}^\to$ and $\cal{D} \subseteq \cal{F}^\to$ and an adjunction
\[
\xymatrix@C+1em{
  \cal{E}
  \ar@<5pt>[r]^{F}
  \ar@{}[r]|{\bot}
&
  \cal{F}
  \ar@<5pt>[l]^{G}
}
\]
then $F(\cal{C}) \subseteq \liftl{\cal{D}}$ if and only if $\cal{C} \subseteq G(\cal{D})$. Our next lemma provides the counterpart of this fact in our setting.




\begin{proposition} \label{lift-of-adjunction} 
Let $u \co \cal{C} \to \cal{E}^{\to}$ and $v \co \cal{D} \to \cal{F}^{\to}$ be functors and consider an adjunction
\[
\xymatrix@C+1em{
  \cal{E}
  \ar@<5pt>[r]^{F}
  \ar@{}[r]|{\bot}
&
  \cal{F}
  \ar@<5pt>[l]^{G}
}
\]
Then, the following are equivalent:
\begin{enumerate}[(i)] 
\item the  functor $F \co \cal{E}^\to \to \cal{F}^\to$ extends to a functor $F' \co \cal{C} \to \liftl{\cal{D}}$ making the following diagram commute:
\[
\xymatrix@C=1.2cm{
  \cal{C}
  \ar[r]^{F'}
  \ar[d]_{u}
&
  \liftl{\cal{D}}
  \ar[d]^{\liftl{v}}
\\
  \cal{E}^{\to}
  \ar[r]_-{F}
&
  \cal{F}^{\to}\, ,}
\]
\item the functor $G \co \cal{F}^\to \to \cal{E}^\to$ extends to a functor $G' \co \cal{D} \to \liftr{\cal{C}}$, making the following diagram commute:
\[
\xymatrix{
  \liftr{\cal{C}}
  \ar[d]_{\liftr{u}}
&
  \cal{D}
  \ar[l]_{G'} 
  \ar[d]^{v}
\\
  \cal{E}^{\to}
&
  \cal{F}^{\to}
  \ar[l]^{G}
}
\]
\end{enumerate}
\end{proposition}

\begin{proof} Giving a functor $F' \co \cal{C} \to \liftl{\cal{D}}$ as above is the same thing as giving fillers for squares of the form
\[
\xymatrix{
FA \ar[d]_{F u_i} \ar[r] & C \ar[d]^{v_j} \\
FB \ar[r] & D }
\]
natural in $i  \in \cal{C}$ and $j \in \cal{D}$. Similarly, giving a functor $G' \co \cal{D} \to \liftl{\cal{C}}$ as above is the same thing as giving fillers for squares 
of the form
\[
\xymatrix{
A \ar[d]_{u_i} \ar[r] & GC \ar[d]^{Gv_j} \\
B \ar[r] & GD }
\]
 natural in $i \in \cal{C}$ and $j \in \cal{D}$. Since $F \dashv G$, these situations coincide.
\end{proof}

We can apply \cref{lift-of-adjunction} to relate right $\cal{C}$-maps and 
right  $\cal{C}_\interval$-maps.

NEED LIFT OF TERNARY ADJUNCTIONS (AND THEIR LEIBNIZ EXTENSIONS) IN ORDER TO ESTABLISH NEXT COROLLARY.



\begin{corollary} \label{prod-exp-general}
Let $\interval = (\interval, \intervall, \intervalr, \varepsilon)$ be an interval in $\catE$ and
 $u \co \cal{C} \to \catE^\to$ be a functor. For every map $f \co X \to Y$ in $\cal{E}$ 
 the following are equivalent: 
\begin{enumerate}[(i)]
\item $f$ admits the structure of a right $\cal{C}_\interval$-map. 
\item $\hatexp(\intervall, f)$ and $\hatexp(\intervalr, f)$ admit the structure of right $\cal{C}$-maps.
\end{enumerate} 
\end{corollary}


 As a special case of \cref{prod-exp-general}, we have
that for a map $p \co X \to Y$ admits the structure of a uniform Kan fibration 
if and only if 
 the maps $\hatexp(\intervall, p)$ and $\hatexp(\intervalr, p)$ admit the structure of
uniform trivial Kan fibrations. 




\subsection*{Orthogonality and slicing} In the classical setting it is well-known that the right orthogonality operation commutes with slicing, while the left orthogonality operation commutes with coslicing.  In order to provide a counterpart of this fact in our setting, we need some auxiliary definitions. Given a functor $u \co \cal{C} \to \catE^{\to}$ and $X \in \catE$, we define the category $\cal{C}/X$
and a functor $u/X \co \cal{C}/X \to (\cal{E}/X)^\to$ as follows. The category $\cal{C}/X$ has as objects pairs consisting of an object $a \in \cal{C}$ and a commutative triangle of the form
\[
\xymatrix{
A_i \ar[dr] \ar[rr]^{u_i} & & B_i \ar[dl] \\
 & X }
 \]
The functor $u/X \co \cal{C}/X \to (\cal{E}/X)^\to$ sends such a pair to $u_i \co A_i \to B_i$, viewed as a morphism in $\cal{E}/X$. This category fits into the
following pullback diagram:
\[
\xymatrix{
  \cal{C}/X
  \ar[r]
  \ar[d]_{u/X}
  \pullback{dr}
&
  \cal{C}
  \ar[d]^{u}
\\
  (\catE/X)^{\to}
  \ar[r]
&
  \catE^{\to}
}
\]
where we used the functor on arrow categories induced by the forgetful functor $\operatorname{dom} \co \catE/X \to \catE$.  Dually, taking the strict pullback along the map on arrows induced by the forgetful functor 
$\operatorname{cod} \co X/\cal{E} \to \catE$ constructs the \emph{coslice} over $X$:
\[
\xymatrix{
  X/\cal{C}
  \ar[r]
  \ar[d]_{X/u}
  \pullback{dr}
&
  \cal{C}
  \ar[d]^{u}
\\
  (X/\catE)^{\to}
  \ar[r]
&
  \catE^{\to}
}
\]
which also admits an explicit description, dual to the one given above for $\cal{C}/X$. With these definitions in place, we can now state the counterpart in our setting of the familiar commutation between slicing and orthogonality operations. 



\begin{proposition} \hfill 
\label{pitchfork-slicing}
\begin{enumerate}[(i)]
\item The right orthogonality functor commutes with slicing, \ie for every $u \co \cal{C} \to \cal{E}$, we have
\[
  \liftr{\cal{C}}/X = \liftr{(\cal{C}/X)}
\]
as categories over $\cal{E}^\to$.
\item The left orthogonality functor commutes with coslicing, \ie for every $u \co \cal{C} \to \cal{E}$, we have
\[
 \liftl{\cal{C}} \backslash X = \liftl{\cal{C} \backslash X}
\]
as categories over $\cal{E}^\to$.
\end{enumerate}
\end{proposition}

\begin{proof} We only consider (i). The claim follows by unfolding definitions, but we describe the objects of the category explicitly for clarity. They are given by 
tuples consisting of an arrow in~$\cal{E}/X$, 
\[
\xymatrix{
X \ar[dr] \ar[rr]^f  &  & Y \ar[dl] \\
 & X & }
 \]
and a function $\phi$ that assigns a diagonal filler to every diagram in $\cal{E}$ of the form
\[
\xymatrix{
A_i \ar[r] \ar[d]_{u_i} & X \ar[d]^{f} \\
B_i \ar[r] & Y}
\]
where $i \in \cal{C}$, subject to a uniformity condition. 
\end{proof}

\medskip

\subsection*{Orthogonality and retract closure} In the ordinary setting, it is well-known that
applying the left (or right) orthogonality operation to a class of morphisms produces the same result as applying it to its retract closure. 
In order to establish a counterpart of this fact, we need again some definitions. 
Given a  functor $u \co \cal{C} \to \catE^{\to}$, we define its retract closure $\overline{u} \co \overline{\cal{C}} \to \catE^{\to}$ as follows. 
An object of $\overline{\cal{C}}$ is a tuple~$(i, e, \sigma, \tau)$ consisting of an object $i \in \cal{C}$, an arrow $e \in \cal{E}^\to$ together with squares $\sigma \co e \Rightarrow u_i$ and $\rho \co u_i \Rightarrow e$,
which exhibit $e$ as a retract of $u_i$ in  $\catE^{\to}$,  \ie such that $\sigma \cc \rho = \id_e$. 
A morphism $(f, \kappa) \co (i, e, \sigma, \tau) \to (i', e', \sigma', \tau')$ of $\overline{\cal{C}}$  consists of a morphism $f \co i \to i'$ in $\cal{C}$ and a square $\kappa \co e \Rightarrow e'$  such that the following diagram in $\cal{E}^\to$ commutes:
\[
\xymatrix{
  e
  \ar[r]^{\sigma}
    \ar[d]_{\kappa}
&
  u_i
  \ar[r]^{\rho}
  \ar[d]^{u_f}
&
  e
  \ar[d]^{\kappa}
\\
  e'
  \ar[r]_{\sigma'}
&
  u_{i'}
  \ar[r]_{\rho'}
&
  e' \, .
}
\]
The functor $\overline{u} \co \overline{\cal{C}} \to \catE^\to$ is then defined  on objects  by letting 
$\overline{u}(i, e, \sigma, \tau) \defeq e$,
and on morphisms by letting $\overline{u}(f, \kappa) \defeq \kappa$. The operation of retract closure gives a monad: for $u \co \cal{C} \to \catE^{\to}$,
the components of the multiplication and the unit, 
\[
\mu_\cal{C} \co \overline{\overline{\cal{C}}} \to \overline{\cal{C}} \, , \quad
\eta_\cal{C} \co \cal{C} \to \overline{\cal{C}} \, ,
\]
are defined by letting
\[
\mu_\cal{C}((i, e, \sigma,  \rho), e', \sigma', \rho') \defeq (i, e', \sigma \cc \sigma', \rho' \cc \rho) \, , \quad
\eta_\cal{C}(i) \defeq (i, u_i, \id_{u_i}, \id_{u_i}) \, .
\]




\begin{proposition}
\label{retract-closure}
The orthogonality functors send the components of the unit and multiplication of the retract closure monad into natural
isomorphisms, and so for every $u \co \cal{C} \to \catE^\to$, we have isomorphisms of categories
\begin{gather*} 
 \liftr{(\overline{\cal{C}})} \iso \liftr{\cal{C}} \, , \quad
 \liftr{(\overline{\overline{\cal{C}}})} \iso \liftr{\overline{\cal{C}}}  \qquad
 \liftl{(\overline{\cal{C}})} \iso \liftl{\cal{C}} \, , \quad
 \liftl{(\overline{\overline{\cal{C}}})} \iso \liftl{\overline{\cal{C}}}
\end{gather*} 
over $\catE^\to$. \qed
\end{proposition}




\begin{remark} Let $\ret$ denote the \emph{walking retract}, \ie the category with objects $\retA, \retB$ and morphisms generated by $s \co \retA \to \retB$ and $r \co \retB \to \retA$ under the relation $r \cc s = \id_{\retA}$. The retract closure of $u \co \cal{C} \to \catE^\to$ fits into the following diagram, involving strict pullback and left composition:
\[
\xymatrix@C+1em{
  \overline{\cal{C}}
  \ar[r]
  \ar[d]
  \ar@/_2em/[dd]_{\overline{u}}
  \pullback{dr}
&
  \cal{C}
  \ar[d]^{u}
\\
  (\catE^{\to})^{\ret}
  \ar[r]^-{(\catE^{\to})^{\retB}}
  \ar[d]^{(\catE^{\to})^{\retA}}
&
  \catE^{\to}
\\
  \catE^{\to}
}
\]
The unit of the monad is formally induced by $(\catE^{\to})^{\canonical} \co \catE^{\to} \to (\catE^{\to})^{\ret}$ being a section to $(\catE^{\to})^{\retB}$.
\end{remark}


\begin{remark}
\label{retract-closure-slicing}
Taking the retract closure commutes with slicing and coslicing.
\end{remark}

\medskip

\subsection*{Orthogonality and left Kan extensions} 
We conclude this section by considering the interaction between the orthogonality functors and  Kan extensions.




\begin{proposition} Let $F \co \cal{C} \to \cal{D}$ be a fully faithful functor. 
\label{kan-extension-closure}
\begin{enumerate}[(i)]
\item Assuming that the pointwise left Kan extension of 
$u \co \cal{C} \to \catE^{\to}$ along $F$ exists
\[
\xymatrix{
  \cal{C}
  \ar[dr]_{u}
  \ar[rr]^{F}
&&
  \cal{D}
  \ar[dl]^{\Lan_F u}
\\&
  \catE^{\to}
}
\]
then the functor $\liftr{F} \co \liftr{\cal{D}} \to \liftr{\cal{C}}$,  fitting in the diagram
\[
\xymatrix{
  \liftr{\cal{C}}
  \ar[dr]_{\liftr{u}}
&&
  \liftr{\cal{D}}
  \ar[ll]_{\liftr{F}}
  \ar[dl]^{\liftr{(\Lan_F u)}}
\\&
  \catE^{\to}
}
\]
is an isomorphism.
\item Assuming that the pointwise right Kan extension of 
$u \co \cal{C} \to \catE^{\to}$ along $F$ exists
\[
\xymatrix{
  \cal{C}
  \ar[dr]_{u}
  \ar[rr]^{F}
&&
  \cal{D}
  \ar[dl]^{\Ran_F u}
\\&
  \catE^{\to}
}
\]
then the functor $\liftl{F} \co \liftl{\cal{D}} \to \liftl{\cal{C}}$, fitting in the diagram
\[
\xymatrix{
  \liftl{\cal{C}}
  \ar[dr]_{\liftl{u}}
&&
  \liftl{\cal{D}}
  \ar[ll]_{\liftl{F}}
  \ar[dl]^{\liftl{(\Ran_F u)}}
\\&
  \catE^{\to}
}
\]
is an isomorphism. \qed
\end{enumerate}
\end{proposition}


Note that the class of decidable monomorphisms is closed under base change.



\section{Pullback and pushforward on categories of orthogonal maps}

\begin{lemma}
\label{slicing-2-functorial}
Let $u \co \cal{C} \to \catE^{\to}$ be a functor and $f \co X \to Y$ a map in $\catE$. Then, 
\begin{enumerate}[(i)]
\item left composition $f_! \co \calE/X \to \calE/Y$ lifts to a functor $f_!$ between slices of $u$,
\[
\xymatrix@C+1em{
  \cal{C}/X
  \ar[r]^-{f_!}
  \ar[d]_{u/X}
&
  \cal{C}/Y
  \ar[d]^{u/Y}
\\
  (\calE/X)^{\to}
  \ar[r]_-{f_!}
&
  (\calE/Y)^{\to}
}
\]
\item right composition $f^! : \calE/Y \to \calE/X$ lifts to a functor $f^!$ between slices of $u$,
\[
\xymatrix@C+1em{
  \cal{C}/X
  \ar[d]_{u/X}
&
  \cal{C}/Y
  \ar[l]_-{f^!}
  \ar[d]^{u/Y}
\\
  (\calE/X)^{\to}
&
  (\calE/Y)^{\to}
  \ar[l]^-{f^!}
}
\]
\end{enumerate}
\end{lemma}

\begin{proposition}
\label{lift-pullback}
Let $f : X \to Y$ be a map in $\catE$ admitting pullback:
\[
\xymatrix@C+1em{
  \catE/X
  \ar@<-5pt>[r]_{f_!}
  \ar@{}[r]|{\top}
&
  \catE/Y
  \ar@<-5pt>[l]_{f^*}
}
\]
Let $u : \cal{C} \to \catE^{\to}$ be a functor.
Then pullback along $f$ lifts to slices of the right orthogonality categories:
\[
\xymatrix@C=1.5cm{
  \liftr{\cal{C}}/X
  \ar[d]_{\liftr{u}/X}
&
  \liftr{\cal{C}}/Y
  \ar@{.>}[l]_{f^*}
  \ar[d]^{\liftr{u}/Y}
\\
  (\catE/X)^{\to}
&
  (\catE/Y)^{\to}
  \ar[l]^{f^*}
}
\]
\end{proposition}

\begin{proof}
Combine \cref{lift-of-adjunction} with part (i) of \cref{slicing-2-functorial}.
\end{proof}

\begin{question}
In fact, $\liftr{C}/\text{--}$ is a cartesian fibration (compare notes-on-awfs).
Do we need that?
\end{question}

The next proposition applies the development in \cref{sec-orthog-functors} to obtain 
a fact about the interaction of orthogonality functors with the pullback and pushforward
functors. This will be useful to establish our main result.

\begin{proposition}
\label{lift-dependent-product}
Let $f \co X \to Y$ be a map in $\catE$ admitting pullback and pushforward:
\[
\xymatrix@C+1em{
  \catE/X
  \ar@<5pt>[r]^{f_*}
  \ar@{}[r]|{\top}
&
  \catE/Y
  \ar@<5pt>[l]^{f^*}
}
\]
Let $u \co \cal{C} \to \catE^{\to}$ be a functor. The following are
equivalent:
\begin{enumerate}[(i)]
\item lifts of the pullback functor $f^*$ of the form
\[
\xymatrix@C=1.5cm{
  \cal{C}/Y
   \ar[r]^{f^*}
  \ar[d]_{u/Y} 
  &
  \liftl{ ( \liftr{\cal{C}}/X ) }
  \ar[d]^{\liftl{(\liftr{u}/X)}}
     \\
     (\catE/Y)^{\to} \ar[r]_{f^*} &
   (\catE/X)^{\to} 
}
\]
\item lifts of pushforward functor $f_*$ of the form
\[
\xymatrix@C=1.5cm{
\liftr{\cal{C}}/X
\ar[r]^{f_*}
  \ar[d]_{u/X}
&
  \liftr{\cal{C}}/Y
  \ar[d]^{\liftr{u}/Y}
\\
  (\catE/X)^{\to}
   \ar[r]_{f_*}
&
  (\catE/Y)^{\to}
 }
\]
\begin{comment}
\item functors $F$ making the following diagram commute:
\[
\xymatrix@C=1.2cm@R=1.5cm{
\liftr{\cal{C}}/X \ar[rr]^F \ar[dr]_{\liftr{u}/X} & &  \liftr{\cal{C}}/ Y \ar[dl]^(.4){\ \liftr{( (u/Y) \cc f^*)}}  \\
 & (\cal{E}/X)^\to & }
\]
\end{comment}
\end{enumerate}
\end{proposition}

\begin{proof}
Recall from \cref{pitchfork-slicing} that slicing commutes with the right orthogonality functor.
%For the first correspondence, apply \cref{lift-of-adjunction} to the adjunction $p^* \dashv p_*$ with $v = \liftr{u}$.
%The last statement is simply the adjunction~\eqref{garner-adjunction}.
Now apply \cref{lift-of-adjunction} to the adjunction $p^* \dashv p_*$ with $u = u/X$ and $v = \liftr{u}/Y$.
\end{proof}

It is possible to show that lifts of the Beck-Chevalley condition, defined in appropriate sense, are equivalent as well, as we shall now see. Let us begin by recalling one of the standard facts about the Beck-Chevalley condition. Consider a
pullback square 
\[
\xymatrix{
  U
  \ar[d]_{s}
  \ar[r]^{g}
  \pullback{dr}
&
  V
  \ar[d]^{t}
\\
  X
  \ar[r]_{f}
&
  Y
}
\]
and assume that $f$ and $g$ admit pullback and pushforward, and that $s$ and $t$ admit pullback. Then,
 the Beck-Chevalley condition for left composition
\[
\xymatrix{
  \catE/U
  \ar[d]_{s_!}
&
  \catE/V
  \ar[l]_{g^*}
  \ar[d]^{t_!}
\\
  \catE/X
&
  \catE/Y
  \ar[l]^{f^*}
}
\]
is equivalent to the Beck-Chevalley condition for pushforward:
\[
\xymatrix{
  \catE/U
  \ar[r]^{g_*}
&
  \catE/V
\\
  \catE/X
  \ar[r]_{f_*}
  \ar[u]^{s^*}
&
  \catE/Y
  \ar[u]_{t^*}
}
\]





\begin{proposition}
\label{lift-pushforward-BC} Let $u \co \cal{C} \to \catE^{\to}$ be a functor and consider a 
pullback square
\[
\xymatrix{
  U
  \ar[d]_{s}
  \ar[r]^{g}
  \pullback{dr}
&
  V
  \ar[d]^{t}
\\
  X
  \ar[r]_{f}
&
  Y
}
\]
Assume now that $f$ and $g$ satisfy the equivalent conditions of \cref{lift-dependent-product}.
Then the following are equivalent:
\begin{enumerate}[(i)]
\item the Beck-Chevalley condition for lifts of pullback, \ie the diagram
\[
\xymatrix{
  \liftl{(\liftr{\cal{C}})}/U
  \ar[d]_{s_!}
&
  \cal{C}/V
  \ar[l]_-{g^*}
  \ar[d]^{t_!}
\\
  \liftl{(\liftr{\cal{C}})}/X
&
  \cal{C}/Y
  \ar[l]^-{f^*}
}
\]
commutes. 
\item the Beck-Chevalley  condition for lifts of pushforward, \ie the diagram
\[
\xymatrix{
  \liftr{\cal{C}}/U
  \ar[r]^{g_*}
&
  \liftr{\cal{C}}/V
\\
  \liftr{\cal{C}}/X
  \ar[r]_{f_*}
  \ar[u]^{s^*}
&
  \liftr{\cal{C}}/Y
  \ar[u]_{t^*}
}
\]
commutes.
\end{enumerate}
\end{proposition}

\begin{proof}
Recall from \cref{pitchfork-slicing} that slicing commutes with the right orthogonality functor.
Now apply \cref{lift-of-adjunction} in the form of a natural correspondence (not just a logical equivalence) with $u = u/V$ and $v = \liftr{u}/X$ while noting that the construction of \cref{lift-of-adjunction} as applied in \cref{lift-pullback} and \cref{lift-dependent-product} composes (meaning the correspondence of \cref{lift-of-adjunction} commutes with composition of adjunctions).
\end{proof}



\section{Two algebraic weak factorisation systems on simplicial sets}
\label{trivial-kan-fibrations}

 
 

 
Classically, a map is a  trivial Kan fibration if and only if it has the right lifting property with respect to the
set of boundary inclusions $i_n \co \partial \Delta^n \to \Delta^n$. We wish to establish a counterpart of 
this fact in our setting, by showing that the category of uniform trivial Kan fibrations can also be 
characterized as the right orthogonal category of  a small category.







\medskip

\newcommand{\yon}{\mathrm{y}} 

In order to do this, let us briefly return to consider the setting of~\cref{sec:ortf} and prove two useful lemmas,
for which we make the further assumption that $\catE$ is a presheaf category, \ie $\catE = \hat{\cat{C}}$, where $\catC$ is some small category. We write $\yon \co \cat{C} \to \catE$ for the Yoneda embedding.

\begin{lemma}
\label{left-kan-extension-of-representables}
Let $\cal{D}$ be a full subcategory of $\catE_{\cart}^{\to}$ closed under base change to representables.
Let $\cal{C}$ denote its restriction to arrows into representables.
\[
\xymatrix{
  \cal{C}
  \ar[rr]
  \ar[dr]
&&
  \cal{D}
  \ar[dl]
\\&
  \catE^{\to}
}
\]
Then, the inclusion $\cal{D} \to \catE^{\to}$ is the left Kan extension of $\cal{C} \to \catE^{\to}$ along $\cal{C} \to \cal{D}$.
\end{lemma}



\begin{proof}
Since $\catE^{\to}$ is cocomplete, we can verify the claim using  the colimit formula for left Kan extensions.
All of the following will be functorial in an object $j \co A \to B$ of $\cal{D}$.
We consider the diagram indexed by cartesian squares of the form
\[
\xymatrix@C=1.2cm{
  A'
  \ar[r]
  \ar[d]_{i}
  \pullback{dr}
&
  A
  \ar[d]^{j}
\\
  \yon(c) 
  \ar[r]_-b 
&
  B
}
\]
with $i \co A' \to \yon(c)$ in $\cal{C}$ and valued $i$.
Our goal is to show that its colimit of this diagram in $\catE^{\to}$ is $j$.
Using the assumption that $\cal{D}$ is closed under pullback to representables, the given diagram
can be described equivalently as the the diagram indexed by maps $b \co \yon(c) \to B$ and valued $b^*(j)$. The claim can then be restated as  $\colim_{b : \yon(c) \to B} b^*(j) \iso j$, which 
holds since pullback commutes with colimits in presheaf categories, and  $\colim_{b : \yon(c) \to B} \yon(c) \iso B$.
\end{proof}


\begin{remark} It would be of interest to prove \cref{left-kan-extension-of-representables} by combining 
the codomain fibration and the corresponding left Kan extension claim for the codomain part
\[
\xymatrix{
  \cat{C}
  \ar[rr]^{y}
  \ar[dr]_{y}
&&
  \hat{\catC}
  \ar[dl]^{\id}
\\&
  \hat{\cat{C}}
}
\]
which holds by the co-Yoneda lemma.
\end{remark}



\begin{lemma}
\label{awfs-on-arrows-into-representables}
Let $\cal{D}$ be a full subcategory of $\catE_{\cart}^{\to}$ closed under base change to representables.
Let $\cal{C}$ denote its restriction to arrows into representables.
\[
\xymatrix{
  \cal{C}
  \ar[rr]
  \ar[dr]
&&
  \cal{D}
  \ar[dl]
\\&
  \catE^{\to}
}
\]
Then $\liftr{\cal{C}} = \liftr{\cal{D}}$.
\end{lemma}

\begin{proof} The result follows by combining \cref{left-kan-extension-of-representables} and part~(i) of \cref{kan-extension-closure}. 
\end{proof}


\begin{theorem} \label{small-gen-triv-kan}
The category of uniform trivial Kan fibrations is isomorphic to the right orthogonality 
category of the following full subcategories of  the category $\cal{M}$ of decidable
monomorphisms and cartesian squares:
\begin{align*}
\cal{M}_1 & = \braces{i \co A \rightarrow \Delta^{n} \ | \ i \text{ is a  decidable monomorphism} } \\ 
\cal{M}_2  & = \braces{i \co A \rightarrow \Delta^{n_1} \times \ldots \times \Delta^{n_k} 
\ | \ i \text{ is a  decidable monomorphism} }  \\
\cal{M}_3  & = \braces{ i \co A \rightarrow B \mid \ i \text{ is a decidable monomorphism and 
$B$ is finite and finite-dimensional}} 
\end{align*}
\end{theorem}

\begin{proof}    \cref{awfs-on-arrows-into-representables} implies that we have that $\liftr{\cal{M}_1}  = \liftr{\cal{M}}$.
For the other equalities, observe that for every full subcategory $\cal{S} \subseteq \cal{M}$ containing $\cal{M}_1$ we have that $\liftr{\cal{M}} = \liftr{\cal{S}}$, since $\cal{M}_1$ is the restriction to maps into representables  of $\cal{M}$. 
\end{proof}

The classes of maps considered in~\cref{small-gen-triv-kan} have different advantages. For example, 
the category $\cal{M}_1$ is small, while other classes have better closure properties. 




\begin{corollary} There exists an algebraic weak factorisation system $(\mathsf{L}, \mathsf{R})$ on
$\SSet$ such that the category of $\mathsf{R}$-algebras is the category of uniform trivial Kan fibrations. 
In particular, there is a functorial factorisation of maps of simplicial sets which sends
a map $f \co X \to Y$ to a diagram of the form
\[
\xymatrix{ 
X \ar[rr]^f \ar[dr]_{i_f}  & & Y \\
 & C_f \ar[ur]_{p_f} }
 \]
 where $p_f$ admits the structure of  a uniform trivial Kan fibration and 
 $i_f$ admits the structure of a $\mathsf{L}$-coalgebra.
\end{corollary}

\begin{proof} Since the category $\cal{M}_1$ defined in~\cref{small-gen-triv-kan} is small and
the inclusion functor $\cal{M}_1 \hookrightarrow \catE^\to$ preserves $\omega$-filtered colimits, 
it is possible to apply Garner's small object argument to
obtain an algebraic weak factorisation system $(\mathsf{L}, \mathsf{R})$.
The fact that the category of $\mathsf{R}$-algebras is the category of uniform trivial Kan fibrations
 follows from \cref{small-gen-triv-kan}.
 \end{proof} 

Note that ~\cref{awfs-on-arrows-into-representables} cannot be used to show 
a map $p \co X \to Y$ of simplicial sets admits the structure of a uniform trivial Kan fibration if and only if it is a trivial Kan fibration in the usual sense since  the class of boundary inclusions  is contained strictly  in~$\cal{C}$. 

\begin{corollary} There exists an algebraic weak factorisation system $(\mathsf{L}, \mathsf{R})$
such that the category of $\mathsf{R}$-algebras is the category of uniform Kan fibrations. 
In particular, there is a functorial factorisation of maps of simplicial sets which sends
a map $f \co X \to Y$ to a diagram of the form
\[
\xymatrix{ 
X \ar[rr]^f \ar[dr]_{i_f}  & & Y \\
 & P_f \ar[ur]_{p_f} }
 \]
 where $p_f$ admits the structure of  a uniform Kan fibration and 
 $i_f$ admits the structure of a $\mathsf{L}$-coalgebra.
\end{corollary} 

\begin{proof} The claim follows from Garner's small object argument, once we find a 
functor $u \co \cal{C} \rightarrow \catE^\to$ such that $\cal{C}$ is small and the
category of uniform Kan fibrations is isomorphic to $\liftr{\cal{C}}$. If we consider
the inclusion $u \co \mathcal{M}_1 \to \cal{E}^\to$ and perform the construction
$u_\interval \co (\mathcal{M}_1)_\interval \to \catE^\to$, the result follows 
by \cref{small-gen-triv-kan} and \cref{prod-exp-general}. 
\end{proof}



\section{Kan fibrations vs. uniform Kan fibrations}
\label{section-kan-fib}

 The aim of this section is to compare the standard notion of a Kan fibration with the
 notion of a uniform Kan fibration: we will show that a map $p \co X \to Y$ of simplicial
 sets is a Kan fibration if and only if it can be equipped with the structure of a uniform
 Kan fibration. In order to do so, we first establish the corresponding result for 
 trivial Kan fibration. For this, it is useful to define the following subcategory $\cal{C} \subseteq \SSet^\to$. The objects are the boundary inclusions
$i_n \co \partial \Delta_n \to \Delta_n$ and the identity maps $\id_{\Delta_n} \co \Delta_n \to \Delta_n$; the
maps are the identity squares and those of the form
 \[
\xymatrix@C=1.2cm{
  \partial \Delta_n
  \ar[r]
  \ar[d]_{i_n}
&
  \Delta_{n-1}
  \ar[d]^{\id_{\Delta_{n-1}}}
\\
  \Delta_n
  \ar[r]_-{s^k_{n-1}}
&
  \Delta_{n-1}
}
\] 


\begin{definition} A \emph{regular trivial Kan fibration} is a right $\cal{C}$-map, \ie a map $p \co X \to Y$ 
equipped with a function that assigns fillers to all squares of the form
\begin{equation}
\label{equ:boundary-filler}
\xycenter{
\partial \Delta_n \ar[d]_{i_n} \ar[r] & X \ar[d]^{p} \\
\Delta_n \ar[r] & Y } 
\end{equation}
subject to the following naturality condition: for every diagram of the form
\begin{equation}
\label{equ:factor-via-id}
{\vcenter{\hbox{\xymatrix@C=1.2cm{
  \partial \Delta_n
  \ar[r]
  \ar[d]_{i_n}
&
  \Delta_{n-1}
  \ar[r]
  \ar[d]^{\id_{\Delta_{n-1}}}
&
  X
  \ar[d]^{p}
\\
  \Delta_n
  \ar[r]_-{s_k^{n-1}}
&
  \Delta_{n-1}
  \ar[r]
&
  Y
}}}}
\end{equation}
the composite filler is coherent with respect to the trivial filler in the right square. 
\end{definition}


\begin{lemma}[ZFC] \label{triv-Kan-is-regular}
Every trivial Kan fibration admits the structure of a regular trivial Kan fibration.
\end{lemma}

\begin{proof} By the axiom of choice, we can choose  designated fillers for squares as in~\eqref{equ:boundary-filler}
 based on (using excluded middle) whether that square factors as in~\eqref{equ:factor-via-id}. Note that it does not matter which degeneracy we choose if multiple are available, since the resulting diagonal filler will be coherent with 
 all possible choices.
\end{proof} 


\begin{lemma} \label{reg-triv-is-unif-Kan}
Every regular trivial Kan fibration admits the structure of a uniform trivial Kan fibration.
\end{lemma}

\begin{proof} Let us consider a map $p \co X \to Y$ equipped with the structure of a 
regular trivial fibration. By \cref{small-gen-triv-kan}, it is sufficient to show that $p$
can be equipped with the structure of a right $\cal{M}_1$-map, where $\cal{M}_1$
is the full subcategory of $\SSet^{\mathbf{2}}_\cart$  spanned by monomorphisms into representables.
So, let us consider a square of the form
\[
\xymatrix{
A \ar[d]_i \ar[r] & X \ar[d]^p \\
\Delta_n \ar[r]  & Y }
\]
where $i$ is a decidable monomorphism. 
We define a diagonal filler by decomposing $i$ into a finite composition of cobase changes of boundary inclusions, filling each of these using~\cref{triv-Kan-is-regular}.
Crucially, this process is independent of the actual order of the boundary fillings (note that this is not true for the analogous situation of horn fillings). In order to prove the naturality condition of uniform trivial Kan fibrations, 
let us consider a diagram of the form
\[
\xymatrix{
  A
  \ar[r]
  \ar[d]_i
  \pullback{dr}
&
  B
  \ar[d]_j 
  \ar[r]
&
  X
  \ar[d]^p 
\\
  \Delta_{n}
  \ar[r]
&
  \Delta_{m}
  \ar[r]
&
  Y
}
\]
where the left-hand side square is a pullback. 
By ``vertical'' induction and the remark on order invariance of boundary fillings, it will suffice to study the case where the middle vertical map is a boundary inclusion $i_n \co \partial \Delta_n \to \Delta_n$.
Working ``horizontally'', it suffices to study the situation where the map $\Delta_{m} \to \Delta_n$ is a face or degeneracy map as $\Delta$ is generated by these.

Let us first examine the case of a face operation.
\[
\xymatrix{
  \Delta_n
  \ar[r]
  \ar[d]
  \pullback{dr}
&
  \partial \Delta_{n+1}
  \ar[d]
  \ar[r]
&
  X
  \ar[d]
\\
  \Delta_n
  \ar[r]_{d^k_{n+1}}
&
  \Delta^{n+1}
  \ar[r]
&
  Y
}
\]
Since the left vertical map is necessarily the identity, the filler for the composite square is uniquely determined, so there is no coherence to be verified.

Let us now examine the case of a degeneracy operation.
\[
\xymatrix{
  2 \times \partial \Delta_n
  \ar[r]
  \ar[d]
  \ar@/^2em/[rr]^(0.3){\pi_2}
  \pullback{dr}
&
  \bigcup_{i \neq k, k+1} \Delta_{[n+1] - i}
  \ar[r]
  \ar[d]
  \pullback{dr}
&
  \partial \Delta_n
  \ar[d]
  \ar[r]
&
  X
  \ar[dd]
\\
  2 \times \Delta_n
  \ar[r]
  \ar@/_2em/[rr]_(0.3){\pi_2}
&
  \partial \Delta_{n+1}
  \ar[r]
  \ar[d]
  \ar@{.>}[urr]
  \pullback{ul}
&
  \Delta_n
  \ar[d]
  \ar@{.>}[ur]
\\&
  \Delta_{n+1}
  \ar[r]_{s_k^n}
  \ar@{.>}[uurr]
&
  \Delta_n
  \ar[r]
  \ar@{.>}[uur]
&
  Y
}
\]
The pullback of the boundary inclusion $\partial \Delta_n \to \Delta_n$ along $s^k_n$ decomposes as a cobase change of two parallel boundary inclusions of dimension $n$ followed by a boundary inclusion of dimension $n+1$,
as indicated.
The two parallel boundary fillings are identical copies of the original right square boundary filling, so they cohere as indicated.
Finally, the filling for the boundary inclusion~$\partial \Delta_{n+1} \to \Delta_{n+1}$ coheres as indicated by how boundary filling was originally defined for degenerate squares.
\end{proof}











\begin{theorem}[ZFC]  \hfill 
\begin{enumerate}[(i)]
\item  Every trivial Kan fibration admits the structure of a uniform trivial Kan fibration.
\item Every Kan fibration admits the structure of a uniform  Kan fibration.
\end{enumerate} 
\end{theorem}

\begin{proof} The claim in (i) follows by \cref{triv-Kan-is-regular}  and \cref{reg-triv-is-unif-Kan}. For (ii), let
$p \co X \to Y$ be a Kan fibration. By the non-algebraic counterpart of \cref{prod-exp-general}, it follows 
that $\hatexp(h_k^1, p)$ is a trivial Kan fibration for $k = 0, 1$. The claim then follows  by \cref{prod-exp-general}. 
\end{proof}




\section{Homotopy equivalences} 


\begin{definition}
\label{def:homotopy}
Let $f, g \co X \to Y$ be maps in $\catE$. A \emph{homotopy} $h$ from $f$ to $g$, denoted $h \co f \sim g$, is a morphism $h \co \interval \otimes X \to Y$ such that the following diagrams commute:
\[
\xymatrix@C=1.2cm{
X \ar[r]^-{\intervall \otimes X} \ar[dr]_{f} & \interval \otimes X \ar[d]^{h} & X \ar[dl]^{g} \ar[l]_-{\intervalr \otimes X}  \\
 & Y & }
 \]
\end{definition}




\begin{definition}
\label{def:homotopy-equivalence}
A map $f \co X \to Y$ is called a \emph{left (right) homotopy equivalence} if there exist $g \co Y \to X$ and homotopies $h \co \id_X \sim g \cc f$ and $k \co
\id_Y \sim f \cc g$ (respectively $h \co g \cc f \sim \id_X$ and $k \co f \cc g \sim \id_Y$). Such a left (right) homotopy equivalence is said to be \emph{strong} if the
following diagram commutes:
\[
\xymatrix{
\interval \otimes X \ar[r]^{\interval \otimes f } \ar[d]_{h} & \interval \otimes Y \ar[d]^{k} \\
X \ar[r]_{f} & Y}
\]
and it is said to be \emph{co-strong} if  the following diagram commutes:
\[
\xymatrix{
\interval \otimes Y \ar[r]^{\interval \otimes g } \ar[d]_{k} & \interval \otimes X \ar[d]^{h} \\
Y \ar[r]_{g} & X \, .}
\]
%
%A \emph{deformation retract} is a homotopy equivalence as above where the homotopy $h$ is trivial (note that this makes $f$ and $g$ into a section-retraction pair).
%Dually, a \emph{co-deformation retract} has the homotopy $k$ trivial (with $g$ and $f$ a section-retraction pair).
\end{definition}

\begin{remark} \hfill 
\begin{enumerate}[(i)]
\item  The notion of homotopy equivalence is symmetric and admits an evident duality, and a homotopy equivalence is strong if 
and only if its dual is co-strong. 
\item The notion of a left or right strong homotopy equivalence is a generalization of the notion of a strong deformation retract, which is obtained by requiring also
that the homotopy $h$ is trivial.
\end{enumerate}
\end{remark}




We wish to give an alternative characterisation of strong homotopy equivalences, which will be useful to establish some of their closure properties. We write $\theta$ for the  commutative square
\begin{equation}
\label{trivial-square}
\begin{gathered}
\xymatrix@C+2em{
  \initial
  \ar[r]^{\hatunit}
  \ar[d]_{\hatunit}
&
  \unit
  \ar[d]^{\intervalr}
\\
  \unit
  \ar[r]_{\intervall}
&
  \interval
}
\end{gathered}
\end{equation}
This square,  which will play an important role in  our development, gives us two maps in the arrow category: 
\[
\thetal \co \hatunit \Rightarrow \intervalr  \, , \quad \thetar \co \hatunit \Rightarrow \intervall \,. 
\]
Using the square $\theta$, we provide the following  characterization of strong homotopy equivalences.

\begin{lemma}
\label{strong-h-equiv-as-section}
Let $f \co X  \to Y$ be a morphism in $\catE$.
\begin{enumerate}[(i)]
\item $f$ is a strong left homotopy equivalence if and only if $\thetal \hatotimes f \co f \Rightarrow \intervalr \hatotimes f$ is a section.
\item $f$ is a strong right homotopy equivalence if and only if $\thetar \hatotimes f \co f \Rightarrow  \intervall \hatotimes f$ is a section.
\end{enumerate}
\end{lemma}

\begin{proof}
By duality, it suffices to exhibit the equivalence in (i). To say that $\thetal \hatotimes f \co f \to \intervalr \hatotimes f$ is a section means that
there is retraction $\rho$, as follows:
\[
\xymatrix@C+1em{
  f
  \ar[r]^-{\thetal \hatotimes f}
  \ar[dr]_{\id_f} &   \intervalr \hattimes f \ar[d]^{\rho} \\
&   f
}
\]
First, standard diagram-chasing shows that giving $\rho \co \intervalr \hattimes f \Rightarrow f$ is equivalent  to giving maps $h \co \interval \otimes X \to X$, $g \co Y \to X$, and $k \co \interval \otimes Y \to Y$ such that the following diagrams commute:
\begin{equation}
\label{equ:first-three}
\xycenter{
X \ar[r]^-{\intervalr \otimes X}  \ar[d]_f & \interval \otimes X \ar[d]^{h} \\
Y \ar[r]_{g} & X}  \qquad
\xycenter{
Y \ar[r]^-{\intervalr \otimes Y} \ar[d]_g & \interval \otimes Y \ar[d]^{k} \\
X \ar[r]_f & B} \qquad
\xycenter{ 
\interval \otimes X \ar[d]_h \ar[r]^{I \otimes f} & \interval \otimes Y \ar[d]^k \\
X \ar[r]_{f} & Y }
\end{equation}
Secondly, requiring that $\rho$ is a section to $\theta \hattimes f$ means that the diagrams
\begin{equation}
\label{equ:second-two}
\xycenter{
X \ar[r]^-{\intervall \otimes X} \ar[dr]_{\id_X} & \interval \otimes X \ar[d]^h \\ 
 & X } \qquad
 \xycenter{
 Y \ar[r]^-{\intervall \otimes Y}  \ar[dr]_{\id_Y} & \interval \otimes Y \ar[d]^{k} \\
  & Y} 
\end{equation}
commute. With reference to \cref{def:homotopy-equivalence}, the equations in~\eqref{equ:first-three} provide right endpoint for $h$, 
right endpoint for $k$, and strength for $h$, respectively; while the equations in~\eqref{equ:second-two} provide left endpoints for~$h$ and~$k$, respectively.
\end{proof}

\cref{strong-h-equiv-as-section} entails the following closure properties of strong homotopy equivalences, which are obtained working entirely at the level of arrow categories.

\begin{proposition}
\label{strong-h-equiv-closed-under-monoidal-prod}
If either $f$ or $g$ is a left (respectively, right) strong homotopy equivalence, then so is $f \hatotimes g$.
\end{proposition}

\begin{proof}
Apply \cref{strong-h-equiv-as-section} and use that functors (in this case the Leibniz monoidal product in one variable) preserve sections.
\end{proof}

\begin{proposition}
\label{strong-h-equiv-closed-under-retract}
Left or right strong homotopy equivalences are closed under retracts.
\end{proposition}

\begin{proof}
Use \cref{strong-h-equiv-as-section},  that functors preserve sections, and that  sections are closed under retracts.
\end{proof}






Recall the notion of an adhesive morphism~\cite{garner-lack:adhesive}.  Let $u \co \cal{C} \to \catE^{\to}$ be a subcategory of adhesive morphisms in $\catE$ with morphisms given by cartesian squares. Assume that the subcategory $\cal{C}$ is closed under the monoidal operations of $\catE^{\to}$, \ie that $(\cal{C}, \hatunit, \hatotimes)$ is itself a monoidal category and $u$ preserves the monoidal structure on the nose. 

\newpage


\begin{definition}
We define a category $\cat{S}_l(\cal{I})$ of strong left homotopy equivalences in $\catE$ relative to some category $\cal{I} : \cat{I} \to \catE^{\to}$ of arrows.
The objects consist of objects $i : \cat{I}$ together with data $(g, h, k)$ making $\cal{I}(i)$ into a strong left homotopy equivalence.
A morphism from $i : \cat{I}$ with $\cal{I}(i) : A \to B$ and data $(g, h, k)$ to $i' : \cat{I}$ with $\cal{I}(i') : A' \to B'$ with data $(g', h', k')$ consists of a map $m : i \to i'$ in $\cat{I}$ such that, writing $\cal{I}(m) = (u, v)$ with maps $u : A \to A'$ and $v : B \to B'$, the following diagrams commute:
\begin{mathpar}
\xymatrix{
  B
  \ar[r]^{g}
  \ar[d]^{v}
&
  A
  \ar[d]^{u}
\\
  B'
  \ar[r]^{g'}
&
  A'
}
\and
\xymatrix{
  \interval \otimes A
  \ar[r]^{h}
  \ar[d]^{\interval \otimes u}
&
  A
  \ar[d]^{\interval \otimes u}
\\
  \interval \otimes A'
  \ar[r]^{h'}
&
  A'
}
\and
\xymatrix{
  \interval \otimes B
  \ar[r]^{k}
  \ar[d]^{\interval \otimes v}
&
  B
  \ar[d]^{\interval \otimes v}
\\
  \interval \otimes B'
  \ar[r]^{k'}
&
  B'
}
\end{mathpar}

We have an obvious forgetful functor $\cat{S}_l(\cal{I}) \to \cat{I}$.
Let $\cal{S}_l(\cal{I}) : \cat{S}_l(\cal{I}) \to \catE^{\to}$ be its composition with $\cal{I} : \cat{I} \to \catE^{\to}$.
\end{definition}

\begin{remark}
\label{strong-h-equiv-as-section-algebraic}
Following the proof of \cref{strong-h-equiv-as-section}, the category $\cat{S}_l(\cal{I})$ can isomorphicly be described at the level of the arrow category $\catE^{\to}$ as follows.
An object consists of $i : \cal{I}$ together with a retraction $\rho$ to $\thetal \hatotimes \cal{I}(i)$:
\[
\xymatrix@C+1em{
  \cal{I}(i)
  \ar[r]^-{\thetal \hatotimes \cal{I}(i)}
  \ar[dr]_{\id}
&
  \intervalr \hatotimes \cal{I}(i) \ar[d]^{\rho}
\\&
  \cal{I}(i)
}
\]
A morphism from $(i, \rho)$ to $(i', \rho')$ consists of $\tau : i \to i'$ that coheres with $\rho$ and $\rho'$ as below:
\[
\xymatrix{
  \intervalr \hatotimes \cal{I}(i)
  \ar[r]^-{\rho}
  \ar[d]^{\intervalr \hatotimes \cal{I}(\tau)}
&
  \cal{I}(i)
  \ar[d]^{\cal{I}(\tau)}
\\
  \intervalr \hatotimes \cal{I}(i')
  \ar[r]^-{\rho'}
&
  \cal{I}(i')
}
\]
With this description, the functor $\cal{S}_l(\cal{I})$ maps $(i, \rho)$ to $i$.
\end{remark}

\begin{remark}
We have a dual construction of a category $\cat{S}_r(\cal{I})$ of strong right homotopy equivalences relative to some functor $\cal{I} : \cat{I} \to \catE^{\to}$.
\end{remark}

\begin{lemma}
\label{she-to-retract-closure}
Assume that $\cal{I}$ is closed under tensoring with $\intervalr$.
Then there is functor $\cal{S}_l(\cal{I}) \to \overline{\intervalr \hatotimes \cal{I}}$ of categories over $\catE^{\to}$.
\end{lemma}

\begin{lemma}
\label{horn-times-gen-to-she}]]
Assume that $\intervalr$ is a strong left homotopy equivalence and that $\cal{I}$ is closed under tensoring with $\intervalr$.
Then there is a functor $\intervalr \hatotimes \cal{I} \to \cal{S}_l(\cal{I})$ of categories over $\catE^{\to}$.
\end{lemma}

\begin{proof}
We work with the characterization of \cref{strong-h-equiv-as-section-algebraic}.

Since $\intervalr$ is assumed a strong left homotopy equivalence, we have a retraction $\rho$ as follows:
\[
\xymatrix@C+1em{
  \intervalr
  \ar[r]^-{\thetal \hatotimes \intervalr}
  \ar[dr]_{\id}
&
  \intervalr \hatotimes \intervalr \ar[d]^{\rho}
\\&
  \intervalr
}
\]

Suppose we are given an object $i : \cat{I}$, lying over the arrow $\intervalr \hatotimes \cal{I}(i)$.
We construct its image $(\intervalr \hatotimes i, \rho \hatotimes \cal{I}(i))$ by tensoring the previous diagram with $\cal{I}(i)$, using that $\cal{I}$ is closed under tensoring with $\intervalr$:
\[
\xymatrix@C+2em{
  \cal{I}(\intervalr \hatotimes i)
  \ar[r]^-{\thetal \hatotimes \cal{I}(\intervalr \hatotimes i)}
  \ar[dr]_{\id}
&
  \intervalr \hatotimes \cal{I}(\intervalr \hatotimes i \ar[d]^{\rho \hatotimes i})
\\&
  \cal{I}(\intervalr \hatotimes \cal{I}(i))
}
\]

Suppose we are given a morphism $\tau : i \to i'$ in $\cat{I}$, lying over map $\intervalr \hatotimes \cal{I}(m)$ of arrows.
We send it to the morphism $\intervalr \hatotimes \tau : (\intervalr \hatotimes i, \rho \hatotimes \cal{I}(i)) \to (\intervalr \hatotimes i', \rho \hatotimes \cal{I}(i'))$, again using that $\cal{I}$ is closed under tensoring with $\intervalr$: 
\[
\xymatrix@C+2em{
  \cal{I}(\intervalr \hatotimes i)
  \ar[r]_-{\thetal \hatotimes \cal{I}(\intervalr \hatotimes i)}
  \ar[d]_{\cal{I}(\intervalr \hatotimes \tau)}
  \ar@/^2em/[rr]^{\id}
&
  \intervalr \hatotimes \cal{I}(\intervalr \hatotimes i)
  \ar[r]_-{\rho \hatotimes \cal{I}(i)}
  \ar[d]^{\intervalr \hatotimes \cal{I}(\intervalr \hatotimes \tau)}
&
  \cal{I}(\intervalr \hatotimes i)
  \ar[d]^{\cal{I}(\intervalr \hatotimes \tau)}
\\
  \cal{I}(\intervalr \hatotimes i')
  \ar[r]^-{\thetal \hatotimes \cal{I}(\intervalr \hatotimes i')}
  \ar@/_2em/[rr]_{\id}
&
  \intervalr \hatotimes \cal{I}(\intervalr \hatotimes i')
  \ar[r]^-{\rho' \hatotimes \cal{I}(i)}
&
  \cal{I}(\intervalr \hatotimes i')
}
\]
Note that the right square commutes by interchange.
\end{proof}

\begin{lemma}
\label{strong-h-equiv-base-change-along-fibration}
Strong homotopy equivalences are stable under base change along morphisms having the right lifting property with respect to $\bot \hattimes \canonical_{0 \to X}$ for $X : \catC$.
\end{lemma}

\begin{proof}
Let $f : A \to B$ be a strong homotopy equivalence and $v : B' \to B$ be a map having the lifting property mentioned in the statement.
We consider the following pullback square:
\[
\xymatrix{
  A'
  \ar[r]^{u}
  \ar[d]^{f'}
  \pullback{dr}
&
  A
  \ar[d]^{f}
\\
  B'
  \ar[r]^{v}
&
  B
}
\]
The goal is to show that $f'$ is a strong homotopy equivalence.

According to \cref{strong-h-equiv-as-retraction}, we must construct $(s', t')$ from $(s, t)$ in the following commutative diagram in the arrow category $\catC^{\to}$:
\[
\xymatrix@C+3em{
  \canonical_{0 \to 1} \hattimes f'
  \ar[r]_{\theta \hattimes f'}
  \ar[d]_{\canonical_{0 \to 1} \hattimes (u, v)}
  \ar@/^2em/[rr]^{\id}
&
  \top \hattimes f'
  \ar@{.>}[r]_{(s', t')}
  \ar[d]^{\top \hattimes (u, v)}
&
  \canonical_{0 \to 1} \hattimes f'
  \ar[d]^{\canonical_{0 \to 1} \hattimes (u, v)}
\\
  \canonical_{0 \to 1} \hattimes f
  \ar[r]^{\theta \hattimes f}
  \ar@/_2em/[rr]_{\id}
&
  \top \hattimes f
  \ar[r]^{(s, t)}
&
  \canonical_{0 \to 1} \hattimes f'
}
\]
Since the codomain fibration has $(u, v)$ as cartesian morphism, it will suffice to solve this problem when projected to its base $\catC$:
\[
\xymatrix@C+2em{
  1 \times B'
  \ar[r]_{\bot \times B'}
  \ar[d]_{1 \times v}
  \ar@/^2em/[rr]^{\id}
&
  I \times B'
  \ar@{.>}[r]_{t'}
  \ar[d]^{I \times v}
&
  1 \times B'
  \ar[d]^{1 \times v}
\\
  1 \times B
  \ar[r]^{\bot \times f}
  \ar@/_2em/[rr]_{\id}
&
  I \times B
  \ar[r]^{t}
&
  1 \times B
}
\]
This is a lifting problem of the form $\bot \hattimes \canonical_{0 \to B'} \pitchfork v$, which is solvable by assumption.
\end{proof}

\begin{theorem} Fix $\cal{E}$ with interval $\interval$. Fix $u \co \cal{C} \to \cal{E}^\to$. For every $(\cal{C}, I)$-fibration $p$,
the pullback functor $p^*$ lifts to a functor $p^* \co \cal{C}_I/X \to \cal{C}_I/Y$. 
\end{theorem}

\begin{corollary} Pushforward lifts.
\end{corollary}

\subsection*{Beck-Chevalley} 


\bibliographystyle{alpha}
\bibliography{../../common/uniform-kan-bibliography}

\end{document}

\ProvidesPackage{uniform-kan-prelude}

%% Math packages
\usepackage{amssymb,amsmath,amsthm}
\usepackage{mathtools}
\usepackage{stmaryrd}
\usepackage{enumerate}
\usepackage[all,cmtip,2cell]{xy} % cmtip gives arrow heads like in Shulman's papers
\usepackage{fouridx}
\usepackage{bbm}

%% General packages
\usepackage[utf8]{inputenc}
\usepackage[usenames,dvipsnames]{xcolor}
\usepackage{geometry}
\usepackage{verbatim}
%\usepackage[draft=false]{hyperref}
\usepackage{url}
\usepackage{cleveref}

%%%%%%%%%%%%%%%%%%%%%%%%%%%%%%%%%%%%%%%%%%%%%%%%%%%%%%%%%%%%%%%%%%


% \renewenvironment{proof}[1][\proofname]{\noindent {\sc #1.}\ }{\qed}

\makeatletter
\renewenvironment{proof}[1][\proofname] {\par\pushQED{\qed}\normalfont\topsep6\p@\@plus6\p@\relax\trivlist\item[\hskip\labelsep\sc#1\@addpunct{.}]\ignorespaces}{\popQED\endtrivlist\@endpefalse}
\makeatother

%\makeatletter

%\renewcommand{\section}{\@startsection
  %{section}% name
  %{1}% level
  %{0mm}% indent
  % {1\baselineskip}% beforeskip
  %{0.5\baselineskip}% afterskip
  % {\centering \sc}}% style

%\makeatother

% \renewcommand{\thepart}{\Roman{part}} 

\makeatletter

\renewcommand{\section}{\@startsection
{section}% name
{1}% level
{0mm}% indent
{-\baselineskip}% beforeskip
{0.5\baselineskip}% afterskip
{\large \bfseries}}% style

 %\renewcommand{\subsection}{\@startsection
 %{subsection}% name
 %{2}% level
  %{0mm}% indent
  %{-\baselineskip}% beforeskip
   % {-2ex}% afterskip
   % {\normalfont\normalsize\bfseries}}% styles

 \makeatother

%% Adapted from the HoTT book sources.

%% Theorem environments
\def\defthm#1#2#3#4{
  \newtheorem{#1}[theorem]{#3}
  \newtheorem*{#1*}{#3}
  \newtheorem{#2}[theorem]{#4}
  \newtheorem*{#2*}{#4}
  \crefname{#1}{#3}{#4}
  \crefname{#2}{#4}{#4}  
}

\numberwithin{equation}{section}

\newtheoremstyle{mythm}% 
{10pt}% Space above 
{}% Space below 
{\itshape}% Body font 
{}% Indent amount 
{\sc}%  Theorem head font 
{.}% Punctuation after theorem head 
{.5em}% Space after theorem head 
{}% 

\newtheoremstyle{mydef}% 
{10pt}% Space above 
{3pt}% Space below 
{}% Body font 
{}% Indent amount 
{\sc}%  Theorem head font 
{.}% Punctuation after theorem head 
{.5em}% Space after theorem head 
{}% 

\newtheoremstyle{myrmk}% 
{10pt}% Space above 
{3pt}% Space below 
{}% Body font 
{}% Indent amount 
{\sc}%  Theorem head font 
{.}% Punctuation after theorem head 
{.5em}% Space after theorem head 
{}% 



\theoremstyle{mythm}

\newtheorem{theorem}{Theorem}[section]
\newtheorem*{theorem*}{Theorem}
\crefname{theorem}{Theorem}{Theorems}

\defthm{corollary}{corollaries}{Corollary}{Corollaries}
\defthm{lemma}{lemmata}{Lemma}{Lemmata}
\defthm{proposition}{propositions}{Proposition}{Propositions}

\theoremstyle{mydef}
\defthm{definition}{definitions}{Definition}{Definitions}

\theoremstyle{myrmk}
\defthm{remark}{remarks}{Remark}{Remarks}
\defthm{example}{examples}{Example}{Examples}
\defthm{question}{questions}{Question}{Questions}

\crefname{section}{Section}{Sections}

\swapnumbers

%%% Lambda abstractions.
% Each variable being abstracted over is a separate argument.  If
% there is more than one such argument, they *must* be enclosed in
% braces.  Arguments can be untyped, as in \lam{x}{y}, or typed with a
% colon, as in \lam{x:A}{y:B}. In the latter case, the colons are
% automatically noticed and (with current implementation) the space
% around the colon is reduced.  You can even give more than one variable
% the same type, as in \lam{x,y:A}.
\def\lam#1{{\lambda}\@lamarg#1:\@endlamarg\@ifnextchar\bgroup{.\,\lam}{.\,}}
\def\@lamarg#1:#2\@endlamarg{\if\relax\detokenize{#2}\relax #1\else\@lamvar{\@lameatcolon#2},#1\@endlamvar\fi}
\def\@lamvar#1,#2\@endlamvar{(#2\,{:}\,#1)}
% \def\@lamvar#1,#2{{#2}^{#1}\@ifnextchar,{.\,{\lambda}\@lamvar{#1}}{\let\@endlamvar\relax}}
\def\@lameatcolon#1:{#1}
\let\lamt\lam
% This version silently eats any typing annotation.
\def\lamu#1{{\lambda}\@lamuarg#1:\@endlamuarg\@ifnextchar\bgroup{.\,\lamu}{.\,}}
\def\@lamuarg#1:#2\@endlamuarg{#1}

%%%%%%%%%%%%%%%%%%%%%%%%%%%%%%%%%%%%%%%%%%%%%%%%%%%%%%%%%%%%%%%%%%%%%%%%%%%%%%%%
%% Nicola's stuff

\newdir{ >}{{}*!/-7pt/@{>}}
\newdir{m}{->}
%\newdir{m}{{}*!/-1pt/@{o}}
\newcommand{\xycenter}[1]{\begin{gathered}\xymatrix{#1}\end{gathered}}
%%% Pullback symbols
\newcommand{\pullback}[1]{\save*!/#1-1.2pc/#1:(-1,1)@^{|-}\restore}
\newcommand{\ulpullback}{\pullback{ul}}
\newcommand{\dlpullback}{\pullback{dl}}
\newcommand{\urpullback}{\pullback{ur}}
\newcommand{\drpullback}{\pullback{dr}}

\newcommand{\ie}{\text{i.e.\ }}
\newcommand{\eg}{\text{e.g.\ }}
\newcommand{\cf}{\text{cf.\,}}
\newcommand{\resp}{\text{resp.\ }}
\newcommand{\etal}{\text{et al.\ }}
\newcommand{\etals}{\text{et al.'s\ }}
\newcommand{\myemph}{\textit} 
\newcommand{\changed}{\todo[noline]{Changed}}

\newcommand{\defeq}{=_{\operatorname{def}}}
\newcommand{\co}{\colon}
\newcommand{\iso}{\cong} 
\newcommand{\rev}{\mathit{\vee}}
\newcommand{\op}{\operatorname{op}}
\newcommand{\catequiv}{\simeq} 
\newcommand{\cateq}{\simeq} 
\newcommand{\coend}{\int}


\newcommand{\cat}[1]{\mathbb{#1}}
\newcommand{\catA}{\cat{A}}
\newcommand{\catB}{\cat{B}}
\newcommand{\catC}{\cat{C}}
\newcommand{\catD}{\cat{D}}
\newcommand{\catK}{\cat{K}}
\newcommand{\catM}{\cat{M}}

%% 

\newcommand{\catE}{\cal{E}}


\newcommand{\SSet}{\mathbf{SSet}}
\newcommand{\CSet}{\mathbf{CSet}}
\newcommand{\UU}{\overline{\mathsf{W}}}
\newcommand{\U}{\mathsf{W}}
\newcommand{\Weq}{\operatorname{Eq}}
\newcommand{\Set}{\mathbf{Set}}

%%%%%%%%%%%%%%%%%%%%%%%%%%%%%%%%%%%%%%%%%%%%%%%%%%%%%%%%%%%%%%%%%%%%%%%%%%%%%%%%
%% My stuff

 \UseAllTwocells

\DeclarePairedDelimiter\parens\lparen\rparen
\DeclarePairedDelimiter\floors\lfloor\rfloor
\DeclarePairedDelimiter\ceils\lceil\rceil
\DeclarePairedDelimiter\bracks\lbrack\rbrack
\DeclarePairedDelimiter\brackss\llbrack\rrbrack
\DeclarePairedDelimiter\verts\lvert\rvert
\DeclarePairedDelimiter\angles\langle\rangle
\DeclarePairedDelimiter\braces\lbrace\rbrace

\newcommand{\arghole}{-}
\newcommand{\brarghole}{({}\mathop{\cdot}{})}

\newcommand{\cc}{\mathbin{\circ}}
\newcommand{\id}{\operatorname{id}}
\newcommand{\const}{\operatorname{const}}
\newcommand{\canonical}{\mathord{!}}
\newcommand{\inl}{\operatorname{inl}}
\newcommand{\inr}{\operatorname{inr}}

\newcommand{\cod}{\operatorname{cod}}
\newcommand{\dom}{\operatorname{dom}}
\newcommand{\colim}{\operatorname{colim}}
\newcommand{\Id}{\operatorname{Id}}
\newcommand{\Lan}{\operatorname{Lan}}
\newcommand{\Ran}{\operatorname{Ran}}
\newcommand{\Alg}{\operatorname{Alg}}
\newcommand{\Coalg}{\operatorname{Coalg}}
\newcommand{\Cat}{\mathbf{Cat}}
\newcommand{\CAT}{\mathbf{CAT}}
\newcommand{\Presheaf}{\operatorname{Presheaf}}
\newcommand{\SSetCart}{\SSet_{\cart}^{\to}}
\newcommand{\cart}{\text{\normalfont{cart}}}
\newcommand{\Mnd}{\operatorname{\mathbf{Mnd}}}


\newcommand{\join}{\mathbin{\star}}
\newcommand{\hatjoin}{\mathbin{\hat{\star}}}
\newcommand{\hattimes}{\mathbin{\hat{\times}}}
\newcommand{\hatexp}{\operatorname{\widehat{exp}}}

\newcommand{\classcofib}{\mathcal{C}}
\newcommand{\classfib}{\mathcal{F}}
\newcommand{\classweq}{\mathcal{W}}
\newcommand{\classleft}{\mathcal{L}}
\newcommand{\classright}{\mathcal{R}}

\newcommand{\ptA}{\mathbf{a}}
\newcommand{\ptB}{\mathbf{b}}


\newcommand{\ret}{\mathbf{R}}
\newcommand{\retA}{\mathit{x}}
\newcommand{\retB}{\mathit{y}}

% Stolen from Mike Shulman.
\newcommand{\xto}{\xrightarrow}

\newcommand{\fib}{\twoheadrightarrow}
\newcommand{\acof}{\mathrel{\mathrlap{\hspace{3pt}\raisebox{4pt}{$\scriptscriptstyle\sim$}}\mathord{\rightarrowtail}}}

\newcommand{\cal}[1]{\mathcal{#1}}
\newcommand{\calA}{\cal{A}}
\newcommand{\calB}{\cal{B}}
\newcommand{\calC}{\cal{C}}
\newcommand{\calD}{\cal{D}}
\newcommand{\calE}{\cal{E}}
\newcommand{\calK}{\cal{K}}
\newcommand{\calM}{\cal{M}}

\newcommand{\sql}{{}^L}
\newcommand{\sqr}{{}^R}
\newcommand{\squ}{{}^U}
\newcommand{\sqd}{{}^D}

\newcommand{\squl}{{}^{UL}}
\newcommand{\squr}{{}^{UR}}
\newcommand{\sqdl}{{}^{DL}}
\newcommand{\sqdr}{{}^{DR}}

\newcommand{\sqhori}{{}^{H}}
\newcommand{\sqvert}{{}^{V}}

\newcommand{\Slice}{\operatorname{Slice}}

\newcommand{\TrivFib}[1][]{\def\arg{#1}\ifx\arg\empty\else\arg\text{-}\fi\mathsf{TrivFib}}
\newcommand{\Fib}[1][]{\def\arg{#1}\ifx\arg\empty\else\arg\text{-}\fi\mathsf{Fib}}

\newcommand{\initial}{\bot}
\newcommand{\terminal}{\top}

\newcommand{\liftl}[1]{{\fourIdx{\pitchfork}{}{}{}{#1}}}
\newcommand{\liftr}[1]{{\fourIdx{}{}{\pitchfork}{}{#1}}}
\newcommand{\gliftl}[1]{{\fourIdx{\pitchfork}{}{}{}{#1}}}
\newcommand{\gliftr}[1]{{\fourIdx{}{}{\pitchfork}{}{#1}}}

\newcommand{\unit}{1}
\newcommand{\hatunit}{\hat{\unit}}
\newcommand{\hatotimes}{\mathbin{\hat{\otimes}}}

\newcommand{\eval}{\operatorname{ev}}
\newcommand{\hateval}{\widehat{\eval}}

\newcommand{\cchat}{\mathbin{\widehat{\cc}}}

\newcommand{\interval}{I}
\newcommand{\intervall}{\ell}
\newcommand{\intervalr}{r}
\newcommand{\intervalc}{\varepsilon}
\newcommand{\thetal}{\theta^0}
\newcommand{\thetar}{\theta^1}
\newcommand{\thetak}{\theta^k}
\newcommand{\thetakinv}{\theta^{1-k}}

\newcommand{\cyl}{C}
\newcommand{\lcyl}{\delta^0}
\newcommand{\rcyl}{\delta^1}
\newcommand{\kcyl}{\delta^k}
\newcommand{\kcylinv}{\delta^{1-k}}
\newcommand{\ccyl}{\varepsilon}


\hyphenation{e-pi-mor-phism}
\hyphenation{e-pi-mor-phisms}
\hyphenation{mo-no-mor-phism}
\hyphenation{mo-no-mor-phisms}


\makeatletter
\def\ignorespacesandallpars{%
  \@ifnextchar\par
    {\expandafter\ignorespacesandallpars\@gobble}%
    {}%
}
\makeatother

%\newcommand{\para}[1]{\smallskip\noindent\textbf{{#1}.\ }\ignorespacesandallpars}

% usage: \note[author]{color}{content}
\newcommand{\note}[3][]{\def\auth{#1}\textcolor{#2}{{[\ifx\auth\empty\else\auth: \fi{#3}]}}}

\newcommand{\notec}{\note{Red}}
\newcommand{\noten}{\note{Blue}}

\newcommand{\Psh}{\mathrm{Psh}}

\newcommand{\yon}{\mathrm{y}}

% Reedy stuff

\newcommand{\R}{\mathrm{R}}
% \newcommand{\deg}{\operatorname{d}}
\newcommand{\Rp}{\R^+}
\newcommand{\Rm}{\R^-}
\newcommand{\Sk}{\operatorname{Sk}}
\newcommand{\Cosk}{\operatorname{Cosk}}
\newcommand{\obj}{\operatorname{Obj}}
\newcommand{\Rhat}{\operatorname{Psh}(\R)}

\newcommand{\fakeslice}{\sslash}